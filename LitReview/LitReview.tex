%!TEX root = ../Main.tex

\chapter{Literature review}
It describes the current status of the wheat genome, genetics and other resources.   

\section{Wheat Breeding}
An overview of how breeding is carried on currently, the different sources of genetic diversity and the relevance of fixing agriculturally important traits. 
\section{Wheat Genetics}
The section describes alleles an the concept of gene, both as a locus in the genome (Quantitative Trait Locus, QTL) and an specific transcript (central dogma of molecular biology). Finally, it discuses traditional Mendelian inheritance and the effect of polyploidy.  
\section{Wheat Genomics}
A description of the current status of the wheat genome (\citet{Mayer2014}, \citet{Chapman2015}), the different available assemblies and and approaches to sort the scaffolds (Genome Zipper, the various genetic maps).  
\section{Sequencing} 
The importance of the selection of the library preparation and the sequencing platforms available. A brief summary of RNA-Seq, Exome capture, Whole Genome Shotgun, etc. and on which cases are more suitable for different experiments.  Mention the new technologies developed during the years of the PhD (Ren-Seq, PacBio?)
\section{Sequence analysis}
This section discusses the criteria to decide analysis done after sequencing, when to do re-alignments or \textit{de novo} assemblies, how to do SNP calling in diploid and polyploid organisims and the bulk frequency ratios.  
\section{Wheat online resources}
A compilation of the currently available resource for whet genetics and genomics. MAS wheat, CeralsDB, Ensembl, etc.  