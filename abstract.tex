
\documentclass[12pt,a4paper,oneside]{article}
\usepackage[utf8]{inputenc}
\usepackage[T1]{fontenc}
\usepackage[pdfencoding=auto,psdextra]{hyperref}
\begin{document}
\setlength\parindent{0pt}
\thispagestyle{empty}
%\begin{abstract}

Ricardo Humberto Ramírez-González \hfill  2016 
\vspace{1cm}
\begin{center}
\Huge
Next generation genomics tools \\ for wheat improvement
\end{center}
\normalsize
\vspace{1cm}
In recent years the amount of genomic resources of wheat has increased to the point where manual analysis is unfeasible. 
The aim of this PhD was to develop bioinformatics tools that help answer biological questions relevant to research and breeding by addressing the complexities associated with the wheat genome. 
I took advantage of resources which became publicly available as the analyses were carried out and I developed new approaches, strategies and tools to help accelerate wheat research. 
Chapter 1 reviews the genomic resources used for the thesis, placing them in historic context with the work and analyses performed. 
Chapter 2 describes the development of PolyMarker, a bioinformatics pipeline to design genome-specific primers in a timely and effective manner. 
Examples of different applications of PolyMarker are also included. 
Chapter 3 describes the analysis of an $F_{2}$ population to generate a genetic map for \textit{Yr15}, a gene that provides resistance to yellow rust. 
The SNP calling was done from bulked segregating samples, sequenced with RNA-Seq as a method of reduced representation. 
Chapter 4 describes expVIP, a tool to integrate RNA-Seq experiments in a relational database.
Data from different studies can be visualised simultaneously, enabling comparisons between them. 
Lastly, in Chapter 5 all the data types used for the analysis on each of the previous chapters is integrated into a relational database. 
The discussion further explores how genetic maps, SNP markers, novel SNPs, gene annotations, gene assemblies and gene expression can be used simultaneously in research and breeding programs. 
All the tools and pipelines described in this thesis are open source and are available on: \url{https://github.com/homonecloco}. 

\end{document}
%\end{abstract}



