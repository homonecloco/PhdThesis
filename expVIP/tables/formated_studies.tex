%!TEX root = ../../Main.tex
\begin{sidewaystable}
\caption{Studies with RNA-Seq replicates, sequenced with Illumina, at the time when expVIP was under development.} 
\label{tab:exp:studiesDetails}
\scriptsize
\begin{tabular}{p{2cm}p{3cm}p{7.4cm}p{10.1cm}}
\toprule
 Study Id & Summary of study  & Brief SRA description  & Manuscript  \\
\midrule
 DRP000768 & Phosphate starvation in roots and shoots & Transcriptome profiles of wheat variety Chinese Spring (CS) in response to Pi starvation (-P) for 10 days. & Characterisation of the wheat (\textit{Triticum aestivum L.}) transcriptome by \textit{de novo} assembly for the discovery of phosphate starvation-responsive genes: gene expression in Pi-stressed wheat \citep{Oono2013}. \\
 ERP003465 & fusarium head blight infected spikelets & Near isogenic wheat lines, differing in the presence of the \textit{Fusarium graminearum} FHB-resistance QTL Fhb1 and Qfhs.ifa-5A, under disease pressure (30 and 50 hai) as well as with mock-inoculation  & Quantitative trait loci-dependent analysis of a gene co-expression network associated with Fusarium head blight resistance in bread wheat \textit{Triticum aestivum L.}. \citep{Kugler2013}.  \\
 ERP004505 & grain tissue-specific developmental timecourse & Analysis of the cell type specific expression of homeologous genes in the developing wheat grain & Genome interplay in the grain transcriptome of hexaploid bread wheat \citep{Pfeifer2014}. \\
 SRP004884 & flag leaf downregulation of GPC & Wild type bread wheat plants and GPC RNAi plants 12 days after anthesis  & Effect of the down-regulation of the high Grain Protein Content (GPC) genes on the wheat transcriptome during monocarpic senescence \citep{Cantu2011b}. \\
 SRP013449 & grain tissue-specific developmental timecourse & Transcriptomes of the aleurone and starchy endosperm tissues of the wheat seed (\textit{Triticum aestivum}) at time points critical to the development of the aleurone layer of 6, 9 and 14 days post anthesis. & Gene expression in the developing aleurone and starchy endosperm of wheat \citep{Gillies2012}. \\
 SRP017303 & stripe rust infected seedlings & Pool of stripe rust infected wheat leaves & Genome analyses of the wheat yellow (stripe) rust pathogen \textit{Puccinia striiformis f. sp. tritici} reveal polymorphic and haustorial expressed secreted proteins as candidate effectors \citep{Cantu2013}. \\
 SRP022869 & \textit{Septoria tritici} infected seedlings   & Molecular mechanisms underlying the interplay between fungal pathogenicity and host responses at specific growth phases and the factors triggering disease transition. & Transcriptional Reprogramming of Wheat and the Hemibiotrophic Pathogen \textit{Septoria tritici} during Two Phases of the Compatible Interaction \citep{Yang2013}. \\
 SRP028357 & shoots and leaves of nulli tetra group 1 and group 5 & RNAseq of nulli-tetrasomic wheat lines (shoots and leaves & Patterns of homoeologous gene expression shown by RNA sequencing in hexaploid bread wheat \citep{Leach2014}. \\
 SRP029372 & grain tissue-specific developmental timecourse & Gene expression profiling of morphological stage of developing wheat grain & Evaluation of Assembly Strategies Using RNA-Seq Data Associated with Grain Development of Wheat (\textit{Triticum aestivum L.}; \citealt{Li2013}). \\
 SRP038912               & comparison of stamen, pistil and pistilloidy expression & Transcriptional profiling of pistillody stamen, pistil and stamen in wheat line HTS-1 & Pistillody mutant reveals key insights into stamen and pistil development in wheat (\textit{Triticum aestivum L.}; \citealt{Yang2015}). \\
 SRP041017 & stripe rust and powdery mildew timecourse of infection in seedlings & Transcriptome Divergence and Overlap for Wheat in Response to Stripe rust and Powdery Mildew Pathogen Stress & Large-scale transcriptome comparison reveals distinct gene activations in wheat responding to stripe rust and powdery mildew. \citep{Zhang2014}. \\
 SRP041022  & developmental time-course of synthetic hexaploid & RNAseq of three tissues of nascent allohexaploid wheat and its following generations, their progenitors, and Chinese Spring & mRNA and Small RNA Transcriptomes Reveal Insights into Dynamic Homoeolog Regulation of Allopolyploid Heterosis in Nascent Hexaploid Wheat \citep{Li2014}). \\
 ERP008767 & grain tissue-specific expression at 12 days post anthesis & Inner pericarp, outer pericarp and endosperm layers from developing grain of bread wheat cv. Holdfast at 12 days post-anthesis. & Heterologous expression and transcript analysis of gibberellin biosynthetic genes of grasses reveals novel functionality in the \textit{GA3ox} family \citep{Pearce2015}. \\
 SRP045409 & drought and heat stress time-course in seedlings & RNAseq of 1-week old wheat seedling leaves subjected to drought stress, heat stress and their combination before (0h) and after stress (1h or 6h) & Temporal transcriptome profiling reveals expression partitioning of homeologous genes contributing to heat and drought acclimation in wheat (\textit{Triticum aestivum L.}; \citealt{Liu2015}). \\
  INRA-RNAseq (ERP004714) & developmental time-course of Chinese Spring & Whole transcriptome sequencing of wheat 3B chromosome  & Structural and functional partitioning of bread wheat chromosome 3B \citep{Choulet2014}. \\
 SRP056412 & grain developmental timecourse with 4A dormancy QTL & This study was to identify candidate genes underlying the 4AL QTL for grain dormancy in wheat. RNA was sequenced from pooled NILs segregating for the QTL & Transcriptomic analysis of wheat near-isogenic linesidentifies  \textit{PM19-A1} an \textit{A2} as candidates for a major dormancy QTL \citep{Barrero2015} \\
\bottomrule
\end{tabular}
\end{sidewaystable}
