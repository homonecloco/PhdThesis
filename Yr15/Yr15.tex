%!TEX root = ../Main.tex

\chapter{Genetic map of \textit{Yr15} with RNA-Seq}
\chaptermark{Genetic map of \textit{Yr15}}
\label{yr15}
%This section describes in detail than the paper of \citet{Ramirez-Gonzalez-2014}
 
%Breeding importance of \textit{Yr15} and original source (an introgression of \textit{T. diccocoides}). 

Wheat breeding programs aim to improve the wheat lines available for production.
One of the traits desired in an elite line is the resistance to pathogens, such as \textit{Puccinia striiformis} f. sp.  \textit{tritici}, the fungi responsible of yellow rust.
A source of resistance genes is are introgressions from other species, such as \textit{Triticum diccocides}. 
In the University of Sydney a collection of \glspl{nil} with introgressions to several Yellow Rust resistance genes on a susceptible background were developed \citep{Wellings1998}. 
On this chapter the NIL for the \textit{Yr15} locus is used to produce a mapping population to improve diagnostic markers. 

%TODO: Paragraph explaining NILs

Line selection can be done with molecular markers that can be used to test if certain allele is present in a line, without the need to do a phenotype.
To find which regions are linked to a trait the use of $F_{2}$ mapping populations is a common practice.
The population is produced by crossing two homozygous parents ($P_1$ and $P_{2}$) with different alleles, A/A (dominant) and a/a (recessive).
When the trait is dominant and has a mendelian segregation, the $F_1$ population show the dominant trait, as it has a copy of each allele (A/a). 
The $F_1$ is then self-crossed to and the population segregates with a ration 1:2:1, dominant:heterozygous:recessive respectively.
This generates a population with a phenotype ratio of 3:1 (dominant:recessive), since the effect of the recessive allele is masked by the dominant gene (\citealt{VanOoijen2013}; Figure \ref{fig:yr15:f2schematic}).  

\begin{SCfigure}
  \centering
    \includegraphics[width=0.4\textwidth]{Yr15/Figures/population/F2schematic.pdf}
  \caption{The cross of two homozygous parents, $P_{1}$ and $P_{2}$, with a dominant and a recessive allele of a gene produce an heterozygous $F_{1}$. The $F_{1}$ crossed with itself produce a segregating $F_{2}$ population with a 1:2:1 ratio (A/A:A/a:a/a). The upper and lower cases represent dominant and recessive alleles  }. 
  \label{fig:yr15:f2schematic}
\end{SCfigure}


\gls{bsa} consists on pooling the DNA of individuals with contrasting phenotypes \citep{Michelmore1991} on a segregating population. 
The bulks show as heterozygous except for the region that is linked to the trait of interest. 
This approach can be used to identify SNPs using High Throughput Sequencing, such as: exome capture \citep{Hodges2007}, RNA-Seq \citep{Pickrell2010}, whole genome resquencing \citep{Schneeberger2009}, among others. 


\begin{figure}
\includegraphics[width=1\textwidth]{Yr15/Figures/bfr.pdf}
\caption{Illustration of a non-informative homoeologous SNP (G181T) present in both parental lines, and an informative allelic SNP (G184A), only present in the resistant progenitor Avocet S + Yr15. The consensus sequences from the parental genotypes include this information in the form of ambiguity codes (K and R, respectively). In the bulks, the individual reads align across the reference sequence, with matches indicated by dots, and polymorphisms at positions 181 and 184 indicated by the corresponding nucleotide variants at those positions. The SNP index is calculated as the frequency of the informative allelic SNP in each bulk. The Bulk Frequency Ratio is the quotient of the resistant and susceptible bulk SNP Indexes. Figure previously published in \citet{Ramirez-Gonzalez2015c}. }
\label{fig:yr15:bfr}
\end{figure}


To Call for SNPs from RNA-Seq a reference transcriptome is used as target when aligning the reads. 
The \gls{bfr} methodology can work on organisms has more than one pseudo genome and that the genes are not necessarily fully characterised independently among homoeologues or paralogues, you can have in a single reference collapsing similar regions. 
The UniGenes database, from NCBI, contains the genes of each species, with all the variations of each gene automatically collapsed and represented as with the longest \acrshort{cdna} \citep{PontiusJUWagnerL2002}. 
The \acrshort{ucw}  genes described in \citet{Krasileva2013} contains 94,177 models from tetraploid and hexaploid wheat, assembled and phased to separate different homoeologues. 
Both gene sets are complement each other, however, the \acrshort{ucw} gene models should provide an improved alignment, since the different homoeologues aren't merged in a single model, a possible side effect of the UniGene pipeline. 

Homoeologous variants, as exemplified by the G$>$T variant at position 181; K in consensus (Figure \ref{fig:yr15:bfr}), will produce the same ambiguity code for both parental consensus sequences and can therefore be excluded. 
Real allelic SNPs between the parental genotypes, exemplified by the G$>$A variant at position 184; R in consensus, are distinguished by the presence in one, but not the other parental consensus sequence. 
The allelic SNPs are then examined further with the alignments of the bulks to identified the SNPs that are enriched on the resistant plants.
The SNP index is the proportion of times an alternative allele is observed over the coverage at certain, in the example the the susceptible bulk has an SNP index of $1/8=0.125$ and $6/8=0.75$ for the resistant bulk \citep{Takagi2013a}. traditional
The \acrshort{bfr} are then calculated by dividing the SNP Index of sample containing the target phenotype (resistance) over the sample without the trait (susceptible), on the example is $0.75/0.125=6$.  
A high BFR suggests that the \acrshort{snp} is linked to the target trait \citep{Trick2012}. 
The results of the BFR analysis on the $F_2$ population are discussed in Section \ref{sec:yr15:bfr}. 

\begin{figure}
  \centering
  \includegraphics[width=1\textwidth]{Yr15/Figures/mapping/layersOfMapping.pdf}
  \caption{Layers of information to do \textit{In Silico} mapping. SNPs are called from gene models. The genes and markers from genetic maps are aligned to scaffolds. The order of the markers in a genetic map can be used to sort the scaffolds.} 
  \label{fig:yr15:layersOfMapping}
\end{figure}

There are several layers of information that can be used to add a context to the SNPs. 
When the SNPs are called from genes like the UniGenes \citep{PontiusJUWagnerL2002} or the UCW gene models \citep{Krasileva2013}, the location of the genes can be assigned by aligning them to a genomic reference, even if it is fragmented. 
A source to get the order of the scaffolds are genetic maps previosly published, such as the genetic map described in \citet{Wang2014}, which has the sequence of the markers available.
The markers and the genes can be aligned to the scaffolds with a high percentage of identity (over $98\%$), to avoid them being assigned to an homoeologue or paralogue region in a different chromosome.
The use of genetic maps to sort genomic sequence is frequently used to produce pseudo-chromosomes on genome wide projects, usually with ad-hoc tools \citep{Tang2015}.
Since the \acrshort{css} assembly is quite fragmented the genetic maps don't have enough resolution to produce a pseudomolecule, however it is enough to sort the scaffolds in bins when several markers map to the same location. 
In this way, it is possible to use the scaffolds as a proxy to map the genes to their genetic position (Figure \ref{fig:yr15:layersOfMapping}).
The results of mapping the genes with SNPs to the CSS assembly and the genetic map are described in Section \ref{sub:yr15:inSilico}. 
For a longer description of resources available for wheat see Section \ref{lit:wheatResourcers}. 
\unsure{To do: section talking about genetic map. }
\unsure{To do: Microsatellites vs SNP markers. }

Finally, the best candidate SNPs where selected to produce a genetic map which lead to a triplet of markers diagnostic to the target locus. 

The steps described in this chapter were first published in \citet{Ramirez-Gonzalez2015c} and the results of this chapter are published in \citet{Ramirez-Gonzalez2015b}.

\section{Mapping population}
\label{sub:mappingPopulation}
%\begin{SCfigure}
\begin{figure}
%\begin{wrapfigure}[17]{R!}{7cm}
    \centering
     
     \begin{subfigure}[b]{0.4\textwidth}
        \caption{}
        \includegraphics[width=1\textwidth]{Yr15/Figures/population/Yr15Photo.png}
        \label{fig:yr15.yr15Photo}
    \end{subfigure}
    ~
    \begin{subfigure}[b]{0.4\textwidth}
        \caption{}
        \includegraphics[width=1\textwidth]{Yr15/Figures/population/AVSPhoto.png}
        \label{fig:yr15:avsPhoto}
    \end{subfigure}

     \begin{subfigure}[b]{0.9\textwidth}
     \caption{}
        \includegraphics[width=1\textwidth]{Yr15/Figures/population/F2Population.pdf} 
    \label{fig:yr15:f2}
	\end{subfigure}

    \caption{Response of (\subref{fig:yr15.yr15Photo}) Avocet + \textit{Yr15} and (\subref{fig:yr15:avsPhoto}) Avocet when inoculated with \textit{Puccinia striiformis} f. sp.  \textit{tritici} at the three leaf stage. (\subref{fig:yr15:f2}) The phenotype of the $F_{2}$ population was used to produce 6 bulks, 3 resistant and 2 susceptible. The RNA was pooled in bulks accordingly. Adapted from \citep{Ramirez-Gonzalez2015b}}

%\end{wrapfigure}
\end{figure}
%\end{SCfigure}

The population was developed by crossing the resistant line \gls{yr15} \citep{Wellings1998}, Figure \ref{fig:yr15.yr15Photo}, to the susceptible line \gls{avs}, Figure \ref{fig:yr15:avsPhoto}. 
\acrshort{yr15} is a \gls{nil} of a 6th generation \gls{bc} and the \acrshort{avs} background is highly succeptible to yellow rust, hence the resistance is coffered by the \acrshort{yr15} locus. 
$F_{2}$ seeds from tree independent $F_{1}$ plants where sown and tissue was collected, before the fungal inoculation to avoid the effect of the response on the gene expression.  
The plants were challenged at the three leaf stage as it is know that \textit{Yr15} confers resistance in seedlings \citep{Gerechter-Amitai1989}.
The expected segregation on an $F_{2}$ population is 3:1 (resistant:susceptible), since \textit{Yr15} is a dominant gene.
From the 232 plants in the $F_{2}$ population that germinated, 187 were resistant and 45 were susceptible, which deviates slightly from the expected ratio ($\chi^{2}=0.049$).
Segregation distortion has been shown for the same \textit{Yr15} donnor \citep{Randhawa2009}, however the decresed number of succeptible plants can be explained by escapes in the virulence essays (i.e. plants scored as resistant without the \textit{Yr15} locus).   For this study we extracted DNA from individual plants in the $F_{2}$ population and we bulked RNA on 6 different bulks: 3 resistant and, 3 succeptible ( Figure \ref{fig:yr15:f2}). 

\section{Sequencing and mapping} 
\label{yr15:sequencing}
\begin{table}
\centering
\caption{Arrangement and number of sequenced base pairs per sample. }
\label{tab:yr15:reads}
\begin{tabular}{rrccccc}
\toprule
Library & name & Bar code & Lane   &  Reads (\e{8} bp)\\ 
\midrule
LIB1715 & Bulk R1 & ATCACG & 1 	& 0.77\\
LIB1716 & Bulk R2 & TAGCTT & 1 		& 1.20\\
LIB1717 & Bulk R3 & ACTTGA & 2 	& 0.96  \\ 
LIB1718 & Bulk S1 & GGCTAC & 2 	& 1.64   \\ 
LIB1719 & Bulk S2 & CGTACG & 2 	& 1.49  \\ 
LIB1720 & Bulk S3 & GTGGCC & 1 	&1.88  \\ 
LIB1721 & AvocetS & N/A & 3 		& 4.13 \\ 
LIB1722 & AvocetS + \textit{Yr15} & N/A & 4 	& 3.99  \\ 
\bottomrule
\end{tabular}
\end{table}

RNA-Seq was used to avoid sequencing the non-coding regions and reduce the search space.  
The sequencing of the bulks and the parents were done on a single Illumina Hi-Seq2000 each.
The bulks were multiplexed and sequenced on a third of a lane each, as shown on Table \ref{tab:yr15:reads}. 
To ensure that the quality of the sequencencing was good, \verb|fastqc-0.10| \citep{fastqc}  was run with its default parameters in each one of the fastq files.  
The GC content was around 52\% in all the samples (Appendx \ref{App:AppendixQCGC}), which is expected as the sample should be of coding regions, and for wheat the reported GC content in genes is around 55\%.  
The quality of the reads is fairly consistent, in general dropping after the base 80 across the samples (Appendix \ref{App:AppendixQCRead}). 


\input{Yr15/SupplementalTables/alignedCoverage}

\begin{figure}
\centering
\begin{subfigure}{0.38\textwidth}
    \caption{}
     \includegraphics[width=1\textwidth]{Yr15/Figures/CoveragePerGene.pdf} 
    \label{fig:yr15:covPerGene}
\end{subfigure}
~
\begin{subfigure}{0.58\textwidth}
    \caption{}
    \includegraphics[width=1\textwidth]{Yr15/Figures/PercentageOfSnps.pdf} 
    
    \label{fig:yr15:SNPper}
\end{subfigure}
\caption{(\subref{fig:yr15:covPerGene}) Box plot distribution of the gene coverage of the parent reads (\acrshort{avs} and \acrshort{yr15}) across the UCW (blue) and the UniGene (red) gene models. The dashed line represents the 209 minimum coverage required for SNP calling. The full line represents the average coverage across all gene models. (\subref{fig:yr15:SNPper}) Percentage of genes exhibiting SNPs across references. The number of \acrshort{snp}s between the parent reads and the corresponding references was calculated (per 100 bp, rounded). The ‘between-parents’ category corresponds to putative SNPs when comparing the consensus sequence between \acrshort{avs} and \acrshort{yr15} Adapted from \citet{Ramirez-Gonzalez2015b} }
\end{figure}


When the analysis was started, the draft genome and the corresponding annotation where not not release yet, hence gene models where used. 
All the samples where aligned to the Unigenes v60 (56,954 genes) and the gene models from UCW \citep{Krasileva2013} using \verb|BWA 0.5.9| \citep{Li2009}. 
The alignment provided showed that a few genes were overly expressed, however we still have have 22,107 and 36,808 genes, on the Unigenes and the UCW gene set respectivley,  with a coverage greater than 20x in the progenitor with \textit{Yr15}. 
Both gene sets performed similarly in terms of the percentage of genes with reads and percentage of aligned reads. 
For \gls{avs} and \gls{yr15}, the percentage of genes with a coverage of at least $20x$ is $45\%$ and $39\%$ respetively across both references (Figure \ref{fig:yr15:covPerGene}).
Since each individual bulk has a lower coverage, the susceptible and resistant reads were merged \textit{in silico} as: (i) susceptible bulks 1 with 2 (S1 + S2) and resistant bulks 1 with 2 (R1 + R2) and (ii) all the susceptible (S1 + S2 + S3) and resistant bulks (R1 + R2 + R3). 
The merged samples increased the percentage of genes with coverage over 20x  to 44\% and 50\% in the resistant and susceptible bulks (Table \ref{app:seqAlnCov}), which is close to the coverage from the progenitors.

\section{SNP Calling}
\label{yr15:snpCalling}

The \acrshort{snp} calling was done on positions with a coverage of at least $20x$ on the progenitor lines against the gene reference. The \acrshort{avs} progenitor had roughly $3\%$ more genes with polymorphisms than \acrshort{yr15}, consistent with the difference in coverage, suggesting that with a higher coverage we could recover more \acrshort{snp}s from \acrshort{yr15}.
The UniGenes have a higher number of \acrshort{snp}s because the \acrshort{ucw} gene models have a higher number of monomorphic genes when compared to the UniGenes. (Figure {\ref{fig:yr15:SNPper}}; Table \ref{app:yr15:cntSNP100bp}). 
The difference in the number of relative monomorphic SNPs between references can be explained by the fact that the UniGenes have homoeologues can be represented as a single sequence, as opposed to the UCW set which are homoeologue-specific, improving the mapping to the correct homoeologue in the genes from the UCW set over the UniGenes.

%!TEX root = ../../Main.tex
\begin{table}
\caption{ Count of SNPs per 100 bp on genes with at least 20x coverage. }
\centering
\label{app:yr15:cntSNP100bp}
\begin{localsize}{10}{12}
\begin{tabular}{lrrrrrrr}
\toprule
 SNPs  & \multicolumn{3}{c}{UCW}  &  &  \multicolumn{3}{c}{UniGene v60 }                                 \\
 \cline{2-4}
 \cline{6-8}
\pbox{1cm}{per 100bp}         & AVS   & \pbox{1.5cm}{\centering AVS+ \textit{Yr15}} & \pbox{1.8cm}{\centering Between progenitors} &      & AVS         & \pbox{1.5cm}{\centering AVS+ \textit{Yr15}} & \pbox{1.8cm}{\centering Between progenitors} \\
\midrule
 0               & 67, 389       & 70,338 & 81,921             &      & 36,210       & 38,339      & 47,097               \\
                 & 71.6\% & 74.7\%      & 87.0\%               &      & 63.6\%       & 67.3\%      & 82.7\%               \\
 \midrule
 1               & 16,111 & 14,770      & 10,107               &      & 10,058       & 9,175       & 8,061                \\
                 & 17.1\% & 15.7\%      & 10.7\%               &      & 17.7\%       & 16.1\%      & 14.2\%               \\
 \midrule
 2               & 8,904  & 7,676       & 1,893                &      & 8,529	         & 7,648       & 1,621                \\
                 & 9.5\%  & 8.2\%       & 2.0\%                &      & 15.0\%       & 13.4\%      & 2.9\%                \\
 \midrule
 3               & 1,517  & 1,192       & 215                  &      & 1,870        & 1,568       & 59       \\
                 & 1.6\%  & 1.3\%       & 0.2\%                &      & 3.3\%        & 2.8\%       & 0.3\%                \\
 \midrule
 4+              & 253    & 198         & 38                   &      & 287          & 224         & 16                  \\
                 & 0.3\%  & 0.2\%       & 0.0\%                &      & 0.5\%        & 0.4\%       & 0.0\%                \\
\bottomrule
\end{tabular}
\end{localsize}
\end{table}

\begin{SCfigure}
    %\centering
    \includegraphics[width=0.4\textwidth]{Yr15/Figures/geneCounts.pdf} 
    \caption{Gene models with putative SNPs in common between the UCW and UniGenes reference. The intersection represnts the genes that are common in both sets. Adapted from \citet{Ramirez-Gonzalez2015b}}
    \label{fig:yr15:geneCount}
\end{SCfigure}

Both gene sets were done from varieties different to \acrshort{avs} and are likely to be incomplete, hence we set a low threshold of at least 20\% of the observed nucleotides on any position to call an \acrshort{snp}. 
To represent cases were more than one consensus base is called we use \gls{iuapc} codes (\citet{Cornish-Bowden1985}; Section \ref{lit:ambiguity}; Figure \ref{fig:yr15:bfr}).  
To focus the analysis on informative \acrshort{snp}s, the common varietal SNPs and variations between homoeologues were removed by finding the cases when the consensus call on both progenitors is the same. 
The \acrshort{snp}s that are unique to a single parental were examined in detail. 
There are 66,426 putative SNPs across 16,022 (17\%) \acrshort{ucw} genes and 52,262 \acrshort{snp}s on 11,056 UniGenes (19.4\%; Figure \ref{fig:yr15:geneCount}).  


\input{Yr15/SupplementalTables/genesAssignedtoArm}

The high number of genes with \acrshort{snp}s was unexpected as a \acrshort{bc}6 \acrshort{nil} used for an $F_2$ mapping population expects to have $<1\%$ of the genetic background segregating. 
The both sets of gene models were aligned with BLAT \citep{Kent2002} to the Chinese Spring Chromosome arm survey sequence (CSS; \citealt{Mayer2014}); the alignment resulted on 80,031 (85.0\%) UCW gene models and 41,118 (72.2\%) UniGenes assigned to a chromosome arm (Table \ref{tab:yr15:genesToCSS}). 
The SNPs found in the mapped genes are evenly distributed across all the chromosomes (Figure \ref{fig:yr15:bfr0}), suggesting that the \acrlong{avs} (JIC, UK) used as parent in the $F_{2}$ is different to the \acrlong{avs} used for the \acrshort{yr15} \acrshort{nil} development (University of Sydney, Australia).  

To confirm that the \acrlong{avs} seed stocks from JIC are distinct to the stocks in Sydney  DNA from both stocks was procured and compared with the iSelect 90k wheat SNP chip. 
Between two independent \acrlong{avs} seeds from JIC only 58 out of 71,972 (0.08\%) valid assays were polymorphic. 
Nonetheless, ther are over 5,000 ($>7.5\%$) assays with polymorphisms between  JIC-\acrlong{avs} and \acrlong{avs} from Sydney. 
The different was not expected originally, but considering that the \acrlong{avs} seeds are coming from different stocks and the fact that in both countries commercial varieties with the same name had been released, it is not surprising. 


\section{Bulk Frequency Ratios}
\label{sec:yr15:bfr}

\input{Yr15/SupplementalTables/scoredSNPs}

The objective was find the \acrshort{snp}s enriched on each bulk and hence linked to the phenotype, variations from \acrshort{yr15} to resistance and from \acrshort{avs} to susceptibility in the segregating population. 
Across individual bulks, it was possible to score between 15,261 (24.5\%) to 31,891(48.0\%) \acrshort{snp}s across both reference sets.
On the \textit{in silico} mixes over $95\%$ of SNPs where scored (Table \ref{app:yr15:scoredSNPs}), suggesting that the coverage of individual bulks is not enough to score all the SNPs.  
The scoring was done with the \acrlong{bfr} (\citealt{Trick2012};Figure \ref{fig:yr15:bfr}; Section \ref{yr15:sub:bfr}), which has a value that increases as the \acrshort{yr15} allele is observed more times relatively to the \acrshort{avs} allele.

%!TEX root = ../../Main.tex
\begin{sidewaystable}
\caption{ SNPs in chromosome group 1S vs total number of SNPs with a minimum BFR from 0 to 10. AVS: SNPs coming from \acrlong{avs}. \textit{Yr15}: SNPs coming from \acrlong{yr15}. }
\centering
\label{app:yr15:bfrThresholds}
\begin{localsize}{6}{7}

\begin{tabular}{llp{1cm}p{1cm}p{1cm}p{1cm}p{1cm}p{1cm}p{1cm}p{1cm}p{1cm}p{1cm}}
\toprule
 Min  BFR   & Gene Set    & R1/S1 \textit{Yr15}        & R1/S1 AVS         & R2/S2 \textit{Yr15}         & R2/S2 AVS          & R3/S3 \textit{Yr15}         & R3/S3 AVS          & S1+2/ R1+2 \textit{Yr15}    & S1+2/ R1+2 AVS     & S1+S2+S3/ R1+R2+R3 \textit{Yr15}   & S1+S2+S3/ R1+R2+R3 AVS   \\
\midrule
 0          & UCW         & 308/8,049 (3.83\%) & 305/8,220 (3.71\%) & 505/14,121 (3.58\%) & 556/15,582 (3.57\%) & 532/14,875 (3.58\%) & 623/17,016 (3.66\%) & 670/18,760 (3.57\%) & 885/25,464 (3.48\%) & 860/24,026 (3.58\%)        & 1,505/40,496 (3.72\%)     \\
            & UniGene v60 & 307/7,823 (3.92\%) & 299/7,438 (4.02\%) & 428/12,409 (3.45\%) & 421/12,734 (3.31\%) & 427/12,050 (3.54\%) & 415/12,498 (3.32\%) & 536/15,672 (3.42\%) & 595/20,026 (2.97\%) & 712/19,358 (3.68\%)        & 901/30,380 (2.97\%)       \\
 \midrule
 1          & UCW         & 214/4,415 (4.85\%) & 194/4,108 (4.72\%) & 325/7,603 (4.27\%)  & 314/7,374 (4.26\%)  & 365/7,920 (4.61\%)  & 415/8,850 (4.69\%)  & 426/10,122 (4.21\%) & 494/12,185 (4.05\%) & 539/13,037 (4.13\%)        & 842/19,466 (4.33\%)       \\
            & UniGene v60 & 207/4,474 (4.63\%) & 194/3,630 (5.34\%) & 269/6,649 (4.05\%)  & 269/6,193 (4.34\%)  & 279/6,511 (4.29\%)  & 272/6,436 (4.23\%)  & 329/8,704 (3.78\%)  & 369/9,343 (3.95\%)  & 446/10,860 (4.11\%)        & 541/14,226 (3.80\%)       \\
 \midrule
 2          & UCW         & 92/651 (14.13\%)   & 75/671 (11.18\%)   & 142/1,372 (10.35\%) & 111/1,101 (10.08\%) & 147/1,162 (12.65\%) & 149/1,411 (10.56\%) & 167/1,324 (12.61\%) & 163/1,478 (11.03\%) & 194/1,370 (14.16\%)        & 207/1,765 (11.73\%)       \\
            & UniGene v60 & 77/568 (13.56\%)   & 58/527 (11.01\%)   & 101/1,017 (9.93\%)  & 81/720 (11.25\%)    & 105/775 (13.55\%)   & 84/867 (9.69\%)     & 122/991 (12.31\%)   & 116/973 (11.92\%)   & 145/1,030 (14.08\%)        & 132/1,210 (10.91\%)       \\
 \midrule
 3          & UCW         & 78/299 (26.09\%)   & 45/295 (15.25\%)   & 118/646 (18.27\%)   & 70/409 (17.11\%)    & 123/577 (21.32\%)   & 85/494 (17.21\%)    & 145/673 (21.55\%)   & 98/563 (17.41\%)    & 168/768 (21.88\%)          & 122/665 (18.35\%)         \\
            & UniGene v60 & 65/254 (25.59\%)   & 26/186 (13.98\%)   & 87/499 (17.43\%)    & 54/294 (18.37\%)    & 93/379 (24.54\%)    & 48/315 (15.24\%)    & 107/525 (20.38\%)   & 66/379 (17.41\%)    & 133/617 (21.56\%)          & 78/489 (15.95\%)          \\
 \midrule
 4          & UCW         & 75/232 (32.33\%)   & 28/160 (17.50\%)   & 109/484 (22.52\%)   & 44/217 (20.28\%)    & 105/416 (25.24\%)   & 44/246 (17.89\%)    & 134/539 (24.86\%)   & 53/277 (19.13\%)    & 149/640 (23.28\%)          & 64/323 (19.81\%)          \\
            & UniGene v60 & 63/192 (32.81\%)   & 17/104 (16.35\%)   & 83/390 (21.28\%)    & 29/155 (18.71\%)    & 82/288 (28.47\%)    & 29/173 (16.76\%)    & 104/431 (24.13\%)   & 40/214 (18.69\%)    & 127/519 (24.47\%)          & 29/266 (10.90\%)          \\
 \midrule
 5          & UCW         & 69/202 (34.16\%)   & 19/108 (17.59\%)   & 95/416 (22.84\%)    & 33/138 (23.91\%)    & 96/354 (27.12\%)    & 23/143 (16.08\%)    & 127/477 (26.62\%)   & 28/175 (16.00\%)    & 140/580 (24.14\%)          & 42/222 (18.92\%)          \\
            & UniGene v60 & 58/163 (35.58\%)   & 11/70 (15.71\%)    & 76/337 (22.55\%)    & 14/102 (13.73\%)    & 70/228 (30.70\%)    & 20/112 (17.86\%)    & 100/389 (25.71\%)   & 23/146 (15.75\%)    & 118/469 (25.16\%)          & 21/178 (11.80\%)          \\
 \midrule
 6          & UCW         & 65/179 (36.31\%)   & 12/85 (14.12\%)    & 86/380 (22.63\%)    & 22/98 (22.45\%)     & 87/299 (29.10\%)    & 11/94 (11.70\%)     & 122/429 (28.44\%)   & 21/130 (16.15\%)    & 126/514 (24.51\%)          & 29/165 (17.58\%)          \\
            & UniGene v60 & 57/151 (37.75\%)   & 7/48 (14.58\%)     & 73/300 (24.33\%)    & 6/71     (8.45\%)   & 65/191 (34.03\%)    & 13/84 (15.48\%)     & 98/358 (27.37\%)    & 20/122 (16.39\%)    & 115/439 (26.20\%)          & 16/143 (11.19\%)          \\
 \midrule
 7          & UCW         & 58/161 (36.02\%)   & 11/73 (15.07\%)    & 77/340 (22.65\%)    & 13/74 (17.57\%)     & 73/248 (29.44\%)    & 7/69 (10.14\%)      & 116/393 (29.52\%)   & 20/111 (18.02\%)    & 114/468 (24.36\%)          & 22/143 (15.38\%)          \\
            & UniGene v60 & 56/132 (42.42\%)   & 4/37 (10.81\%)     & 68/273 (24.91\%)    & 5/58    (8.62\%)    & 60/171 (35.09\%)    & 9/64 (14.06\%)      & 94/334 (28.14\%)    & 18/103 (17.48\%)    & 113/412 (27.43\%)          & 16/124 (12.90\%)          \\
 \midrule
 8          & UCW         & 58/149 (38.93\%)   & 10/62 (16.13\%)    & 68/310 (21.94\%)    & 12/59 (20.34\%)     & 66/214 (30.84\%)    & 6/56 (10.71\%)      & 104/359 (28.97\%)   & 17/102 (16.67\%)    & 108/429 (25.17\%)          & 16/119 (13.45\%)          \\
            & UniGene v60 & 55/126 (43.65\%)   & 3/33    (9.09\%)   & 64/255 (25.10\%)    & 5/50 (10.00\%)      & 55/150 (36.67\%)    & 9/55 (16.36\%)      & 91/313 (29.07\%)    & 14/89 (15.73\%)     & 105/376 (27.93\%)          & 15/108 (13.89\%)          \\
 \midrule
 9          & UCW         & 54/135 (40.00\%)   & 8/53 (15.09\%)     & 63/289 (21.80\%)    & 8/51 (15.69\%)      & 61/182 (33.52\%)    & 5/49 (10.20\%)      & 100/331 (30.21\%)   & 15/91 (16.48\%)     & 100/387 (25.84\%)          & 13/106 (12.26\%)          \\
            & UniGene v60 & 53/117 (45.30\%)   & 1/30    (3.33\%)   & 62/244 (25.41\%)    & 5/46 (10.87\%)      & 50/136 (36.76\%)    & 9/48 (18.75\%)      & 88/291 (30.24\%)    & 13/83 (15.66\%)     & 97/345 (28.12\%)           & 12/99 (12.12\%)           \\
 \midrule
 10         & UCW         & 52/126 (41.27\%)   & 8/50 (16.00\%)     & 62/279 (22.22\%)    & 8/50 (16.00\%)      & 56/165 (33.94\%)    & 4/45    (8.89\%)    & 96/309 (31.07\%)    & 14/82 (17.07\%)     & 91/355 (25.63\%)           & 13/100 (13.00\%)          \\
            & UniGene v60 & 50/105 (47.62\%)   & 1/28    (3.57\%)   & 60/226 (26.55\%)    & 5/39 (12.82\%)      & 43/119 (36.13\%)    & 7/45 (15.56\%)      & 85/272 (31.25\%)    & 13/82 (15.85\%)     & 92/318 (28.93\%)           & 12/97 (12.37\%)           \\
\bottomrule
\end{tabular}
\end{localsize}
\end{sidewaystable}


When increasing the minimum BFR threshold, enrichment of SNPs was observed in the short arm of the group 1 chromosomes (1S). 
Without taking in account the BFR, $~3.6\%$ of the SNPs are located in the 1S group, similar to the number of SNPs located in other groups \ref{tab:yr15:genesToCSS}. 
However, when increasing the threshold  (between $BFR > 5 $ and $BFR > 7$) the relative number of SNPs in group 1S increases. 
After $BFR>7$ the gains in relative enrichment only improves marginally, but the number of called SNPs is reduced (Table \ref{app:yr15:bfrThresholds}; Figure \ref{fig:yr15:bfrChange}).
For that reason, SNPs with a $BFR>6$ were selected for further validation. 
The method described by \citet{Trick2012} was extended by including cases where there is a complete lack of coverage in one of the samples ($BFR=\infty$), which is an ideal case where the linkage between the SNP and the phenotype is perfect. 
A total of 1,582 SNPs across 1,173 genes had a $BFR>6$.



\begin{figure}
\includegraphics[width=1\textwidth]{Yr15/Figures/bfrChanges.pdf}
\caption{ Effect of BFR threshold on the number of SNPs across the short arm of chromosome group 1. Figure previously published in \citet{Ramirez-Gonzalez2015b}. }
\label{fig:yr15:bfrChange}
\end{figure}

\section{\textit{In silico} mapping}
\label{sub:yr15:inSilico}
%Mapping of the gene models to the IWGSC CSS \citet{Mayer2014} reference and the location of the SNPs using the genetic map from \citet{Wang2014}.
From the mapped SNPs with a $BFR>6$, 872 of 1470 ($\sim60\%$) were assigned to the chromosomes in group 1 of hexaploid wheat, being the only group with more than $4\%$ of the SNPs assigned to it (Table \ref{app:yr15:bfr6Mapping}). 
From the group 1, the B genome contained the higher proportion of SNPs mapped ($54\%$), having 255 ($54\%$) and 214 ($46\%$) assigned to the long and short arms respectively (Figure \ref{fig:yr15:snpsBFR6Group1}).  
This results are expected since previous studies have located \acrshort{yr15} near the centromere in the short arm of chromosome 1B and, the \acrshort{yr15} introgression contains regions from the long and short arm from \textit{T. diccocoides} \citep{Murphy2009,Peng2000,Grama1997}. 

\begin{SCfigure}
  \centering
    \includegraphics[width=0.5\textwidth]{Yr15/Figures/mapping/snpsBFR6Group1.pdf}
  \caption{Location of SNPs with $BFR>6$ according to the best alignment of the UniGene (red) and UCW (blue) gene models to the flow-sorted group 1 chromosomes from the Chinese Spring Survey sequence (CSS) \citep{Mayer2014}. Figure adapted from \citet{Ramirez-Gonzalez2015b}.} 
  \label{fig:yr15:snpsBFR6Group1}
\end{SCfigure}

\input{Yr15/SupplementalTables/bfr6Mapping}

\begin{figure}
  \centering
  \begin{subfigure}{0.6\textwidth}
  \caption{}
   \label{fig:yr15:snpsBFR6Chr1B}
   \includegraphics[width=1\textwidth]{Yr15/Figures/mapping/snpsBFR6crh1B.pdf}
  \end{subfigure}
  ~
  \begin{subfigure}{0.35\textwidth}
  \caption{}
   \label{fig:yr15:BFRValues1BS}
   \includegraphics[width=1\textwidth]{Yr15/Figures/mapping/BFRValues1BS.pdf}
  \end{subfigure}
\caption{(\subref{fig:yr15:snpsBFR6Chr1B}) Number of SNPs with $BFR>6$ per cM in chromosome 1B. (\subref{fig:yr15:BFRValues1BS}) BFRs of mapped genes with SNPs on chromosome 1B. The area of the circle represents the number of SNPs clustered by location (windows size: 10 cM) and BFR (window size: 5cM). R11 is the only marker near the \acrshort{yr15} locus that had a corresponding position in the genetic map. The percentage of genes with SNPs per cM is also illustrated based on UCW (blue) and UniGene (red) gene models. The centromere is imputed by the centre of a window of 10 cM where the short arm switches to the long in the genetic map. BFRs correspond to those from the mixed in silico bulk S1 + S2 + S3/R1 + R2 + R3. Adapted from \citep{Ramirez-Gonzalez2015b}.} 
\label{fig:yr15:chr1}
\end{figure}

The \acrshort{css} assembly was used as a common reference between the reference genes and the SNPs  40,266 SNP markers published at the time when this analysis was done \citep{Wang2014} to locate the SNPs with a $BFR>6$ (including $BFR=\infty$) in a genomic position (Figures \ref{fig:yr15:chr1}, \ref{fig:yr15:bfrs:0-6}).  
From the 1,582 SNPS across 1,173 genes,  only 678 SNPs ($43\%$, 474 genes) were successfully located in the genetic map. 
Since the \acrshort{css} assembly is quite fragmented, the low percentage of located SNPs can be because not all candidate SNPs had a corresponding scaffold that has at least one of the 40,266 markers in the genetic map. 
Even if the number of located SNPs was not enough to give a position for over $50\%$ of the SNPs from the parental line, the resolution of the genetic position SNPs that were assigned improved over just having the chromosome arm information from the CSS assembly. 
The mapping position further confirmed an enrichment of SNPs near the centromere of chromosome 1B with 325 out of 678 SNPs. 
Furthermore, 311 of those where located within an interval of 30cM (Figures \ref{fig:yr15:bfr6}, \ref{fig:yr15:snpsBFR6Chr1B}). 

Studies in diploid organismis using \acrshort{qtl}-Seq \citep{Takagi2013} or other \acrshort{ngs}-enable genetic approaches \citep{James2013} have shown smooth curves with a defined peak in the region linked to the studied trait. 
In practice, we only observe clusters of SNPs with  enriched \acrshortpl{bfr} near the centromere of chromosome 1B (Figures \ref{fig:yr15:snpsBFR6Chr1B}, \ref{fig:yr15:bfr6}). 

\begin{figure}
	\centering
	\begin{subfigure}{0.4\textwidth}
	\caption{}
	\label{fig:yr15:bfr0}
	\includegraphics[height=0.55\textheight]{Yr15/Figures/mapping/snpsBFR0.pdf}
	\end{subfigure}
	~
	\begin{subfigure}{0.45\textwidth}
	\caption{}
	\label{fig:yr15:bfr6}
	\includegraphics[height=0.55\textheight]{Yr15/Figures/mapping/snpsBFR6.pdf}
	\end{subfigure}
	\caption{Genetic location of genes with SNPs between AVS and Yr15. The colour scale indicates the percentage of genes with SNPs per centi-Morgan (cM) across the 21 wheat chromosomes. The location of the genes was determined by the best alignment to the CSS scaffolds, and the location of these was determined by their position on a genetic map \citep{Wang2014} (\subref{fig:yr15:bfr0}). All the SNPS between progenitors. Note the lack of enrichment across any individual chromosome. (\subref{fig:yr15:bfr6}) SNPs with BFR$>$6. Note the enrichment in Chromosome 1B }
	\label{fig:yr15:bfrs:0-6}
\end{figure}

The location of the clusters with an enrichment of SNPs near the centromere is not expected on a random selection of genes, as the gene density increases with the distance to the centromere \citep{Akhunov2003}. 
This suggests that the experiment was successful on finding \acrshortpl{snp} linked to \acrshort{yr15}. 
There are several factor that prevent a clear peak; like the biases induced by the differential expression, the fragmented reference sequence with scaffolds that are not long enough to go across genetic positions. 
Since there are several SNPs with a high BFR and the genetic map is not enough to locate a single region linked to \acrshort{yr15},  multiple criteria was needed to prioritise SNPs that were more likely to yield on successful genetic markers.

\section{Assay selection} 
\label{yr15:assaySelection}
\begin{figure}
\centering
\includegraphics[width=1\textwidth]{Yr15/Figures/selection/snpSets.pdf}
\caption{Selection criteria for marker design. Venn diagrams based on the three selection criteria (SNP in the short arm of chromosome group 1; SNP has a $BFR>6$; and SNP is from the \acrshort{yr15} parent) for the UCW (blue) and UniGene (red) gene models. The centre diagram shows the intersection between common genes matching all three criteria across both data sets. Note that the numbers are not directly additive as in cases, multiple models from one reference set will relate to a single gene model in the other values. Published in \citep{Ramirez-Gonzalez2015b} }
\label{fig:yr15:snpset}
\end{figure}

Three independent criteria were use to prioritize the SNPs for marker development and validation: 

\begin{description}
\item[High BFR.] SNPs with a $BFR>6$ in at least two independent bulk replicates or in either of the \textit{in silico} mixes were selected to ensure consistency and recover SNPs with a low coverage on a particular bulk. 
\item[Group 1S.] SNPs that are in \acrshort{css} scaffolds in the short arm of chromosome group 1 were selected.
This is to be consistent with the \textit{in silico} genetic map and with previous studies \citep{Murphy2009,Peng2000,Grama1997}.
\item[\acrshort{yr15} parent.] The SNPs should originate from the \acrshort{yr15} parent to ensure that the SNP is coming from the \textit{T. diccocoides} introgression and not from a SNP in the \acrshort{avs} genetic background, who would be less useful in breeding programs with a different background.
\end{description}

Only SNPs meeting the three criteria were selected for further analysis. 

\begin{figure}
\centering
\includegraphics[width=1\textwidth]{Yr15/Figures/selection/selectionDetals.pdf}
\caption{Bulk frequency ratio (BFR) of selected SNPs across the individual bulks and in silico mixes (UCW, red; UniGene, blue). The dotted line represents the BFR threshold of 6 (logarithmic scale). Left: Distribution of the BFRs for each selection criteria and the selected SNPs for validation. The circles on the top of each plot represent the percentage of SNPs with $BFR=\infty$. The Selection may include SNPs with $BFR<6$ when the same SNP has a higher score on the complementing reference (ie. $BFR>6$ on UCW, but $BFR<6$ on UniGenes). Right: The BFR values of selected SNPs were sorted in descending order across the different bulks and according to their origin. Validated SNPs are indicated by open triangles, and SNPs corresponding to markers R5, R8 and R11 are labelled across different bulks and mixes. Note that some SNPs are below the threshold in a specific bulk as they meet the BFR criteria across others. }
\label{fig:yr15:bfrDetailScore}
\end{figure}

With the multiple criteria the number of genes with a putative SNP went down from $>27,000$ to just 175; 77 and 98 from the UniGene and UCW gene sets respectively. 
The selected genes from both references were aligned between references, as they come from independent sources an overlap in the selection between them is expected and, as expected, around half of the genes between gene sets overlap (Figure \ref{fig:yr15:snpset}). 
The 50 SNPs with the highest BFRs, out of the 175 genes, were selected for validation, 15 of them were redundant between references, resulting on 35 SNPs to validate. 

The separate bulks and the \textit{in silico} mixes were evaluated in detail to understand the behaviour and value of having multiple bulks. 
The initial expectation was that as the number of SNPs with $BFR=\infty$ should drop in the mixes, as the improved coverage should reduce the instances were the absence of an allele is because of the lack of coverage on a particular sample. 
However, the opposite happened, the additional coverage in the \textit{in silico} mixes recovered SNPs in genes with a low expression at the time of the sampling (Figure \ref{fig:yr15:bfrDetailScore}).  
Some SNPs were present across al the samples, however the value of the BFR changed depending on the sample(marker R5). 
On some cases a SNP are missing in an individual bulk, but present in the rest of them and in the mixes (marker R8). 
The main reason affecting the scoring is the coverage in the sample for each particular gene, hence an strategy with a consistent coverage would be preferred for this kind of analysis.  
Previous studies have shown that a coverage of $<5x$ is enough to call for SNPs in model organisms with a high-quality reference \citep{Schneeberger2011}.
However, the results on this study are in line with other studies using populations for SNP calling \citep{Abe2012,Takagi2013}. 
The non-uniform distribution of the coverage in RNA-Seq experiments affects the number of reads that can be used to call for SNPs, specially on genes with a low expression level \citep{Mortazavi2008}. 

%!TEX root = ../../Main.tex
\begin{table}
\caption{ Number of genes (and SNPs) with a unique hit ($>99\%$ sequence identity) to a single wheat survey sequence scaffold. }
\label{tab:yr15:mappedGenes}
\centering
\begin{localsize}{9}{11}
\begin{tabular}{llrr@{\extracolsep{6pt}}rr@{\extracolsep{6pt}}rr}

\toprule
 \multicolumn{2}{l}{Chromosome 1}            &  \multicolumn{2}{c}{All SNPs} &  \multicolumn{2}{c}{BFR\ensuremath{>}6 }  &    \multicolumn{2}{c}{ \%  BFR\ensuremath{>}6 }       \\
  \cline{3-4}
  \cline{5-6}
  \cline{7-8}
                &            & SNP        & Genes  & SNP     & Genes  & SNPs               & Genes  \\
\midrule
 UCW            & Unique     & 5,283      & 1,245  & 311     & 214    & 5.89\%              & 17.19\% \\
                & Total      & 8,086      & 1,954  & 486     & 330    & 6.01\%              & 16.89\% \\
                & Percentage & 65.34\%     & 63.72\% & 63.99\%  & 64.85\% &                    &        \\
 \midrule
 UniGene        & Unique     & 3,687      & 745    & 213     & 139    & 5.78\%              & 18.66\% \\
                & Total      & 6,422      & 1,318  & 386     & 246    & 6.01\%              & 18.66\% \\
                & Percentage & 57.41\%     & 56.53\% & 49.17\%  & 56.07\% &                    &        \\
 \midrule
 UCW  & Unique     & 8,970      & 1,990  & 524     & 353    & 5.84\%              & 17.74\% \\
+              & Total      & 14,508     & 3,272  & 872     & 576    & 6.01\%              & 17.60\% \\
 UniGene       & Percentage & 61.83\%     & 60.82\% & 60.09\%  & 61.28\% &                    &        \\
\bottomrule
                &            &            &        &         &        &                    &        \\
\toprule
 All SNPs       &            &  \multicolumn{2}{c}{All SNPs} &  \multicolumn{2}{c}{BFR\ensuremath{>}6 }  &    \multicolumn{2}{c}{ \% BFR\ensuremath{>}6 }         \\
\cline{3-4}
\cline{5-6}
\cline{7-8}
                &            & SNP        & Genes  & SNP     & Genes  & SNPs               & Genes  \\
 \midrule
 UCW            & Unique     & 39,247     & 9,585  & 481     & 368    & 1.23\%              & 3.84\%  \\
                & Total      & 66,426     & 16,022 & 859     & 643    & 1.29\%              & 4.01\%  \\
                & Percentage & 59.08\%     & 59.82\% & 56.00\%  & 57.23\% &                    &        \\
 \midrule
 UniGene        & Unique     & 27,292     & 5,698  & 344     & 252    & 1.26\%              & 4.42\%  \\
                & Total      & 52,262     & 11,056 & 723     & 530    & 1.38\%              & 4.79\%  \\
                & Percentage & 52.22\%     & 51.54\% & 47.58\%  & 47.55\% &                    &        \\
 \midrule
 UCW  & Unique     & 66,539     & 15,283 & 825     & 620    & 1.24\%              & 4.06\%  \\
 +               & Total      & 118,688    & 27,078 & 1,582   & 1,173  & 1.33\%              & 4.33\%  \\
 UniGene       & Percentage & 56.06\%     & 56.44\% & 52.15\%  & 52.86\% &                    &        \\
\bottomrule
\end{tabular}
\end{localsize}
\end{table}


%TODO: maybe move to the discussion. 
Around $60\%$ of the gene models, across both references, had a unique hit with $>99\%$ sequence identity to a single \acrshort{css} scaffold (Table \ref{tab:yr15:mappedGenes}). 
This is likely because there is no unique homoeologue in the gene models, leading to reads, from two different homoeologues, mapping to the same region.
To reduce the number of spurious SNPs we used IUAPC ambiguity codes (Section \ref{lit:ambiguity}, \citet{Cornish-Bowden1985}) when two different alleles were observed.
This had as side effect that in order to keep only high confidence SNPs we required a higher coverage ($>20x$). 
On the original study introducing the BFR in tetraploid wheat, the authors show that increasing the coverage, from $8x$ to $16x$, reduces the putative SNPs by $60\%$, but the validated SNPs increas from $57\%$ to $83\%$ \citep{Trick2012}. 
Hence, a compromise between increasing the minimum coverage at the cost of reducing the SNP candidates has to be reached in line with the objectives and available resources for a particular study. 

\section {SNP Validation}
\begin{figure}
\begin{subfigure}{0.31\textwidth}
\caption{}
\label{fig:yr15:r2}
\includegraphics[width=1\textwidth]{Yr15/Figures/selection/R2.pdf}
\end{subfigure}
~
\begin{subfigure}{0.31\textwidth}
\caption{}
\label{fig:yr15:r8}
\includegraphics[width=1\textwidth]{Yr15/Figures/selection/R8.pdf}
\end{subfigure}
~
\begin{subfigure}{0.31\textwidth}
\caption{}
\label{fig:yr15:r8f2}
\includegraphics[width=1\textwidth]{Yr15/Figures/selection/R8f2.pdf}
\end{subfigure}

\caption{KASP output from the wheat variety panel with (Ochre, Boston, Cortez) and without (Robigus, Cadenza and Shamrock) \acrshort{yr15}. Marker  R2 (\subref{fig:yr15:r2}) is monomorphic while R8 (\subref{fig:yr15:r8})  is polymorphic between varieties know to carry the gene.  Marker R8 results for the F2 population (\subref{fig:yr15:r8f2}) showing three distinct clusters. The central cluster (light green) is comprised of heterozygous individuals, whereas clusters near the axes are homozygous for either AVS (VIC; orange) or \acrshort{yr15} (FAM; dark green).}
\end{figure}



KASP assays were designed to validate and generate a genetic map of the \acrshort{yr15} locus for the 35 selected SNPs. 
To automate the design of genome-specific primers for polyploid organisms PolyMarker was developed (Chapter \ref{cha:polymarker}).
Out of the 35 assays to design, 17 were design as specific, 9 as semi-specific to chormosome 1BS, and 9 were not specific because there was no information for the homoeolouges on the \acrshort{css} scaffolds. 
PolyMarker also identified putative homoeologous variants (between genomes, as opposed to between varieties) that were in the list of candidate SNPs, but were not identified previously (Figure \ref{fig:poly:mask}; Table \ref{tab:yr15:polymarker}). 

%!TEX root = ../../Main.tex
\begin{sidewaystable}
\caption{Primer details for the markers to validate. }
\centering
\label{tab:yr15:polymarker}
\begin{localsize}{6}{9}

\begin{tabular}{lllllll}
\toprule
 Assay ID   & Polymorphism\_type   & AVS-specific primer         & Yr15-specific primer        & Common primer               & Specificity             & Orientation   \\
\midrule
 R1         & non-homeologous     & aactggtaatggtgcagCgG        & aactggtaatggtgcagCgC        & ttcaggataacacAggagatgtT     & chromosome\_semispecific & reverse       \\
 R2         & non-homeologous     & acatcaattcttcaggaaagctctaC  & acatcaattcttcaggaaagctctaT  & gcacagcttctcgtgttcTT        & chromosome\_specific     & forward       \\
 R3         & non-homeologous     & acgtggagaacctagattgcG       & acgtggagaacctagattgcC       & ccttttaggtgcgccaactT        & chromosome\_semispecific & reverse       \\
 R4         & non-homeologous     & agactctttgggcagtggatC       & agactctttgggcagtggatT       & cctcgggcgatctattctcT        & chromosome\_specific     & forward       \\
 R5         & non-homeologous     & agtcaacttggattacactgaagtT   & agtcaacttggattacactgaagtC   & agatatcacactgaacatactgatgaG & chromosome\_specific     & reverse       \\
 R6         & non-homeologous     & caagatgaagatgaagaggaatatgaT & caagatgaagatgaagaggaatatgaC & gCttgaccctgtaatcatactcG     & chromosome\_semispecific & forward       \\
 R7         & non-homeologous     & caccaccaTggaggccaC          & caccaccaTggaggccaT          & cgccgtggtagtgtccgG          & chromosome\_specific     & forward       \\
 R8         & non-homeologous     & cagatccccggttctctcaaG       & cagatccccggttctctcaaA       & cccccaaatgatcgagaata        & chromosome\_inspecific   & reverse       \\
 R9         & non-homeologous     & caggtgctgaaatgcatcC         & caggtgctgaaatgcatcT         & cggcctatcttcaggtctgt        & chromosome\_inspecific   & reverse       \\
 R10        & non-homeologous     & cattcgtcgcgccttctacG        & cattcgtcgcgccttctacA        & tcctaactcatatgcatgactcAC    & chromosome\_specific     & reverse       \\
 R11        & non-homeologous     & ccattctgatcaaggtcactgtcG    & ccattctgatcaaggtcactgtcA    & ttctgtaTggcaaCgggagC        & chromosome\_specific     & reverse       \\
 R12        & homeologous         & cttagccagtgaaccAggcC        & cttagccagtgaaccAggcT        & ggctgtttgttacCgtggaG        & chromosome\_specific     & reverse       \\
 R14        & non-homeologous     & gacTacAggtgcgatcccC         & gacTacAggtgcgatcccT         & ctcgcctgccagtcgTaT          & chromosome\_specific     & forward       \\
 R15        & homeologous         & gactagggctaccAttgttgA       & gactagggctaccAttgttgC       & agccctgCtaacaatggcaA        & chromosome\_specific     & reverse       \\
 R16        & non-homeologous     & gatgtaagcTAtgactggCgC       & gatgtaagcTAtgactggCgT       & tgcaactgatctttagcaggC       & chromosome\_semispecific & reverse       \\
 R17        & non-homeologous     & gcaAcaacaaCaaCaagtggT       & gcaAcaacaaCaaCaagtggC       & cctcaacctgcttgttgttgT       & chromosome\_specific     & forward       \\
 R19        & non-homeologous     & gcctgatttttaattcgctccaG     & gcctgatttttaattcgctccaA     & agagcactgatgatgacccC        & chromosome\_specific     & reverse       \\
 R20        & non-homeologous     & gctgtatcctcttgaaaaaggcT     & gctgtatcctcttgaaaaaggcC     & ttaggcatgtcagaaatgtagaaaa   & chromosome\_semispecific & forward       \\
 R21        & non-homeologous     & gcttcaaacatgccggctG         & gcttcaaacatgccggctT         & cggtctttttcaaccagggC        & chromosome\_semispecific & forward       \\
 R22        & homeologous         & gctTgtCttaaagccAtttccA      & gctTgtCttaaagccAtttccG      & gcctatcgttCgctaaactctaacT   & chromosome\_specific     & reverse       \\
 R23        & non-homeologous     & gctttaggcactatggattcAcC     & gctttaggcactatggattcAcT     & caggtttctgttcgacctcA        & chromosome\_specific     & forward       \\
 R24        & non-homeologous     & ggaggtcctacacgcgtctT        & ggaggtcctacacgcgtctG        & ctccaaaagaggggcatcattT      & chromosome\_semispecific & forward       \\
 R25        & non-homeologous     & gggttcctcacctgcgcC          & gggttcctcacctgcgcT          & ctctTtgcaatcggccagc         & chromosome\_inspecific   & reverse       \\
 R26        & non-homeologous     & gtCttcgcCggcacCacC          & gtCttcgcCggcacCacT          & agtggatcttgccgatctcg        & chromosome\_inspecific   & forward       \\
 R28        & non-homeologous     & tagatgagaccttggaCggA        & tagatgagaccttggaCggG        & cagtcatctaatgcggaacattcA    & chromosome\_semispecific & reverse       \\
 R29        & non-homeologous     & TatggtGtggccTtccccG         & TatggtGtggccTtccccA         & cgagctcgctgatgaacttG        & chromosome\_specific     & forward       \\
 R30        & non-homeologous     & tcagcagcccttttaacccaA       & tcagcagcccttttaacccaT       & agtaaatcgggcacggttgt        & chromosome\_inspecific   & reverse       \\
 R31        & homeologous         & tcatccatgtatatGaaTccaagcC   & tcatccatgtatatGaaTccaagcA   & tcacgcctgcaacAttcaaaT       & chromosome\_specific     & reverse       \\
 R32        & homeologous         & tccaatcttatggctttgcttctG    & tccaatcttatggctttgcttctT    & caggtgatgtagatgctgagaC      & chromosome\_semispecific & reverse       \\
 R33        & non-homeologous     & tccttcctgctatagctgaaagG     & tccttcctgctatagctgaaagT     & ccctttgcctgccatgtaga        & chromosome\_inspecific   & forward       \\
 R34        & non-homeologous     & tctgagatgatgatactTtgtggG    & tctgagatgatgatactTtgtggA    & actggggatgccctctgtat        & chromosome\_inspecific   & forward       \\
 R35        & non-homeologous     & tgaaagagtggaatttcttgttgT    & tgaaagagtggaatttcttgttgC    & ctttTagctgcttaattctattgcttC & chromosome\_specific     & forward       \\
 R36        & non-homeologous     & tgaaatgccttgtcaatgccA       & tgaaatgccttgtcaatgccG       & ATGCGAATTGGGGAATTAAA        & chromosome\_inspecific   & reverse       \\
 R37        & non-homeologous     & tgcatatgcctgaagagactcG      & tgcatatgcctgaagagactcA      & tgtccacctactcaagtctgc       & chromosome\_inspecific   & reverse       \\
 R38        & non-homeologous     & tgGcCaagTtTttctgcaagaT      & tgGcCaagTtTttctgcaagaG      & tgtaggaaggaactcCgaagtA      & chromosome\_specific     & forward       \\
 R40        & non-homeologous     & tgcatatgcctgaagagactcA      & tgcatatgcctgaagagactcG      & agtccgctaaagcattgcct        & chromosome\_nonspecific  & reverse       \\
 R43        & non-homeologous     & tcgctgatttcatcatgtcccA      & tcgctgatttcatcatgtcccG      & tcaggtgctgcaaatttgagG       & chromosome\_semispecific & forward       \\
\bottomrule
\end{tabular}
\end{localsize}
\end{sidewaystable}


To validate if the 35 SNPs were polymorphic across the parents and, diagnostic to \acrshort{yr15} we tested them in the progenitors plus six commercial varieties, three containing \acrshort{yr15} (Ochre, Boston and, Cortez) and three without it (Shamrock, Robigus and, Cadenza).
Two of the lines without \acrshort{yr15} have \textit{T. diccocoides} in their pedigree (Shamrock and Robigus), as it is the donor species of \textit{Yr15} \citep{mcintosh1995}. 
This test panel allows to test if the SNPs are only diagnostic to \textit{T. diccocoides} instead of \acrshort{yr15}.
On the test panel, 28 ($80\%$) SNPs were polymorphic across the parents and three of them where diagnostic to \textit{yr15} (R5, R8, R33).
From the five homoeologous SNPs, three of them were monomorphic and two polymorphic, suggesting that PolyMarker is effective on detecting which assays are less likely to work (Table \ref{tab:yr15:markersToTest}; Figure \ref{fig:yr15:r2},\subref{fig:yr15:r8}).
The segregation of the SNPs in the full $F_{2}$ population (Section \ref{sub:mappingPopulation}, Figure \ref{fig:yr15:r8f2}) and a genetic map was produced (Section \ref{yr15:geneticMap}).   

%!TEX root = ../../Main.tex
\begin{sidewaystable}
\caption{Results of validation of primers on the progenitors (\acrshort{avs} and \acrshort{yr15}, varieties known to contain \acrshort{yr15} (Cortez, Ochre and, Boston) } and, varieties without \acrshort{yr15} (Robigus, Cadenza and, Shamrock). Shamrock and Robigus have \textit{T. dicoccoides} introgressions. The bold markers are diagnostic in the panel (R5, R8, R88) or in the genetic map (R11). 

\label{tab:yr15:markersToTest}
\begin{localsize}{6}{9}
\centering 
\begin{tabular}{llllllll!{\extracolsep{4pt}}lllllll}
\toprule
Assay &             &                    &        & \begin{sideways}Yr15\end{sideways}    & \begin{sideways}Ochre\end{sideways}  & \begin{sideways}Boston\end{sideways}    & \begin{sideways}Cortez\end{sideways}    & \begin{sideways}Shamrock \end{sideways}    & \begin{sideways}Robigus\end{sideways}    & \begin{sideways}Cadenza\end{sideways}    & \begin{sideways}AVS\end{sideways} & \begin{sideways}Polymorphic\end{sideways} & \begin{sideways}Linked \acrshort{yr15}\end{sideways}  &\\
\cline{5-8}
\cline{9-12}
 ID   & Gene set    & Gene model name    & SNP    & \multicolumn{4}{c}{\acrshort{yr15}+ } & \multicolumn{4}{c}{\acrshort{yr15}- } &   &      & comment                 \\
\midrule
 R1         & UCW         & UCW\_Tt-k55\_contig\_8830;tt-k21\_contig\_10204                      & C341G  & A      & H         & A        & A        & A            & -         & A         & B     & Yes           & * &   segregation distortion                      \\
 R2         & UniGene v60 & gnl$|$UG$|$Ta\#S13126619                                             & C491T  & B      & B         & B        & B        & B            & B         & B         & B     & No            & -                      &                         \\
 R3         & UCW         & contig95240                                                     & C220G  & H      & B         & B        & B        & B            & B         & B         & B     & Yes           & Yes                    &                         \\
 R4         & UCW         & contig105384                                                    & C1227T & A      & B         & B        & B        & B            & B         & B         & B     & Yes           & Yes                    &                         \\
 \textbf{R5}         & UniGene v60 & gnl$|$UG$|$Ta\#S58861868                                             & A214G  & \textbf{A}      & \textbf{A}         & \textbf{A}        & \textbf{A}        & \textbf{B}            & \textbf{B}         & \textbf{B}         & \textbf{B}    & Yes           & Yes                    &                         \\
 R6         & UCW         & KukriC706\_1                                                     & T2979C & A      & H         & B        & B        & B            & B         & H         & H     & Yes           & No                     &                         \\
 R7         & UniGene v60 & gnl$|$UG$|$Ta\#S37932863                                             & C281T  & H      & A         & A        & A        & B            & B         & A         & B     & Yes           & No                     &                         \\
 \textbf{R8}         & UniGene v60 & gnl$|$UG$|$Ta\#S58863387                                             & T241C  & \textbf{B}      & \textbf{B}         & \textbf{B}        & \textbf{B}        & \textbf{A}            & \textbf{A}         & \textbf{A}         & \textbf{A}    & Yes           & Yes                    &                         \\

 R9         & UniGene v60 & gnl$|$UG$|$Ta\#S58892239                                             & C303T  & H      & B         & A        & B        & B            & B         & H         & B     & Yes           & No                     &                         \\
 R10        & UCW         & UCW\_Tt-k63\_contig\_79829                                         & C207T  & H      & A         & B        & A        & B            & B         & B         & B     & Yes           & Yes                    &                         \\
 \textbf{R11}        & UCW         & UCW\_Tt-k45\_contig\_39011                                         & C726T  & \textbf{A}       & \textbf{A}          & \textbf{A}         & \textbf{H}         & \textbf{-}             & \textbf{B}          & \textbf{B}         & \textbf{B}     & Yes           & Yes                    &                         \\
  R12        & UCW         & contig50308                                                     & G587A  & -      & H         & H        & H        & B            & B         & A         & B     & Yes           & Yes                    &                         \\
 R14        & UniGene v60 & gnl$|$UG$|$Ta\#S44692929                                             & C549T  & A      & A         & -        & A        & A            & B         & -         & B     & Yes           & Yes                    &                         \\
 R15        & UCW         & UCW\_Tt-k51\_contig\_2344;tt-k55\_contig\_2091                       & T686G  & A      & A         & A        & A        & A            & A         & A         & A     & No            & -                      &                         \\
 R16        & UniGene v60 & gnl$|$UG$|$Ta\#S17898149                                             & G227A  & A      & A         & B        & A        & B            & B         & B         & B     & Yes           & Yes                    &                         \\
 R17        & UCW         & CL3339Contig1                                                   & T509C  & H      & H         & H        & H        & H            & H         & H         & H     & No            & -                      &                         \\
 R19        & UCW         & UCW\_Tt-k21\_contig\_8407;tt-k61\_contig\_5972                       & C1405T & A      & B         & B        & B        & B            & B         & B         & B     & Yes           & Yes                    &                         \\
 R20        & UCW         & UCW\_Tt-k21\_contig\_8407;tt-k61\_contig\_5972                       & T1102C & A      & B         & B        & B        & -            & B         & B         & B     & Yes           & Yes                    &                         \\
 R21        & UCW         & UCW\_Tt-k31\_contig\_53804;tt-k41\_contig\_31582                     & G1810T & H      & B         & B        & B        & B            & B         & B         & B     & Yes           & Yes                    &                         \\
 R22        & UCW         & UCW\_Tt-k31\_contig\_14966                                         & T408C  & A      & A         & A        & A        & A            & A         & A         & B     & Yes           & Yes                    &                         \\
 R23        & UCW         & UCW\_Tt-k51\_contig\_12731;tt-k55\_contig\_13077;tt-k61\_contig\_18734 & C50T   & A      & H         & H        & H        & H            & -         & H         & B     & Yes           & Yes                    &                         \\
 R24        & UCW         & UCW\_Tt-k55\_contig\_8830;tt-k21\_contig\_10204                      & T3005G & H      & H         & B        & H        & B            & B         & B         & B     & Yes           & Yes                    &                         \\
 R25        & UCW         & UCW\_Tt-k63\_contig\_79829                                         & G184A  & A      & A         & A        & A        & H            & H         & A         & A     & No            & -                      &                         \\
 R26        & UCW         & UCW\_Tt-k21\_contig\_3794                                          & C702T  & H      & A         & B        & A        & B            & B         & B         & B     & Yes           & Yes                    &                         \\
 R28        & UCW         & KukriC3701\_1                                                    & T1053C & A      & A         & B        & A        & B            & B         & B         & B     & Yes           & Yes                    &                         \\
 R29        & UCW         & UCW\_Tt-k55\_contig\_8640;tt-k41\_contig\_8875                       & G783A  & H      & A         & B        & A        & B            & B         & B         & B     & Yes           & Yes                    &                         \\
 R30        & UCW         & UCW\_Tt-k55\_contig\_8830;tt-k21\_contig\_10204                      & T2184A & A      & A         & B        & A        & B            & B         & A         & B     & Yes           & Yes                    &                         \\
 R31        & UCW         & UCW\_Tt-k45\_contig\_22098                                         & G683T  & A      & B         & A        & B        & B            & B         & A         & B     & Yes           & Yes                    &                         \\
 R32        & UCW         & UCW\_Tt-k21\_contig\_33188;tt-k25\_contig\_30647                     & C596A  & H      & A         & A        & A        & A            & A         & A         & H     & No            & -                      &                         \\
 \textbf{R33}        & UniGene v60 & gnl$|$UG$|$Ta\#S58861868                                             & G486T  & \textbf{A}      & \textbf{A}         & \textbf{A}        & \textbf{A}        & \textbf{B}            & \textbf{B}         & \textbf{B}         & \textbf{B}     & Yes           & Yes                    &                         \\
 R34        & UCW         & UCW\_Tt-k31\_contig\_34099                                         & G1713A & H      & A         & B        & -        & B            & B         & B         & B     & Yes           & No                     &                         \\
 R35        & UniGene v60 & gnl$|$UG$|$Ta\#S58900202                                             & T889C  & A      & B         & B        & B        & B            & B         & B         & B     & Yes           & Yes                    &                         \\
 R36        & UCW         & UCW\_Tt-k55\_contig\_8830;tt-k21\_contig\_10204                      & T2349C & H      & H         & H        & H        & H            & -         & H         & B     & Yes           & Yes                    &                         \\
 R37        & UCW         & UCW\_Tt-k31\_contig\_34099                                         & C846T  & B      & B         & B        & B        & B            & B         & B         & B     & No            & -                      &                         \\
 R38        & UniGene v60 & gnl$|$UG$|$Ta\#S58840501                                             & T179G  & B      & B         & B        & B        & B            & B         & B         & B     & No            & -                      &                         \\
 R40        & UCW         & UCW\_Tt-k31\_contig\_34099                                         & C846T  & A      & H         & B        & A        & -            & B         & B         & B     & Yes           & No                     & based on barley synteny \\
 R43        & UniGene v60 & gnl$|$UG$|$Ta\#S58843705                                             & G268A  & A      & B         & B        & -        & -            & B         & -         & B     & Yes           & Yes                    & based on barley synteny \\
\bottomrule
\end{tabular}
\end{localsize}
\end{sidewaystable}


\section{Genetic map} 
\label{yr15:geneticMap}


Initially, the 28 polymorphic markers were used to genotype a subset of 66 plants from the $F_{2}$ population. 
From those, 23 (82\%) were linked to \acrshort{yr15} and several markers fall in a small interval around \acrshort{yr15} (Figure \ref{fig:yr15:initialMap}; Table \ref{tab:yr15:markersToTest}), confirming that the multiple-criteria strategy(Section \ref{yr15:assaySelection}) for selecting candidate SNPs was effective. 
Then, the complete $F_{2}$ population was assessed with:
\begin{itemize}	
	\item  the seven markers that were most linked to \acrshort{yr15}, including two of the diagnostic markers from the variety panel (R5 and R8),
	\item The flanking \acrshort{ssr} microsatellite markers used by UK breeders for germoplasm selection (Xbarc8 and Xgwm413).  
	\item A marker based on barley-wheat synteny (R43) which met the selection criteria, but wasn't on the original set of 50 markers with high BFR. 
\end{itemize}

The $F_{2}$ population consisted on 232 plants with phenotypic information, of those 196 where genotyped reliably (no more than one data point missing). 
Using the eight SNP markers and 2 SRRs, the \acrshort{yr15} locus was mapped to an interval of 0.77cM, with R8/xgwm413 0.26cM distal, and R5/R11 0.77cM proximal from \acrshort{yr15} (Figure \ref{fig:yr15:finalMap},\subref{fig:yr15:mapDetails}). 




\begin{figure}
	\centering
	\begin{subfigure}{0.45\textwidth}
	\caption{}
	\label{fig:yr15:initialMap}
	\includegraphics[height=0.45\textheight]{Yr15/Figures/selection/initialMap.pdf}
	\end{subfigure}
	~
	\begin{subfigure}{0.45\textwidth}
	\caption{}
	\label{fig:yr15:finalMap}
	\includegraphics[height=0.45\textheight]{Yr15/Figures/selection/fineMap.pdf}
	\end{subfigure}

	\begin{subfigure}{1\textwidth}
	\caption{}
	\label{fig:yr15:mapDetails}
	\includegraphics[width=1\textwidth]{Yr15/Figures/selection/mapDetails.pdf}
	\end{subfigure}
	

	\caption{Genetic maps for \acrshort{yr15}. (\subref{fig:yr15:initialMap}) Genetic map of the test panel from 50 individuals. (\subref{fig:yr15:finalMap}) Genetic map from 196 individuals from the full population only with the 8 markers previously identified as closer to the \acrshort{yr15} locus. (\subref{fig:yr15:mapDetails})Graphical genotype of the 196 $F_{2}$ individuals used to develop the genetic map. The alleles are abbreviated according to their origin: A: AVS; B: \textit{Yr15} and H: Heterozygous. Missing calls are indicated by a hyphen.}
\end{figure}

The sub-cM resolution is expected on an $F_{2}$ population of 196 individuals, as 392 gametes provide a resolution of 0.26{}cM. 
Despite the fact that none of the selected markers have perfect linkage to \acrshort{yr15}, the produce genetic map is an improvement in the resolution of the map for the locus and it enables the shift to SNP markers from microsatellites, which has become the preferred marker system in \acrshort{mas} pipelines in breeding programmes. 


\section{Bioinformatic methods}

\subsection{Alignment reads to gene models}

The raw output from the Illumina HiSeq 2000 was processed with Casava v1.8 \citep{casavaBCL}. 
Lanes 1 and 2, containing multiplexed bulks (Table \ref{tab:yr15:reads}) was demultiplexed with a tolerance of 1 mismatch in the barcode. 
Lanes 3 and 4 contained the parental sequences without a barcode. 
The FastQ files where left compressed and in chunks of 40,000, as teh default for the BCL conversion pipeline from Casava to allow parallel processing in a cluster environment. 
The quality of the sequencing lanes was assessed with FastQC v0.10.1 \citep{fastqc}. 
The RNA-Seq reads were aligned with BWA 0.5.9 \citep{Li2009} to the wheat UniGene database v60 \citep{PontiusJUWagnerL2002} and to the UCW gene models \citep{Krasileva2013}, including the \textit{T. turgidum} and complementary ORFs \citep{MASWheat2013}.
The alignments where sorted and stored as single BAM files to have random access \citep{Li2009a}. 

%TODO: Snippet with submission of the alignments. 

\subsection{Bulk Frequency Ratios}
\label{yr15:sub:bfr}

%To identify variant candidates, the methodology described in Trick et al. (2012) was used and extended to work with BAM files and to consider cases when a variant is completely absent from the parental sequences. The consensus from the two parents is called, allowing for ambiguities to take into account possible homoeo- logues when at least 20% of the bases differ from the reference. The consensus is stored with standard IUPAC codes (Cornish- Bowden, 1985). On the bases where the consensus is different between the parents, the bulk frequency ratio (BFR) was calculated as in Trick et al. (2012). The algorithm was imple- mented using Ruby on the top of BioRuby (Goto et al., 2010) and extending the functionality of bio-samtools (Ramirez-Gonzalez et al., 2012). The BFRs were calculated independently on each of the three comparisons (Bulk 1: S1–R1, Bulk 2: S2–R2 and Bulk 3: S3–R3) and with in silico mixes of bulks 1 and 2; and bulks 1, 2 and 3 produced by merging the BAM files using samtools (Li et al., 2009). A list was produced with all the scores in a table format containing the BFR, coverage, ratio of the SNP base and the the parent containing the SNP (Data S2–S4).

\lstinputlisting[language=Ruby,firstline=17, lastline=28, caption=Method to find best BLAT alignment]{Yr15/code/find_best_blat_hit.rb}

\lstinputlisting[language=Ruby,firstline=33, lastline=43, caption=Extensions to Bio::Blat::Report::Hit to get the percentage of coverage]{Yr15/code/find_best_blat_hit.rb}
%The UniGenes, UCW gene models and the contigs containing the 46 977 genetically mapped SNPs from Wang et al. (2014) were aligned with BLAT (Kent, 2002) to the CSS scaffolds (IWGSC, 2014). To assign the genes to a specific scaffold, we selected the best hit from the BLAT alignment, provided that at least 60% of the gene was covered in the scaffold with at least 90% of identity. This selection criterion reduces the number of spurious hits in repetitive motifs in the gene and allows the assignment to a homoeologous chromosome even when the correct homoeologue is missing from the CSS scaffolds. The origin of the scaffold was used to assign a putative chromosome arm. In addition, the gene models were also aligned to the cDNA of Hordeum vulgare (International Barley Genome Sequencing Consortium, 2012) from Ensembl! Plants, release 16 (Kersey et al., 2012). The genetic position of the wheat contigs with SNPs was used to calculate the density of SNPs with a BFR > 6 and for all SNPs between AVS and Yr15. After selection using the criteria outlined in the results section, the selected genes were sorted by BFR in the mix between samples to select the top 50 SNPs for further validation via KASP markers.

\subsection{Alignment between gene models}
%\section{Assembly of the transcriptome} 
%A comparison between thef known unigenes and the transcript from the progenitors. Since \textit{Yr15} comes from an introgression with \textit{T. diccocoides}, some novel transcripts can be extracted. Analysis of the gels from Mitaly? 

\section{Discussion} 
%Remarks on how this techinque can be used to do fine-mapping and that if I were to start the project now I would  use exome capture or Ren-Seq. 

Resequencing the $\sim17$Gbp genome of hexaploid wheat is costly and approaches to reduce the required sequenced volume to effectively call for SNPs had been evolving since the conception of this project. 
The RNA and DNA extraction and the sequencing for this project was carried on before the beginning of my PhD (before October 2012). 
At that point the exome capture was already established for genotyping humans \citep{Ng2009}, however the first exome capture on wheat was just recently published, with probes coming from unassembled 454 reads \citep{Winfield2012}, and an probe designed from transcripts \citep{Henry2014} was not published after the analysis of this section was completed and validated.
An even more targeted capture for resistance genes (RenSeq) was published while this study was executed \citep{Jupe2013}.
On the other hand RNA-Seq was already tested for \acrlong{bsa} on tetraploid wheat \citep{Trick2012}.  
Hence, the decision of reducing the sequenced space with RNA-Seq was appropriate at the time (Figure \ref{fig:intro:timeline}). 
Unfortunately, one of the shortcomings of RNA-Seq used to call for SNPs is that the coverage is not uniform and the genes that have low expression don't have enough coverage to call for SNPs (Section  \ref{yr15:sequencing}).
If a similar study is to be started today, a better alternative would be to use exome capture in general from a segregating population for any trait, or RenSeq if the target gene is a resistance gene.  

The quality and completeness  of the reference genome or gene models directly affects the mapping \acrshort{ngs}. 
This is particularly true on polyploid organisms: if one of the homoeologues is absent, the reads are likely to map to the wrong genome if the parameters of the aligner are relaxed or; not map at all if the the required identity is high.
When the bioinformatic analysis of this project started, the only available genomic reference was the whole genome shotgun 454 sequencing, unassembled \citep{Brenchley2012}; the \gls{css} assembly was being finished\citep{Mayer2014}; the longer scaffolds from \citet{Chapman2015} were not public yet and; the efforts to make a whole genome shotgun assembly were being planned independently by the International Wheat Genome Sequencing consortium  \citep{Pozniak2016} and TGAC \citep{Clark2016}.  
Because contiguous assembly with the corresponding annotation wasn't available at the time of the analysis and the fact that the data available was from a transcriptome, the use of gene models as a reference for the alignment was a suitable approach. 

In terms of available of gene references when the analysis started, the canonical reference was the UniGenes from the NCBI \citep{PontiusJUWagnerL2002}. 
The UniGenes are done with an automated pipeline that clusters all the \acrshortpl{est} deposited in the NCBI by identity and selects the longest transcript, which can yield in having homoeologous transcript merged as a single reference.
Shortly after I started the bioinformatic analysis, two aditional gene models were available, the draft annotation for the \acrshort{css} assembly (MIPSv1) in January 2013 and the UCW gene models \citep{Krasileva2013} in May 2013. 
I selected the UCW gene models, as they were already mature, were phased to distinguish between genomes and already published, over the the MIPSv1 genes that were still being refined as it was an initial approach lifting proteins from related organisms and a few RNA-Seq experiments.  
The MIPS gene models were improved by removing duplications in the assembly in a later stage and the nomenclature before the release of the assembly \citep{Mayer2014}, but at that point the results of this project where already submitted for publication (Figure \ref{fig:intro:timeline}; \citealt{Ramirez-Gonzalez2015b}). 

\unsure{Should I talk briefly about barley? If I have time I'll add a section of synteny, but I don't consider it critical.}
To locate the gene models in the chromosome arms and see if there was an enrichment on the called SNPs the use of a high resolution consensus map is needed, as the genome assemblies available during the analysis are fragmented. 
Timely, a genetic map with $>42,000$ markers was published \citep{Wang2014}. 
I was able to use it to locate several \acrshort{css} scaffolds before it was published, as I collaborated in their assembly. 
The located scaffolds were used as proxy to sort just under half of the reference genes in their chromosomal position (Section \ref{sub:yr15:inSilico}). 
Despite the resolution not being enough to find a single point of enrichment, it was enough to confirm that the SNPs were in the expected location, including one of the SNP candidates flanking the \textit{Yr15} locus (SNP R11, Figure \ref{fig:yr15:BFRValues1BS}).  
If the analysis was to be done today, the genetic map from \citet{Chapman2015} along with their longer scaffolds, or the scaffolds from TGACv1 or the NRGene should provide a better resolution. 
Even without having all the \acrshort{css} scaffolds sorted, the fact that they come from individual chromosome arms they enabled the assignment of the genes to a chromosome. 


The original expectation was to have a \gls{nil} for the \acrshort{bsa}, however the number of SNPs called in the progenitors suggested that the background, \acrlong{avs}, was not the same.  
This happened because despite both succeptible lines being called the same and having the same response to the pathogen, they are different lines from different countries (Section \ref{yr15:snpCalling}). 
This highlights the importance of genotyping the material used when developing mapping populations, specially if the source of the seeds come from different seed banks. 


Despite this shortcomings, the use of the \glspl{bfr} to score the putative SNPs was effective as most of the SNPs with a high score  mapped in chromosome 1B, as expected from previous studies ($BFR>6$, Section \ref{yr15:assaySelection}).
Using the extra criteria of only selecting SNPs from the resistant progenitor and in the expected chromosome arm I was able to produce a high resolution genetic map (Section \ref{yr15:geneticMap}). 
The genetic was of the expected resolution for the size of the population (0.26cM on 196 individuals).
Since the mapping population contained only one critical recombinant between \textit{Yr15} and the flanking markers, the population couldn't yield to a better map. 
From the two critical recombinants it should be possible to make an extra cross and repeat the analysis, but sequencing with either exome capture or RenSeq, to improve the genetic map. 
\unsure{Talk about Why is R33 diagnostic on the varieties, but maps away?. }

\begin{figure}
\includegraphics[width=1\textwidth]{Yr15/Figures/breedersTest.pdf}
\caption{Haplotype analysis and phenotypic evaluation of the 113 doubled haploid lines used in the study. The TGC haplotype corresponds to that originally identified in the \textit{Yr15} parent and which was diagnostic across 112 of the 113 lines studied.}
\label{fig:yr15:breeders}
\end{figure}

As described in \citet{Ramirez-Gonzalez2015b}, the markers
 R11, R5 and R8 were tested "across 122 doubled haploid (DH) lines. 
These DH lines were derived from crosses crosses between five different UK varieties/breeding lines to Yr15 derivatives known to carry the resistance gene. 
The expected \textit{Yr15} haplotype corresponded to T, G and C alleles at markers R11, R5 and R8, respectively (TGC haplotype). 
The DH lines were tested at seedling stage for reaction to \textit{P. striiformis}, with 84 showing complete resistance and 34 presenting an intermediate or completely susceptible reaction.
The resistant lines all carried the complete \textit{Yr15} haplotype (TGC, Figure \ref{fig:yr15:breeders}) across the three SNP markers with the exception of five lines which had a single missing data point, but were otherwise consistent. 
This compared favourably with the most diagnostic in-house SNP markers available within the breeding programmes. 
Using the three in-house markers, 79 resistant lines carried the expected haplotype, but five completely resistant DH lines were scored as false negative due to the presence of the non-\textit{Yr15} haplotype. 
Within the intermediate and susceptible DH lines, all but one had a non-\textit{Yr15} haplotype (CAT or TAT) across R11, R5 and R8 (Figure \ref{fig:yr15:breeders}). This single DH line was scored as a false positive as it carried the TGC \textit{Yr15} haplotype, but was found to have an intermediate (chlorotic) reaction to \textit{P. striiformis}. This line was also the only one scored as a false positive using the three in-house markers". 
The fact that the developed markers perform better than the markers developed by breeders show the value of this particular experiment and further confirms that \acrshort{bsa} combined with \acrshort{ngs} is an effective way to develop novel markers. 
\unsure{ Mention other people using a similar strategy since this was published. }
