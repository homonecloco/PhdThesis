%!TEX root = ../Main.tex

\chapter{Genetic mapping of \textit{Yr15}}

This section describes in detail than the paper of \citet{Ramirez-Gonzalez-2014}


\section{(Introduction)\textit{Yr15}} 
Breeding importance of \textit{Yr15} and original source (an introgression of \textit{T. diccocoides}). 

\section{Segregating population and resistance essays}
A description of the starting material and how the population was generated.  

\section{Sequencing and mapping} 

RNA-Seq and the decision to call SNPs on gene models rather than the whole reference.  Details of the mapping against the Wheat UniGenes \cite{PontiusJUWagnerL2002} and the UCW. \cite{Krasileva2013} gene models.  

\section{SNP Calling}. 
\verb|Ruby| implementation of the methodology described by \citet{Trick2012}. 

\section{Bulk Frequency Ratios} 
Results of the simple SNP calls from the progenitors and how the score of the Bulk Frequency Ratios(BFR) improve the location of the SNPs. 

\section{\textit{In silico} mapping}
Mapping of the gene models to the IWGSC CSS \cite{Mayer2014} reference and the location of the SNPs using the genetic map from \citet{Wang2014}.

\section{Assay selection}. 
The selection criteria to decide which SNPs where selected to produce the genetic map: BFR$>$6, in the short arm of chromosome group 1 and from the \textit{Yr15} progenitor.

\section{Genetic map} 
\label{yr15:geneticMap}
The three versions of the genetic map: With a subset of the F\textsubscript{2} population

\section{Assembly of the transcriptome} 
A comparison between thef known unigenes and the transcript from the progenitors. Since \textit{Yr15} comes from an introgression with \textit{T. diccocoides}, some novel transcripts can be extracted. Analysis of the gels from Mitaly? 

\section{Conclusions} 
Remarks on how this techinque can be used to do fine-mapping and that if I were to start the project now I would  use exome capture or Ren-Seq. 