\documentclass[12pt,a4paper,oneside]{book}
\usepackage[T1]{fontenc}
\usepackage[english]{babel}
\usepackage[sharp]{easylist}
\usepackage[pdftex]{graphicx}
\usepackage{natbib}
\usepackage[font={small}]{caption}
\usepackage{longtable}
\usepackage{subcaption}
\usepackage{pdflscape} 
\usepackage[locale=UK]{siunitx} % Formats the units and values
\usepackage{rotating}
\usepackage{pdflscape}
\usepackage[inner=4cm,outer=4cm,top=3cm,bottom=3cm]{geometry}
\usepackage{inconsolata}
\usepackage{datatool}
\usepackage{blindtext}
\usepackage{scrextend}
\usepackage[inline]{enumitem}
\usepackage{wrapfig}
\usepackage{sidecap}
\usepackage{setspace}
\usepackage{algorithm2e}
\usepackage{algpseudocode}
\usepackage{pbox}
\usepackage{listings}
\usepackage[acronym]{glossaries}
\usepackage{xcolor}
\usepackage[pdfencoding=auto,psdextra]{hyperref}
\usepackage{xargs}   
\usepackage[all]{nowidow}
\usepackage{pdfpages}
\usepackage{coloremoji}
\usepackage[colorinlistoftodos,prependcaption,textsize=tiny]{todonotes}
\newcommandx{\unsure}[2][1=]{\todo[linecolor=red,backgroundcolor=red!25,bordercolor=red,#1]{#2}}
\newcommandx{\change}[2][1=]{\todo[linecolor=blue,backgroundcolor=blue!25,bordercolor=blue,#1]{#2}}
\newcommandx{\info}[2][1=]{\todo[linecolor=OliveGreen,backgroundcolor=OliveGreen!25,bordercolor=OliveGreen,#1]{#2}}
\newcommandx{\improvement}[2][1=]{\todo[linecolor=Plum,backgroundcolor=Plum!25,bordercolor=Plum,#1]{#2}}
\newcommandx{\thiswillnotshow}[2][1=]{\todo[disable,#1]{#2}}


\lstloadlanguages{Ruby}
%\lstset{ 
%basicstyle=\footnotesize\color{black},
%commentstyle = \footnotesize\color{red},
%keywordstyle=\footnotesize\color{blue},
%stringstyle=\footnotesize\color{orange}
%}

\lstnewenvironment{code}[1][]%
  {\noindent\minipage{\linewidth}\medskip 
   \lstset{
   basicstyle=\footnotesize\ttfamily, % Standardschrift
         numbers=left,               % Ort der Zeilennummern
         numberstyle=\tiny,          % Stil der Zeilennummern
         %stepnumber=2,               % Abstand zwischen den Zeilennummern
         numbersep=5pt,              % Abstand der Nummern zum Text
         tabsize=2,                  % Groesse von Tabs
         extendedchars=true,         %
         breaklines=true,            % Zeilen werden Umgebrochen
         keywordstyle=\color{blue},
            frame=b,         
 %        keywordstyle=[1]\textbf,    % Stil der Keywords
 %        keywordstyle=[2]\textbf,    %
 %        keywordstyle=[3]\textbf,    %
 %        keywordstyle=[4]\textbf,   \sqrt{\sqrt{}} %
         stringstyle=\color{red}\ttfamily, % Farbe der String
         showspaces=false,           % Leerzeichen anzeigen ?
         showtabs=false,             % Tabs anzeigen ?
         xleftmargin=17pt,
         framexleftmargin=17pt,
         framexrightmargin=5pt,
         framexbottommargin=4pt,
         %backgroundcolor=\color{lightgray},
         showstringspaces=false      % Leerzeichen in Strings anzeigen ?  
   %basicstyle=\ttfamily\footnotesize,
        frame=single,
   #1}}
  {\endminipage}

\DeclareCaptionFont{white}{\color{white}}
\DeclareCaptionFormat{listing}{\colorbox[cmyk]{0.43, 0.35, 0.35,0.01}{\parbox{\textwidth}{\hspace{15pt}#1#2#3}}}
\captionsetup[lstlisting]{  format=listing,labelfont=white,textfont=white, singlelinecheck=false, margin=0pt, font={bf,footnotesize}
}

\lstnewenvironment{code_2}[1][]%
  {\noindent
  \minipage{\linewidth}
   \medskip 
   \lstset{
   basicstyle=\tiny\ttfamily, % Standardschrift
         numbers=left,               % Ort der Zeilennummern
         numberstyle=\tiny,          % Stil der Zeilennummern
         %stepnumber=2,               % Abstand zwischen den Zeilennummern
         numbersep=5pt,              % Abstand der Nummern zum Text
         tabsize=2,                  % Groesse von Tabs
         extendedchars=true,         %
         breaklines=true,            % Zeilen werden Umgebrochen
         keywordstyle=\color{blue},
            frame=b,         
         stringstyle=\color{red}\ttfamily, % Farbe der String
         showspaces=false,           % Leerzeichen anzeigen ?
         showtabs=false,             % Tabs anzeigen ?
         xleftmargin=17pt,
         framexleftmargin=17pt,
         framexrightmargin=5pt,
         framexbottommargin=4pt,
         %backgroundcolor=\color{lightgray},
         showstringspaces=false      % Leerzeichen in 
        frame=single,
   #1}}
  {
  \endminipage
  }

\DeclareCaptionFont{white}{\color{white}}
\DeclareCaptionFormat{listing}{\colorbox[cmyk]{0.43, 0.35, 0.35,0.01}{\parbox{\textwidth}{\hspace{15pt}#1#2#3}}}
\captionsetup[lstlisting]{  format=listing,labelfont=white,textfont=white, singlelinecheck=false, margin=0pt, font={bf,footnotesize}
}


\newenvironment{blockquote}{%
  \par%
  \medskip
  \leftskip=4em\rightskip=2em%
  \noindent\ignorespaces}{%
  \par\medskip}


\captionsetup[subfigure]{position=top, labelfont=bf,textfont=normalfont,singlelinecheck=off,justification=raggedright}

\SetInd{0.5em}{0.5em}
\linespread{1.25}
\addtokomafont{labelinglabel}{\sffamily}
\usepackage{booktabs} % For \toprule, \midrule and \bottomrule
% start every dtl table with \toprule from booktabs
\renewcommand{\dtldisplaystarttab}{\small}

% likewise for \midrule and \bottomrule from booktabs 
\renewcommand{\dtldisplayafterhead}{\midrule}
\renewcommand{\dtldisplayendtab}{\\\bottomrule}


\newenvironment{processtable}[3]{\setbox\temptbox=\hbox{{\tablesize #2}}%
\tempdime\wd\temptbox\@processtable{#1}{#2}{#3}{\tempdime}}
{\relax}


\providecommand{\e}[1]{\ensuremath{\times 10^{#1}}}


% Setup siunitx:
\sisetup{
  round-mode          = places, % Rounds numbers
  round-precision     = 2, % to 2 places
  group-separator={,}
}

\newenvironment{localsize}[2]
{%
  \clearpage
  \let\orignewcommand\newcommand
  \let\newcommand\renewcommand
  \makeatletter
  \fontsize{#1pt}{#2pt}\selectfont
  %\tiny
  %\input{bk#1.clo}%
  \makeatother
  \let\newcommand\orignewcommand
}
{%
  \clearpage
}

\newenvironment{changemargin}[2]{%
\begin{list}{}{%
\setlength{\topsep}{0pt}%
\setlength{\leftmargin}{#1}%
\setlength{\rightmargin}{#2}%
\setlength{\listparindent}{\parindent}%
\setlength{\itemindent}{\parindent}%
\setlength{\parsep}{\parskip}%
}%
\item[]}{\end{list}}


\author{Ricardo Humberto Ramirez Gonzalez}
\title{Thesis}

\makeglossaries
\begin{document}
\tableofcontents
\listoffigures
\listoftables
\lstlistoflistings
\printglossaries

%%!TEX root = ../Main.tex

\chapter{Introduction}

It defines the objectives and the importance of the research. It focus on the the application of Next Generation Sequencing to molecular biology, wheat genetics and ultimately to breeding programs. It also mentions the current status of the wheat reference genome and other resources (genetic maps, markers) the need of tools to query them effectively. 

%!TEX root = ../Main.tex

\chapter{Literature review}
%!TEX root = ../Main.tex

\chapter{PolyMarker: A fast polyploid primer design pipeline}
\section{Introduction} 
Explain how the SNP markers are designed without the tool and an overview. 


\begin{figure}![ht]
\includegraphics[width=1\textwidth]{PolyMarker/Figures/pipeline.pdf}
        \caption{PolyMarker Pipeline}
        \label{fig:poly:pipeline}
\end{figure}


\begin{figure}
    \centering
    \begin{subfigure}[b]{0.4\textwidth}
        \includegraphics[width=1\textwidth]{PolyMarker/Figures/scaffoldsSearch.pdf}
        \caption{Global search}
        \label{fig:poly:globalSearch}
    \end{subfigure}
    ~ %add desired spacing between images, e. g. ~, \quad, \qquad, \hfill etc. 
      %(or a blank line to force the subfigure onto a new line)
    \begin{subfigure}[b]{0.4\textwidth}
        \raisebox{10mm} { \includegraphics[width=1\textwidth]{PolyMarker/Figures/scaffoldsFound.pdf} }
        \caption{Found regions by the local alignment}
        \label{fig:poly:globalFound}
    \end{subfigure}
    \caption{Local alignments}\label{fig:global}
\end{figure}


\section{Global alignment} 
Search of the contigs with the sequence in the CSS reference and the importance of being able to distinguish between homoeologous regions. 



\section{Local alignment} 
Once the region with the primer has been selected, make a local alignment. This section discusses why the local alignment is needed. 

\section{Primer design tools} 
In this section, the principles of \textit{in silico} primer design are discussed, and why not simply selecting a genomic variation is enough (thermal stability, primers folding on themselves)

\section{Primer selection algorithms} 
Different algorithms to select the \"best primer\". 

\subsection{Regular markers}  
Algorithm to select the two primers with a geneome-specific variation. For amplicons/capillary sequencing. 

\subsection{KASP markers} 
For KASP markers, the product should be as short as possible with the mutation in the first three bases. 

\subsection{Deletion algorithms}
Algorithm to produce KASP for deletions in polyploids. 

\section{Designed markers} Details of the generated primers for the 80k iSelect chip and the 820k axiom chip. This section also include counts on how many are genome specific, semi-specific and non specific. Also an analysis of how many are repeated or map to more than one chromosome perfectly.

\section{Conclusions} Remarks on the importance of getting the primers right, and the time saved by automating the primer selection. Also mention other primer design tools that have been inspired by polymarker: \cite{Ma2015}, \cite{Wang2016}


%!TEX root = ../Main.tex

\chapter{Genetic map of \textit{Yr15} with RNA-Seq.}
\glsresetall
\chaptermark{Genetic map of \textit{Yr15}}
\label{yr15}
%This section describes in detail than the paper of \citet{Ramirez-Gonzalez-2014}
 
%Breeding importance of \textit{Yr15} and original source (an introgression of \textit{T. diccocoides}). 
\section{Background.}
Wheat breeding programs aim to improve the wheat lines available for production.
One of the traits desired in an elite line is the resistance to pathogens, such as\gls{pst}, the fungi responsible of yellow rust.
A source of resistance genes are introgressions from other species, such as \textit{Triticum dicoccoides} (emmer, Figure \ref{fig:lit:polyplody}). 
In the University of Sydney a collection of \glspl{nil} with introgressions to several yellow rust resistance genes on a susceptible background were developed \citep{Wellings1998}. 
In this chapter the \gls{nil} for the \textit{Yr15} locus is used to produce a mapping population to produce a mapping population, which when combined with mapping by sequencing approaches, results in improved diagnostic markers. 

%TODO: Paragraph explaining NILs
\subsection{Segregation on \texorpdfstring{$F_{2}$}{F2} populations.}
\label{yr15:f2}
Molecular markers can be used to select lines by testing if certain allele is present in a line, without the need to phenotype  the given line.
To find which regions are linked to a trait the use of $F_{2}$ mapping populations is a common practice, especially for major single gene traits.
The population is produced by crossing two (usually homozygous) parents ($P_1$ and $P_{2}$) with different alleles, A/A (dominant, resistant if containing \textit{Yr15}) and a/a (recessive, susceptible in our experiment).
When the trait is dominant and has a mendelian segregation, the $F_1$ population should exhibit the dominant trait, as it has a copy of each allele (A/a). 
The $F_1$ is then self-pollinated to produce and $F_2$ population which should segregate with a ratio of 1:2:1, dominant:heterozygous:recessive respectively.
This generates a population with a phenotypic ratio of 3:1 (resistant:susceptible), since the effect of the recessive allele is masked by the dominant allele (\citealt{VanOoijen2013}; Figure \ref{fig:yr15:f2schematic}).  

\begin{figure}
  \centering
   \begin{subfigure}{0.45\textwidth}
   \caption{}
   \label{fig:yr15:f2schematic}
   \centering
   \includegraphics[height=0.3\textheight]{Yr15/Figures/population/F2schematic.pdf}
  \end{subfigure}
  ~
   \begin{subfigure}{0.5\textwidth}
   \caption{}
   \label{fig:yr15:BSAschematic}
   \centering
   \includegraphics[height=0.3\textheight]{Yr15/Figures/BSA.pdf}
  \end{subfigure}
   \caption[Alleles on $F_2$ population and Bulk Segregant Analysis.]{Alleles on $F_2$ population and Bulk Segregant Analysis. The $\otimes$ represent self-pollination. (\subref{fig:yr15:f2schematic}) The cross of two homozygous parents, $P_{1}$ and $P_{2}$, with a dominant and a recessive allele of a gene produces an heterozygous $F_{1}$. The $F_{1}$ crossed with itself produce a segregating $F_{2}$ population with a 1:2:1 ratio (A/A:A/a:a/a). The upper and lower cases represent dominant and recessive alleles, respectively. (\subref{fig:yr15:BSAschematic}) Bulk segregant Analysis consist on pooling DNA from the $F_{2}$ population. The DNA is mixed in bulks coming from plants with a shared phenotype. For a dominant resistance gene, an R sulk contains only resistant individuals (with A/A and A/ genotype) and, an S bulk with the succeptible individuals (with a/a genotype). } 
  
\end{figure}

\subsection{SNP calling}
\gls{bsa} consists on pooling the DNA of individuals from a segregating population with contrasting phenotypes \citep{Michelmore1991} in a segregating population. 
By combining multiple independent individuals with similar phenotypes, one can identify regions which are over-represented or enriched in the corresponding bulks. 
Regions which are not linked to the trait of interest show up as heterozygous in the bulks, whereas regions which are linked to the trait of interest will be enriched for either parental allele.
Here one would expect an enrichment of the resistant allele A with respect to the susceptible allele a in the resistant bulk. 
Analogously, one would expect the absence of the resistant allele A in the susceptible bulk (Figure \ref{fig:yr15:BSAschematic}). 
This approach can be used to identify SNPs using \gls{ngs}, such as: exome capture \citep{Hodges2007}, RNA-Seq \citep{Pickrell2010}, whole genome resequencing \citep{Schneeberger2009}, among others. 

To find SNPs linked to the trait segregating in an $F_{2}$ population from \gls{ngs} there several options. 
In organisms with a contiguous reference genome a normalized count of the times each allele is observed is enough to find the region linked to the trait, this simple ratio is called SNP-Index \citep{Takagi2013a}.
However, wheat is a poliploid organisms, with an identity between homoeologues over 98\%. 
Because of the high identity, and in cases where the reference for some of the homoeologues is absent, reads coming from different homoeologues may map to the same reference. 
The \gls{bfr} \citep{Trick2012} methodology can work on organisms that have more than one pseudo genome with not all the genes, homoeologues or paralogues, characterised independently; it works with a single reference collapsing similar regions. 
Both methodologies relay on an enrichment of the alleles linked to the trait in the corresponding locus. 


Homoeologous variants between two sub-genomes of wheat are exemplified by the G$>$T variant at position 181; K in consensus (Figure \ref{fig:yr15:bfr}) and will produce the same ambiguity code for both parental consensus sequences and can therefore be excluded. 
Real allelic varietal SNPs between the parental genotypes, exemplified by the G$>$A variant at position 184; R in consensus, are distinguished by the presence in one, but not the other parental consensus sequence. 
The allelic SNPs are then examined further with the alignments of the bulks to identify the SNPs that are enriched on the resistant plants.
The SNP index is the proportion of times an alternative allele is observed over the coverage at certain position, in the example the susceptible bulk has an SNP index of $1/8=0.125$ and $6/8=0.75$ for the resistant bulk \citep{Takagi2013a}. 
The \glspl{bfr} are then calculated by dividing the SNP Index of the sample containing the target phenotype (resistance) over the sample without the trait (susceptible), For this example it would be $0.75/0.125=6$.  
A high BFR suggests that the \acrshort{snp} is linked to the target trait \citep{Trick2012}. 
The implementation of the BFR analysis is detailed in Section \ref{yr15:sub:bfr} and the results on the $F_2$ population are discussed in Section \ref{sec:yr15:bfr}. 
The Bulk Frequency Ratio (BFR) methodology
can work on organisms that have more than one pseudo genome
with not all the genes, homoeologues or paralogues, characterised independently;
it works with a single reference collapsing similar regions.

\begin{figure}
\includegraphics[width=1\textwidth]{Yr15/Figures/bfr.pdf}
\caption[BFR formula]{BFR formula. Illustration of a non-informative homoeologous SNP (G181T) present in both parental lines, and an informative allelic SNP (G184A), only present in the resistant progenitor Avocet S + Yr15. The consensus sequences from the parental genotypes include this information in the form of ambiguity codes (K and R, respectively). In the bulks, the individual reads align across the reference sequence, with matches indicated by dots, and polymorphisms at positions 181 and 184 indicated by the corresponding nucleotide variants at those positions. The SNP index is calculated as the frequency of the informative allelic SNP in each bulk. The Bulk Frequency Ratio is the quotient of the resistant and susceptible bulk SNP Indexes. Figure previously published in \citet{Ramirez-Gonzalez2015c}. }
\label{fig:yr15:bfr}
\end{figure}

To call for SNPs from RNA-Seq a reference transcriptome is used as target to align the reads. 
The UniGenes database, from NCBI, contains the genes of each species with all the variations of each gene automatically collapsed and represented with the longest \acrshort{cdna} \citep{PontiusJUWagnerL2002}. 
The \acrshort{ucw}  genes described in \citet{Krasileva2013} contains 94,177 models from tetraploid and hexaploid wheat, assembled and phased to separate different homoeologues. 
Both gene sets complement each other, however, the \acrshort{ucw} gene models should provide an improved alignment, since the different homoeologues aren't merged in a single model, a possible side effect of the UniGene pipeline. 

\subsection{\textit{In Silico} mapping.}
There are several layers of information that can be used to add a context to the SNPs. 
When the SNPs are called from genes like the UniGenes \citep{PontiusJUWagnerL2002} or the UCW gene models \citep{Krasileva2013}, the location of the genes can be assigned by aligning them to a genomic reference, even if it is fragmented. 
A source to get the order of the scaffolds are previously published genetic maps, such as the genetic map described in \citet{Wang2014}, which has the sequence of the markers available.
The markers and the genes can be aligned to the scaffolds with a high percentage of identity (over $98\%$), to avoid them being assigned to a homoeologue or paralogue on a different chromosome.
The use of genetic maps to sort genomic sequence is frequently used to produce pseudo-chromosomes on genome wide projects, usually with ad-hoc tools \citep{Tang2015}.
Since the \acrshort{css} assembly is quite fragmented the genetic maps don't have enough resolution to produce a pseudomolecule, however it is enough to sort the scaffolds in bins when several markers map to the same location. 
In this way, it is possible to use the scaffolds as a proxy to map the genes to their genetic position (Figure \ref{fig:yr15:layersOfMapping}).
The results of mapping the genes with SNPs to the CSS assembly and the genetic map are described in Section \ref{sub:yr15:inSilico}. 
For a longer description of resources available for wheat see Section \ref{lit:wheatResourcers}. 
%\unsure{To do: section talking about genetic map. }
%\unsure{To do: Microsatellites vs SNP markers. }

\begin{figure}
  \centering
  \includegraphics[width=1\textwidth]{Yr15/Figures/mapping/layersOfMapping.pdf}
  \caption[Layers of information to do \textit{In Silico} mapping.]{Layers of information to do \textit{In Silico} mapping. SNPs are called from gene models. The genes and markers from genetic maps are aligned to scaffolds. The order of the markers in a genetic map can be used to sort the scaffolds.} 
  \label{fig:yr15:layersOfMapping}
\end{figure}

Finally, the best candidate SNPs where selected to produce a genetic map which lead to a triplet of markers diagnostic to the target locus. 

The steps described in this chapter were first published in \citet{Ramirez-Gonzalez2015c} and the results of this chapter are published in \citet{Ramirez-Gonzalez2015b}.

\section{Mapping population.}

The population was developed by crossing the resistant line \gls{yr15} \citep{Wellings1998}, Figure \ref{fig:yr15.yr15Photo}, to the susceptible line \gls{avs}, Figure \ref{fig:yr15:avsPhoto}. 
\gls{yr15} is a \gls{nil} of a 6th generation \gls{bc} and the \gls{avs} background is highly succeptible to yellow rust, hence the resistance is coffered by the \gls{yr15} locus. 
$F_{2}$ seeds from three independent $F_{1}$ plants where sown and tissue was collected, before the fungal inoculation to avoid the effect of the disease resistance response on the gene expression. 
Sampling after inoculation could have led to associations in the bulks due to expression of genes downstream of \gls{yr15} and not due to the gene itself.
Seedlings were challenged at the three leaf stage as it is know that \textit{Yr15} confers resistance in seedlings \citep{Gerechter-Amitai1989}.
The expected segregation on an $F_{2}$ population is 3:1 (resistant:susceptible), since \textit{Yr15} is a dominant gene.
From the 232 plants in the $F_{2}$ population that germinated, 187 were resistant and 45 were susceptible, which deviates slightly from the expected ratio ($\chi^{2}=0.049$).
Segregation distortion has been shown for the same \textit{Yr15} donnor \citep{Randhawa2009}, however the decresed number of succeptible plants can be explained by escapes in the virulence essays (i.e. plants scored as resistant without the \textit{Yr15} locus).   For this study we extracted DNA from individual plants in the $F_{2}$ population and we bulked RNA on 6 different bulks: 3 resistant and, 3 succeptible ( Figure \ref{fig:yr15:f2}). 


\label{sub:mappingPopulation}
%\begin{SCfigure}
\begin{figure}
%\begin{wrapfigure}[17]{R!}{7cm}
    \centering
     
     \begin{subfigure}[b]{0.4\textwidth}
        \caption{}
        \includegraphics[width=1\textwidth]{Yr15/Figures/population/Yr15Photo.png}
        \label{fig:yr15.yr15Photo}
    \end{subfigure}
    ~
    \begin{subfigure}[b]{0.4\textwidth}
        \caption{}
        \includegraphics[width=1\textwidth]{Yr15/Figures/population/AVSPhoto.png}
        \label{fig:yr15:avsPhoto}
    \end{subfigure}

     \begin{subfigure}[b]{0.9\textwidth}
     \caption{}
        \includegraphics[width=1\textwidth]{Yr15/Figures/population/F2Population.pdf} 
    \label{fig:yr15:f2}
  \end{subfigure}

    \caption[Avocet + \textit{Yr15} $F_{2}$  mapping population.]{Avocet + \textit{Yr15} $F_{2}$  mapping population. Response of (\subref{fig:yr15.yr15Photo}) Avocet + \textit{Yr15} and (\subref{fig:yr15:avsPhoto}) Avocet when inoculated with \textit{Puccinia striiformis} f. sp.  \textit{tritici} at the three leaf stage. (\subref{fig:yr15:f2}) The phenotype of the $F_{2}$ population was used to produce 6 bulks, 3 resistant and 2 susceptible. The RNA was pooled in bulks accordingly. Adapted from \citep{Ramirez-Gonzalez2015b}}

%\end{wrapfigure}
\end{figure}
%\end{SCfigure}


\section{Sequencing and mapping.} 

RNA-Seq was used as a reduced representation method and thus avoid sequencing the non-coding regions and reduce the search space.  
The sequencing of the bulks and the parents were done on a single Illumina Hi-Seq2000 each.
The bulks were multiplexed and sequenced on a third of a lane each, as shown on Table \ref{tab:yr15:reads}. 
To ensure that the quality of the sequencencing was good, \verb|fastqc-0.10| \citep{fastqc}  was run with its default parameters in each one of the fastq files.  
The GC content was around 52\% in all the samples (Appendix \ref{App:AppendixQCGC}), which is expected as the sample should be of coding regions, and for wheat the reported GC content in genes is around 55\%.  
The quality of the reads is fairly consistent, in general dropping after the base 80 across the samples (Appendix \ref{App:AppendixQCRead}). 
\label{yr15:sequencing}

\begin{table}
\centering
\caption{Arrangement and number of sequenced base pairs per sample. }
\label{tab:yr15:reads}
\begin{tabular}{rrccccc}
\toprule
Library & name & Bar code & Lane   &  Reads (\e{8} bp)\\ 
\midrule
LIB1715 & Bulk R1 & ATCACG & 1  & 0.77\\
LIB1716 & Bulk R2 & TAGCTT & 1    & 1.20\\
LIB1717 & Bulk R3 & ACTTGA & 2  & 0.96  \\ 
LIB1718 & Bulk S1 & GGCTAC & 2  & 1.64   \\ 
LIB1719 & Bulk S2 & CGTACG & 2  & 1.49  \\ 
LIB1720 & Bulk S3 & GTGGCC & 1  &1.88  \\ 
LIB1721 & AvocetS & N/A & 3     & 4.13 \\ 
LIB1722 & AvocetS + \textit{Yr15} & N/A & 4   & 3.99  \\ 
\bottomrule
\end{tabular}
\end{table}



\input{Yr15/SupplementalTables/alignedCoverage}

When the analysis was started, the draft genome and the corresponding annotation where not not release yet, hence gene models where used. 
All the samples where aligned to the UniGenes v60 (56,954 genes) and the gene models from UCW \citep{Krasileva2013} using \verb|BWA 0.5.9| \citep{Li2009}. 
The alignment  showed that a few genes were very highly expressed, howeverthere was still sufficient coverage of over 20x in \gls{yr15} across 22,107 and 36,808 genes, on the UniGenes and the UCW gene , respectively. 
Both gene sets performed similarly in terms of the percentage of genes with reads and percentage of aligned reads. 
For \gls{avs} and \gls{yr15}, the percentage of genes with a coverage of at least $20x$ is $45\%$ and $39\%$, respectively, across both references (Figure \ref{fig:yr15:covPerGene}).
Since each individual bulk has a lower coverage, the susceptible and resistant reads were merged \textit{in silico} as: (i) susceptible bulks 1 with 2 (S1+S2) and resistant bulks 1 with 2 (R1+R2) and (ii) all the susceptible (S1+S2+S3) and resistant bulks (R1+R2+R3). 
The merged samples increased the percentage of genes with coverage over 20x  to 44\% and 50\% in the resistant and susceptible bulks (Table \ref{app:seqAlnCov}), which is close to the coverage from the progenitors.
We treated bulk 3 slightly differently since these bulks included a few lines which were borderline with respect to their phenotype. 
Therefore exclusion of bulk 3 plants in the S1+S2 and R1+R2 bulk would provide the "cleanest" possible data, whereas inclusion in the second set of bulks would allow us to evaluate the effect of possible noise within the system. 

\begin{figure}
\centering
\begin{subfigure}{0.38\textwidth}
    \caption{}
     \includegraphics[width=1\textwidth]{Yr15/Figures/CoveragePerGene.pdf} 
    \label{fig:yr15:covPerGene}
\end{subfigure}
~
\begin{subfigure}{0.58\textwidth}
    \caption{}
    \includegraphics[width=1\textwidth]{Yr15/Figures/PercentageOfSnps.pdf} 
    
    \label{fig:yr15:SNPper}
\end{subfigure}
\caption[Coverage and SNPs between progenitors]{Aoverage and SNPs between progenitors. (\subref{fig:yr15:covPerGene}) Box plot distribution of the gene coverage of the parent reads (\gls{avs} and \gls{yr15}) across the UCW (blue) and the UniGene (red) gene models. The dashed line represents the 209 minimum coverage required for SNP calling. The full line represents the average coverage across all gene models. (\subref{fig:yr15:SNPper}) Percentage of genes exhibiting SNPs across references. The number of \gls{snp}s between the parent reads and the corresponding references was calculated (per 100 bp, rounded). The ‘between-parents’ category corresponds to putative SNPs when comparing the consensus sequence between \gls{avs} and \gls{yr15} Adapted from \citet{Ramirez-Gonzalez2015b} }
\end{figure}



\section{SNP Calling}
\label{yr15:snpCalling}

The \gls{snp} calling was done on positions with a coverage of at least $20x$ on the progenitor lines against the gene reference. The \gls{avs} progenitor had roughly $3\%$ more genes with polymorphisms than \gls{yr15}, consistent with the difference in coverage, suggesting that with a higher coverage we could recover more \gls{snp}s from \gls{yr15}.
The UniGenes have a higher number of \gls{snp}s because the \gls{ucw} gene models have a higher number of monomorphic genes when compared to the UniGenes (Figure \ref{fig:yr15:SNPper}; Table \ref{app:yr15:cntSNP100bp}). 
The difference in the number of relative monomorphic SNPs between references can be explained by the fact that the UniGenes have homoeologues that can be represented as a single sequence, as opposed to the UCW set which are homoeologue-specific, improving the mapping to the correct homoeologue in the genes from the UCW set over the UniGenes.

%!TEX root = ../../Main.tex
\begin{table}
\caption{ Count of SNPs per 100 bp on genes with at least 20x coverage. }
\centering
\label{app:yr15:cntSNP100bp}
\begin{localsize}{10}{12}
\begin{tabular}{lrrrrrrr}
\toprule
 SNPs  & \multicolumn{3}{c}{UCW}  &  &  \multicolumn{3}{c}{UniGene v60 }                                 \\
 \cline{2-4}
 \cline{6-8}
\pbox{1cm}{per 100bp}         & AVS   & \pbox{1.5cm}{\centering AVS+ \textit{Yr15}} & \pbox{1.8cm}{\centering Between progenitors} &      & AVS         & \pbox{1.5cm}{\centering AVS+ \textit{Yr15}} & \pbox{1.8cm}{\centering Between progenitors} \\
\midrule
 0               & 67, 389       & 70,338 & 81,921             &      & 36,210       & 38,339      & 47,097               \\
                 & 71.6\% & 74.7\%      & 87.0\%               &      & 63.6\%       & 67.3\%      & 82.7\%               \\
 \midrule
 1               & 16,111 & 14,770      & 10,107               &      & 10,058       & 9,175       & 8,061                \\
                 & 17.1\% & 15.7\%      & 10.7\%               &      & 17.7\%       & 16.1\%      & 14.2\%               \\
 \midrule
 2               & 8,904  & 7,676       & 1,893                &      & 8,529	         & 7,648       & 1,621                \\
                 & 9.5\%  & 8.2\%       & 2.0\%                &      & 15.0\%       & 13.4\%      & 2.9\%                \\
 \midrule
 3               & 1,517  & 1,192       & 215                  &      & 1,870        & 1,568       & 59       \\
                 & 1.6\%  & 1.3\%       & 0.2\%                &      & 3.3\%        & 2.8\%       & 0.3\%                \\
 \midrule
 4+              & 253    & 198         & 38                   &      & 287          & 224         & 16                  \\
                 & 0.3\%  & 0.2\%       & 0.0\%                &      & 0.5\%        & 0.4\%       & 0.0\%                \\
\bottomrule
\end{tabular}
\end{localsize}
\end{table}


Both gene sets were done from varieties different to \gls{avs} and are likely to be incomplete, hence we set a low threshold of at least 20\% of the observed nucleotides on any position to call an \gls{snp}. 
To represent cases were more than one consensus base is called we use \gls{iuapc} codes (\citet{Cornish-Bowden1985}; Section \ref{lit:ambiguity}; Figure \ref{fig:yr15:bfr}).  
To focus the analysis on informative \gls{snp}s, the common varietal SNPs and variations between homoeologues were removed by finding the cases when the consensus call on both progenitors is the same. 
The \gls{snp}s that are unique to a single parental were examined in detail. 
There are 66,426 putative SNPs across 16,022 (17\%) \gls{ucw} genes and 52,262 \acrshortpl{snp} on 11,056 UniGenes (19.4\%; Figure \ref{fig:yr15:geneCount}).  

\begin{SCfigure}
    %\centering
    \includegraphics[width=0.4\textwidth]{Yr15/Figures/geneCounts.pdf} 
    \caption[Gene models with putative SNPs]{Gene models with putative SNPs in common between the UCW and UniGenes reference. The intersection represnts the genes that are common in both sets. Adapted from \citet{Ramirez-Gonzalez2015b}}
    \label{fig:yr15:geneCount}
\end{SCfigure}

\input{Yr15/SupplementalTables/genesAssignedtoArm}

The high number of genes with \gls{snp}s was unexpected as a \gls{bc}6 \gls{nil} used for an $F_2$ mapping population expects to have $<1\%$ of the genetic background segregating. 
Both sets of gene models were aligned with BLAT \citep{Kent2002} to the \gls{css} assembly \citep{Mayer2014}; the alignment resulted on 80,031 (85.0\%) UCW gene models and 41,118 (72.2\%) UniGenes assigned to a chromosome arm (Table \ref{tab:yr15:genesToCSS}). 
The SNPs found in the mapped genes are evenly distributed across all the chromosomes (see Section \ref{sub:yr15:inSilico}), suggesting that the \gls{avs} (\gls{jic}, UK) used as parent in the $F_{2}$ is different to the \gls{avs} used for the \acrshort{yr15} \acrshort{nil} development (University of Sydney, Australia).  

To confirm that the \gls{avs} seed stocks from \gls{jic} are distinct to the stocks in Sydney, DNA from both stocks was procured and compared with the iSelect 90k wheat SNP chip. 
Between two independent \gls{avs} seeds from \gls{jic} only 58 out of 71,972 (0.08\%) valid assays were polymorphic. 
Nonetheless, there are over 5,000 ($>7.5\%$) assays with polymorphisms between  \gls{jic}-\gls{avs} and \gls{avs} from Sydney. 
The different was not expected originally, but considering that the \gls{avs} seeds are coming from different stocks and the fact that in both countries commercial varieties with the same name had been released, it is not surprising. 


\section{Bulk Frequency Ratios}
\label{sec:yr15:bfr}

\input{Yr15/SupplementalTables/scoredSNPs}

The objective was to find the \acrshort{snp}s enriched  (or depleted) on each bulk and hence linked to the phenotype.  \Glspl{snp} originating from \gls{yr15} would be expected to be linked to resistance whereas those from AVS to susceptibility. from \gls{avs} to susceptibility in the segregating population. 
Across individual bulks, it was possible to score between 15,261 (24.5\%) to 31,891(48.0\%) \glspl{snp} across both reference sets.
On the \textit{in silico} mixes over $95\%$ of SNPs where scored (Table \ref{app:yr15:scoredSNPs}), suggesting that the coverage of individual bulks is not enough to score all the SNPs.  
The scoring was done with the \acrlong{bfr} (\citealt{Trick2012};Figure \ref{fig:yr15:bfr}; Section \ref{yr15:sub:bfr}), which has a value that increases as the \acrshort{yr15} allele is observed more times relatively to the \acrshort{avs} allele.

%!TEX root = ../../Main.tex
\begin{sidewaystable}
\caption{ SNPs in chromosome group 1S vs total number of SNPs with a minimum BFR from 0 to 10. AVS: SNPs coming from \acrlong{avs}. \textit{Yr15}: SNPs coming from \acrlong{yr15}. }
\centering
\label{app:yr15:bfrThresholds}
\begin{localsize}{6}{7}

\begin{tabular}{llp{1cm}p{1cm}p{1cm}p{1cm}p{1cm}p{1cm}p{1cm}p{1cm}p{1cm}p{1cm}}
\toprule
 Min  BFR   & Gene Set    & R1/S1 \textit{Yr15}        & R1/S1 AVS         & R2/S2 \textit{Yr15}         & R2/S2 AVS          & R3/S3 \textit{Yr15}         & R3/S3 AVS          & S1+2/ R1+2 \textit{Yr15}    & S1+2/ R1+2 AVS     & S1+S2+S3/ R1+R2+R3 \textit{Yr15}   & S1+S2+S3/ R1+R2+R3 AVS   \\
\midrule
 0          & UCW         & 308/8,049 (3.83\%) & 305/8,220 (3.71\%) & 505/14,121 (3.58\%) & 556/15,582 (3.57\%) & 532/14,875 (3.58\%) & 623/17,016 (3.66\%) & 670/18,760 (3.57\%) & 885/25,464 (3.48\%) & 860/24,026 (3.58\%)        & 1,505/40,496 (3.72\%)     \\
            & UniGene v60 & 307/7,823 (3.92\%) & 299/7,438 (4.02\%) & 428/12,409 (3.45\%) & 421/12,734 (3.31\%) & 427/12,050 (3.54\%) & 415/12,498 (3.32\%) & 536/15,672 (3.42\%) & 595/20,026 (2.97\%) & 712/19,358 (3.68\%)        & 901/30,380 (2.97\%)       \\
 \midrule
 1          & UCW         & 214/4,415 (4.85\%) & 194/4,108 (4.72\%) & 325/7,603 (4.27\%)  & 314/7,374 (4.26\%)  & 365/7,920 (4.61\%)  & 415/8,850 (4.69\%)  & 426/10,122 (4.21\%) & 494/12,185 (4.05\%) & 539/13,037 (4.13\%)        & 842/19,466 (4.33\%)       \\
            & UniGene v60 & 207/4,474 (4.63\%) & 194/3,630 (5.34\%) & 269/6,649 (4.05\%)  & 269/6,193 (4.34\%)  & 279/6,511 (4.29\%)  & 272/6,436 (4.23\%)  & 329/8,704 (3.78\%)  & 369/9,343 (3.95\%)  & 446/10,860 (4.11\%)        & 541/14,226 (3.80\%)       \\
 \midrule
 2          & UCW         & 92/651 (14.13\%)   & 75/671 (11.18\%)   & 142/1,372 (10.35\%) & 111/1,101 (10.08\%) & 147/1,162 (12.65\%) & 149/1,411 (10.56\%) & 167/1,324 (12.61\%) & 163/1,478 (11.03\%) & 194/1,370 (14.16\%)        & 207/1,765 (11.73\%)       \\
            & UniGene v60 & 77/568 (13.56\%)   & 58/527 (11.01\%)   & 101/1,017 (9.93\%)  & 81/720 (11.25\%)    & 105/775 (13.55\%)   & 84/867 (9.69\%)     & 122/991 (12.31\%)   & 116/973 (11.92\%)   & 145/1,030 (14.08\%)        & 132/1,210 (10.91\%)       \\
 \midrule
 3          & UCW         & 78/299 (26.09\%)   & 45/295 (15.25\%)   & 118/646 (18.27\%)   & 70/409 (17.11\%)    & 123/577 (21.32\%)   & 85/494 (17.21\%)    & 145/673 (21.55\%)   & 98/563 (17.41\%)    & 168/768 (21.88\%)          & 122/665 (18.35\%)         \\
            & UniGene v60 & 65/254 (25.59\%)   & 26/186 (13.98\%)   & 87/499 (17.43\%)    & 54/294 (18.37\%)    & 93/379 (24.54\%)    & 48/315 (15.24\%)    & 107/525 (20.38\%)   & 66/379 (17.41\%)    & 133/617 (21.56\%)          & 78/489 (15.95\%)          \\
 \midrule
 4          & UCW         & 75/232 (32.33\%)   & 28/160 (17.50\%)   & 109/484 (22.52\%)   & 44/217 (20.28\%)    & 105/416 (25.24\%)   & 44/246 (17.89\%)    & 134/539 (24.86\%)   & 53/277 (19.13\%)    & 149/640 (23.28\%)          & 64/323 (19.81\%)          \\
            & UniGene v60 & 63/192 (32.81\%)   & 17/104 (16.35\%)   & 83/390 (21.28\%)    & 29/155 (18.71\%)    & 82/288 (28.47\%)    & 29/173 (16.76\%)    & 104/431 (24.13\%)   & 40/214 (18.69\%)    & 127/519 (24.47\%)          & 29/266 (10.90\%)          \\
 \midrule
 5          & UCW         & 69/202 (34.16\%)   & 19/108 (17.59\%)   & 95/416 (22.84\%)    & 33/138 (23.91\%)    & 96/354 (27.12\%)    & 23/143 (16.08\%)    & 127/477 (26.62\%)   & 28/175 (16.00\%)    & 140/580 (24.14\%)          & 42/222 (18.92\%)          \\
            & UniGene v60 & 58/163 (35.58\%)   & 11/70 (15.71\%)    & 76/337 (22.55\%)    & 14/102 (13.73\%)    & 70/228 (30.70\%)    & 20/112 (17.86\%)    & 100/389 (25.71\%)   & 23/146 (15.75\%)    & 118/469 (25.16\%)          & 21/178 (11.80\%)          \\
 \midrule
 6          & UCW         & 65/179 (36.31\%)   & 12/85 (14.12\%)    & 86/380 (22.63\%)    & 22/98 (22.45\%)     & 87/299 (29.10\%)    & 11/94 (11.70\%)     & 122/429 (28.44\%)   & 21/130 (16.15\%)    & 126/514 (24.51\%)          & 29/165 (17.58\%)          \\
            & UniGene v60 & 57/151 (37.75\%)   & 7/48 (14.58\%)     & 73/300 (24.33\%)    & 6/71     (8.45\%)   & 65/191 (34.03\%)    & 13/84 (15.48\%)     & 98/358 (27.37\%)    & 20/122 (16.39\%)    & 115/439 (26.20\%)          & 16/143 (11.19\%)          \\
 \midrule
 7          & UCW         & 58/161 (36.02\%)   & 11/73 (15.07\%)    & 77/340 (22.65\%)    & 13/74 (17.57\%)     & 73/248 (29.44\%)    & 7/69 (10.14\%)      & 116/393 (29.52\%)   & 20/111 (18.02\%)    & 114/468 (24.36\%)          & 22/143 (15.38\%)          \\
            & UniGene v60 & 56/132 (42.42\%)   & 4/37 (10.81\%)     & 68/273 (24.91\%)    & 5/58    (8.62\%)    & 60/171 (35.09\%)    & 9/64 (14.06\%)      & 94/334 (28.14\%)    & 18/103 (17.48\%)    & 113/412 (27.43\%)          & 16/124 (12.90\%)          \\
 \midrule
 8          & UCW         & 58/149 (38.93\%)   & 10/62 (16.13\%)    & 68/310 (21.94\%)    & 12/59 (20.34\%)     & 66/214 (30.84\%)    & 6/56 (10.71\%)      & 104/359 (28.97\%)   & 17/102 (16.67\%)    & 108/429 (25.17\%)          & 16/119 (13.45\%)          \\
            & UniGene v60 & 55/126 (43.65\%)   & 3/33    (9.09\%)   & 64/255 (25.10\%)    & 5/50 (10.00\%)      & 55/150 (36.67\%)    & 9/55 (16.36\%)      & 91/313 (29.07\%)    & 14/89 (15.73\%)     & 105/376 (27.93\%)          & 15/108 (13.89\%)          \\
 \midrule
 9          & UCW         & 54/135 (40.00\%)   & 8/53 (15.09\%)     & 63/289 (21.80\%)    & 8/51 (15.69\%)      & 61/182 (33.52\%)    & 5/49 (10.20\%)      & 100/331 (30.21\%)   & 15/91 (16.48\%)     & 100/387 (25.84\%)          & 13/106 (12.26\%)          \\
            & UniGene v60 & 53/117 (45.30\%)   & 1/30    (3.33\%)   & 62/244 (25.41\%)    & 5/46 (10.87\%)      & 50/136 (36.76\%)    & 9/48 (18.75\%)      & 88/291 (30.24\%)    & 13/83 (15.66\%)     & 97/345 (28.12\%)           & 12/99 (12.12\%)           \\
 \midrule
 10         & UCW         & 52/126 (41.27\%)   & 8/50 (16.00\%)     & 62/279 (22.22\%)    & 8/50 (16.00\%)      & 56/165 (33.94\%)    & 4/45    (8.89\%)    & 96/309 (31.07\%)    & 14/82 (17.07\%)     & 91/355 (25.63\%)           & 13/100 (13.00\%)          \\
            & UniGene v60 & 50/105 (47.62\%)   & 1/28    (3.57\%)   & 60/226 (26.55\%)    & 5/39 (12.82\%)      & 43/119 (36.13\%)    & 7/45 (15.56\%)      & 85/272 (31.25\%)    & 13/82 (15.85\%)     & 92/318 (28.93\%)           & 12/97 (12.37\%)           \\
\bottomrule
\end{tabular}
\end{localsize}
\end{sidewaystable}


When increasing the minimum BFR threshold, enrichment of SNPs was observed in the short arm of the group 1 chromosomes (1S). 
Without taking into account the BFR, $~3.6\%$ of the SNPs are located in the 1S group, similar to the number of SNPs located in other groups \ref{tab:yr15:genesToCSS}. 
However, when increasing the threshold  (between $BFR > 5 $ and $BFR > 7$) the relative number of SNPs in group 1S increases. 
After $BFR>7$ the gains in relative enrichment only improves marginally, but the number of called SNPs is reduced (Table \ref{app:yr15:bfrThresholds}; Figure \ref{fig:yr15:bfrChange}).
For that reason, SNPs with a $BFR>6$ were selected for further validation. 
The method described by \citet{Trick2012} was extended to include cases where there is a complete lack of coverage in one of the samples ($BFR=\infty$), which is an ideal case where the linkage between the SNP and the phenotype is perfect. 
A total of 1,582 SNPs across 1,173 genes had a $BFR>6$.



\begin{figure}
\includegraphics[width=1\textwidth]{Yr15/Figures/bfrChanges.pdf}
\caption[Effect of BFR threshold on the number of SNPs]{Effect of BFR threshold on the number of SNPs across the short arm of chromosome group 1. Figure previously published in \citet{Ramirez-Gonzalez2015b}. }
\label{fig:yr15:bfrChange}
\end{figure}

\section{\textit{In silico} mapping}
\label{sub:yr15:inSilico}
From the mapped SNPs with a $BFR>6$, 872 of 1,582 ($\sim60\%$) were assigned to the chromosomes in group 1 of hexaploid wheat, being the only group with more than $4\%$ of the SNPs assigned to it (Table \ref{app:yr15:bfr6Mapping}). 
From the group 1, the B genome contained the higher proportion of SNPs mapped ($54\%$), having 255 ($54\%$) and 214 ($46\%$) assigned to the long and short arms respectively (Figure \ref{fig:yr15:snpsBFR6Group1}).  
This results are expected since previous studies have located \acrshort{yr15} near the centromere in the short arm of chromosome 1B and, the \acrshort{yr15} introgression contains regions from the long and short arm from \textit{T. diccocoides} \citep{Murphy2009,Peng2000,Grama1997}. 

\begin{SCfigure}
  \centering
    \includegraphics[width=0.5\textwidth]{Yr15/Figures/mapping/snpsBFR6Group1.pdf}
  \caption[Location of SNPs with BFR\textgreater6.]{Location of SNPs with $BFR>6$ according to the best alignment of the UniGene (red) and UCW (blue) gene models to the flow-sorted group 1 chromosomes from the \gls{css} \citep{Mayer2014}. Figure adapted from \citet{Ramirez-Gonzalez2015b}.} 
  \label{fig:yr15:snpsBFR6Group1}
\end{SCfigure}

\input{Yr15/SupplementalTables/bfr6Mapping}

\begin{figure}
  \centering
  \begin{subfigure}{0.6\textwidth}
  \caption{}
   \label{fig:yr15:snpsBFR6Chr1B}
   \includegraphics[width=1\textwidth]{Yr15/Figures/mapping/snpsBFR6crh1B.pdf}
  \end{subfigure}
  ~
  \begin{subfigure}{0.35\textwidth}
  \caption{}
   \label{fig:yr15:BFRValues1BS}
   \includegraphics[width=1\textwidth]{Yr15/Figures/mapping/BFRValues1BS.pdf}
  \end{subfigure}
\caption[\textit{In silico} location of SNPs with BFR\textgreater6]{\textit{In silico} location of SNPs with $BFR>6$. (\subref{fig:yr15:snpsBFR6Chr1B}) Number of SNPs with $BFR>6$ per cM in chromosome 1B. (\subref{fig:yr15:BFRValues1BS}) BFRs of mapped genes with SNPs on chromosome 1B. The area of the circle represents the number of SNPs clustered by location (windows size: 10 cM) and BFR (window size: 5cM). R11 is the only marker near the \acrshort{yr15} locus that had a corresponding position in the genetic map. The percentage of genes with SNPs per cM is also illustrated based on UCW (blue) and UniGene (red) gene models. The centromere is imputed by the centre of a window of 10 cM where the short arm switches to the long in the genetic map. BFRs correspond to those from the mixed in silico bulk S1 + S2 + S3/R1 + R2 + R3. Adapted from \citep{Ramirez-Gonzalez2015b}.} 
\label{fig:yr15:chr1}
\end{figure}

The \acrshort{css} assembly was used as a common reference between the reference genes and the 40,266 SNP markers published at the time \citep{Wang2014} to locate the SNPs with a $BFR>6$ (including $BFR=\infty$) in a genomic position (Figures \ref{fig:yr15:chr1}, \ref{fig:yr15:bfrs:0-6}).  
From the 1,582 SNPS across 1,173 genes,  only 678 SNPs ($43\%$, 474 genes) were successfully located in the genetic map. 
Since the \acrshort{css} assembly is quite fragmented, the low percentage of located SNPs can be because not all candidate SNPs had a corresponding scaffold that has at least one of the 40,266 markers in the genetic map. 
Even if the number of located SNPs was not enough to give a position for over $50\%$ of the SNPs from the parental line, the resolution of the genetic position SNPs that were assigned improved over just having the chromosome arm information from the CSS assembly. 
The mapping position further confirmed an enrichment of SNPs near the centromere of chromosome 1B with 325 out of 678 SNPs. 
Furthermore, 311 of those where located within an interval of 30cM (Figures \ref{fig:yr15:bfr6}, \ref{fig:yr15:snpsBFR6Chr1B}). 

Studies in diploid organismis using \acrshort{qtl}-Seq \citep{Takagi2013} or other \acrshort{ngs}-enable genetic approaches \citep{James2013} have shown smooth curves with a defined peak in the region linked to the studied trait. 
In practice, we only observe clusters of SNPs with  enriched \acrshortpl{bfr} near the centromere of chromosome 1B (Figures \ref{fig:yr15:snpsBFR6Chr1B}, \ref{fig:yr15:bfr6}). 

\begin{figure}
	\centering
	\begin{subfigure}{0.4\textwidth}
	\caption{}
	\label{fig:yr15:bfr0}
	\includegraphics[height=0.55\textheight]{Yr15/Figures/mapping/snpsBFR0.pdf}
	\end{subfigure}
	~
	\begin{subfigure}{0.45\textwidth}
	\caption{}
	\label{fig:yr15:bfr6}
	\includegraphics[height=0.55\textheight]{Yr15/Figures/mapping/snpsBFR6.pdf}
	\end{subfigure}
	\caption[Genetic location of genes with SNPs between AVS and Yr15.]{Genetic location of genes with SNPs between AVS and Yr15. The colour scale indicates the percentage of genes with SNPs per centi-Morgan (cM) across the 21 wheat chromosomes. The location of the genes was determined by the best alignment to the CSS scaffolds, and the location of these was determined by their position on a genetic map \citep{Wang2014} (\subref{fig:yr15:bfr0}). All the SNPS between progenitors. Note the lack of enrichment across any individual chromosome. (\subref{fig:yr15:bfr6}) SNPs with BFR$>$6. Note the enrichment in Chromosome 1B }
	\label{fig:yr15:bfrs:0-6}
\end{figure}

The location of the clusters with an enrichment of SNPs near the centromere is not expected on a random selection of genes, as the gene density increases with the distance to the centromere \citep{Akhunov2003}. 
This suggests that the experiment was successful on finding \acrshortpl{snp} linked to \acrshort{yr15}. 
There are several factor that prevent a clear peak; these include the biases induced by the differential expression and the fragmented reference sequence with scaffolds that are not long enough to go across genetic positions. 
Since there are several SNPs with a high BFR and the genetic map is not enough to locate a single region linked to \acrshort{yr15},  multiple criteria were needed to prioritise SNPs that were more likely to yield on successful genetic markers.

\section{Assay selection} 
\label{yr15:assaySelection}
\begin{figure}
\centering
\includegraphics[width=1\textwidth]{Yr15/Figures/selection/snpSets.pdf}
\caption[Selection criteria for marker design]{Selection criteria for marker design. Venn diagrams based on the three selection criteria (SNP in the short arm of chromosome group 1; SNP has a $BFR>6$; and SNP is from the \acrshort{yr15} parent) for the UCW (blue) and UniGene (red) gene models. The centre diagram shows the intersection between common genes matching all three criteria across both data sets. Note that the numbers are not directly additive as in cases, multiple models from one reference set will relate to a single gene model in the other values. Published in \citep{Ramirez-Gonzalez2015b} }
\label{fig:yr15:snpset}
\end{figure}

Three independent criteria were use to prioritize the SNPs for marker development and validation: 

\begin{description}
\item[High BFR.] SNPs with a $BFR>6$ in at least two independent bulk replicates or in either of the \textit{in silico} mixes were selected to ensure consistency and recover SNPs with a low coverage on a particular bulk. 
\item[Group 1S.] SNPs that are in \acrshort{css} scaffolds in the short arm of chromosome group 1 were selected.
The selection included \glspl{snp} from the A, B and D genomes because the best hit to each gene model may be missing from the \gls{css} assembly, hence locating the gene in the wrong chromosome, where an homoeologues exist. 
Likewise, if the gene model corresponding in 1B, the reads would have mapped to one of the homoeologues.
This is to be consistent with the \textit{in silico} genetic map and with previous studies \citep{Murphy2009,Peng2000,Grama1997}.
\item[\acrshort{yr15} parent.] The SNPs should originate from the \acrshort{yr15} parent to ensure that the SNP is coming from the \textit{T. diccocoides} introgression and not from a SNP in the \acrshort{avs} genetic background, who would be less useful in breeding programs with a different background.
\end{description}

Only SNPs meeting the three criteria were selected for further analysis. 

\begin{figure}
\centering
\includegraphics[width=1\textwidth]{Yr15/Figures/selection/selectionDetals.pdf}
\caption[BFRs of selected SNPs across bulks.]{BFRs of selected SNPs across the individual bulks and in silico mixes (UCW, red; UniGene, blue). The dotted line represents the BFR threshold of 6 (logarithmic scale). Left: Distribution of the BFRs for each selection criteria and the selected SNPs for validation. The circles on the top of each plot represent the percentage of SNPs with $BFR=\infty$. The Selection may include SNPs with $BFR<6$ when the same SNP has a higher score on the complementing reference (ie. $BFR>6$ on UCW, but $BFR<6$ on UniGenes). Right: The BFR values of selected SNPs were sorted in descending order across the different bulks and according to their origin. Validated SNPs are indicated by open triangles, and SNPs corresponding to markers R5, R8 and R11 are labelled across different bulks and mixes. Note that some SNPs are below the threshold in a specific bulk as they meet the BFR criteria across others. }
\label{fig:yr15:bfrDetailScore}
\end{figure}

With the multiple criteria the number of genes with a putative SNP went down from $>27,000$ to just 175; 77 and 98 from the UniGene and UCW gene sets respectively. 
The selected genes from both references were aligned between references, as they come from independent sources an overlap in the selection between them is expected and, as expected, around half of the genes between gene sets overlap (Figure \ref{fig:yr15:snpset}). 
The 50 SNPs with the highest BFRs, out of the 175 genes, were selected for validation, 15 of them were redundant between references, resulting on 35 SNPs to validate. 

The separate bulks and the \textit{in silico} mixes were evaluated in detail to understand the behaviour and value of having multiple bulks. 
The initial expectation was that as the number of SNPs with $BFR=\infty$ should drop in the mixes, as the improved coverage should reduce the instances were the absence of an allele is because of the lack of coverage on a particular sample. 
However, the opposite happened, the additional coverage in the \textit{in silico} mixes recovered SNPs in genes with a low expression at the time of the sampling (Figure \ref{fig:yr15:bfrDetailScore}).  
Some SNPs were present across all the samples, however the value of the BFR changed depending on the sample (marker R5). 
In some cases a \gls{snp} is missing in an individual bulk, but present in the rest of them and in the mixes (marker R8). 
The main reason affecting the scoring is the coverage in the sample for each particular gene, hence a strategy with a consistent coverage would be preferred for this kind of analysis.  
Previous studies have shown that a coverage of $<5x$ is enough to call for SNPs in model organisms with a high-quality reference \citep{Schneeberger2011}.
However, the results on this study are in line with other studies using populations for SNP calling \citep{Abe2012,Takagi2013}. 
The non-uniform distribution of the coverage in RNA-Seq experiments affects the number of reads that can be used to call for SNPs, specially on genes with a low expression level \citep{Mortazavi2008}. 

%!TEX root = ../../Main.tex
\begin{table}
\caption{ Number of genes (and SNPs) with a unique hit ($>99\%$ sequence identity) to a single wheat survey sequence scaffold. }
\label{tab:yr15:mappedGenes}
\centering
\begin{localsize}{9}{11}
\begin{tabular}{llrr@{\extracolsep{6pt}}rr@{\extracolsep{6pt}}rr}

\toprule
 \multicolumn{2}{l}{Chromosome 1}            &  \multicolumn{2}{c}{All SNPs} &  \multicolumn{2}{c}{BFR\ensuremath{>}6 }  &    \multicolumn{2}{c}{ \%  BFR\ensuremath{>}6 }       \\
  \cline{3-4}
  \cline{5-6}
  \cline{7-8}
                &            & SNP        & Genes  & SNP     & Genes  & SNPs               & Genes  \\
\midrule
 UCW            & Unique     & 5,283      & 1,245  & 311     & 214    & 5.89\%              & 17.19\% \\
                & Total      & 8,086      & 1,954  & 486     & 330    & 6.01\%              & 16.89\% \\
                & Percentage & 65.34\%     & 63.72\% & 63.99\%  & 64.85\% &                    &        \\
 \midrule
 UniGene        & Unique     & 3,687      & 745    & 213     & 139    & 5.78\%              & 18.66\% \\
                & Total      & 6,422      & 1,318  & 386     & 246    & 6.01\%              & 18.66\% \\
                & Percentage & 57.41\%     & 56.53\% & 49.17\%  & 56.07\% &                    &        \\
 \midrule
 UCW  & Unique     & 8,970      & 1,990  & 524     & 353    & 5.84\%              & 17.74\% \\
+              & Total      & 14,508     & 3,272  & 872     & 576    & 6.01\%              & 17.60\% \\
 UniGene       & Percentage & 61.83\%     & 60.82\% & 60.09\%  & 61.28\% &                    &        \\
\bottomrule
                &            &            &        &         &        &                    &        \\
\toprule
 All SNPs       &            &  \multicolumn{2}{c}{All SNPs} &  \multicolumn{2}{c}{BFR\ensuremath{>}6 }  &    \multicolumn{2}{c}{ \% BFR\ensuremath{>}6 }         \\
\cline{3-4}
\cline{5-6}
\cline{7-8}
                &            & SNP        & Genes  & SNP     & Genes  & SNPs               & Genes  \\
 \midrule
 UCW            & Unique     & 39,247     & 9,585  & 481     & 368    & 1.23\%              & 3.84\%  \\
                & Total      & 66,426     & 16,022 & 859     & 643    & 1.29\%              & 4.01\%  \\
                & Percentage & 59.08\%     & 59.82\% & 56.00\%  & 57.23\% &                    &        \\
 \midrule
 UniGene        & Unique     & 27,292     & 5,698  & 344     & 252    & 1.26\%              & 4.42\%  \\
                & Total      & 52,262     & 11,056 & 723     & 530    & 1.38\%              & 4.79\%  \\
                & Percentage & 52.22\%     & 51.54\% & 47.58\%  & 47.55\% &                    &        \\
 \midrule
 UCW  & Unique     & 66,539     & 15,283 & 825     & 620    & 1.24\%              & 4.06\%  \\
 +               & Total      & 118,688    & 27,078 & 1,582   & 1,173  & 1.33\%              & 4.33\%  \\
 UniGene       & Percentage & 56.06\%     & 56.44\% & 52.15\%  & 52.86\% &                    &        \\
\bottomrule
\end{tabular}
\end{localsize}
\end{table}


%TODO: maybe move to the discussion. 
Around $60\%$ of the gene models, across both references, had a unique hit with $>99\%$ sequence identity to a single \acrshort{css} scaffold (Table \ref{tab:yr15:mappedGenes}). 
This is likely because there is no unique homoeologue in the gene models, leading to reads, from two different homoeologues, mapping to the same region.
To reduce the number of spurious SNPs we used IUAPC ambiguity codes (Section \ref{lit:ambiguity}, \citet{Cornish-Bowden1985}) when two different alleles were observed.
This had as side effect that in order to keep only high confidence SNPs we required a higher coverage ($>20x$). 
On the original study introducing the BFR in tetraploid wheat, the authors show that increasing the coverage, from $8x$ to $16x$, reduces the putative SNPs by $60\%$, but the validated SNPs increas from $57\%$ to $83\%$ \citep{Trick2012}. 
Hence, a compromise between increasing the minimum coverage at the cost of reducing the SNP candidates had to be reached in line with the objectives and available resources for this particular study. 

\section {SNP Validation}

KASP assays were designed to validate and generate a genetic map of the \acrshort{yr15} locus for the 35 selected SNPs. 
To automate the design of genome-specific primers for polyploid organisms PolyMarker was developed (Chapter \ref{cha:polymarker}).
Out of the 35 assays to design, 17 were design as specific, 9 as semi-specific to chormosome 1BS, and 9 were not specific because there was no information for the homoeolouges on the \acrshort{css} scaffolds. 
PolyMarker also identified putative homoeologous variants (between genomes, as opposed to between varieties) that were in the list of candidate SNPs, but were not identified previously (Figure \ref{fig:poly:mask}; Table \ref{tab:yr15:polymarker}). 


\begin{figure}[b!]
\begin{subfigure}{0.31\textwidth}
\caption{}
\label{fig:yr15:r2}
\includegraphics[width=1\textwidth]{Yr15/Figures/selection/R2.pdf}
\end{subfigure}
~
\begin{subfigure}{0.31\textwidth}
\caption{}
\label{fig:yr15:r8}
\includegraphics[width=1\textwidth]{Yr15/Figures/selection/R8.pdf}
\end{subfigure}
~
\begin{subfigure}{0.31\textwidth}
\caption{}
\label{fig:yr15:r8f2}
\includegraphics[width=1\textwidth]{Yr15/Figures/selection/R8f2.pdf}
\end{subfigure}

\caption[KASP output from the wheat variety panel]{KASP output from the wheat variety panel with (Ochre, Boston, Cortez) and without (Robigus, Cadenza and Shamrock) \acrshort{yr15}. Marker  R2 (\subref{fig:yr15:r2}) is monomorphic while R8 (\subref{fig:yr15:r8})  is polymorphic between varieties know to carry the gene.  Marker R8 results for the F2 population (\subref{fig:yr15:r8f2}) showing three distinct clusters. The central cluster (light green) is comprised of heterozygous individuals, whereas clusters near the axes are homozygous for either AVS (VIC; orange) or \acrshort{yr15} (FAM; dark green).}
\end{figure}


To validate if the 35 SNPs were polymorphic across the parents and, diagnostic to \acrshort{yr15} we tested them in the progenitors plus six commercial varieties, three containing \acrshort{yr15} (Ochre, Boston and, Cortez) and three without it (Shamrock, Robigus and, Cadenza).
Two of the lines without \acrshort{yr15} have \textit{T. diccocoides} in their pedigree (Shamrock and Robigus), as it is the donor species of \textit{Yr15} \citep{mcintosh1995}. 
This test panel allows to test if the SNPs are only diagnostic to \textit{T. diccocoides} instead of \acrshort{yr15}.
On the test panel, 28 ($80\%$) SNPs were polymorphic across the parents and three of them where diagnostic to \textit{Yr15} (R5, R8, R33).
From the five homoeologous SNPs, three of them were monomorphic and two polymorphic, suggesting that PolyMarker is effective on detecting which assays are less likely to work (Table \ref{tab:yr15:markersToTest}; Figure \ref{fig:yr15:r2},\subref{fig:yr15:r8}).
The segregation of the SNPs in the full $F_{2}$ population (Section \ref{sub:mappingPopulation}, Figure \ref{fig:yr15:r8f2}) and a genetic map was produced (Section \ref{yr15:geneticMap}).   



%!TEX root = ../../Main.tex
\begin{sidewaystable}
\caption{Primer details for the markers to validate. }
\centering
\label{tab:yr15:polymarker}
\begin{localsize}{6}{9}

\begin{tabular}{lllllll}
\toprule
 Assay ID   & Polymorphism\_type   & AVS-specific primer         & Yr15-specific primer        & Common primer               & Specificity             & Orientation   \\
\midrule
 R1         & non-homeologous     & aactggtaatggtgcagCgG        & aactggtaatggtgcagCgC        & ttcaggataacacAggagatgtT     & chromosome\_semispecific & reverse       \\
 R2         & non-homeologous     & acatcaattcttcaggaaagctctaC  & acatcaattcttcaggaaagctctaT  & gcacagcttctcgtgttcTT        & chromosome\_specific     & forward       \\
 R3         & non-homeologous     & acgtggagaacctagattgcG       & acgtggagaacctagattgcC       & ccttttaggtgcgccaactT        & chromosome\_semispecific & reverse       \\
 R4         & non-homeologous     & agactctttgggcagtggatC       & agactctttgggcagtggatT       & cctcgggcgatctattctcT        & chromosome\_specific     & forward       \\
 R5         & non-homeologous     & agtcaacttggattacactgaagtT   & agtcaacttggattacactgaagtC   & agatatcacactgaacatactgatgaG & chromosome\_specific     & reverse       \\
 R6         & non-homeologous     & caagatgaagatgaagaggaatatgaT & caagatgaagatgaagaggaatatgaC & gCttgaccctgtaatcatactcG     & chromosome\_semispecific & forward       \\
 R7         & non-homeologous     & caccaccaTggaggccaC          & caccaccaTggaggccaT          & cgccgtggtagtgtccgG          & chromosome\_specific     & forward       \\
 R8         & non-homeologous     & cagatccccggttctctcaaG       & cagatccccggttctctcaaA       & cccccaaatgatcgagaata        & chromosome\_inspecific   & reverse       \\
 R9         & non-homeologous     & caggtgctgaaatgcatcC         & caggtgctgaaatgcatcT         & cggcctatcttcaggtctgt        & chromosome\_inspecific   & reverse       \\
 R10        & non-homeologous     & cattcgtcgcgccttctacG        & cattcgtcgcgccttctacA        & tcctaactcatatgcatgactcAC    & chromosome\_specific     & reverse       \\
 R11        & non-homeologous     & ccattctgatcaaggtcactgtcG    & ccattctgatcaaggtcactgtcA    & ttctgtaTggcaaCgggagC        & chromosome\_specific     & reverse       \\
 R12        & homeologous         & cttagccagtgaaccAggcC        & cttagccagtgaaccAggcT        & ggctgtttgttacCgtggaG        & chromosome\_specific     & reverse       \\
 R14        & non-homeologous     & gacTacAggtgcgatcccC         & gacTacAggtgcgatcccT         & ctcgcctgccagtcgTaT          & chromosome\_specific     & forward       \\
 R15        & homeologous         & gactagggctaccAttgttgA       & gactagggctaccAttgttgC       & agccctgCtaacaatggcaA        & chromosome\_specific     & reverse       \\
 R16        & non-homeologous     & gatgtaagcTAtgactggCgC       & gatgtaagcTAtgactggCgT       & tgcaactgatctttagcaggC       & chromosome\_semispecific & reverse       \\
 R17        & non-homeologous     & gcaAcaacaaCaaCaagtggT       & gcaAcaacaaCaaCaagtggC       & cctcaacctgcttgttgttgT       & chromosome\_specific     & forward       \\
 R19        & non-homeologous     & gcctgatttttaattcgctccaG     & gcctgatttttaattcgctccaA     & agagcactgatgatgacccC        & chromosome\_specific     & reverse       \\
 R20        & non-homeologous     & gctgtatcctcttgaaaaaggcT     & gctgtatcctcttgaaaaaggcC     & ttaggcatgtcagaaatgtagaaaa   & chromosome\_semispecific & forward       \\
 R21        & non-homeologous     & gcttcaaacatgccggctG         & gcttcaaacatgccggctT         & cggtctttttcaaccagggC        & chromosome\_semispecific & forward       \\
 R22        & homeologous         & gctTgtCttaaagccAtttccA      & gctTgtCttaaagccAtttccG      & gcctatcgttCgctaaactctaacT   & chromosome\_specific     & reverse       \\
 R23        & non-homeologous     & gctttaggcactatggattcAcC     & gctttaggcactatggattcAcT     & caggtttctgttcgacctcA        & chromosome\_specific     & forward       \\
 R24        & non-homeologous     & ggaggtcctacacgcgtctT        & ggaggtcctacacgcgtctG        & ctccaaaagaggggcatcattT      & chromosome\_semispecific & forward       \\
 R25        & non-homeologous     & gggttcctcacctgcgcC          & gggttcctcacctgcgcT          & ctctTtgcaatcggccagc         & chromosome\_inspecific   & reverse       \\
 R26        & non-homeologous     & gtCttcgcCggcacCacC          & gtCttcgcCggcacCacT          & agtggatcttgccgatctcg        & chromosome\_inspecific   & forward       \\
 R28        & non-homeologous     & tagatgagaccttggaCggA        & tagatgagaccttggaCggG        & cagtcatctaatgcggaacattcA    & chromosome\_semispecific & reverse       \\
 R29        & non-homeologous     & TatggtGtggccTtccccG         & TatggtGtggccTtccccA         & cgagctcgctgatgaacttG        & chromosome\_specific     & forward       \\
 R30        & non-homeologous     & tcagcagcccttttaacccaA       & tcagcagcccttttaacccaT       & agtaaatcgggcacggttgt        & chromosome\_inspecific   & reverse       \\
 R31        & homeologous         & tcatccatgtatatGaaTccaagcC   & tcatccatgtatatGaaTccaagcA   & tcacgcctgcaacAttcaaaT       & chromosome\_specific     & reverse       \\
 R32        & homeologous         & tccaatcttatggctttgcttctG    & tccaatcttatggctttgcttctT    & caggtgatgtagatgctgagaC      & chromosome\_semispecific & reverse       \\
 R33        & non-homeologous     & tccttcctgctatagctgaaagG     & tccttcctgctatagctgaaagT     & ccctttgcctgccatgtaga        & chromosome\_inspecific   & forward       \\
 R34        & non-homeologous     & tctgagatgatgatactTtgtggG    & tctgagatgatgatactTtgtggA    & actggggatgccctctgtat        & chromosome\_inspecific   & forward       \\
 R35        & non-homeologous     & tgaaagagtggaatttcttgttgT    & tgaaagagtggaatttcttgttgC    & ctttTagctgcttaattctattgcttC & chromosome\_specific     & forward       \\
 R36        & non-homeologous     & tgaaatgccttgtcaatgccA       & tgaaatgccttgtcaatgccG       & ATGCGAATTGGGGAATTAAA        & chromosome\_inspecific   & reverse       \\
 R37        & non-homeologous     & tgcatatgcctgaagagactcG      & tgcatatgcctgaagagactcA      & tgtccacctactcaagtctgc       & chromosome\_inspecific   & reverse       \\
 R38        & non-homeologous     & tgGcCaagTtTttctgcaagaT      & tgGcCaagTtTttctgcaagaG      & tgtaggaaggaactcCgaagtA      & chromosome\_specific     & forward       \\
 R40        & non-homeologous     & tgcatatgcctgaagagactcA      & tgcatatgcctgaagagactcG      & agtccgctaaagcattgcct        & chromosome\_nonspecific  & reverse       \\
 R43        & non-homeologous     & tcgctgatttcatcatgtcccA      & tcgctgatttcatcatgtcccG      & tcaggtgctgcaaatttgagG       & chromosome\_semispecific & forward       \\
\bottomrule
\end{tabular}
\end{localsize}
\end{sidewaystable}

%!TEX root = ../../Main.tex
\begin{sidewaystable}
\caption{Results of validation of primers on the progenitors (\acrshort{avs} and \acrshort{yr15}, varieties known to contain \acrshort{yr15} (Cortez, Ochre and, Boston) } and, varieties without \acrshort{yr15} (Robigus, Cadenza and, Shamrock). Shamrock and Robigus have \textit{T. dicoccoides} introgressions. The bold markers are diagnostic in the panel (R5, R8, R88) or in the genetic map (R11). 

\label{tab:yr15:markersToTest}
\begin{localsize}{6}{9}
\centering 
\begin{tabular}{llllllll!{\extracolsep{4pt}}lllllll}
\toprule
Assay &             &                    &        & \begin{sideways}Yr15\end{sideways}    & \begin{sideways}Ochre\end{sideways}  & \begin{sideways}Boston\end{sideways}    & \begin{sideways}Cortez\end{sideways}    & \begin{sideways}Shamrock \end{sideways}    & \begin{sideways}Robigus\end{sideways}    & \begin{sideways}Cadenza\end{sideways}    & \begin{sideways}AVS\end{sideways} & \begin{sideways}Polymorphic\end{sideways} & \begin{sideways}Linked \acrshort{yr15}\end{sideways}  &\\
\cline{5-8}
\cline{9-12}
 ID   & Gene set    & Gene model name    & SNP    & \multicolumn{4}{c}{\acrshort{yr15}+ } & \multicolumn{4}{c}{\acrshort{yr15}- } &   &      & comment                 \\
\midrule
 R1         & UCW         & UCW\_Tt-k55\_contig\_8830;tt-k21\_contig\_10204                      & C341G  & A      & H         & A        & A        & A            & -         & A         & B     & Yes           & * &   segregation distortion                      \\
 R2         & UniGene v60 & gnl$|$UG$|$Ta\#S13126619                                             & C491T  & B      & B         & B        & B        & B            & B         & B         & B     & No            & -                      &                         \\
 R3         & UCW         & contig95240                                                     & C220G  & H      & B         & B        & B        & B            & B         & B         & B     & Yes           & Yes                    &                         \\
 R4         & UCW         & contig105384                                                    & C1227T & A      & B         & B        & B        & B            & B         & B         & B     & Yes           & Yes                    &                         \\
 \textbf{R5}         & UniGene v60 & gnl$|$UG$|$Ta\#S58861868                                             & A214G  & \textbf{A}      & \textbf{A}         & \textbf{A}        & \textbf{A}        & \textbf{B}            & \textbf{B}         & \textbf{B}         & \textbf{B}    & Yes           & Yes                    &                         \\
 R6         & UCW         & KukriC706\_1                                                     & T2979C & A      & H         & B        & B        & B            & B         & H         & H     & Yes           & No                     &                         \\
 R7         & UniGene v60 & gnl$|$UG$|$Ta\#S37932863                                             & C281T  & H      & A         & A        & A        & B            & B         & A         & B     & Yes           & No                     &                         \\
 \textbf{R8}         & UniGene v60 & gnl$|$UG$|$Ta\#S58863387                                             & T241C  & \textbf{B}      & \textbf{B}         & \textbf{B}        & \textbf{B}        & \textbf{A}            & \textbf{A}         & \textbf{A}         & \textbf{A}    & Yes           & Yes                    &                         \\

 R9         & UniGene v60 & gnl$|$UG$|$Ta\#S58892239                                             & C303T  & H      & B         & A        & B        & B            & B         & H         & B     & Yes           & No                     &                         \\
 R10        & UCW         & UCW\_Tt-k63\_contig\_79829                                         & C207T  & H      & A         & B        & A        & B            & B         & B         & B     & Yes           & Yes                    &                         \\
 \textbf{R11}        & UCW         & UCW\_Tt-k45\_contig\_39011                                         & C726T  & \textbf{A}       & \textbf{A}          & \textbf{A}         & \textbf{H}         & \textbf{-}             & \textbf{B}          & \textbf{B}         & \textbf{B}     & Yes           & Yes                    &                         \\
  R12        & UCW         & contig50308                                                     & G587A  & -      & H         & H        & H        & B            & B         & A         & B     & Yes           & Yes                    &                         \\
 R14        & UniGene v60 & gnl$|$UG$|$Ta\#S44692929                                             & C549T  & A      & A         & -        & A        & A            & B         & -         & B     & Yes           & Yes                    &                         \\
 R15        & UCW         & UCW\_Tt-k51\_contig\_2344;tt-k55\_contig\_2091                       & T686G  & A      & A         & A        & A        & A            & A         & A         & A     & No            & -                      &                         \\
 R16        & UniGene v60 & gnl$|$UG$|$Ta\#S17898149                                             & G227A  & A      & A         & B        & A        & B            & B         & B         & B     & Yes           & Yes                    &                         \\
 R17        & UCW         & CL3339Contig1                                                   & T509C  & H      & H         & H        & H        & H            & H         & H         & H     & No            & -                      &                         \\
 R19        & UCW         & UCW\_Tt-k21\_contig\_8407;tt-k61\_contig\_5972                       & C1405T & A      & B         & B        & B        & B            & B         & B         & B     & Yes           & Yes                    &                         \\
 R20        & UCW         & UCW\_Tt-k21\_contig\_8407;tt-k61\_contig\_5972                       & T1102C & A      & B         & B        & B        & -            & B         & B         & B     & Yes           & Yes                    &                         \\
 R21        & UCW         & UCW\_Tt-k31\_contig\_53804;tt-k41\_contig\_31582                     & G1810T & H      & B         & B        & B        & B            & B         & B         & B     & Yes           & Yes                    &                         \\
 R22        & UCW         & UCW\_Tt-k31\_contig\_14966                                         & T408C  & A      & A         & A        & A        & A            & A         & A         & B     & Yes           & Yes                    &                         \\
 R23        & UCW         & UCW\_Tt-k51\_contig\_12731;tt-k55\_contig\_13077;tt-k61\_contig\_18734 & C50T   & A      & H         & H        & H        & H            & -         & H         & B     & Yes           & Yes                    &                         \\
 R24        & UCW         & UCW\_Tt-k55\_contig\_8830;tt-k21\_contig\_10204                      & T3005G & H      & H         & B        & H        & B            & B         & B         & B     & Yes           & Yes                    &                         \\
 R25        & UCW         & UCW\_Tt-k63\_contig\_79829                                         & G184A  & A      & A         & A        & A        & H            & H         & A         & A     & No            & -                      &                         \\
 R26        & UCW         & UCW\_Tt-k21\_contig\_3794                                          & C702T  & H      & A         & B        & A        & B            & B         & B         & B     & Yes           & Yes                    &                         \\
 R28        & UCW         & KukriC3701\_1                                                    & T1053C & A      & A         & B        & A        & B            & B         & B         & B     & Yes           & Yes                    &                         \\
 R29        & UCW         & UCW\_Tt-k55\_contig\_8640;tt-k41\_contig\_8875                       & G783A  & H      & A         & B        & A        & B            & B         & B         & B     & Yes           & Yes                    &                         \\
 R30        & UCW         & UCW\_Tt-k55\_contig\_8830;tt-k21\_contig\_10204                      & T2184A & A      & A         & B        & A        & B            & B         & A         & B     & Yes           & Yes                    &                         \\
 R31        & UCW         & UCW\_Tt-k45\_contig\_22098                                         & G683T  & A      & B         & A        & B        & B            & B         & A         & B     & Yes           & Yes                    &                         \\
 R32        & UCW         & UCW\_Tt-k21\_contig\_33188;tt-k25\_contig\_30647                     & C596A  & H      & A         & A        & A        & A            & A         & A         & H     & No            & -                      &                         \\
 \textbf{R33}        & UniGene v60 & gnl$|$UG$|$Ta\#S58861868                                             & G486T  & \textbf{A}      & \textbf{A}         & \textbf{A}        & \textbf{A}        & \textbf{B}            & \textbf{B}         & \textbf{B}         & \textbf{B}     & Yes           & Yes                    &                         \\
 R34        & UCW         & UCW\_Tt-k31\_contig\_34099                                         & G1713A & H      & A         & B        & -        & B            & B         & B         & B     & Yes           & No                     &                         \\
 R35        & UniGene v60 & gnl$|$UG$|$Ta\#S58900202                                             & T889C  & A      & B         & B        & B        & B            & B         & B         & B     & Yes           & Yes                    &                         \\
 R36        & UCW         & UCW\_Tt-k55\_contig\_8830;tt-k21\_contig\_10204                      & T2349C & H      & H         & H        & H        & H            & -         & H         & B     & Yes           & Yes                    &                         \\
 R37        & UCW         & UCW\_Tt-k31\_contig\_34099                                         & C846T  & B      & B         & B        & B        & B            & B         & B         & B     & No            & -                      &                         \\
 R38        & UniGene v60 & gnl$|$UG$|$Ta\#S58840501                                             & T179G  & B      & B         & B        & B        & B            & B         & B         & B     & No            & -                      &                         \\
 R40        & UCW         & UCW\_Tt-k31\_contig\_34099                                         & C846T  & A      & H         & B        & A        & -            & B         & B         & B     & Yes           & No                     & based on barley synteny \\
 R43        & UniGene v60 & gnl$|$UG$|$Ta\#S58843705                                             & G268A  & A      & B         & B        & -        & -            & B         & -         & B     & Yes           & Yes                    & based on barley synteny \\
\bottomrule
\end{tabular}
\end{localsize}
\end{sidewaystable}


\section{Genetic map} 
\label{yr15:geneticMap}


Initially, the 28 polymorphic markers were used to genotype a subset of 66 plants from the $F_{2}$ population. 
From those, 23 (82\%) were linked to \acrshort{yr15} and several markers map in a small interval around \acrshort{yr15} (Figure \ref{fig:yr15:initialMap}; Table \ref{tab:yr15:markersToTest}), confirming that the multiple-criteria strategy(Section \ref{yr15:assaySelection}) for selecting candidate SNPs was effective. 
Then, the complete $F_{2}$ population was assessed with:
\begin{itemize}	
	\item  the seven markers that were most closely linked to \acrshort{yr15}, including two of the diagnostic markers from the variety panel (R5 and R8),
	\item The flanking \acrshort{ssr} microsatellite markers used by UK breeders for germoplasm selection (\textit{Xbarc8} and \textit{Xgwm413}).  These correspond to the best markers available to breeders at the time of the study.
	\item A marker based on barley-wheat synteny (R43) which met the selection criteria, but wasn't on the original set of 50 markers with high BFR. 
\end{itemize}

The $F_{2}$ population consisted on 232 plants with phenotypic information, of those 196 where genotyped reliably (no more than one data point missing). 
Using the eight SNP markers and 2 SRRs, the \acrshort{yr15} locus was mapped to an interval of 0.77cM, with R8/xgwm413 0.26cM distal, and R5/R11 0.77cM proximal from \acrshort{yr15} (Figure \ref{fig:yr15:finalMap},\subref{fig:yr15:mapDetails}). 




\begin{figure}
	\centering
	\begin{subfigure}{0.45\textwidth}
	\caption{}
	\label{fig:yr15:initialMap}
	\includegraphics[height=0.45\textheight]{Yr15/Figures/selection/initialMap.pdf}
	\end{subfigure}
	~
	\begin{subfigure}{0.45\textwidth}
	\caption{}
	\label{fig:yr15:finalMap}
	\includegraphics[height=0.45\textheight]{Yr15/Figures/selection/fineMap.pdf}
	\end{subfigure}

	\begin{subfigure}{1\textwidth}
	\caption{}
	\label{fig:yr15:mapDetails}
	\includegraphics[width=1\textwidth]{Yr15/Figures/selection/mapDetails.pdf}
	\end{subfigure}
	

	\caption[Genetic maps for \acrshort{yr15}.]{Genetic maps for \acrshort{yr15}. (\subref{fig:yr15:initialMap}) Genetic map of the test panel from 50 individuals. (\subref{fig:yr15:finalMap}) Genetic map from 196 individuals from the full population only with the 8 markers previously identified as closer to the \acrshort{yr15} locus. (\subref{fig:yr15:mapDetails})Graphical genotype of the 196 $F_{2}$ individuals used to develop the genetic map. The alleles are abbreviated according to their origin: A: AVS; B: \textit{Yr15} and H: Heterozygous. Missing calls are indicated by a hyphen.}
\end{figure}

The sub-cM resolution is expected in a $F_{2}$ population of 196 individuals, as 392 gametes should provide n average resolution of 0.26{}cM. 
Despite the fact that none of the selected markers have perfect linkage to \acrshort{yr15}, the produce genetic map is an improvement in the resolution of the map for the locus and it enables the shift to SNP markers from microsatellites, which has become the preferred marker system in \acrshort{mas} pipelines in breeding programmes. 


\section{Methods}
\label{yr15:methods}

\begin{figure}
\includegraphics[width=1\textwidth]{Yr15/Figures/pipeline.pdf}
\caption{Steps used to go from the $F_{2}$ population to the genetic map.}
\end{figure}

The data analysis for this PhD required the use of some standard tools and custom developed code. 
All the code produced for this project is available and updated on the a github repository: \url{https://github.com/TGAC/bioruby-polyploid-tools}. 
For clarity, the snippets of code on this section had been simplified by removing the exception handling, type checks and caching mechanism.

\subsection{Base-call and Quality Control of sequencing reads}
The raw output from the Illumina HiSeq 2000 was processed with Casava v1.8 \citep{casavaBCL}. 
Lanes 1 and 2, containing multiplexed bulks (Table \ref{tab:yr15:reads}) was demultiplexed with a tolerance of 1 mismatch in the barcode. 
Lanes 3 and 4 contained the parental sequences without a barcode. 
The FastQ files where left compressed and in chunks of 40,000, as the default for the BCL conversion pipeline from Casava to allow parallel processing in a cluster environment. 
The quality of the sequencing lanes was assessed with FastQC v0.10.1 \citep{fastqc}. 

\subsection{Alignment reads to gene models}
The RNA-Seq reads were aligned with BWA 0.5.9 \citep{Li2009} to the wheat UniGene database v60 \citep{PontiusJUWagnerL2002} and to the UCW gene models \citep{Krasileva2013}, including the \textit{T. turgidum} and complementary ORFs \citep{MASWheat2013}.
The alignments where sorted and stored as single BAM files to have random access \citep{Li2009a}. 
%\unsure{ Snippet with submission of the alignments. However I haven't got access to the old cluster files.}


\subsection{Bulk Frequency Ratios and SNP calling}
\label{yr15:sub:bfr}
To avoid the creation of several temporary files with the coverage information on all the bases I developed a \texttt{Ruby} pipeline based on the \texttt{bio-samtools} library \citep{Ramirez-Gonzalez2012}, and some of the improvements to work with pileups were published as a followup on the library \citep{Etherington2015}.
To call for the consensus, the function \texttt{Bio::DB::Sam::mpileup} is called to generate the pileup of each gene. 
As the pileups are used several times during the analysis, a function that caches the current pileup is implemented.
The consensus is called by counting how many times each base appears, and if the number of bases is higher than \texttt{minumum\_ratio\_for\_iuap\_consensus} the base is added to the set of possible bases \citep{Cornish-Bowden1985} 
If there is no coverage at a certain position, the reference base is used, and set as lowercase. 
If the set of called bases is not empty, the ambiguity code for the observed bases is called, and set as upper case (Listing \ref{lst:yr15:consensus}).
The minimum ratio was done on 0.2 (20\%), that allows for calling for a consensus even when more than one homoeologue is mapping to the same reference.

\begin{code}[language=Ruby,caption=Method to call for the consensus on progenitors from a pileup, label=lst:yr15:consensus]
def consensus_iuap(minumum_ratio_for_iuap_consensus)
  minumum_ratio_for_iup_consensus
  @consensus_iuap = self.ref_base.downcase
  bases = self.bases
  tmp = String.new
  bases.each do |k,v|
    if v/self.coverage > minumum_ratio_for_iup_consensus
      tmp << k[0].to_s
    end 
    if tmp.length > 0
      @consensus_iuap = Bio::NucleicAcid.to_IUAPC(tmp)
    end
  end 
  @consensus_iuap.upcase
end
\end{code}

Then, to calculate the \acrshortpl{bfr} as shown on Figure \ref{fig:yr15:bfr} extra extensions for the \texttt{Bio::DB::Pileup} were added to get the actual number of bases in the pile (to exclude short insertions and deletions; Listing \ref{lst:yr15:cov}), and to calculate the SNP-Index (Listing \ref{lst:yr15:ratio}).  

\begin{code}[language=Ruby,caption=\texttt{base\_coverage} gets the number of bases called from a single pileup., label=lst:yr15:cov]
def base_coverage
  total = 0
  @bases.each do |k,v|
    total += v  
  end
  total
end
\end{code}

\begin{code}[language=Ruby,caption=\texttt{base\_ratios} gets the SNP-Index on a single pileup., label=lst:yr15:ratio]
def base_ratios
  return @base_ratios if @base_ratios
  bases = self.bases
  @base_ratios = Hash.new
  bases.each do |k,v| 
    @base_ratios[k] = v.to_f/self.base_coverage.to_f 
  end
  @base_ratios
 end
\end{code}

To calculate \acrshortpl{bfr} the class \texttt{Bio::BFRTools::Container} was implemented to contain all the \texttt{BIO:DB:Sam} objects corresponding to the progenitors and the bulks. 
The class \texttt{Bio::BFRTools::BFRRegion}  was implemented to contain the ratios and consensus sequences of each region. 
The method \texttt{bfr} uses the calculated SNP-Indices on every position, from the point of view of both progenitors (lines 15-16: Listing \ref{lst:yr15:bfr}, and in the case of lack of coverage the value is set to 0 or \texttt{Infinity} (lines 8-13), depending on the progenitor where the base is not called at all. 
Using this design were the values of each region are calculated at once increases reduces the number of times the pileup needs to be generated for each sample and, allows to have in a single place in memory all the elements to calculate the \acrshortpl{bfr} without having to write any temporary files on disc. 
Also, the fact that the calculation of each region is independent to other regions, it is possible to use a computing cluster to distribute the analysis on several nodes.

The code produces a table with the SNP-Indices and BFRs for all the SNPs found in the progenitors. 
The program was used to calculate the \glspl{bfr} on the independent conditions (Bulk 1: S1-R1, Bulk 2: S2-R2 and Bulk 3: S3-R3); the \textit{in silico} mixes of bulks 1 and 2; and bulks 1, 2 and 3. 


\begin{code}[language=Ruby,caption=Section of the code that , label=lst:yr15:bfr]
for i in (0..self.size-1)
  ratios_1 = @ratios_bulk_1[i]
  ratios_2 = @ratios_bulk_2[i]
  BASES.each do |base| 
    if ratios_1[base] == 0 and ratios_2[base] == 0
      bfr1 = 0
      bfr2 = 0
    elsif ratios_1[base] == 0
      bfr1 = 0
      bfr2 = Float::INFINITY
    elsif ratios_2[base] == 0
      bfr1 = Float::INFINITY
      bfr2 = 0
    else
      bfr1 = ratios_1[base] / ratios_2[base]
      bfr2 = ratios_2[base] / ratios_1[base]
    end
    @BFRs[:first][base] << bfr1
    @BFRs[:second][base] << bfr2
  end
end
\end{code}

\subsection{\textit{In Silico} mapping}
\label{yr15:met:inSilico}

To find the chromosomal position of the SNPs with a high BFR the sequence of the markers with a genetic position from \citet{Wang2014} were aligned with BLAT \citep{Kent2002} to the \acrshort{css} scaffolds \citep{Mayer2014}. 
To find the best hit for each query was kept using a \texttt{Ruby script}. 
Briefly, the class \texttt{Bio::Blat::Report} from \texttt{BioRuby} \citep{Goto2010} was extended to include an iterator only for the best alignment of each query: 
First, the whole file is iterated (line 5); the alignment with the best score is stored in a hash (lines 7-9) and; the hash is iterated (line 11).
The script found 46,977 scaffolds that contained at least one marker from the map. 

\begin{code}[language=Ruby, caption={[\texttt{Bio::Blat::Report.each\_best\_hit}] Extension to \texttt{Bio::Blat::Report} that selects the best alignment from a \texttt{psl file from BLAT}}, label=lst:yr15:bestHit]
def self.each_best_hit(text = '')
  emptyHit = Bio::Blat::Report::Hit.new
  emptyHit.score = 0
  best_aln = Hash.new(emptyHit)
  self.each_hit(text) do |hit|
    current_score = hit.score
    if current_score > best_aln[current_name].score
      best_aln[current_name] = hit 
    end
  end
  best_aln.each_value { |val| yield  val }
end
\end{code}

Then, the UniGenes and the UCW gene models were also aligned with BLAT to the scaffolds that were located in the genetic map.  
The class \texttt{Bio::Blat::Report::Hit}  was extended to calculate how many bases are covered in the alignment and the percentage of covered bases in both, the target and query sequences (Listing \ref{lst:yr15:hit}).
Only the genes that align over 60\% of covered bases with an identity of at least 90\% were considered. 
This removes spurious mappings from repetitive regions while retaining a location to an homoeologue in case that the correct scaffold is not in the genetic map. 
The genes were also align to the full \acrshort{css} reference, to be able to locate the genes to a chromosome arm, even when it is not possible to assign a position in the genetic map and, to the cDNA of \textit{Hordeum vulgare} \citep{Mayer2011} as deposited in Ensembl! Plants, release 16 \citep{Kersey2012}. 
\unsure{Include code on how the coordinates where extracted, with the patch to the Ensembl package}
The genetic position of the contigs was used to calculate the density of SNPs between \acrshort{avs} and \acrshort{yr15} in the genetic bins for Figure \ref{fig:yr15:bfrs:0-6}. 
This information was used to select the SNPs with high BFR to validate.  

\begin{code}[language=Ruby, caption=Extension to \texttt{Bio::Blat::Report::Hit} for filtering of spurious alignments., label=lst:yr15:hit]
class Bio::Blat::Report::Hit
  def covered
    match + mismatch
  end
  def query_percentage_covered
    covered * 100.0 / query_len.to_f
  end
  def target_percentage_covered
    covered * 100.0 / target_len.to_f
  end
end
\end{code}

\subsection{Primer design and KASP assays}
The primer design for KASP were designed with PolyMarker as described in Chapter \ref{cha:polymarker}. 
The only difference with the default settings is that instead of using a template sequence, the sequence for each allele is calculated from the consensus of the alignments. 
As described in as described in \citet{Ramirez-Gonzalez2015b}, the primers 
\begin{blockquote} were ordered from Sigma-Aldrich (Gillingham, UK), with primers carrying standard FAM or HEX compatible tails (FAM tail: 5' GAAGGTGACCAAGTTCATGCT 3'; HEX tail: 5' GAAGGTCGGAGTCAACGGATT 3') with the target SNP at the 30 end. 
Primer mix was set up as recommended by LGC [46 $\mu$L dH$_{2}$O, 30 $\mu$L common primer (100 $\mu$M) and 12 $\mu$L of each tailed primer (100 $\mu$M)] \citep{LGC}
Assays were tested in 384-well format and set up as 4-$\mu$L reactions [2-$\mu$L template (10–20 ng of DNA), 1.944 $\mu$L of V4 2 $\times$ Kaspar mix and 0.056 $\mu$L primer mix]. 
PCR cycling was performed on a Eppendorf Mastercycler pro 384 using the following protocol: hotstart at 95 °C for 15 min, followed by ten touchdown cycles (95°C for 20 s; touchdown 65°C, $-1$ °C per cycle, 25 s) then followed by 30 cycles of amplification (95°C 10 s; 57°C 60 s).
As KASP amplicons are smaller than 120 bp, an extension step is unnecessary in the PCR protocol. 384-well optically clear plates (Cat. No. E10423000; Starlab Milton Keynes, UK) were read on a Tecan Safire plate reader. 
Fluorescence was detected detected at ambient temperature. If the signature genotyping clusters had not formed after the initial amplification, additional amplification cycles (usually 5–10) were conducted, and the samples were read again. Data analysis was performed manually using Klustercaller software (version 2.22.0.5; LGC Hoddesdon, UK).
\end{blockquote}


\subsection{Genetic map}
As described in \citet{Ramirez-Gonzalez2015b}:
\begin{blockquote}
 JoinMap version 3 \citep{vanOoijen2002} was used for linkage analysis and genetic map construction, using default settings. The linkage to \textit{Yr15} was determined using a divergent log-of-odds (LOD) threshold of 3.0, and genetic distances were computed based on recombination frequency. 
\end{blockquote}

\section{Discussion} 


Resequencing the $\sim17$Gbp genome of hexaploid wheat is costly and approaches to reduce the required sequenced volume to effectively call for SNPs had been evolving since the conception of this project. 
The RNA and DNA extraction and the sequencing for this project was carried out before the beginning of my PhD (before October 2012). 
At that time exome capture was already established for genotyping humans \citep{Ng2009}, however the first exome capture on wheat was just recently published, with probes coming from unassembled 454 reads \citep{Winfield2012}, and a probe designed from transcripts \citep{Henry2014} was not published after the analysis of this section was completed and validated (Figure \ref{fig:yr15:timeline}).
An even more targeted capture for resistance genes (RenSeq), by capturing genes with the NBS-LRR motif, was published while this study was executed \citep{Jupe2013}.
On the other hand RNA-Seq was already tested for \acrlong{bsa} on tetraploid wheat \citep{Trick2012}.  
Hence, the decision of reducing the sequenced space with RNA-Seq was appropriate at the time (Figure \ref{fig:yr15:timeline}). 
Unfortunately, one of the shortcomings of RNA-Seq used to call for SNPs is that the coverage is not uniform and the genes that have low expression don't have enough coverage to call for SNPs (Section  \ref{yr15:sequencing}).
If a similar study was to be started today, a better alternative would be to use exome capture in general from a segregating population for any trait, or RenSeq if the target gene is a resistance gene. 

\begin{figure}
\includegraphics[width=1\textwidth]{expVIP/Figures/Timeline.pdf}
\caption{Timeline of resources used, or potentially used for \gls{yr15}.}
\label{fig:yr15:timeline}
\end{figure}

The quality and completeness  of the reference genome or gene models directly affects the mapping of \gls{ngs} reads. 
This is particularly true on polyploid organisms: if one of the homoeologues is absent, the reads are likely to map to the wrong genome if the parameters of the aligner are relaxed or; not map at all if the required identity is too high.
When the bioinformatic analysis of this project started, the only available wheat genomic reference was a whole genome shotgun 454 sequencing, unassembled \citep{Brenchley2012}; the \gls{css} assembly was being finished \citep{Mayer2014}; the longer scaffolds from \citet{Chapman2015} were not public yet and; the efforts to make a whole genome shotgun assembly were being planned independently by the International Wheat Genome Sequencing consortium  \citep{Pozniak2016} and TGAC (\citealt{Clark2016} ;Figure \ref{fig:yr15:timeline}).  
Because a contiguous assembly with the corresponding annotation wasn't available at the time of the analysis and the fact that the data available was from a transcriptome, the use of gene models as a reference for the alignment was a suitable approach. 

In terms of available gene sets when the analysis started, the canonical reference was the UniGenes from the NCBI \citep{PontiusJUWagnerL2002}. 
The UniGenes are produced with an automated pipeline that clusters all the \glspl{est} deposited in the NCBI by identity and selects the longest transcript, which can merge homoeologous transcripts as a single reference.
Shortly after I started the bioinformatic analysis, two aditional gene models were available, the draft annotation for the \gls{css} assembly (MIPSv1) in January 2013 and the UCW gene models \citep{Krasileva2013} in May 2013. 
I selected the UCW gene models, as they were more mature and were phased to distinguish between genomes and already published, over the MIPSv1 genes, still being refined from an initial approach lifting proteins from related organisms and a few RNA-Seq experiments.  
The MIPS gene models were improved by removing duplications in the assembly in a later stage and the nomenclature before the release of the assembly \citep{Mayer2014}, but at that point the results of this project where already submitted for publication (Figure \ref{fig:intro:timeline}; \citealt{Ramirez-Gonzalez2015b}). 
 
To locate the gene models in the chromosome arms and see if there was an enrichment on the called SNPs the use of a high resolution consensus map is needed, as the genome assemblies available during the analysis are fragmented. 
Initially, I used barley to locate the gene models because the genetic map used to locate the \gls{css} scaffolds was not released yet and barley has a conserved syntheny across the wheat genomes. 
It was thus timely that a genetic map with $>42,000$ markers was published \citep{Wang2014}. 
I was able to use it to locate several \acrshort{css} scaffolds before the assembly was published during the development of this project, as I collaborated in the project. 
The located scaffolds were used as proxy to sort just under half of the reference genes in their chromosomal position (Section \ref{sub:yr15:inSilico}). 
Despite the resolution not being enough to find a single point of enrichment, it was enough to confirm that the SNPs were in the expected location, including one of the SNP candidates flanking the \textit{Yr15} locus (SNP R11, Figure \ref{fig:yr15:BFRValues1BS}).  
If the analysis was to be done today, the genetic map from \citet{Chapman2015} along with their longer scaffolds, or the scaffolds from TGACv1 or the NRGene should provide a better resolution. 
Even without having all the \acrshort{css} scaffolds sorted, the fact that they come from individual chromosome arms they enabled the assignment of the genes to a chromosome. 

The original expectation was to have a \gls{nil} for the \gls{bsa} to simplify the SNP discovery and analysis since the majority (if not all ) SNPs should be restricted to the region immediately flanking Yr15. However the number of \glspl{snp} called in the progenitors suggested that the background, \acrlong{avs}, was not the same.  
This happened because despite both succeptible lines being called the same and having the same response to the pathogen, they are different lines from different countries (Section \ref{yr15:snpCalling}). 
This highlights the importance of genotyping the material used when developing mapping populations, specially if the source of the seeds come from different seed banks. 

Despite this shortcomings, the use of the \glspl{bfr} to score the putative SNPs was effective as most of the SNPs with a high score  mapped in chromosome 1B, as expected from previous studies ($BFR>6$, Section \ref{yr15:assaySelection}).
Using the extra criteria of only selecting SNPs from the resistant progenitor and in the expected chromosome arm I was able to produce a high resolution genetic map (Section \ref{yr15:geneticMap}). 
The genetic map was of the expected resolution for the size of the population (0.26cM on 196 individuals).
Since the mapping population contained only one critical recombinant between \textit{Yr15} and the flanking markers, the population could not yield a better map. 
To improve the map, a cross from the two critical recombinants could be use to repeat a similar analysis, but sequencing with either exome capture or RenSeq. 
%\unsure{Talk about Why is R33 diagnostic on the varieties, but maps away?. }

\begin{figure}
\includegraphics[width=1\textwidth]{Yr15/Figures/breedersTest.pdf}
\caption[Haplotype and phenotype of 113 doubled haploid lines]{Haplotype analysis and phenotypic evaluation of the 113 doubled haploid lines used in the study. The TGC haplotype corresponds to that originally identified in the \textit{Yr15} parent and which was diagnostic across 112 of the 113 lines studied. Figure from \citep{Ramirez-Gonzalez2015b}}
\label{fig:yr15:breeders}
\end{figure}

As described in \citet{Ramirez-Gonzalez2015b}:  
\begin{blockquote} The markers
 R11, R5 and R8 were tested across 122 doubled haploid (DH) lines. 
These DH lines were derived from crosses crosses between five different UK varieties/breeding lines to \textit{Yr15} derivatives known to carry the resistance gene. 
The expected \textit{Yr15} haplotype corresponded to T, G and C alleles at markers R11, R5 and R8, respectively (TGC haplotype). 
The DH lines were tested at seedling stage for reaction to \textit{P. striiformis}, with 84 showing complete resistance and 34 presenting an intermediate or completely susceptible reaction.
The resistant lines all carried the complete \textit{Yr15} haplotype (TGC, Figure \ref{fig:yr15:breeders}) across the three SNP markers with the exception of five lines which had a single missing data point, but were otherwise consistent. 
This compared favourably with the most diagnostic in-house SNP markers available within the breeding programmes. 
Using the three in-house markers, 79 resistant lines carried the expected haplotype, but five completely resistant DH lines were scored as false negative due to the presence of the non-\textit{Yr15} haplotype. 
Within the intermediate and susceptible DH lines, all but one had a non-\textit{Yr15} haplotype (CAT or TAT) across R11, R5 and R8 (Figure \ref{fig:yr15:breeders}). This single DH line was scored as a false positive as it carried the TGC \textit{Yr15} haplotype, but was found to have an intermediate (chlorotic) reaction to \textit{P. striiformis}. This line was also the only one scored as a false positive using the three in-house markers.
\end{blockquote}
%\unsure{This is a very long quote, but I'm finding hard to shorten it. } 
The fact that the developed markers perform better than the markers developed by breeders show the value of this particular experiment and further confirms that \acrshort{bsa} combined with \acrshort{ngs} is an effective way to develop novel markers. 
%\unsure{ Mention other people using a similar strategy since this was published. }

%TODO: Discuss other people using the mark 

In this chapter the integration of different levels of data helped to improve the selection of the candidate SNPs. 
The main criteria for selecting \acrshortpl{snp}  was the\acrshort{bfr} score.
However, thanks to the genetic map from \citet{Wang2014} and the \acrshort{css} scaffolds \citep{Mayer2014}) enabled to confirm that the high scores were in the expected region. 
As the reference genome for wheat improves, the location of SNPs linked to a trait will be easier. 
With a continous reference between two markers flanking a locus and an improve annotation, a more focused set of gene candidates will be possible. 


%!TEX root = ../Main.tex

\chapter[expVIP]{expVIP: a customisable RNA-seq data analysis and visualisation platform.}
\label{cha:exp}


\section{Background.}



%Describe the list of previously published expression experiments and how they can potentially be used as a framework for new experiments.  

%Co-expression of homoeologus varies from triplet to triplet \citep{Pfeifer2014}
%The silencing is mostly regulated by epigenetic changes as the hybridization events are recent.  \citep{Bottley2006}


%\begin{table}

%\caption{Studies included in expVIP for wheat}
%\label{tab:exp:studysDetails}

%\end{table}

\subsection{Alternative expression browsers.}
\label{exp:alternative}
%Expression experiments are done routinely in model organisms, such as arabidopsis.

To the best of my knowledge, the only alternative expression browser developed for specifically for wheat is WheatExp \citep{Pearce2015b}. 
This expression browser contains information of 6 studies from diploid, tetraploid and hexaploid wheat. 
The studies were selected to be complementary among them; " a broad study of five different tissues across multiple timepoints \citep{Choulet2014}, a study of seedling photomorphogenesis \citep{Fox2014}, a study of drought and heat stress in wheat seedlings \citep{Liu2015}, a study of wheat grain layers at a single timepoint \citep{Pearce2015}, a senescing leaf timecourse \citep{Pearce2014} and a timecourse of different grain tissue layers during development \citep{Pfeifer2014}".
The expression quantification is produced by aligning to the gene transcripts using BWA \citep{Li2010}, and the counts are extracted with HTSeq \citep{Anders2015}. 
The Expression Atlas from \acrshort{ebi} is a public resource that collects expression experiments of several species \citep{Petryszak2016}. 
The samples are processed automatically from the reads deposited ArrayExpress \citep{Kolesnikov2015}, containing data from microarrays and RNA-Seq.
The studies included are manually curated and annotated with the relevant ontologies. 
As of September 2016, 202 differential and 82 base experiments had been included. 
The visualisation is designed to explore each gene individually, showing all the related ontologies and details on the expression by factor, or by pairs of factor on a heatmap. 
The quantification is calculated with HTSeq \citep{Anders2015}, from alignments produced with Tophat \citep{Ling2013}.

%\unsure{introduce BioGPS. It is not wheat related, but is feature rich. }


\subsection{Public wheat RNA-Seq experiments.}

Table \ref{tab:exp:studiesDetails} contains the 16 studies used during the development of the \gls{expvip}. 
This studies were categorized by developmental time courses, tissues, pathogen infections, and abiotic stresses. 
The use of divers studies demonstrate the utility of an integrated platform to generate novel hypotheses. 

%!TEX root = ../../Main.tex
\begin{sidewaystable}
\caption{Studies with RNA-Seq replicates, sequenced with Illumina, at the time when expVIP was under development.} 
\label{tab:exp:studiesDetails}
\scriptsize
\begin{tabular}{p{2cm}p{3cm}p{7.4cm}p{10.1cm}}
\toprule
 Study Id & Summary of study  & Brief SRA description  & Manuscript  \\
\midrule
 DRP000768 & Phosphate starvation in roots and shoots & Transcriptome profiles of wheat variety Chinese Spring (CS) in response to Pi starvation (-P) for 10 days. & Characterisation of the wheat (\textit{Triticum aestivum L.}) transcriptome by \textit{de novo} assembly for the discovery of phosphate starvation-responsive genes: gene expression in Pi-stressed wheat \citep{Oono2013}. \\
 ERP003465 & fusarium head blight infected spikelets & Near isogenic wheat lines, differing in the presence of the \textit{Fusarium graminearum} FHB-resistance QTL Fhb1 and Qfhs.ifa-5A, under disease pressure (30 and 50 hai) as well as with mock-inoculation  & Quantitative trait loci-dependent analysis of a gene co-expression network associated with Fusarium head blight resistance in bread wheat \textit{Triticum aestivum L.}. \citep{Kugler2013}.  \\
 ERP004505 & grain tissue-specific developmental timecourse & Analysis of the cell type specific expression of homeologous genes in the developing wheat grain & Genome interplay in the grain transcriptome of hexaploid bread wheat \citep{Pfeifer2014}. \\
 SRP004884 & flag leaf downregulation of GPC & Wild type bread wheat plants and GPC RNAi plants 12 days after anthesis  & Effect of the down-regulation of the high Grain Protein Content (GPC) genes on the wheat transcriptome during monocarpic senescence \citep{Cantu2011b}. \\
 SRP013449 & grain tissue-specific developmental timecourse & Transcriptomes of the aleurone and starchy endosperm tissues of the wheat seed (\textit{Triticum aestivum}) at time points critical to the development of the aleurone layer of 6, 9 and 14 days post anthesis. & Gene expression in the developing aleurone and starchy endosperm of wheat \citep{Gillies2012}. \\
 SRP017303 & stripe rust infected seedlings & Pool of stripe rust infected wheat leaves & Genome analyses of the wheat yellow (stripe) rust pathogen \textit{Puccinia striiformis f. sp. tritici} reveal polymorphic and haustorial expressed secreted proteins as candidate effectors \citep{Cantu2013}. \\
 SRP022869 & \textit{Septoria tritici} infected seedlings   & Molecular mechanisms underlying the interplay between fungal pathogenicity and host responses at specific growth phases and the factors triggering disease transition. & Transcriptional Reprogramming of Wheat and the Hemibiotrophic Pathogen \textit{Septoria tritici} during Two Phases of the Compatible Interaction \citep{Yang2013}. \\
 SRP028357 & shoots and leaves of nulli tetra group 1 and group 5 & RNAseq of nulli-tetrasomic wheat lines (shoots and leaves & Patterns of homoeologous gene expression shown by RNA sequencing in hexaploid bread wheat \citep{Leach2014}. \\
 SRP029372 & grain tissue-specific developmental timecourse & Gene expression profiling of morphological stage of developing wheat grain & Evaluation of Assembly Strategies Using RNA-Seq Data Associated with Grain Development of Wheat (\textit{Triticum aestivum L.}; \citealt{Li2013}). \\
 SRP038912               & comparison of stamen, pistil and pistilloidy expression & Transcriptional profiling of pistillody stamen, pistil and stamen in wheat line HTS-1 & Pistillody mutant reveals key insights into stamen and pistil development in wheat (\textit{Triticum aestivum L.}; \citealt{Yang2015}). \\
 SRP041017 & stripe rust and powdery mildew timecourse of infection in seedlings & Transcriptome Divergence and Overlap for Wheat in Response to Stripe rust and Powdery Mildew Pathogen Stress & Large-scale transcriptome comparison reveals distinct gene activations in wheat responding to stripe rust and powdery mildew. \citep{Zhang2014}. \\
 SRP041022  & developmental time-course of synthetic hexaploid & RNAseq of three tissues of nascent allohexaploid wheat and its following generations, their progenitors, and Chinese Spring & mRNA and Small RNA Transcriptomes Reveal Insights into Dynamic Homoeolog Regulation of Allopolyploid Heterosis in Nascent Hexaploid Wheat \citep{Li2014}). \\
 ERP008767 & grain tissue-specific expression at 12 days post anthesis & Inner pericarp, outer pericarp and endosperm layers from developing grain of bread wheat cv. Holdfast at 12 days post-anthesis. & Heterologous expression and transcript analysis of gibberellin biosynthetic genes of grasses reveals novel functionality in the \textit{GA3ox} family \citep{Pearce2015}. \\
 SRP045409 & drought and heat stress time-course in seedlings & RNAseq of 1-week old wheat seedling leaves subjected to drought stress, heat stress and their combination before (0h) and after stress (1h or 6h) & Temporal transcriptome profiling reveals expression partitioning of homeologous genes contributing to heat and drought acclimation in wheat (\textit{Triticum aestivum L.}; \citealt{Liu2015}). \\
  INRA-RNAseq (ERP004714) & developmental time-course of Chinese Spring & Whole transcriptome sequencing of wheat 3B chromosome  & Structural and functional partitioning of bread wheat chromosome 3B \citep{Choulet2014}. \\
 SRP056412 & grain developmental timecourse with 4A dormancy QTL & This study was to identify candidate genes underlying the 4AL QTL for grain dormancy in wheat. RNA was sequenced from pooled NILs segregating for the QTL & Transcriptomic analysis of wheat near-isogenic linesidentifies  \textit{PM19-A1} an \textit{A2} as candidates for a major dormancy QTL \citep{Barrero2015} \\
\bottomrule
\end{tabular}
\end{sidewaystable}


\subsection{Expression quantification with Kallisto.}
\label{exp:kallisto}
Differential expression experiments try to elucidate which genes change under different conditions. 
To do that, a quantification of the levels of expressions is needed. 
When using RNA-Seq, the expression analysis usually consists of: aligning the reads to the genome or transcriptome reference and quantifying the expression according to how many reads map to a region. 
However, this process usually takes around 6 hours per sample. 
Aligners such as \texttt{bwa} or \texttt{bowtie} produce a detailed alignment of each read, which is useful find polymorphisms (see Chapter \ref{yr15}) or to find novel alternative splices \citep{Trapnell2012}. 

For expression analysis only the count of how many reads with a transcript is required, calculating the best local alignment and the output of each read is unnecessary. 
\verb|Kallisto| is a tool that generates an index based on overlapping k-mers (sequences of size $k$), which are connected sequentially to represent each transcript  on a \gls{tdbg}. 
For alternative splicings of the same gene, were some sequence overlap between transcripts, the connections produce two different sets of connections between k-mers. 
The k-mers on each read are then used to find the compatible transcripts across the T-DBG and those are counted. 
Finally, the program estimates the abundance of the transcript in the sample (Figure \ref{fig:exp:kallisto}; \citealt{Bray2016}).  


\begin{figure}
\includegraphics[width=1\textwidth]{LitReview/Figures/kallisto.pdf}
\caption[Overview of kallisto.]{"Overview of kallisto. The input consists of a reference transcriptome and reads from an RNA-seq experiment. (a) An example of a read (in black) and three overlapping transcripts with exonic regions as shown. (b) An index is constructed by creating the transcriptome de Bruijn Graph (T-DBG) where
nodes (v1, v2, v3, ... ) are k-mers, each transcript corresponds to a colored path as shown and the path cover of the transcriptome induces a k-compatibility class for each k-mer. (c) Conceptually, the k-mers of a read are hashed (black nodes) to find the k-compatibility class of a read. (d) Skipping (black dashed lines) uses the information stored in the T-DBG to skip k-mers that are redundant because they have the same k-compatibility class. (e) The k-compatibility class of the read is determined by taking the intersection of the k-compatibility classes of its constituent k-mers" \citep{Bray2016}.}
\label{fig:exp:kallisto}
\end{figure}


\subsection{Relational databases.}
A relational database is a set of structured tables that have relationships between each other. 
The tables correspond to the data that is essential for the represented concept (domain).
For example, in a table representing several species, the common name and the scientific name belong to the same domain (ie name: Bread wheat; scientific name \textit{Triticum aestivum}). 
Tables in the same relational database form relationships between each other. 
Continuing with the example, a species can have several scientific studies related to them. 
The domain of a study can be formed by the accession, a title, a corresponding manuscript and the species it is concerned with. The two tables can be connected by the usage of a common ID signifying the species. It is therefore common practice to create an extra column containing unique integer values, to act as connecting keys between two or more tables. Such a column is defined as the primary key of the table; although, strictly speaking, any set of columns whose combination of values is unique for each row can act as primary key. 
In our example an extra \texttt{id} column is added allowing to create a stable relationship between the \verb|Species| and the \verb|Studies| tables.
 (Figure \ref{fig:expvip:miniER}; \citealt{Codd1970}).
The tables \ref{exp:tab:species} and \ref{exp:tab:studies} have the content of their corresponding domains. 

\begin{figure}
\includegraphics{expVIP/Figures/miniER.pdf}
\caption[Example of a relationship between tables.]{Example of a relationship between tables. The tables Species and Studies are related. Each study has one species and each species can have several studies.}
\label{fig:expvip:miniER}
\end{figure}

\begin{table}
\caption[Species]{Example content for the table \texttt{species}}
\label{exp:tab:species}
\centering
\begin{tabular}{rll}
\toprule
   id & name                          & scientific\_name                                         \\
\midrule
    1 & Bread wheat                 & Triticum aestivum                                     \\
    2 & Yellow rust                 & Puccinia striiformis              \\
    3 & wheat and rust & T.aestivum,S.tritici  \\
\bottomrule
\end{tabular}

\end{table}

\begin{table}
\caption[Studies]{Example content for the table \texttt{studies}}
\label{exp:tab:studies}
\centering
\begin{tabular}{rllr}
\toprule
   id & accession                                          & manuscript                              &   species\_id \\
\midrule
    1 & DRP000768                                          &  10.1186/1471-2164-14-77          &            1 \\
    2 & ERP003465                                          & 10.1186/1471-2164-14-728            &            1 \\
    3 & ERP004505                                          &  10.1126/science.1250091          &            1 \\
    4 & SRP004884                                          & 10.1186/1471-2164-12-492            &            1 \\
    5 & SRP013449                                          &  10.1111/j.1467-7652.2012.00705.x &            1 \\
    6 & SRP017303                                          & 10.1186/1471-2164-14-270            &            2 \\
    7 & SRP022869                                          &  10.1371/journal.pone.0081606     &            3 \\
\bottomrule
\end{tabular}

\end{table}

\subsection{SQL.}
\gls{sql} is a common language to retrieve information from relational databases. 
\acrshort{sql} has operations to select columns and rows, join tables, group repeated values and order the results. 
Those simple operations are enough to retrieve information between tables \citep{Oracle2014}. 
The following list shows a brief description of some commands  that can be used to build a query.  

\begin{description}
\item[\texttt{SELECT <EXPRESSIONS> }]. A list of columns or an expression that will be displayed, separated by commas (\texttt{,}). To display all the columns, the \texttt{*} character  all the columns in the table. The order of the columns will be the same as the order given in this part of the command
\item[\texttt{FROM <TABLE>}]. follows the column names to add a list of tables to select from.
\item[\texttt{JOIN <TABLE> ON <EXPRESSION> }]. is used to join the table from the left side of the statement with the \texttt{<EXPRESSION>} given after the \texttt{ON} clause.  
\item[\texttt{WHERE <EXPRESSION>}] filters the rows by the \texttt{<EXPRESSION>}
\item[\texttt{ORDER BY <COLUMNS>}]. The rows will be sorted by the natural order of the given \texttt{<COLUMNS>}.  
\item[\texttt{GROUP BY <COLUMNS>}]. The rows are merged by the columns stated. This can be used to get an unique set of values and apply a function to all the rows that have the same value, as a count.
\end{description} 

Expressions can be values, operators or functions like:

\begin{description}
\item[\texttt{COLUMN}] The value of a column. 
\item[ \texttt{<EXPRESSION> = <EXPRESSION>}] \texttt{TRUE} when the left and right \texttt{<EXPRESSION>} are equal. \texttt{FALSE} otherwise 
\item[ \texttt{<EXPRESSION> > <EXPRESSION>}] \texttt{TRUE} when the left  \texttt{<EXPRESSION>} is greater than the  right \texttt{<EXPRESSION>} are equal. \texttt{FALSE} otherwise 
\item[ \texttt{<EXPRESSION> < <EXPRESSION>}] \texttt{TRUE} when the left  \texttt{<EXPRESSION>} is less than the  right \texttt{<EXPRESSION>} are equal. \texttt{FALSE} otherwise
\item[ \texttt{COUNT(*)}] The count of rows that have the same values, as selected in the \texttt{GROUP BY} clause.
\end{description}

A simple query to join the  \texttt{species} and \texttt{studies} tables and displaying only the species name, scientific name and accession of the study is shown in Listing \ref{exp:lst:joinExample}. 
The results of the query are in Table \ref{exp:tab:joinExample}. 

\begin{code}[language=sql, label=exp:lst:joinExample,caption=Join example query]
SELECT 
	species.name, 
	species.scientific_name, 
	studies.accession, 
FROM species
JOIN studies on species.id = studies.species_id;
\end{code}

\begin{table}[h!]

\caption[Join example]{Join of the \texttt{species} and \texttt{studies} table. }
\label{exp:tab:joinExample}
\centering
\begin{tabular}{lll}
\toprule
 name                          & scientific\_name                                         & accession                                                                       \\
\midrule
 Bread wheat                 & Triticum aestivum                                     & DRP000768                                          \\ 
 Bread wheat                 & Triticum aestivum                                     & ERP003465                                          \\ 
 Bread wheat                 & Triticum aestivum                                     & ERP004505                                          \\ 
 Bread wheat                 & Triticum aestivum                                     & SRP004884                                          \\ 
 Bread wheat                 & Triticum aestivum                                     & SRP013449                                          \\ 
 Yellow rust                 & Puccinia striiformis            & SRP017303                                          \\ 
 wheat and rust & T.aestivum,S.tritici & SRP022869                                          \\ 
\bottomrule
\end{tabular}

\end{table}


The relationships between tables can be of the following types:

\begin{description}
\item[one-to-one.] When rows on a table can be related to a row in a second table. On the diagrams they are represented by a straight line.
\item[many-to-many.] Rows on a table can have many corresponding rows in a second table, resented with lines with whiskers on both sides of the line.  
\item[one-to-many.] Rows on a table can be related to many rows on the second table, represented with whiskers only on one side of the line. 
\end{description}

An important feature of a database is the ability to store the data consistently.
A transaction is a set of related operations that need to be performed at the same time; as such, it needs to follow the principles of \gls{acid} \citep{Haerder1983}.


\begin{description}
\item[Atomicity.] All the operations or none have to be performed. If any of them fails or an error happens while the transaction is executed, the data has to be restored to the original status. 
\item[Consistency.] The changes in the database have to be valid before and after the transaction. 
\item[Isolation.] If more than one transaction is being executed at the same time, the result must be the same as if the transactions were executed one after the other. 
\item[Durability.] The result of the transaction is stored even if the server is restarted.
\end{description}  

\unsure{Add explanaiton of inserts. }

Several \gls{rdbms} implement \acrshort{sql}, with various levels of compliance to the standard and different licenses.  
A popular \acrshort{rdbms} is \texttt{MySQL}.
From the beginning \texttt{MySQL} aimed to be a lightweight and easy to install open source product \citep{Oracle2014}. 
This characteristics made it popular on the web and it is currently the \gls{rdbms} behind ensembl! \citep{Flicek2012}. 

\subsection{Model-View-Controller}

\begin{SCfigure} 
  \centering
  \includegraphics[width=0.7\textwidth]{LitReview/Figures/MVC.pdf}
  \caption{\acrshort{mvc} interaction between components.}
  \label{fig:poly:mvc}
\end{SCfigure}
The \gls{mvc} is a metaphor to isolate the user interactions from the underlying data. 
The models hold the data on logical their domains.
The views contain the layout on how the models are displayed to the user.
The controllers receive the requests from the users and modify the models hold the data as mapped to their logical domains accordingly and send a view back for display. 
The \acrshort{mvc} metaphor allows the development of independent parts of the system and helps to structure the underlying representation of the domains.  (Figure \ref{fig:poly:mvc}; \citealt{Krasner1988}).

\gls{ror} is a framework to develop web applications heavily influenced by the \acrshort{mvc} metaphor. 
It is based on the Rails language and provides several tools designed to facilitate development, such as automated tasks designed to create models with their corresponding views and controllers. 
On the top of that, it provides the tools to manage the connection and queries to the \acrshort{rdbms}, allowing the developer to focus on functionality \citep{RailsGuide2016}. 

\subsection{Data visualisation}

In the last couple of decades the amount of information available in any given field has been growing exponentially, in part thanks to the internet. 
A standing challenge is therefore to produce tools that help interpretation. 
An effective way to communicate large amount of data is through visualisation, but it has to have the following properties to be usable \citep{Myatt2011}:

\begin{description}
\item[Time to learn.] The user needs to take little time to learn how to use the system to extract the information they need. Also, all the features should be easy to find.
\item[Performance.] The tool needs to be quick to access and transform the visualisation as the user requests.
\item[Accuracy.] The tool has to perform as the user expects, so if the tool is prone to make users commit mistakes, it is not accurate. 
\item[Memorability.] Once the user learns to use the system, is it easy to remember how to use it? Systems that change their interface often, or between windows are not as easy to remember. 
\item[Satisfaction.] The user responds positively to the use of the system and the time is spent actually exploring the data, rather than trying to make the system work. 
\end{description}

\subsection{Aims}
\label{exp:aims}
The aims of expVIP are to:
\begin{enumerate}
\item Integrate RNA-Seq experiments from several sources in a single database (Section \ref{exp:DB}). 
\item Automate the calculation of the expression values and load them in to the database (Section \ref{exp:pipeline}).
\item Produce a tool  visualisation for said expression values with a short time to learn, good perfomance, memorability, accuracy and satisfaction s (Section \ref{exp:gui}).
\item Make the system available to the community (Section \ref{exp:gui}).
\end{enumerate}

The software developed in this chapter is published in \cite*{Borrill2016}. 

\section{General design}

One of the main objectives of \gls{expvip} is to make the public expression datasets easily accessible and explorable for the target community (currently wheat, but not limited to it). 
A web interface is an effective way to reach a global audience. 
A web service requires to have a server to run the application, and a browser to connect to the server and display the application to the user (ie Internet Explorer, Chrome, Firefox).
The web server technology used for expVIP is \acrshort{ror}, as it abstracts the \acrshort{mvc} metaphor and it is designed to speed development \citep{RailsGuide2016}. 
In order to display the expression data to the users,  expVIP relies on a BioJS component (\citealt{Yachdav2015}, Section \ref{exp:gui}) developed for expVIP. 
All the data is stored in a MySQL database (Section \ref{exp:DB}) and it is accessed through models developed under  \acrshort{ror} (Figure \ref{fig:poly:archDesign}).  

\begin{figure}[b] 
  \centering
  \includegraphics[width=0.8\textwidth]{expVIP/Figures/archDesign.pdf}
  \caption{General design of expVIP}
  \label{fig:poly:archDesign}
\end{figure}

\section{Database design} 
\label{exp:DB}

To address the different types of conditions over different experiments, expVIP is designed around a relational database. 
The design comprises two core groups of tables and two auxiliary tables that take care of different species and homoeologues (Figure \ref{fig:expvip:dbDesign}).


\begin{sidewaysfigure}
\centering
\includegraphics[width=0.85\textwidth]{expVIP/Figures/dbDesign.pdf}
\caption[expVIP database design]{Database design. The block on the top stores the meta-data about the experiments and the studies. The bottom block consist on the tables related to the expression values. Species and homoeologues are outside the main blocks as they are not core to the groups. The whiskers in the connections show the cardinality of the relationships.}
\label{fig:expvip:dbDesign}
\end{sidewaysfigure}

\begin{description}
\item[Metadata] The tables in this group contain the information of each one of the studies. 
\begin{description}
\item[Studies] holds general information for a study, which contain several experiments. The table also contains the reference to the paper where the data is published and the accession number for the study. 
\item[Experiment group] keeps together all the individual experiments that come from the same study and that were taken on the same condition (ie. replicates). 
\item[Factors] holds all the possible factors used to group the experiments. Each experiment group has many factors and each factor group has many experiment groups. As the experiment does not have a fixed number of columns representing each factor, it is possible to have any arbitrary number factors to group. 
\item[Experiment] holds the information of each individual experiment, with the corresponding accession. 
\end{description}
\item[Expression values.] The tables on this block contain the information for each genes and their expression values. 
\begin{description}
\item[Types of value] keeps a list of different units that are stored. On the original design \acrshort{tpm} and raw counts are set up, but as the units are not hard coded it is possible to use \acrshort{fpkm}, \acrshort{rpkm} or any other unit. 
\item[Gene Set] contains the name of a reference gene set for the analysis. On the original version of expVIP, the gene models from the \acrshort{iwgsc} as deposited in Ensembl release 26 were used \citep{Mayer2014}. However, this table allows to use several reference gene models on the same database. 
\item[Genes] are related to a \texttt{gene set}, so even if two genes coming from two different datasets share the same name it is possible to distinguish them and avoid ID collisions. This situation is unlikely to occur when using published references, but might arise when joining several \textit{de novo} gene model datasets. 
\item[Meta Experiment] allows to have the same data analysed with different tools. By default expVIP uses Kallisto \citep{Bray2016}. However other tools, or different versions of the same tool, can be used to repeat the analysis.   
\item[Values] have a domain that includes the \texttt{meta experiment}, \texttt{gene} and, \texttt{type of value}.
\end{description}
\item[Homoeologues] contains the relationship between genes. This allows to get the expression values of several related genes. 
\item[Species] contains the target species of a study. It is not linked to the gene models to allow the direct comparison between related species using the same gene models (ie, \texttt{T. aestivum} vs \texttt{T. turgidum}). 
\end{description}

In the cases where a relationship between tables is not unique, such as \texttt{experiment\_groups} having many \texttt{factors} and the \texttt{factors} having many \texttt{experiment\_groups}, storing of relationships is done with an auxiliary table (ie. \texttt{ExperimentGroups\_Factors}, not explicitly shown in Figure \ref{fig:expvip:dbDesign}, but implicit by the lines with whiskers). 

Once all the data is stored, the tables can be queried together to make clear the relationship between specific rows. 
One of the core tasks of expVIP is to get all the factors that define each experiment, in order to be able to merge similar studies. 
To retrieve the \texttt{experiments} and \texttt{factors} of an \texttt{experiment group}, the auxiliary tables \texttt{ExperimentGroups\_Factors}  and \texttt{experiment\_groups\_experiments} are used in the query. (Listing \ref{lst:exp:queryMetadata} and Table \ref{tab:exp:queryMetadata}).


\begin{code}[language=sql, caption={[Query experiments and factors]Query experiments and factorsQuery experiments and factors from accession 'DRR003148'},label=lst:exp:queryMetadata]
SELECT
	experiments.accession,  
	factors.factor,
	factors.description, 
	experiment_groups.name as expriment_group 
FROM factors 
JOIN ExperimentGroups_Factors 
	ON factors.id = ExperimentGroups_Factors.factor_id
JOIN experiment_groups 
	ON experiment_groups.id = ExperimentGroups_Factors.experiment_group_id
JOIN experiment_groups_experiments 
	ON experiment_groups_experiments.experiment_group_id = experiment_groups.id
JOIN experiments 
	ON experiments.id = experiment_groups_experiments.experiment_id
WHERE accession =  'DRR003148'
\end{code}

%\pagebreak
\begin{table}[h]
\caption[Results of query for metadata]{Results of querying the metadata for accession 'DRR003148' (Listing \ref{lst:exp:queryMetadata})}
\label{tab:exp:queryMetadata}
\input{expVIP/tables/metadataDRR003148}
\end{table}

Likewise, to get the \texttt{expression\_values} for a \texttt{gene} with the corresponding unit (\texttt{type\_of\_values}) and \texttt{experiment} a simple query joining the four tables is used. 
The Listing \ref{lst:exp:queryExpValues} retrieves the \texttt{expression\_values} for the \texttt{gene} 'Traes\_5BS\_0AFC3F795.1', and the result is on Listing \ref{tab:exp:queryExpValues}

\begin{code}[language=sql, caption={[Query values for gene and experiment group] Query values from 'Group1' and gene 'Traes\_5BS\_0AFC3F795.1' },label=lst:exp:queryExpValues]
SELECT 
	genes.name as gene, 
	expression_values.value,
	experiments.accession,
	type_of_values.name as unit
FROM expression_values
JOIN genes 
	ON expression_values.gene_id = genes.id
JOIN type_of_values 
	ON type_of_values.id = expression_values.type_of_value_id
JOIN experiments 
	ON experiments.id = expression_values.experiment_id
WHERE 
	genes.name = 'Traes_5BS_0AFC3F795.1' 
\end{code}

\begin{table}[h]
\caption[Results of query for values]{Results of query to get the values for gene 'Traes\_5BS\_0AFC3F795.1' (Listing \ref{lst:exp:queryExpValues}), only 'Group1' is displayed from the output. The three values with the same unit correspond to replicates on the same study.}
\label{tab:exp:queryExpValues}
\begin{tabular}{lrlll}
\toprule
 gene                  &    value & accession   & experiment   & unit   \\
                    &     &    & group   &    \\
\midrule
 Traes\_5BS\_0AFC3F795.1 & 136.995  & DRR003148   & Group1             & count  \\
 Traes\_5BS\_0AFC3F795.1 & 120.683  & DRR003149   & Group1             & count  \\
 Traes\_5BS\_0AFC3F795.1 & 140.94   & DRR003150   & Group1             & count  \\
 Traes\_5BS\_0AFC3F795.1 &  24.2277 & DRR003148   & Group1             & tpm    \\
 Traes\_5BS\_0AFC3F795.1 &  23.9739 & DRR003149   & Group1             & tpm    \\
 Traes\_5BS\_0AFC3F795.1 &  24.9835 & DRR003150   & Group1             & tpm    \\
\bottomrule
\end{tabular}

\end{table}

With those two queries is enough to retrieve all the information required to do sub-groupings. 

The database is implemented using the \acrshort{rdbms} \texttt{MySQL 5.5}. 



\section{Data integration pipeline} 
\label{exp:pipeline}


To prepare the database, expVIP requires to have all the metadata for the experiments to integrate. 
ExpVIP contains tasks to load all the metadata and a wrapper for Kallisto that can be run from expVIP. 
Alternatively, the expression values can be calculated with another tool and loaded as a single file, this approach is preferred for a large set of samples (Figure \ref{fig:exp:loadPipeline}). 
Details on how to load the files in the database are in the expVIP tutorial (Appendix \ref{exp:tutorial}). 

\begin{figure}
\centering
\includegraphics[width=1\textwidth]{expVIP/Figures/loadDataPipeline.pdf}
\caption[expVIP load data]{The pipeline for loading the data into expVIP. The black lines represent a border of tasks that are not required to be executed in a particular order.}
\label{fig:exp:loadPipeline}
\end{figure}

The required files for the metadata are:
\begin{description}
\item[Factors.] The file contains all the possible factors that can be used to group all the experiments. The file must contain the following columns (Table \ref{tab:exp:factors}):
\begin{description}
\item[factor] The category were the factor belongs. In the case of the initial dataset used in expVIP, the grouping factors are: Age, stress-disease, tissue and a corresponding 'High level' for each factor. The metadata file must contain a column corresponding to each one of this factors. 
\item[order] The default order in which to display each factor. This ensures that the age of the plants is sorted chronologically. 
\item[name] Long description of each level for the category. The values in this column must match the values in the metadata file (see below).
\item[short] Is a short name, used when the space to display the full description of the factor is not enough.
\end{description}
\begin{table}
\centering
\caption[Factors file]{Factors file. The table must be saved as a text file, with columns separated by tabs}
\label{tab:exp:factors}
\begin{tabular}{llll}
\toprule
factor & order & name & short \\
\midrule
Age & 1 & 7 days & 7d \\
Age & 2 & seedling stage & see \\
Age & 3 & 14 days & 14d \\
Age & 4 & three leaf stage & 3\_lea \\
Age & 5 & 24 days & 24d \\
High level age & 1 & seedling & see \\
High level age & 2 & vegetative & veg \\
High level age & 3 & reproductive & repr \\
High level stress-disease & 1 & none & none \\
High level stress-disease & 2 & disease & dis \\
High level stress-disease & 3 & abiotic & abio \\
High level stress-disease & 4 & transgenic & trans \\
High level tissue & 1 & spike & spike \\
High level tissue & 2 & grain & grain \\
... & & \\
\bottomrule
\end{tabular}
\end{table}
\item[metadata] The metadata file is the file that contains the information related to each study and the corresponding experiments. 
Each study contains several experiment groups (replicates), which in turn contain every individual experiment. 
The factors must be shared across experimental groups. 
\begin{description}
\item[secondary\_study\_accession] The accession number for experiments carried as part of a single study. This is usually the high level BioProject or SRA number. 
\item[run\_accession] The accession of the individual run. 
\item[scientific\_name] of the species. 
\item[experiment\_title] A description for the individual RNA-seq sample.
\item[study\_title] A description of the general study.
\item[Manuscript] The DOI of the study.
\item[Group\_for\_averaging] A description of the experiment. This must be the same all the replicates in the same study. 
\item[Group\_number\_for\_averaging] A short name for replicated experiments.  
\item[Total reads] (optional)
\item[Mapped reads] (optional)
\end{description}
Besides the main fields, each factor has a a corresponding column Variety, Tissue,Age, Stress-disease, High level variety, High level tissue, High level age and High level stress-disease

\item[Gene set] The gene set is provided as a single \verb|fasta| file. 
The file may contain alternative transcripts from the same gene. To identify this, the \verb|fasta| header may include the optional fields \texttt{gene} and \texttt{transcript}. 
In the absence of this, the only stored value is the name derived from the header, going from the \textgreater character to the first space on the line (Listing \ref{lst:poly:geneFa}). 

\begin{code}[label=lst:poly:geneFa, caption={[Gene set fasta file] A fasta entry on of the gene set.}]
>Traes_5BL_3FC5BA305.1 cdna:novel scaffold:IWGSC2:IWGSC_CSS_5BL_scaff_1082268:5:199:-1 gene:Traes_5BL_3FC5BA305 transcript:Traes_5BL_3FC5BA305.1
TGCTGCTGCTAGGCTTGAAGAGGTTGCTGGCAAGCTCCAGTCTGCTC
GGCAGCTCATTCAGAGGGGCTGTGAGGAGTGCCCCAAGAACGAGGAT
GTTTGGTTCGAGGCATGCCGGTTGGCTAGCCCAGATGAGTCAAAGGC
AGTAATTGCCAGGGGTGTGAAGGCAATTCCCAACTCTGTGAAGCTGT
GGCTGCA
\end{code}
\item[homoeologues] A file containing the homoeologues for the  A, B and D genomes. Currently these are the only supported default names. The file also includes a column with the gene name and to which Group (ie 1, 2, 3 ... 7) and Genome (ie A, B or D) it belongs (Table \ref{tab:exp:hom}).  

\begin{sidewaystable}
\centering
\caption[Homoeology file]{Example tabular file containing the homoeology across the three genomes. }
\label{tab:exp:hom}
\begin{tabular}{llllll}
\toprule
Gene & A & B & D & Group & Genome \\
\midrule
Traes\_5BS\_0AFC3F795 & Traes\_5AS\_0AFCC204EBA & Traes\_5BS\_0AFC3F795 & Traes\_5DS\_C204EBAA9 & 5 & B \\
Traes\_5DS\_C204EBAA9 & Traes\_5DS\_0AFCC204EBA & Traes\_5BS\_0AFC3F795 & Traes\_5DS\_C204EBAA9 & 5 & D \\
Traes\_7DL\_82360D4EE1 & & Traes\_7BL\_3F7958204 & Traes\_7DL\_82360D4EE1 &  7 & D \\
Traes\_2AL\_1368BE0AD & Traes\_2AL\_1368BE0AD & & Traes\_2BL\_CD459994C1 & 2 & A \\
... & & & & &  \\
\bottomrule 
\end{tabular}
\end{sidewaystable}
\end{description} 

expVIP includes several tasks to load the different files. 
For example, to load the factors the \verb|load_data:factor| starts a transaction (Listing \ref{lst:exp:loadFactor}; line 2) to ensure that all the data is loaded, and if for some reason the load fails, the database is restored to the previous status.
In the transaction, the file is read row by row using the \verb|csv| library (line 3).
The function \verb|find_or_create_by| is a function that \acrshort{ror} provides on models to create an entry in the table, or update it if already exists.
Each row is used to create or update a \verb|Factor| (lines 374-376). 
A similar strategy is used for all the files that are regular tables. 

\begin{code}[label=lst:exp:loadFactor, language=ruby, caption={[Load factors]Task that loads factors}]
task :factor, [:filename] => :environment do |t, args|
  ActiveRecord::Base.transaction do 
    CSV.foreach(args[:filename], :headers => true, :col_sep => "\t") do |row|
      factor = Factor.find_or_create_by(:factor=>row["factor"],  :description=>row["name"],  :name=>row["short"])
      factor.order = row["order"].to_i
      factor.save!
    end
  end
end
\end{code}

The gene sets are loaded slightly differently, as the input is a \verb|fasta| file, as opposed to tabular file. 
The reader for the \verb|FastaFormat| from BioRuby \citep{Goto2010} is used to read the file (Listing \ref{lst:exp:loadGenes}; line 4). 
Since expVIP only records the name of the genes, only the id of the fasta sequence is extracted (lines 6-79). 
The name is stored in the name and cDNA columns. 
The parser for entries from ensembl, such the one in Listing \ref{lst:poly:geneFa} include code to load the \acrshort{cdna} and transcript fields correctly. 

\begin{code}[language=ruby, caption={[Load genes from Fasta]Task that load genes from a fasta file}, label=lst:exp:loadGenes]
task :de_novo_genes, [:gene_set,:filename] => :environment do |t, args|
  ActiveRecord::Base.transaction do
    gene_set = GeneSet.find_or_create_by(:name=>args[:gene_set])
    Bio::FlatFile.open(Bio::FastaFormat, args[:filename]) do |ff|
      ff.each do |entry| 
        arr = entry.definition.split(/\s+/)
        name = arr[0]
        g = Gene.new 
        g.gene_set = gene_set
        g.name = name
        g.cdna = name
        g.save!
      end
    end
  end
end
\end{code}

There are two options to load the expression values from the database: a matrix with all the expression values and running \verb|Kallisto| from expVIP. 

The task in Listing \ref{lst:exp:loadValues} loads the expression values from a tabular file with the genes as rows and the values as columns. 
The exception handling and messages are removed 
The task requires the following arguments:

\begin{description}
\item[\texttt{meta\_experiment}.] A name for the analysis. This can be the name of the tool used for the expression quantification combined with the name of the reference, as a single text variable. 
\item[\texttt{gene\_set}.] The reference used for the analysis. 
\item[\texttt{value\_type}.] The unit of the file (ie.  \acrshort{tpm}, count)
\item[\texttt{filename}.] The file that is going to be loaded in the database. 
\end{description}

The steps to load the values are:
\begin{enumerate}
\item A transaction is initiated at the beginning of the task, to ensure that if any step fails and the execution is aborted the database will stay in a consistent state (Listing \ref{lst:exp:loadValues}; line 2).
\item The connection is assigned to the variable \verb|conn|, to be able to execute queries directly to the database (line 3).   
\item The \verb|meta_experimet|, \verb|gene_set| and \verb|value_type| are loaded and stored to get the corresponding IDs in the insertion (line 4).
\item All the \verb|Genes| and \verb|Experiments| are loaded in their corresponding hash table, to be able to get the IDs when the actual values are inserted (lines 7-11). 
\item The file is read with the \verb|CSV| library from Ruby, keeping the headers to be able to assign the correct experiment (line 14). 
\item The first column is named \verb|target_id| and contains the gene name. The ID of the gene  in the database is retrieved from the previously loaded hash (lines 15-16)
\item Each column is iterated and the values needed to execute the insertion to the database are concatenated. 
\item Whenever the number of queued insertions reaches 1,000, the command to perform the insertions is executed (line 25). 
\item As the number of genes is not usually a multiple of 1,000, when the process finished reading the file an extra insertion is executed to empty the queue (line 30).
\end{enumerate}

The decision to execute the insertions in batches of 1,000 objects is to reduce the number of processes running in the database, while keeping low the memory usage of the application. 
This approach is faster than using the functions for insertions \acrshort{ror} on multiple values. 
For trivial operations, the functions from the framework are used, as they are easier to maintain (compare insertion in line 4 to the block of code from line 18 to 26). 

\begin{code}[language=ruby, label=lst:exp:loadValues, caption={[Load expression values from file] Task to load the expression values from a tabular file.}]
task :values, [:meta_experiment, :gene_set, :value_type, :filename ] => :environment do |t, args| 
ActiveRecord::Base::transaction do
conn = ActiveRecord::Base.connection
meta_exp = MetaExperiment.find_or_create_by(:name=>args[:meta_experiment])
gene_set = GeneSet.find_by(:name=>args[:gene_set])
value_type = TypeOfValue.find_or_create_by(:name=>args[:value_type])
experiments = Hash.new
meta_exp.gene_set = gene_set
genes = Hash.new
Gene.find_by_sql("SELECT * FROM genes where gene_set_id='#{gene_set.id}'").each {|g| { genes[g.name] = g.id}
Experiment.find_each{|e| experiments[e.accession] = e.id}
count = 0
inserts = Array.new
CSV.foreach(args[:filename], :headers => true, :col_sep => "\t") do |row|
	gene_name = row["target_id"]
	gene = genes[gene_name]
	row.delete("target_id")
	row.to_hash.each_pair do |name, val| 
		val = val.to_f 
		str = "(#{experiments[name]},#{gene},#{meta_exp.id},#{value_type.id},#{val},NOW(),NOW())"
		inserts.push str          
	end
	count += 1
	if count % 1000 == 0 
		sql = "INSERT INTO expression_values (`experiment_id`,`gene_id`, `meta_experiment_id`, `type_of_value_id`, `value`,`created_at`, `updated_at`) VALUES #{inserts.join(", ")}"
		conn.execute sql
		inserts = Array.new
	end
end
sql = "INSERT INTO expression_values (`experiment_id`,`gene_id`, `meta_experiment_id`, `type_of_value_id`, `value`,`created_at`, `updated_at`) VALUES #{inserts.join(", ")}"
conn.execute sql
end
end
\end{code}


\begin{wrapfigure}{R}{0.45\textwidth}
\centering
%\rule{3cm}{7cm}
\includegraphics[width=0.45\textwidth]{LitReview/Figures/runKallisto.pdf}
\caption  {Steps to run and load Kallisto}
\label{fig:exp:runKallisto}
\end{wrapfigure}

Alternatively, expVIP can execute \verb|Kallisto| on all the samples loaded in the database. 
For this purpose, expVIP stores the raw reads in FastQ format, organized in directories named with the same accessions as in the metadata (ie a directory named \verb|DRR003148| contains the reads for the metadata displayed in table \ref{tab:exp:queryMetadata}). 
expVIP takes all the accessions for the experiments in the database and searches for the corresponding folder. 
If the folder exists and if it contains the \verb|fastq| files, then it is deemed as valid. 
If the folder already has the \verb|kallisto| output, the next folder is evaluated, otherwise \verb|kallisto| is executed with its default settings and the results are loaded into the database. 
This process is repeated for all the accessions (Figure \ref{fig:exp:runKallisto}).  
This pipeline allows to populate the database partially, in case that not all the experiments are ready from the beginning. 

New experiments can be added to the existing metadata file  or to a new file; the expVIP loading procedure then can be run again to update the list of experiments and the corresponding expression values. 
This design allows to keep the database updated as more experiments become available. 
The fact that the loading is done in transactions ensures that the database is kept consistent, regardless of potential errors in the input files. 

\section{Graphical interface}
\label{exp:gui}  
%How the expression can be displayed filtered, and sorted
With the expression across experiments integrated in a single database, the next challenge was to make the data accessible to a wide audience. 
\acrshort{ror} has tasks to automate the construction of controllers and views from the \acrshort{mvc} metaphor, helping on the retrieval and formatting of the raw data. 
However, the main objective was to visualise the data from all the experiments as intuitively as possible. 
To make the visualisation dynamic in a browser, the use of \verb|JavaScript| is necessary, as it is the only widely adopted programming language used for web content. 
Among the tools built on the top of \verb|JavaScript|, \verb|D3| is a framework designed explicitly to do dynamic visualisations \citep{Bostock2011}. 

Usability was a top priority on the design of the visualisation component for expVIP. 
The time needed to learn, performance are accuracy were taken into account when designing expVIP. 
As memorability and satisfaction are subjective, they were not directly tested for during development.  


\begin{description}
\item[Time to learn.] The controls are located in two blocks, one for global controls (ie. unit, save plot) and one to modify the factors, close to the factor they affect (Figure \ref{fig:exp:layout}). 
\item[Performance.] All the data for the genes being displayed is loaded from the database in a single transaction. The data is available in the cache of web browser and whenever the user changes anything in the visualisation, new values are calculated locally. 
\item[Accuracy.] As knowing what the users will do is not obvious, the visualisation was given to a panel of potential users (other members of the Uauy lab) for comments. In previous versions the legends were a confusing and the location of the buttons was too distant from the aspect of visualisation that they controlled. After reviewing the feedback, the accuracy was improved.  Also, the position of the controls don't change, regardless of the representation of the data (bar plot or heatmap). 
\end{description}

\begin{figure}
\includegraphics[width=1\textwidth]{expVIP/Figures/UIsketch.pdf}
\caption[\gls{expvip} \gls{gui} layout]{expVIP \acrlong{gui} components. The top bar shows a short description of what is displayed. The General and Visualisation controls contain the buttons and menus that enable the interaction with the figure. The Headers, Labels and Plot are the actual components of the visualisation}
\label{fig:exp:layout}
\end{figure}

The elements in the graphical are shown in Figures \ref{fig:exp:tutorial1}, \ref{fig:exp:tutorial2} and \ref{fig:exp:tutorial3}, with a description of each element of the \acrshort{gui} listed below. 

\begin{figure}
\includegraphics[width=1\textwidth]{expVIP/tutorial/images/Figure1.png} 
\caption{Features on expVIP}
\label{fig:exp:tutorial1}
\end{figure}
\begin{enumerate}
\item
  \textbf{\lstinline!Search box!}: at any point it is possible to type a new gene name (based on Ensembl Plants nomenclature) and generate a new set of expression data.
\item
  \textbf{\lstinline!Compare box!}: it is possible to add a second gene and press the \lstinline!Compare! button to generate two expression graphs drawn at the same scale.
\item
  \textbf{\lstinline!Menu options!}: has several links on the details of the study and tutorial. The menu can be edited to customize instances of expVIP. 
\item
  \textbf{\lstinline!Gene!}: shows the currently displayed gene, with a link to Ensembl to the corresponding description. 
\item
  \textbf{\lstinline!Expression unit!}: selects the expression unit to visualise. This can be either \gls{tpm} or  estimated counts. If other units are loaded in the database, they will appear in this field. 
\item
  \textbf{\lstinline!Save graph!}: these two buttons allow users to save the current graphs in either \lstinline!SVG! (to work on Adobe Illustrator) or as \lstinline!PNG! files. 
  The export process selects the graphical elements from the visualization and binds them together on a single \verb|SVG| file. The reasons to follow this process are: to ensure that the plot reflects the user selection; remove the elements that do not have a meaning in a static context and; allow the user to format the image with publication quality (Figure \ref{fig:exp:exportImage}).

\begin{figure}

\begin{subfigure}{1\textwidth}
\caption{}
\label{fig:exp:exportLayout}
\includegraphics[width=1\textwidth]{expVIP/Figures/Exportsketch.pdf}
\end{subfigure}

\begin{subfigure}{1\textwidth}
\caption{}
\label{fig:exp:exportImageSample}
\includegraphics[width=1\textwidth]{expVIP/Figures/tutorialSampleOutput.png}
\end{subfigure}

\caption[expVIP export image]{expVIP export image. (\ref{fig:exp:exportLayout}) \acrshort{gui} components components exported in image} (\ref{fig:exp:exportImageSample}) The exported components from Figure \ref{fig:exp:tutorial1}) 
\label{fig:exp:exportImage}.
\end{figure}

\item
  \textbf{\lstinline!Save data!}: downloads a \lstinline!csv! file with the data with the current selection and
  order of factors as displayed on the screen. The data will include the standard errors and the number of samples that make up each value. 
  An example of how the output looks is in Listing \ref{lst:exp:exportSample}. 


\item
  \textbf{\lstinline!Homoeologues!}: this button displays the
 graphs of known homoeologues of the original primary gene. This gene is highlighted in bold and the homoeologous graphs will be displayed according to A, B, D genome ordering. 
 The scale of all the homoeologues is the same to facilitate the comparison between them. 
\item
  \textbf{\lstinline!Gene names!}: each displayed gene is labelled on this block.  
  If the list of genes is too long, the gene names are rotated for readability. 
  In case that a gene name is too long to fit, the font is scaled to the largest size that will fit in the designated area. 
\item
  \textbf{\lstinline!Expression level!}: the expression level adjusts according to the expression of each set of gene homoeologues. The scale remains consistent across homoeologues to allow comparison. 
  The values are based on the unit selected in the \lstinline!expression unit! box (see point 5 above).

\begin{figure}
\includegraphics[width=1\textwidth]{expVIP/tutorial/images/Figure2.png} 
\caption{Features on expVIP (continued)}
\label{fig:exp:tutorial2}
\end{figure}

\item
  \textbf{\lstinline!Filter!}: opens a pop-up window revealing the levels of the selected category. 
  By default, all the levels are selected, but the users can decide to exclude from the visualisation experiments containing any level. 
  The order of the levels can be modified by dragging the levels on the the pop-up window. 
\item
  \textbf{\lstinline!Display/hide category!}: Each category
  can be displayed or hidden by pressing the \lstinline!+/-! button.
  As categories are added or removed the expression graphs show the new values for the new groups. 
  Data is not removed when changing the displayed categories, instead the values are distributed according to the new groups (the number of samples remains the same). 
  The colours below the category correspond to each level, and the plot is coloured according to the sorting category. 
\item
  \textbf{\lstinline!Expression bars!}: These bars represent the expression level for the \verb|n| grouped samples according to
  the chosen categories (11 and 12 above).
  When hovering over the bar with the mouse a small tooltip will appear, containing the expression level (\lstinline!tpm! or \lstinline!counts!) and the standard error (sem) used for the error bars (see 14)
\item
  \textbf{\lstinline!Error bars!}: Standard error of the means for the \verb|n| expression values on which the bar graph is based.
\item
  \textbf{\lstinline!Factors!}: The colour of the rectangles represents the displayed categories, according to the selection criteria (11 and 12 above). 
  To clarify the meaning of the colour, hovering above the rectangle displays a tooltip with the long name of the examined level. 
\item
  \textbf{\lstinline!Description!}: Textual description of the grouped factors, according to the selection criteria (11 and 12 above). 
  The number of grouped samples is also displayed.

\begin{figure}
\includegraphics[width=1\textwidth]{expVIP/tutorial/images/Figure2a.png} 
\caption{Features on expVIP for Multiple gene comparisons}
\label{fig:exp:tutorial3}
\end{figure}

\item
  \textbf{\lstinline!Expression unit!}: For heatmaps, the default unit log2(tpm) as the logarithmic scale provides a better context for comparisons acorss several genes.
\item
  \textbf{\lstinline!Heatmap!}: To compare several genes, the values are represented as a heatmap. The sorting and filtering is done with the same controls as for single genes.
  Up to 50 genes are displayed in the heatmap, as more genes will degrade the performance of the database, and the visual clutter makes the plot hard to interpret. 
  This view allows to visualise several candidate genes for a trait expressed under certain conditions and quickly asses which one is a good candidate for further research. 
\item
  \textbf{\lstinline!Scale!}: The scale is calculated according to the highest displayed value in the current heatmap. 
  Since logarithmic values below 1 result in negative values and anything with a \acrshort{tpm} under 2 is considered as very low expressed, every value lower than 1 is plotted as 0.
\end{enumerate}
For a comprehensive user manual see Appendix \ref{exp:tutorial}. 

%%\section{Virtual Machine}
%%\label{exp:vm}
\unsure{If I have time, I'll add the section about the virtual machine, as I would also need to add something in the background on virtualisation, it can potentially be a sink of time}

\begin{landscape}
\begin{code_2}[label=lst:exp:exportSample, caption={[Export data example] Export data example, corresponding to the data plot in Figure \ref{fig:exp:tutorial1}}]
High level age	seedling, vegetative, reproductive, 
High level stress-disease	no stress, disease, abiotic, transgenic, 
High level tissue	spike, grain, leaves/shoots, roots, 
High level variety	Chinese Spring, other, Nullitetra Chinese Spring, 
	tpm	SEM	tpm	SEM	tpm	SEM	
	Traes_2AL_173FE664B.2	Traes_2AL_173FE664B.2	Traes_2BL_2141AFC9E.1	Traes_2BL_2141AFC9E.1	Traes_2DL_099F54442.1	Traes_2DL_099F54442.1	
roots, vegetative, no stress(n=62)	4.96	0.99	1.77	0.49	2.08	0.47
roots, vegetative, abiotic(n=3)	7.26	1.95	3.00	0.13	2.85	0.66
leaves/shoots, vegetative, no stress(n=65)	1.59	1.50	0.91	0.55	0.87	0.56
leaves/shoots, vegetative, abiotic(n=3)	2.33	0.38	1.55	0.07	1.29	0.09
spike, reproductive, disease(n=30)	6.85	0.90	1.19	0.22	2.99	0.38
spike, reproductive, no stress(n=43)	6.01	1.67	1.32	0.43	2.58	0.81
grain, reproductive, no stress(n=147)	3.47	1.69	0.76	0.55	1.51	0.77
leaves/shoots, reproductive, transgenic(n=4)	2.15	0.40	0.88	0.18	1.11	0.25
leaves/shoots, reproductive, no stress(n=7)	2.69	0.67	1.37	0.89	1.30	0.62
leaves/shoots, seedling, disease(n=23)	3.25	1.50	0.85	0.71	1.64	0.72
leaves/shoots, seedling, no stress(n=13)	2.10	0.67	0.55	0.23	1.07	0.29
leaves/shoots, seedling, abiotic(n=12)	0.99	0.39	0.25	0.12	0.74	0.34
roots, seedling, no stress(n=2)	4.96	0.32	1.49	0.17	2.07	0.11
roots, reproductive, no stress(n=2)	5.06	1.46	1.62	0.32	2.10	0.14
spike, vegetative, no stress(n=2)	6.55	0.39	2.27	0.27	2.43	0.03
\end{code_2}
\end{landscape}

\section{Discussion.} 


%The use of previously published studies is a valuable resource. 

In model organisms there are several on-line resources that aggregate the raw data and meta analysis of several studies. 
For example, the Expression Atlas \acrshort{ebi}  includes over 2,000 studies for \textit{Arabidopsis thaliana} inclusive of other technologies besides RNA-Seq (ie Affymetrix expression arrays). 
However, when I started the development of expVIP the Expression Atlas only included a couple of baseline experiments and WheatExp had not been published yet (Figure \ref{fig:exp:timeline}).  

\begin{figure}
\includegraphics[width=1\textwidth]{expVIP/Figures/Timeline.pdf}
\caption[expVIP resources time line]{expVIP resources time line. In bold the line of the time for development of expVIP and the annotation used. The Expression Atlas has a line starting on the initial study deposited on it till the last update, as of September 2016.}
\label{fig:exp:timeline}
\end{figure}

%When I started the development of expVIP none of the expression browsers mentioned in Section \ref{exp:alternative} had data for wheat. 


WheatExp was developed roughly at the same time of expVIP and contains 6 studies (against 16 on expVIP; \citealt{Pearce2015b}). 
Four of the studies in WheatExp are on hexaploid wheat and are included in expVIP as well. 
In WheatExp, the expression for each gene is displayed on the context of the original experiment without a direct comparison between studies. 
However, because the studies are kept independently the levels of the categories are a direct reflection of the original studies. 
%\unsure{Why is this fact a detriment for WheatEXP?}
WheatExp allows to search genes by sequence, a feature not implemented in expVIP. 

The Expression Atlas from \acrshort{ebi}, as of September 2016, contains 10 studies, 4 baseline and 6 for differential expression \citep{Petryszak2016}. 
ExpVIP contains studies released before \acrshort{ebi} started to upload expression experiments for wheat. 
Even if some of the most recent studies in the Expression Atlas are not included in expVIP yet, we are in the process of updating our tool to include studies published after the initial release.
In terms of visualisation, the \acrshort{ebi} included a heatmap to compare to different factors at the same time for given gene (ie. tissue vs stress); the same information can be displayed on expVIP by sorting by two factors.


Most of the RNA-Seq studies report their results in terms of \acrshort{rpkm}. 
This normalisation, which is computed for done for each feature $g$ in the reference $G$, requires the count of the number of reads  $r_{g}$, the feature length $\textrm{fl}_{g}$ and the total number of mapped reads $R$ (Figure \ref{fig:exp:units}; equations \ref{eq:exp:R} and  \ref{eq:exp:rpkm} \citealt{Mortazavi2008}).
As the denominator of \ref{eq:exp:rpkm} is based on the number of mapped reads rather than the number of nucleotides covered, RPKM does not allow to compare correctly results obtained between different samples, or even results from the same sample when the average read length changes due to variations in the experimental protocol \citealt{Wagner2012}.
%\unsure{The original sentence was not completely clear; I hope this version elucidates more}

\begin{figure}
\includegraphics[width=1\textwidth]{expVIP/Figures/Units.pdf}
\caption[Units used for expression normalisation]{Units used for expression normalisation}
\label{fig:exp:units}
\end{figure} 


\begin{equation}
\label{eq:exp:R}
R = \displaystyle\sum_{g \in G} r_{g} 
\end{equation}

\begin{equation}
\label{eq:exp:rpkm}
  \textrm{RPKM}_{g} = \frac{r_{g}\times10^9}{\textrm{fl}_{g}\times R}
\end{equation}

The \acrshort{tpm} is an alternative to \acrshort{rpkm} that approximates the total number of transcripts $T$ as a normalisation factor (equation \ref{eq:exp:T}). 
Besides the previously described parameters, the \acrshort{tpm} also includes the read length $\textrm{rl}$, which is dependent on the study. 
This formula assumes that each read corresponds to a full observed transcript (equation \ref{eq:exp:tpm}; \citealt{Wagner2012}). 
%\unsure{Do you mean here that the formula assumes that each read has been assigned *unambiguously* to a single transcript? The sentence is unclear.}

\begin{equation}
\label{eq:exp:T}
T = \displaystyle\sum_{g \in G} \frac{r_{g} \times \textrm{rl}}{\textrm{fl}_{g}}
\end{equation}

\begin{equation}
\label{eq:exp:tpm}
  \textrm{TPM}_{g} = \frac{r_{g} \times \textrm{rl}\times10^6}{\textrm{fl}_{g} \times T }
\end{equation}


One of the aims in developing TPM was to be able to compare samples from different studies; as it is more stable across different experiments, we decided to use it as the main unit of comparison in expVIP. 
After we took the decision of which unit to use, we found a couple of tools, Kallisto \citep{Bray2016} and Sailfish \citep{Patro2014}, that could calculate the TPMs directly from mapping the reads to a reference, without producing a precise alignment.
The main advantage of only doing mapping, without aligning is a significant reduction in both the computational resources and the time needed to analyse a sample.

%\unsure{These lists would really benefit from being multi-line, not all inline like they are displayed at the moment.}
The traditional pipeline to quantify expression from RNA-Seq consists on the following steps:
\begin{enumerate}
  \item Index the reference. Only done once, as the same index can be used for all the samples. 
  \item Align the reads to the reference. 
  \item Sort the alignment and remove duplicates. 
  \item Quantify the expression.  
\end{enumerate}
This is the prevailing pipeline for expression analysis. 
In my experience, on a computing cluster it takes between 6 to 8 hours to process each wheat sample on a computing cluster, using multiprocessing and up to 24 Gb of RAM (Figure \ref{fig:exp:alnPipeline}). 
This pipeline is usually implemented by aligning the reads with \verb|BWA| \citep{Li2009} or \verb|tophat|  \citep{Trapnell2012} and the quantification is performed with tools like \verb|HTSeq| \citep{Anders2015} or \verb|cufflinks| \citep{Trapnell2012}.

%\unsure{Again, this list would be clearer as one item per line rather than inline.}
An advantage of using a mapper is that the quantification pipeline is shorter:
\begin{enumerate}
  \item Index the reference. Only done once, as the same index can be used for all the samples. 
  \item Map the reads to the reference and quantify the expression in a single program.   
\end{enumerate}
Since mapping does not require the precise alignment of every single base on the read and the output is only the quantification for each gene, as opposed to the alignment of each read, the programs implementing mapping take around 15 minutes to run. 
RNA-Seq mapping algorithms are implemented by \verb|Kallisto|  \citep{Bray2016} and \verb|Sailfish| \citep{Patro2014}.

\begin{figure}
\begin{subfigure}{1\textwidth}
\caption{}
\label{fig:exp:alnPipeline}
\includegraphics[width=1\textwidth]{expVIP/Figures/alnpipeline.pdf}
\end{subfigure}

\begin{subfigure}{1\textwidth}
\caption{}
\label{fig:exp:mapPipeline}
\includegraphics[width=1\textwidth]{expVIP/Figures/mappipeline.pdf}
\end{subfigure}

\caption[Alignment vs mapping pipelines]{Alignment vs mapping pipelines for RNA quantification. (\subref{fig:exp:alnPipeline}) Alignment pipeline. (\subref{fig:exp:mapPipeline}) Mapping pipeline.}
\label{fig:exp:alnVSmap}
\end{figure} 

I decided to integrate \verb|Kallisto|  into expVIP because the algorithm is able to walk through different splicing junctions via a \acrshort{tdbg} (see Section \ref{exp:kallisto}; \citealt{Bray2016}), as opposed to Sailfish, which is based on counting k-mers only \citep{Patro2014}. 
Since homoeologues with a high level of identity may form a bubble in the graph, in a similar way to small indels or SNPs \citep{Leggett2013}, \verb|Kallisto|  should be able to assign the reads to the correct k-compatibility class for their corresponding homoeologue.  
\unsure{Make a diagrame with the graph. Unfortunatly, I don't have in hand my diagrams for cortex an Richard did the diagrams for Legget et al.}

Some features that could improve expVIP in the long term and that already present in other expression browsers would include searching by sequence, by gene ontology and a heatmap of a particular gene displaying two different factors on each axis . 
A feature that expVIP could leverage from other genomic resources is the retrieval of genes flanking a gene of interest, or between genetic markers. 
The upcoming assemblies , from the IWGSC \citep{Clark2016} and TGACv1 \citep{Pozniak2016}  with longer scaffolds in conjunction with the high resolution genetic maps from \citet{Wang2014} and \citet{Chapman2015} can enable such kind of queries (resources described in Section \ref{lit:wheatResourcers}). 

For the TGACv1 assembly, an improved wheat annotation was made available after the initial release of expVIP (Figure \ref{fig:exp:timeline}). The \acrshort{iwgsc} will provide an updated annotation for their genome as well. 
In the near future, expVIP will be updated to include those annotations and some development will be needed to allow the comparison between annotations, at least while the community adopts a canonical reference. 

The same mechanisms to compare expression between different references can be used to compare the expression between different organisms. 
However, the current implementation uses the homology table with one column for each genome group in hexaploid wheat (A, B, and D; Figure \ref{fig:exp:currentGeneRelations}). 
To be able to allow the inclusion of different polyploids with different genome names, to compare homologoues and paralogues effectively and to add any arbitrary relationship between genes the homoeologues table needs to be updated. 
Instead of representing triplets, the table should contain binary relationships; each gene pair will have a type to be able to distinguish between relationships. 
Furthermore, each gene set should be linked to a species. 
With that explicit relationship, equivalent genes from the same species, but coming from different gene models can be identified. 
Likewise, genes known to be conserved across relatives (ie Barley vs Wheat genes) can be compared through these modifications to the database (Figure \ref{fig:exp:futureGeneRelations}).  

\begin{figure}
\centering
\begin{subfigure}{0.45\textwidth}
\centering
\caption{}
\label{fig:exp:currentGeneRelations}
\includegraphics[width=0.85\textwidth]{expVIP/Figures/currentGeneRelations.pdf}
\end{subfigure}
~
\begin{subfigure}{0.45\textwidth}
\centering
\caption{}
\label{fig:exp:futureGeneRelations}
\includegraphics[width=0.85\textwidth]{expVIP/Figures/futureGeneRelations.pdf}
\end{subfigure}
\caption[Changes to improve gene comparasions]{Changes to improve gene comparisons. (\subref{fig:exp:currentGeneRelations}) Current implementation of homoeology. (\subref{fig:exp:currentGeneRelations}) Proposed implementation to extend the types of relationships between genes.  }
\end{figure}

To the best of my knowledge, none of the expression browsers available for wheat, or other polyploid species, allow the direct comparison between homoeologues. However, the effect of different related genes is a topic of active research in polyploids. 
Making the relationship of the expression between related genes easily accessible can provide some initial evidence of having an uniform expression across homoeologues or of a triplet with a dominant gene. 
Likewise, when the update to the table that keeps arbitrary relationships between genes is completed, it will be possible to find if related genes have a conserved expression pattern. 

Since its conception, I wanted expVIP to be open source and available to the community. 
As part of the project, I released the visualisation component as a biojs component  (\texttt{bio-vis-expression-bar}; \url{http://biojs.io/d/bio-vis-expression-bar}; \citealt{Yachdav2015}).
In a collaboration with the Earlham Institute and eLife, another PhD student is working to integrate the visualisation plugin as a live figure. 
If I had kept the code closed, this potential use of the component would have never happened. 

The webserver is also open and hosted on github (\url{https://github.com/homonecloco/expvip-web}), with a tutorial on how to load the data on a personal server (Appendix \ref{exp:tutorial}). 
As the quantification tool used by expVIP has a relatively low memory requirement, I was able to prepare and preconfigure two virtual machines, one without any data loaded and one with all the data used in the original article \url{http://www.wheat-expression.com}. 
This allows the comparison of private projects in the context of previous studies. 
On the Norwich Research Park, at least two groups have shown interest on using expVIP to make their data available to the community on other species.
We had also been contacted by both the team behind the annotation of TGACv1 and the IWGSC to include the upcoming annotations to the public instance of expVIP. 

The public server has received visitors from several research institutes around the globe (Figure \ref{fig:exp:users}). 
Since April, when the google analytics tracker was setup in the website, over 1,500 users have visited the site. 
Most of the users are based in the UK (34\%), and within the nation the majority of the traffic originates from Norwich, Cambridge, Harpenden, London and Dundee. 
Those cities have research institutes that work on wheat, so it is very likely that the users are real. 
Most of the international users are coming from the US, China, Australia, India, Germany and Canada. 
In those countries, most of the visitors also come from cities where wheat institutes are based. 
Furthermore, around half of the sessions stay on the website to access either the the individual genes or the heatmap. 
Around 25\% of the sessions consult their genes using the heatmap, suggesting that they have a list of candidate genes for a trait of interest to select by comparing their expression.

\begin{figure}
\includegraphics[width=1\textwidth]{expVIP/Figures/WorldSessions.png}
\caption{Heatmap of the number of unique sessions of expVIP in the world.}
\label{fig:exp:users}
\end{figure}

Overall, expVIP has met its aims (Section \ref{exp:aims}). 
I designed a relational database capable of storing several RNA-Seq experiments with their corresponding metadata. 
The expression analysis has been automated, to facilitate the process of running Kallisto, the selected quantification tool. 
The stored data can be visualised to compare the expression across several conditions, and the visualisation tool allows the arbitrary grouping and selection of studies. 
Finally, the open source code and the virtual machines facilitate the adoption of expVIP on other communities, not necessarily working on wheat. 

%!TEX root = ../Main.tex

\chapter{General discussion and final remarks}
\label{cha:discussion}
%This section wraps up by showing the relationship and importance of a comprehensive approach to data analysis, from the field, genetics, molecular biology and genomics. I will also remark how the technology and the resources have changed in the last 4 years. As at the references used at beginning where superseded during the PhD. 

%Biology is becoming interdisciplinary

%\section{Analysis and tool development for polyploid organisms}

Knowledge from  computer science can be applied to produce software for specific needs, and which can also be useful for the comunity. 
One of the limitations though is that most approaches are developed for diploid systems and are sometimes not compatible with polyploid species, such as wheat. 
Polyploidy adds an extra level of complexity (due to homoeologs) and in the case of wheat the large genome size also hinders certain approaches to genome analyses. Therefore bespoke tools are required to deal with this barriers. 

In this thesis we have taken advantage of developments in the technology and in resources to generate a series of solutions to many of the problems associated with polyploidy. These methods and approaches are not restricted to wheat, but can be applied and implemented to other polyploid systems.

I have discussed individual elements of project at the end of each chapter. However, looking forward my interest is to integrate this data into a single scheme. I outline this in Figure \ref{fig:discussion:allTables} and discuss it below. 

\begin{sidewaysfigure}
\includegraphics[width=1\textwidth]{Conclusions/Figures/CompleteDatabase.pdf}
\caption[Relational database integrating all the datasets.]{Relational database integrating all the datasets. The boxes represent the tables that contain information related to each chapter.}
\label{fig:discussion:allTables}
\end{sidewaysfigure}


\section{Integration of different genetic maps}

Genetic maps are a common starting point to start the search for a locus linked to a trait. 
For example, in Chapter \ref{yr15} previous genetics map had already identified the short arm of chromosome 1B as the locus for \acrshort{yr15} \citep{Murphy2009}.
Furthermore, I was able to confirm an enrichment of SNPs linked in the expected region thanks to the mapping of the markers included in the genetic map from \citet{Wang2014} to the \acrshort{css} scaffolds \citep{Mayer2014}. Since my initial analysis, other genetic maps with a higher resolution have been published \citep{Chapman2015, Allen2016,Winfield2016}. 
The relationship between the tables used in the genetic map is shown in Figure \ref{fig:discussion:geneticMapsTables}

\begin{figure}
\includegraphics[width=1\textwidth]{Conclusions/Figures/GenetiMapTables.pdf}
\caption{Tables to store information about genetic maps.}
\label{fig:discussion:geneticMapsTables}
\end{figure}

Genetic maps are produced from the genotype of several markers (marker sets). 
Those markers can be developed explicitly for the genetic map or from an already published marker set, usually in the form of an SNP array. 
A database containing several genetic maps should be able to distinguish the origin of the used markers.
However, a genetic map may contain markers from several marker sets.
For those reasons, a genetic map is not connected directly to a marker set, but the relationship is maintained trough the markers and their position in the genetic map.

To be able to use the genetic marker in a genomic context, the sequence of the markers must be mapped to a reference. 
The assembly may be in a single pseudomolecule molecule or in separated scaffolds. 
If the assembly is fragmented, the  map can be used to anchor the scaffolds to a genetic position. 
If the assembly consist on long scaffolds the genetic map and the positions can be used to find if the lines used for the map have a rearrangement event. 
If a rearrangement is present, the collinearity between the genomic and genetic reference is not conserved. 
Having the markers and scaffolds in the database simplify this kind of analysis, as all the needed data is readily available. 

Furthermore, PolyMarker (Chapter \ref{cha:polymarker}) has been used to design KASP assays for most of the primers in the 90k \citep{Wang2014} and 820k \citet{Winfield2016} SNP arrays. 
The primers for the assays can be integrated in the database. 
\unsure{Do we have any paper on the lab where this approach is used?}
This allow a use case were known flanking markers are queried and the database can return a list of possible markers with the primers already designed to be validated on a mapping population. 

\section{Integration of different genome references. }

The efforts to produce a wheat reference genome had been focused on the \gls{cs} variety. 
\acrshort{cs} is only cultivated as a research line, as it is susceptible to several pathogens and its yield is inferior to modern varieties.  
The reason for \acrshort{cs} to be the selected cultivar to be sequenced as reference is historic: it has long been a variety  for research.
\acrshort{cs} was originally chosen because it was able to cross with rye. 
It has also been used to produce lines with chromosomic aberrations, useful to find if any particular chromosome is responsible for certain traits \citep{Sears1985}. 

New assemblies are required to address the shortcomings of the use of \acrshort{cs} as a genome reference and to include the diversity from other lines \citep{Allen2016,Winfield2016}. 
The assemblies may include their own annotation, and that should be reflected in the database too. 

As of september 2016 there are four sets of genomic sequence used by the wheat community:
\begin{enumerate}
	\item  A 454 whole genome shotgun sequence that is unassembled, and each read is treated as a scaffold \citep{Brenchley2012}.
	\item The genome assembly and annotation from the \gls{css} done by the \acrshort{iwgsc} \citep{Mayer2014}.
	\item A whole genome shotgun sequence from a syntetic wheat, without a corresponding annotation \citep{Chapman2015}.
	\item The whole genome shotgun sequencing and annotation from \acrshort{cs} (TGACv1; \citealt{Clark2016}).
\end{enumerate}

All those references can be aligned to each other to find corresponding region. 
The corresponding regions can be stored in the scaffold mapping table. 
Furthermore, the scaffolds can be mapped to relative species, such as \textit{Brachypodium distachyon}, \textit{Oryza sativa}, \textit{Sorghum bicolor}, and \textit{Hordeum vulgare} to find synthenic blocks. 

As each genome assemblies are usually annotated with their genes and other features. 
To include the annotation, each scaffold can contain several features. 
As some features consist on sub-features, like genes conformed by several exons, a recursive relationship is included in the features tables. 
With the support of the scaffold mapping, the different annotations can be projected on different references. 
Gene models, like the ones described in \citep{Krasileva2013}, and  genetic markers can be aligned to any of the reference genomes.

With all those relationships available, the \textit{In silico} mapping in Chapter \ref{yr15} could be produced on several references at once. 
Also, the relationship between different gene models could be simplified, as the corresponding features will share coordinates.
Likewise, the co-expression of genes located in the same region could be an useful feature for expVIP (Chapter \ref{cha:exp}). 

Figure \ref{fig:discussion:assemblyTables} include the tables used to store assemblies, the relationships between them and their annotation. 

\begin{figure}
\includegraphics[width=1\textwidth]{Conclusions/Figures/assemblyTables.pdf}
\caption{Tables to store information about genome assemblies and their annotation.}
\label{fig:discussion:assemblyTables}
\end{figure}

\section{Integration with other services}
Currently, the publicly available wheat resources are scattered as supplemental materials on their corresponding publication or available as ad hoc systems focused on a particular field.
For example ensembl has two different views for every organism: from the genomic point of view and from the expression data. 
The Collaborative Open Plant Omics (COPO; \citealt{Davey2015}) is trying to integrate different sources and types of data by connecting the data providers. 
This approach requires the cooperation of the service providers, which have their own view of what is important. 
I believe that in order to effectively integrate the resources it is necessary to understand how the users are likely to interact with the data.


\section{Side projects}

PolyInDel / . 

\section{Final remarks}

In order to produce bioinformatic software that is powerful and usable it is required an understanding of both: the biological processes to solve and; the computational methods and software development practices.


%\newacronym{rnaseq}{RNA-Seq}{RNA Sequencing, a technique to only sequence trh transcriptime}
\newacronym{bfr}{BFR}{Bulk Frequency Ratio}
\newacronym{bsa}{BSA}{Bulk Segregant Analysis}
\newacronym{snp}{SNP}{Single Nucleotide Polymorphism}
\newacronym{cdna}{cDNA}{coding deoxyribonucleic acid}
\newacronym{cna}{DNA}{deoxyribonucleic acid}
\newacronym{nil}{NIL}{Near Isogenic Line}
\newacronym{avs}{AVS}{Avocet S}
\newacronym{yr15}{\textit{Yr15}}{Avocet + \textit{Yr15}}
\newacronym{ucw}{UCW}{University of California Wheat}
\newacronym{iuapc}{IUPAC}{International Union of Pure and Applied Chemistry}
\newacronym{bc}{BC}{Back-cross}
\newacronym{css}{CSS}{Chinese Spring Chromosome arm survey sequence}
\newacronym{ngs}{NGS}{Next Generation Sequencing}
\newacronym{qtl}{QTL}{Quantitative Trait Locus}
\newacronym{ssr}{SSR}{Simple Sequence Repeat}
\newacronym{mas}{MAS}{Marker Assisted Selection}
\newacronym{iwgsc}{IWGSC}{International Wheat Genome Sequencing 
Consortium}
\newacronym{est}{EST}{Encoding Sequence Tag}
\newacronym{dh}{DH}{Double Haploid}
\newacronym{pcr}{PCR}{Polymerase Chain Reaction}
\newacronym{indels}{indels}{insertions and deletions}

\appendix
\chapter{Supplemental tables}
%!TEX root = ../Main.tex

\section{Genetic map of \textit{Yr15} with RNA-Seq supplemental tables.}

\input{Yr15/SupplementalTables/alignedCoverage}


%\begin{tabular}{rlrll|cc|cc|ccc|cc}
\toprule
          &              &  &  & & \multicolumn{2}{c}{Cluster I isolates}        & \multicolumn{2}{c}{Cluster II isolates}        & \multicolumn{3}{c}{Cluster III isolates}         & \multicolumn{2}{c}{Cluster IV isolates}        \\
Assay & Contig       & Position &  X &  Y & 13/26              & 13/123 & CL1                 & T-13/3 & 13/09                & 13/23 & 13/182 & 13/36               & 13/40 \\
 \midrule
  1  & PST130\_14470 & 268      & C        & T        & X:Y                & X:Y    & X:X                 & X:X    & X:X                  & X:X   & X:X    & X:X                 & X:X   \\
  2  & PST130\_8160  & 11876    & C        & T        & Y:Y                & Y:Y    & X:Y                 & X:Y    & X:Y                  & X:Y   & X:Y    & X:Y                 & X:Y   \\
  3  & PST130\_14628 & 1712     & A        & C        & X:Y                & -      & X:X                 & X:X    & X:X                  & X:X   & X:X    & X:X                 & X:X   \\
  4  & PST130\_14898 & 503      & G        & A        & X:X                & X:X    & X:Y                 & X:Y    & X:Y                  & X:Y   & -      & X:Y                 & X:Y   \\
  5  & PST130\_28344 & 2372     & A        & G        & Y:Y                & Y:Y    & X:Y                 & X:Y    & Y:Y                  & Y:Y   & Y:Y    & Y:Y                 & Y:Y   \\
  6  & PST130\_7634  & 3463     & A        & C        & Y:Y                & Y:Y    & X:Y                 & X:Y    & Y:Y                  & Y:Y   & Y:Y    & Y:Y                 & Y:Y   \\
  7  & PST130\_7629  & 11699    & G        & A        & Y:Y                & Y:Y    & X:Y                 & X:Y    & Y:Y                  & Y:Y   & Y:Y    & Y:Y                 & Y:Y   \\
  8  & PST130\_10943 & 2979     & C        & T        & X:Y                & X:Y    & X:Y                 & X:Y    & X:X                  & X:X   & X:X    & X:Y                 & X:Y   \\
  9  & PST130\_10126 & 6216     & G        & T        & Y:Y                & Y:Y    & X:X                 & X:X    & X:X                  & X:X   & -      & Y:Y                 & Y:Y   \\
  10 & PST130\_22010 & 172      & C        & T        & Y:Y                & Y:Y    & Y:Y                 & Y:Y    & X:Y                  & X:Y   & -      & X:Y                 & X:Y   \\
  11 & PST130\_16961 & 1098     & C        & T        & X:X                & X:X    & X:Y                 & X:Y    & Y:Y                  & Y:Y   & Y:Y    & X:Y                 & X:Y   \\
  12 & PST130\_6915  & 2710     & A        & T        & Y:Y                & Y:Y    & Y:Y                 & Y:Y    & Y:Y                  & X:Y   & X:Y    & Y:Y                 & Y:Y   \\
  13 & PST130\_12479 & 1428     & C        & T        & X:X                & X:X    & Y:Y                 & Y:Y    & X:X                  & X:X   & X:X    & Y:Y                 & X:X   \\
  14 & PST130\_7634  & 3883     & C        & G        & X:X                & X:X    & X:Y                 & X:Y    & X:X                  & X:X   & X:Y    & X:Y                 & X:X   \\
  15 & PST130\_14470 & 456      & T        & C        & Y:Y                & Y:Y    & X:Y                 & X:Y    & Y:Y                  & Y:Y   & X:Y    & Y:Y                 & Y:Y   \\
\bottomrule
\end{tabular}


%\begin{sidewaystable}
\centering
\caption[Validation of homozygous deletions on line Cadenza0423.]{Validation of homozygous deletions on line Cadenza0423. X represents a mutant call, Y an wild type call and H an heterozygous call. In the case of the deletions, heterozygous calls are expected in the control lines (C), as two genomes are amplifying. }
\label{app:poly:homDelCad0423}
\begin{localsize}{6}{7}
\begin{tabular}{llllrllllllllllllllll}
\toprule
 Marker                                   & Deletion           & chr   &     cM & 1   & 2   & 3   & 4   & 5   & 6   & 7   & 8   & 9   & 10   & 11   & 12   & C   & C   & C   & C   & Result       \\
\midrule
 5BS\_2297308\_Cadenza0423\_12664\_C12664T    & -          & 5B    &  4.551 &  X   & X   & -   & X   & X   & X   & X   & X   & X   & X    & -    & X    & Y   & Y   & Y   & Y   & HOM Mutation \\
 5BL\_10812849\_Cadenza0423\_5664\_G5664T     & -          & 5B    & 38.769 &  X   & X   & -   & X   & X   & X   & X   & X   & X   & X    & -    & X    & Y   & Y   & Y   & Y   & HOM Mutation \\
 5BL\_10825062\_Cadenza0423\_7917\_G7917A     & -          & 5B    & 38.769 &  X   & X   & -   & X   & X   & X   & X   & X   & X   & X    & -    & X    & Y   & Y   & Y   & Y   & HOM Mutation \\
 IWGSC\_CSS\_5BL\_scaff\_10847976:27068-27231 & +          & 5B    & 38.769 &  X   & X   & -   & X   & X   & X   & X   & X   & X   & X    & -    & X    & H   & H   & H   & H   & Hom Deletion \\
 IWGSC\_CSS\_5BL\_scaff\_10847976:28118-28674 & +          & 5B    & 38.769 &  X   & X   & -   & X   & X   & X   & X   & X   & X   & X    & -    & X    & H   & H   & H   & H   & Hom Deletion \\
 IWGSC\_CSS\_5BL\_scaff\_10865441:15863-15946 & +          & 5B    & 38.769 &  X   & X   & -   & X   & X   & X   & X   & X   & X   & X    & -    & X    & H   & H   & H   & H   & Hom Deletion \\
 5BL\_10837222\_Cadenza0423\_4616\_G4616A     & -          & 5B    & 39.905 &  X   & X   & -   & X   & X   & X   & X   & X   & X   & X    & -    & X    & Y   & Y   & Y   & Y   & HOM Mutation \\
 5BL\_10891320\_Cadenza0423\_18847\_C18847T   & -          & 5B    & 45.594 &  Y   & Y   & -   & Y   & H   & X   & X   & Y   & H   & Y    & -    & H    & Y   & Y   & Y   & Y   & HET Mutation \\
\bottomrule
\end{tabular}
\end{localsize}
\end{sidewaystable}


%\begin{landscape}
%\begin{localsize}{6}{7}
%\begin{longtable}{llrlllllll}
\caption{Validation of mutations on $M_{4}$ on Cadenza}\\
\label{app:PolyMarkerM4ValidationCadenza}\\
\toprule
 IWGSC contig                 & Line       &   Pos & WT   & Mut   & Predicted   & $M_{4}$      & Primer 1 (Cadenza)        & Primer 2 (mutant)         & Common Primer             \\
\midrule
\endfirsthead
\toprule
 IWGSC contig                 & Line       &   Pos & WT   & Mut   & Predicted   & $M_{4}$      & Primer 1 (Cadenza)        & Primer 2 (mutant)         & Common Primer             \\
\midrule
\endhead
\bottomrule
\endfoot
\bottomrule
\endlastfoot
 IWGSC\_CSS\_3B\_scaff\_10445294  & Cadenza1772 &       6019 & C         & T        & het            & het         & caggatAgtGggactgtcaaaG    & caggatAgtGggactgtcaaaA    & ggagacGGctGtggacatT       \\
 IWGSC\_CSS\_3DL\_scaff\_6955403  & Cadenza1772 &       2418 & C         & T        & het*           & hom         & tcagCggattgtcgggatG       & tcagCggattgtcgggatA       & tgtcCatgaaTcttgtccacG     \\
 IWGSC\_CSS\_4AL\_scaff\_7106846  & Cadenza1772 &      11277 & G         & A        & hom            & hom         & tgggatccatgcctacactG      & tgggatccatgcctacactA      & gatggtGgatttgccgctA       \\
 IWGSC\_CSS\_4AS\_scaff\_5991335  & Cadenza1772 &      15710 & G         & A        & hom            & hom         & ctggccctgcgctgctaC        & ctggccctgcgctgctaT        & gtggaaGttcagaaggaccaG     \\
 IWGSC\_CSS\_4BS\_scaff\_4956646  & Cadenza1772 &        252 & G         & A        & het*           & hom         & gcaggttgacttcccggaG       & gcaggttgacttcccggaA       & tGaggtacgaGcTaaagAaagC    \\
 IWGSC\_CSS\_4DS\_scaff\_1715962  & Cadenza1772 &       1225 & G         & A        & hom            & hom         & cagctgtggTatctcaactgG     & cagctgtggTatctcaactgA     & CcCtGaaACACcGtttggaT      \\
 IWGSC\_CSS\_5AL\_scaff\_2763407  & Cadenza1772 &       2119 & G         & A        & hom            & hom         & gcgacGaacctcgagatctG      & gcgacGaacctcgagatctA      & gaTggcaAtcgtCgtgcA        \\
 IWGSC\_CSS\_5AS\_scaff\_1548786  & Cadenza1772 &      12625 & C         & T        & het            & het         & AtaggcacattgctagactgaG    & AtaggcacattgctagactgaA    & ggattgggtgttgcacgC        \\
 IWGSC\_CSS\_5BL\_scaff\_10849226 & Cadenza1772 &       2289 & C         & T        & het*           & hom         & cctgacatcattgttcacgatC    & cctgacatcattgttcacgatT    & cactccgaggtgtccatgaT      \\
 IWGSC\_CSS\_5BS\_scaff\_2270737  & Cadenza1772 &       2262 & G         & A        & hom            & ---         & attcCTgtgttgtggCaaatgaG   & attcCTgtgttgtggCaaatgaA   & taaGcacaaAccctccagctgG    \\
 IWGSC\_CSS\_1AL\_scaff\_3022915  & Cadenza1661 &        891 & C         & T        & hom            & hom         & ccacagtgagactcctattgaCG   & ccacagtgagactcctattgaCA   & atgtctgattcGtcGtagtcC     \\
 IWGSC\_CSS\_1AS\_scaff\_3297240  & Cadenza1661 &       1970 & C         & T        & het            & het         & catcccgccGtttcctcC        & catcccgccGtttcctcT        & gctcgccgatgaagagcT        \\
 IWGSC\_CSS\_1BL\_scaff\_3828996  & Cadenza1661 &       1340 & G         & A        & hom            & hom         & agccggatgttagtgttaacC     & agccggatgttagtgttaacT     & agcagcttgTcgcgttaaC       \\
 IWGSC\_CSS\_1DS\_scaff\_1884529  & Cadenza1661 &      10575 & G         & A        & hom            & hom         & aCagatacaAttgtcatgcaggC   & aCagatacaAttgtcatgcaggT   & acctgggTTgtccaatacttC     \\
 IWGSC\_CSS\_2AL\_scaff\_6318370  & Cadenza1661 &      19142 & C         & T        & het            & ---         & cgtggcCgaatCtcGacG        & cgtggcCgaatCtcGacA        & ttcttgtgggagccgggC        \\
 IWGSC\_CSS\_2AS\_scaff\_5213460  & Cadenza1661 &       1358 & G         & A        & hom            & hom         & gtcacgaaCccgctcagG        & gtcacgaaCccgctcagA        & aggaaagagaggaaaagaGcG     \\
 IWGSC\_CSS\_2BS\_scaff\_5179331  & Cadenza1661 &       5604 & G         & A        & het            & het         & actctcgtcaagaactgatacaG   & actctcgtcaagaactgatacaA   & gcaGagaatgttcttgcaacT     \\
 IWGSC\_CSS\_2DS\_scaff\_5341235  & Cadenza1661 &       4673 & G         & A        & het            & het         & ggtgaggatctcggagctG       & ggtgaggatctcggagctA       & gcgcggtcgtacgagttG        \\
 IWGSC\_CSS\_3AL\_scaff\_4250995  & Cadenza1661 &       7046 & G         & A        & hom            & hom         & cCaagaaacgggtggtccaG      & cCaagaaacgggtggtccaA      & ctgcagctgtcccatcatcgT     \\
 IWGSC\_CSS\_3B\_scaff\_10404421  & Cadenza1661 &       4303 & G         & A        & het            & het         & ccttcgtcgaCaggacctG       & ccttcgtcgaCaggacctA       & GCcagtactCacAtgctctC      \\
 IWGSC\_CSS\_5DL\_scaff\_2390496  & Cadenza1538 &       2125 & C         & T        & hom            & het         & gcagttttatcctcagtagtcttgG & gcagttttatcctcagtagtcttgA & ttctgagaaTgtaatgtgcGatG   \\
 IWGSC\_CSS\_6AL\_scaff\_5753680  & Cadenza1538 &       3920 & C         & T        & hom            & hom         & tgctccaaatttgagcacaaTaaC  & tgctccaaatttgagcacaaTaaT  & aaatgcaaggggtaagtttttgT   \\
 IWGSC\_CSS\_6AS\_scaff\_4425792  & Cadenza1538 &       4307 & G         & A        & hom            & het         & agatgcttgtCggGccaG        & agatgcttgtCggGccaA        & gctgaagcaacgcgatcaaT      \\
 IWGSC\_CSS\_6BS\_scaff\_3003630  & Cadenza1538 &       6933 & C         & T        & het            & het         & ggcagtaatgtggtgctgagC     & ggcagtaatgtggtgctgagT     & tTgaCttctggtttggtggcA     \\
 IWGSC\_CSS\_6DL\_scaff\_3246988  & Cadenza1538 &       9186 & G         & A        & het            & het         & gctaaagaagagcttgagagaattC & gctaaagaagagcttgagagaattT & aatttctgaagagaggtgttgtatG \\
 IWGSC\_CSS\_7AL\_scaff\_4480114  & Cadenza1538 &       3446 & C         & T        & het            & ---         & gatatctcccacacggcgG       & gatatctcccacacggcgA       & tgagccactcttgcagtttT      \\
 IWGSC\_CSS\_7AS\_scaff\_4193541  & Cadenza1538 &       8359 & C         & T        & hom            & het         & agcaattctttggctatcaattagC & agcaattctttggctatcaattagT & tcatctGtcttaactctactgctG  \\
 IWGSC\_CSS\_7BL\_scaff\_6721572  & Cadenza1538 &       9223 & C         & T        & het            & het         & gctCagggaggaagacaagaaG    & gctCagggaggaagacaagaaA    & tgctatgaagaattccgacctC    \\
 IWGSC\_CSS\_7BS\_scaff\_3152545  & Cadenza1538 &       3960 & G         & A        & hom            & ---         & tcagcaaaatcacctgcCgC      & tcagcaaaatcacctgcCgT      & gCtgccccatcatcgtttaT      \\
 IWGSC\_CSS\_7DS\_scaff\_3963838  & Cadenza1538 &       2913 & G         & A        & het            & het         & tCgttgcaagcCttTtgtgC      & tCgttgcaagcCttTtgtgT      & agaGttaTcaagcTactgtcacA   \\
 IWGSC\_CSS\_1AL\_scaff\_3903380  & Cadenza1469 &       6193 & G         & A        & hom            & hom         & ctcttcAgagatgaacgcgG      & ctcttcAgagatgaacgcgA      & tcGtGagatgGtggtttGTtA     \\
 IWGSC\_CSS\_1AS\_scaff\_3287728  & Cadenza1469 &       3817 & C         & T        & het*           & hom         & ccgaccaAttcactaaccgG      & ccgaccaAttcactaaccgA      & accctctttcccAgacatgaT     \\
 IWGSC\_CSS\_1BL\_scaff\_3815304  & Cadenza1469 &        513 & G         & A        & hom            & hom         & aacatttgcctTaCcaaaacGC    & aacatttgcctTaCcaaaacGT    & acacagcaagttataatgCAAgC   \\
 IWGSC\_CSS\_1DL\_scaff\_2266648  & Cadenza1469 &       5926 & C         & T        & het            & het         & caacatgagacacaacaccttC    & caacatgagacacaacaccttT    & gtcaacgcgtgaggattgtC      \\
 IWGSC\_CSS\_1DS\_scaff\_1906671  & Cadenza1469 &       3697 & C         & T        & hom            & hom         & tggTGtagacacttggcgaG      & tggTGtagacacttggcgaA      & catggcgaccaccAcctG        \\
 IWGSC\_CSS\_2AL\_scaff\_6337088  & Cadenza1469 &       7334 & G         & A        & het*           & hom         & acaatgccAagttgacaggttG    & acaatgccAagttgacaggttA    & gggagtgttggttCagaacaT     \\
 IWGSC\_CSS\_2BL\_scaff\_7972799  & Cadenza1469 &       8995 & C         & T        & het            & hom         & gTgCtcctcGgcatccttC       & gTgCtcctcGgcatccttT       & gatccgGgcaaactacgTG       \\
 IWGSC\_CSS\_2DL\_scaff\_9832343  & Cadenza1469 &       3262 & G         & A        & het            & het         & TtgtctaAcagcacCGcagG      & TtgtctaAcagcacCGcagA      & agatctcggtcagcctttcT      \\
 IWGSC\_CSS\_2DS\_scaff\_5327939  & Cadenza1469 &       3889 & G         & A        & het            & het         & ttttTgccttatgtgactctagtaC & ttttTgccttatgtgactctagtaT & gaggccatcacagatagcG       \\
 IWGSC\_CSS\_3B\_scaff\_10395219  & Cadenza1469 &       1292 & G         & A        & hom            & ---         & aggtgcttgtgcttgctgG       & aggtgcttgtgcttgctgA       & cctcttctgggggctttataC     \\
 IWGSC\_CSS\_3B\_scaff\_10592217  & Cadenza0580 &       2994 & C         & T        & het            & ---         & acagcagtatcaagcccctC      & acagcagtatcaagcccctT      & tgatactgttgTggCggagG      \\
 IWGSC\_CSS\_3DS\_scaff\_2596771  & Cadenza0580 &       1037 & G         & A        & het            & het         & tggttatgCAcaggataatCagG   & tggttatgCAcaggataatCagA   & tggcaaatgtgatgtcattaggT   \\
 IWGSC\_CSS\_4AL\_scaff\_7093953  & Cadenza0580 &       9881 & C         & T        & hom            & hom         & GacaggaagccggtaacaC       & GacaggaagccggtaacaT       & ctccAgcaggcatgggaT        \\
 IWGSC\_CSS\_4BL\_scaff\_7037448  & Cadenza0580 &       1837 & C         & T        & hom            & hom         & CgttgaaaaGctgcaagaacttaaC & CgttgaaaaGctgcaagaacttaaT & cagttcttccTtCaGagcagataT  \\
 IWGSC\_CSS\_4BS\_scaff\_4929479  & Cadenza0580 &      10668 & G         & A        & hom            & ---         & tggattttcccgcactgttC      & tggattttcccgcactgttT      & gtaaacaaggcatttcaagagtcA  \\
 IWGSC\_CSS\_4DL\_scaff\_14359838 & Cadenza0580 &       1408 & G         & A        & hom            & ---         & gCtcAttcagggatTGTcCtaTatG & gCtcAttcagggatTGTcCtaTatA & tgaCagaacagttggtcatacT    \\
 IWGSC\_CSS\_4DS\_scaff\_2276484  & Cadenza0580 &       8034 & G         & A        & hom            & hom         & gccgtggttgatggAgaG        & gccgtggttgatggAgaA        & cgtccagattactgatacttgcA   \\
 IWGSC\_CSS\_5AL\_scaff\_2756579  & Cadenza0580 &       5278 & G         & A        & het            & het         & tgaatggatttttcgtcccgttC   & tgaatggatttttcgtcccgttT   & ggAAtCCTATgCAgaAgAaaCTG   \\
 IWGSC\_CSS\_5BL\_scaff\_10787208 & Cadenza0580 &      10627 & G         & A        & het            & ---         & gcctctcacatgcggagaC       & gcctctcacatgcggagaT       & acgatgtcAggtggGcgT        \\
 IWGSC\_CSS\_5BS\_scaff\_2282179  & Cadenza0580 &       5267 & G         & A        & het            & ---         & tgatgggctacgacgtgC        & tgatgggctacgacgtgT        & tcggcgcccttgaaAtcC        \\
 IWGSC\_CSS\_5DL\_scaff\_4498073  & Cadenza0423 &       4937 & C         & T        & hom            & hom         & gcaccctctggttggtcatC      & gcaccctctggttggtcatT      & tgagcagcaAagcagccG        \\
 IWGSC\_CSS\_5DS\_scaff\_2738970  & Cadenza0423 &       2319 & C         & T        & het            & ---         & cgtgaggtgggtgatttgC       & cgtgaggtgggtgatttgT       & tggaactagttacactgcagtTC   \\
 IWGSC\_CSS\_6AL\_scaff\_5757109  & Cadenza0423 &       2788 & G         & A        & hom            & hom         & caggaGcctggcaaataaaGG     & caggaGcctggcaaataaaGA     & ctttcGcagtctcttagtttcG    \\
 IWGSC\_CSS\_6AS\_scaff\_4387871  & Cadenza0423 &       2543 & G         & A        & hom            & hom         & gcatgctaacaggcgaaaagG     & gcatgctaacaggcgaaaagA     & ctcatgctcctgatcttaaggtT   \\
 IWGSC\_CSS\_6BL\_scaff\_4271391  & Cadenza0423 &       4660 & C         & T        & hom            & hom         & tacgtgcatgatgtggtagtcgtaC & tacgtgcatgatgtggtagtcgtaT & gtttgaagtgcatcagatgTaccA  \\
 IWGSC\_CSS\_6DS\_scaff\_1880206  & Cadenza0423 &       9159 & G         & A        & het            & het         & ctgCgaaggctccacaaG        & ctgCgaaggctccacaaA        & ggatgagaagtttgcattgctC    \\
 IWGSC\_CSS\_7AS\_scaff\_4227506  & Cadenza0423 &        952 & G         & A        & het            & ---         & ccatgtgtttccaatgttagagC   & ccatgtgtttccaatgttagagT   & tgccctagctggtatgcT        \\
 IWGSC\_CSS\_7BL\_scaff\_6681782  & Cadenza0423 &       1486 & C         & T        & hom            & hom         & agtaagCGtgacagcaatggG     & agtaagCGtgacagcaatggA     & AtgtctTtgGtggaagtacatcA   \\
 IWGSC\_CSS\_7BS\_scaff\_3160328  & Cadenza0423 &       7801 & C         & T        & het            & het         & tgttaaatGatacagCctgcagC   & tgttaaatGatacagCctgcagT   & tggaatggtgCgttgttttT      \\
 IWGSC\_CSS\_7DS\_scaff\_407428   & Cadenza0423 &       2051 & G         & A        & het            & het         & gtcGCgccatcctgacaG        & gtcGCgccatcctgacaA        & actcatcAggtcagcccaA       \\
 IWGSC\_CSS\_3AL\_scaff\_442479   & Cadenza0364 &       3198 & C         & T        & het            & het         & gagtcaTtaagttggtaagattggC & gagtcaTtaagttggtaagattggT & GCaGaTaaCaacaggatcacG     \\
 IWGSC\_CSS\_3AL\_scaff\_4447942  & Cadenza0364 &      11917 & G         & A        & het            & het         & gtcataaagattgctcctgtgaaG  & gtcataaagattgctcctgtgaaA  & ctcGgatgtgggaggaagA       \\
 IWGSC\_CSS\_3AS\_scaff\_1557483  & Cadenza0364 &       2547 & C         & T        & het            & het         & aaagtcacatcatgcttaccataaG & aaagtcacatcatgcttaccataaA & cgaaatccaacgcctcatcA      \\
 IWGSC\_CSS\_3AS\_scaff\_2648747  & Cadenza0364 &       2688 & G         & A        & het            & het         & tggAagcAcaaggggccC        & tggAagcAcaaggggccT        & GccgccgatggagactcG        \\
 IWGSC\_CSS\_3AS\_scaff\_3304956  & Cadenza0364 &       1017 & G         & A        & het            & het         & gtcccttgcacacagctttG      & gtcccttgcacacagctttA      & cctgctggactacaacttcaaT    \\
 IWGSC\_CSS\_3AS\_scaff\_3321091  & Cadenza0364 &       4585 & C         & T        & het            & het         & caagaatgATgctgatgttggaG   & caagaatgATgctgatgttggaA   & acatgctgaatcgccgaatC      \\
 IWGSC\_CSS\_3AS\_scaff\_3371333  & Cadenza0364 &        538 & G         & A        & het            & het         & gggaaaCgAgAcgagcgG        & gggaaaCgAgAcgagcgA        & ccgtgccttcctcacccT        \\
 IWGSC\_CSS\_3AS\_scaff\_3371815  & Cadenza0364 &       1061 & C         & T        & het            & het         & atccccacggcacagagG        & atccccacggcacagagA        & aAttggcccttggtgattcC      \\
 IWGSC\_CSS\_3AS\_scaff\_3440912  & Cadenza0364 &       4498 & G         & A        & het            & het         & ccgtaaaactttctgtgcttgC    & ccgtaaaactttctgtgcttgT    & atActgacaaactacatgatgtgC  \\
 IWGSC\_CSS\_3B\_scaff\_10343586  & Cadenza0364 &       2242 & G         & A        & het            & ---         & ggttcTgTcctctcttccactG    & ggttcTgTcctctcttccactA    & tgtgttgaacccgcaagcA       \\
IWGSC\_CSS\_3AL\_scaff\_442479   & Cadenza0364 &       3198 & C         & T        & het            & het         & gagtcaTtaagttggtaagattggC & gagtcaTtaagttggtaagattggT & GCaGaTaaCaacaggatcacG     \\
 IWGSC\_CSS\_3AL\_scaff\_4447942  & Cadenza0364 &      11917 & G         & A        & het            & het         & gtcataaagattgctcctgtgaaG  & gtcataaagattgctcctgtgaaA  & ctcGgatgtgggaggaagA       \\
 IWGSC\_CSS\_3AS\_scaff\_1557483  & Cadenza0364 &       2547 & C         & T        & het            & het         & aaagtcacatcatgcttaccataaG & aaagtcacatcatgcttaccataaA & cgaaatccaacgcctcatcA      \\
 IWGSC\_CSS\_3AS\_scaff\_2648747  & Cadenza0364 &       2688 & G         & A        & het            & het         & tggAagcAcaaggggccC        & tggAagcAcaaggggccT        & GccgccgatggagactcG        \\
 IWGSC\_CSS\_3AS\_scaff\_3304956  & Cadenza0364 &       1017 & G         & A        & het            & het         & gtcccttgcacacagctttG      & gtcccttgcacacagctttA      & cctgctggactacaacttcaaT    \\
 IWGSC\_CSS\_3AS\_scaff\_3321091  & Cadenza0364 &       4585 & C         & T        & het            & het         & caagaatgATgctgatgttggaG   & caagaatgATgctgatgttggaA   & acatgctgaatcgccgaatC      \\
 IWGSC\_CSS\_3AS\_scaff\_3371333  & Cadenza0364 &        538 & G         & A        & het            & het         & gggaaaCgAgAcgagcgG        & gggaaaCgAgAcgagcgA        & ccgtgccttcctcacccT        \\
 IWGSC\_CSS\_3AS\_scaff\_3371815  & Cadenza0364 &       1061 & C         & T        & het            & het         & atccccacggcacagagG        & atccccacggcacagagA        & aAttggcccttggtgattcC      \\
 IWGSC\_CSS\_3AS\_scaff\_3440912  & Cadenza0364 &       4498 & G         & A        & het            & het         & ccgtaaaactttctgtgcttgC    & ccgtaaaactttctgtgcttgT    & atActgacaaactacatgatgtgC  \\
 IWGSC\_CSS\_3B\_scaff\_10343586  & Cadenza0364 &       2242 & G         & A        & het            & ---         & ggttcTgTcctctcttccactG    & ggttcTgTcctctcttccactA    & tgtgttgaacccgcaagcA       \\
 IWGSC\_CSS\_5DL\_scaff\_242342   & Cadenza0281 &       2433 & C         & T        & hom            & hom         & catggCgacggtGtcctG        & catggCgacggtGtcctA        & aAccctcatTTtggCTACTtCT    \\
 IWGSC\_CSS\_5DL\_scaff\_4538822  & Cadenza0281 &       1208 & G         & A        & hom            & ---         & acgtcagaacaaccgtttgaC     & acgtcagaacaaccgtttgaT     & ttaaattggttggcgccacC      \\
 IWGSC\_CSS\_6AL\_scaff\_5813297  & Cadenza0281 &       4532 & C         & T        & hom            & ---         & gggagagggacgtctcgG        & gggagagggacgtctcgA        & ttcttctgccaacgattccG      \\
 IWGSC\_CSS\_6AS\_scaff\_4378990  & Cadenza0281 &       6748 & C         & T        & hom            & hom         & cccaggttctgcttcttttcC     & cccaggttctgcttcttttcT     & caagtatcaagaaaatgaagggTgT \\
 IWGSC\_CSS\_6BL\_scaff\_4360781  & Cadenza0281 &       5426 & C         & T        & het            & het         & aCtactcaaatggcttGgtgtaG   & aCtactcaaatggcttGgtgtaA   & tcagtccaacatgTcaagagatT   \\
 IWGSC\_CSS\_7AL\_scaff\_4488310  & Cadenza0281 &       3808 & G         & A        & hom            & hom         & gttctcttgtagtagcagccG     & gttctcttgtagtagcagccA     & ggcgctttcttcggcctA        \\
 IWGSC\_CSS\_7BL\_scaff\_6696509  & Cadenza0281 &       9232 & G         & A        & het            & het         & gctctaggGgtggcaaAagG      & gctctaggGgtggcaaAagA      & ggcttGaGgtcGcagtgT        \\
 IWGSC\_CSS\_7BS\_scaff\_3143575  & Cadenza0281 &       1866 & C         & T        & het            & het         & agatgttgagagggcgcttC      & agatgttgagagggcgcttT      & gcttggAtggtggcaagtT       \\
 IWGSC\_CSS\_7DL\_scaff\_3346250  & Cadenza0281 &       1663 & G         & A        & het            & het         & acgtgcagcaacatcctaaC      & acgtgcagcaacatcctaaT      & TttcccaccaggcccaagA       \\
 IWGSC\_CSS\_7DS\_scaff\_3933917  & Cadenza0281 &       1243 & C         & T        & het            & het         & tgCtgagcCttTcaccttgC      & tgCtgagcCttTcaccttgT      & agaggtttggttccatcGG       \\
 IWGSC\_CSS\_3B\_scaff\_10626860  & Cadenza0148 &       7847 & G         & A        & het            & het         & gcagctctgggaaggagG        & gcagctctgggaaggagA        & gttaatgtacCTcctagcctcG    \\
 IWGSC\_CSS\_3DL\_scaff\_6915683  & Cadenza0148 &       6904 & C         & T        & het            & het         & cgtcaaCctgtgggcaattG      & cgtcaaCctgtgggcaattA      & tcatgctcataatgTcatagggT   \\
 IWGSC\_CSS\_4AS\_scaff\_5929057  & Cadenza0148 &       4238 & G         & A        & hom            & hom         & gcgcaacgtagCacctacC       & gcgcaacgtagCacctacT       & ttatctggtgaagtgacaggttCA  \\
 IWGSC\_CSS\_4AS\_scaff\_5950625  & Cadenza0148 &      10590 & C         & T        & het            & het         & agaTattCaaaTcggtggAttggC  & agaTattCaaaTcggtggAttggT  & cctgCtcccctcacgtcC        \\
 IWGSC\_CSS\_4AS\_scaff\_5967119  & Cadenza0148 &      11626 & C         & T        & hom            & hom         & cgtGgacaccccgagctG        & cgtGgacaccccgagctA        & gacgacgcactgcacgaC        \\
 IWGSC\_CSS\_4DL\_scaff\_14455742 & Cadenza0148 &       1946 & C         & T        & hom            & hom         & gCctgagggagatcgcgC        & gCctgagggagatcgcgT        & aaccgGtAaCTGtGgGcA        \\
 IWGSC\_CSS\_4DS\_scaff\_2318993  & Cadenza0148 &       4000 & C         & T        & hom            & hom         & tccagtttgacacagattgaatggG & tccagtttgacacagattgaatggA & tgagaTtctgtttcctttcacAttG \\
 IWGSC\_CSS\_5AL\_scaff\_2750707  & Cadenza0148 &       4603 & G         & A        & het            & het         & ccttggtgctagccatttcaagTaG & ccttggtgctagccatttcaagTaA & ccaggaTgcAgtgcaatatttcaaG \\
 IWGSC\_CSS\_5BL\_scaff\_10794137 & Cadenza0148 &       9235 & C         & T        & hom            & hom         & gaagctgcttctgcgttG        & gaagctgcttctgcgttA        & agtatcccttccatataagcagtG  \\
 IWGSC\_CSS\_5BS\_scaff\_1646558  & Cadenza0148 &       2916 & C         & T        & het            & het         & gccGtacactcacctAtcctttG   & gccGtacactcacctAtcctttA   & gcaaTgtccacttAtcatcccT    \\
 IWGSC\_CSS\_1AL\_scaff\_3883106  & Cadenza0110 &      27536 & C         & T        & het            & het         & accttccatcactggctgG       & accttccatcactggctgA       & gtgaagaacaacaggttgaagC    \\
 IWGSC\_CSS\_1BL\_scaff\_3812829  & Cadenza0110 &      10770 & G         & A        & het*           & hom         & cccccactccattccagG        & cccccactccattccagA        & gGatgttgttctgtgctggaA     \\
 IWGSC\_CSS\_1DL\_scaff\_2266648  & Cadenza0110 &       6156 & G         & A        & het            & het         & actgcgtggttatgggacC       & actgcgtggttatgggacT       & ccccatcactgaacacaacA      \\
 IWGSC\_CSS\_1DS\_scaff\_1889435  & Cadenza0110 &       8826 & C         & T        & hom            & hom         & aaccatgaattactcggacagG    & aaccatgaattactcggacagA    & gccctgaagaattgtatcaaaacaG \\
 IWGSC\_CSS\_2AS\_scaff\_5268634  & Cadenza0110 &       4636 & G         & A        & het            & het         & gatccatgtgattggcatgtttG   & gatccatgtgattggcatgtttA   & TgctgtTggatatgcagttacT    \\
 IWGSC\_CSS\_2BL\_scaff\_7965110  & Cadenza0110 &      15801 & C         & T        & hom            & hom         & cattgaagcAtacacAattgcAtaC & cattgaagcAtacacAattgcAtaT & gccagagtatccagataaggTttA  \\
 IWGSC\_CSS\_2DL\_scaff\_9852812  & Cadenza0110 &      13788 & G         & A        & hom            & hom         & atttttgtatggtctcaatcttcgC & atttttgtatggtctcaatcttcgT & gaacgtTcattcttgtacttgcT   \\
 IWGSC\_CSS\_2DS\_scaff\_5371379  & Cadenza0110 &       2166 & C         & T        & hom            & hom         & agacacaaaactagtGatgcgC    & agacacaaaactagtGatgcgT    & gctgctgagaatgttTtgtatttG  \\
 IWGSC\_CSS\_3AL\_scaff\_4384278  & Cadenza0110 &       1276 & C         & T        & het            & het         & agcTgaactgccccTgtaG       & agcTgaactgccccTgtaA       & agggacctCgGtggatgaA       \\
 IWGSC\_CSS\_3AS\_scaff\_3340122  & Cadenza0110 &       1467 & C         & T        & hom            & hom         & attcctAgtgttgtcggaacatG   & attcctAgtgttgtcggaacatA   & gagaagactagaaagttttcAgcaT \\
 IWGSC\_CSS\_5DL\_scaff\_4554222  & Cadenza2103 &       6528 & C         & T        & het*           & hom         & gctgccctacaaagaaacaaaattG & gctgccctacaaagaaacaaaattA & aTcccaactatCGaTtttgtcataC \\
 IWGSC\_CSS\_6AL\_scaff\_5833640  & Cadenza2103 &       7346 & C         & T        & hom            & hom         & aagaaaagccacaatggtttctC   & aagaaaagccacaatggtttctT   & aCTctgTcagtgtttcccagC     \\
 IWGSC\_CSS\_6AS\_scaff\_4429974  & Cadenza2103 &       3867 & G         & A        & hom            & hom         & GagatgaAtttattgagcatgtggC & GagatgaAtttattgagcatgtggT & ggttccggctgcataagT        \\
 IWGSC\_CSS\_6DL\_scaff\_3307626  & Cadenza2103 &       4970 & C         & T        & hom            & hom         & tgcagatgttgtcctgtgtaG     & tgcagatgttgtcctgtgtaA     & ctaggaaggtgattttgtactGtC  \\
 IWGSC\_CSS\_6DS\_scaff\_2059604  & Cadenza2103 &       5224 & G         & A        & het            & ---         & gctcaatgcatgcTgagtgG      & gctcaatgcatgcTgagtgA      & tgtcaagtattattttcctgctctG \\
 IWGSC\_CSS\_7AL\_scaff\_4552322  & Cadenza2103 &       1412 & C         & T        & het            & het         & gcaaaggcTgatactccaacaG    & gcaaaggcTgatactccaacaA    & ggcAAGccAgtataaaagtaaGC   \\
 IWGSC\_CSS\_7BS\_scaff\_3147455  & Cadenza2103 &       4607 & G         & A        & het            & ---         & gcaccttaggatgtgagTtatgC   & gcaccttaggatgtgagTtatgT   & gcatgtagggtttatttgactgttA \\
 IWGSC\_CSS\_7DL\_scaff\_3382467  & Cadenza2103 &       3473 & C         & T        & hom            & ---         & GGTtctgCaGTTCATAActcatC   & GGTtctgCaGTTCATAActcatT   & attgaatcaactgatacGaaGactC \\
 IWGSC\_CSS\_3B\_scaff\_10457010  & Cadenza0277 &      10599 & G         & A        & het            & het         & aaccttggccgcagaacaC       & aaccttggccgcagaacaT       & actggctgcacgagaggG        \\
 IWGSC\_CSS\_3B\_scaff\_10593852  & Cadenza0277 &      10124 & C         & T        & het            & het         & tgacaggggacgctatacaG      & tgacaggggacgctatacaA      & gtctaaCTtACattAcccatcagC  \\
 IWGSC\_CSS\_3DS\_scaff\_2583390  & Cadenza0277 &        663 & G         & A        & hom            & hom         & actgcactcatacaatActtCtgC  & actgcactcatacaatActtCtgT  & tcCacctggacagcaagtG       \\
 IWGSC\_CSS\_4AL\_scaff\_7093953  & Cadenza0277 &      10004 & C         & T        & hom            & hom         & ccttgtattcaatggaTtgTtttgG & ccttgtattcaatggaTtgTtttgA & ttccccaaaTaaaaaggaagagC   \\
 IWGSC\_CSS\_4AL\_scaff\_7176064  & Cadenza0277 &       6220 & C         & T        & het            & het         & gtgccgtaTtcCgcctgG        & gtgccgtaTtcCgcctgA        & atgttcgaggggatgggG        \\
 IWGSC\_CSS\_4DL\_scaff\_14122349 & Cadenza0277 &       1010 & C         & T        & hom            & hom         & gtcgctgctgCttgtgaG        & gtcgctgctgCttgtgaA        & ggaacaggcccaaggagG        \\
 IWGSC\_CSS\_5AL\_scaff\_2736916  & Cadenza0277 &       4296 & G         & A        & het            & het         & aagaactATgAaaGtaacacacgaC & aagaactATgAaaGtaacacacgaT & ttcGcTttTaagGcAttCtcG     \\
 IWGSC\_CSS\_5BL\_scaff\_10883744 & Cadenza0277 &       2080 & C         & T        & hom            & hom         & gcctctttCtgttTagcctcaG    & gcctctttCtgttTagcctcaA    & cgacaaggttcgtgatTgcA      \\
 IWGSC\_CSS\_1AL\_scaff\_3932013  & Cadenza0548 &      11765 & C         & T        & hom            & hom         & accgccaaCccaagacaG        & accgccaaCccaagacaA        & cccattaGccgTgcAacG        \\
 IWGSC\_CSS\_1BS\_scaff\_3417505  & Cadenza0548 &        373 & C         & T        & het            & het         & gtggtgaggaGGgtgGaG        & gtggtgaggaGGgtgGaA        & tggtcgGccagttgttgA        \\
 IWGSC\_CSS\_2AS\_scaff\_5305619  & Cadenza0548 &       2786 & C         & T        & hom            & hom         & atacagatgccctAAgtggTtC    & atacagatgccctAAgtggTtT    & ggaagacaAtGctccaggtaC     \\
 IWGSC\_CSS\_2AS\_scaff\_5306489  & Cadenza0548 &      46953 & T         & G        & het            & wt          & aggttccatgtccatagaagGT    & aggttccatgtccatagaagGG    & aggctaTAgactcctgtACAgT    \\
 IWGSC\_CSS\_2BL\_scaff\_7984123  & Cadenza0548 &      11660 & G         & A        & het            & het         & cattgtggcatagtaatcagtacaG & cattgtggcatagtaatcagtacaA & aatacattgaggaatcaaagccC   \\
 IWGSC\_CSS\_2DL\_scaff\_9907477  & Cadenza0548 &       1363 & C         & T        & hom            & hom         & tgcctccctttgccagaaC       & tgcctccctttgccagaaT       & ggcaaacctgatgtggcatC      \\
 IWGSC\_CSS\_2DS\_scaff\_5330886  & Cadenza0548 &       5449 & G         & A        & hom            & hom         & gcatgtccatttatactgaaCgtG  & gcatgtccatttatactgaaCgtA  & catgctgcttcttctggacC      \\
 IWGSC\_CSS\_3AL\_scaff\_4449951  & Cadenza0548 &        633 & C         & T        & het            & het         & tccaaacctaacagtctaacactaG & tccaaacctaacagtctaacactaA & gtctgcagTGCaatgtgC        \\
 IWGSC\_CSS\_3B\_scaff\_10479889  & Cadenza0097 &       3339 & C         & T        & hom            & ---         & ttgTttctGgagaagatgcCG     & ttgTttctGgagaagatgcCA     & ggtgctcattcaAcGgcA        \\
 IWGSC\_CSS\_3B\_scaff\_10562262  & Cadenza0097 &       7819 & C         & T        & het            & het         & agaggggtgctatccatAttgG    & agaggggtgctatccatAttgA    & agcgatgccaaggcttcC        \\
 IWGSC\_CSS\_4AL\_scaff\_7040796  & Cadenza0097 &      10772 & G         & A        & hom            & hom         & acacaacattgccaccagaG      & acacaacattgccaccagaA      & CAatCgattgcttgctTctcC     \\
 IWGSC\_CSS\_4AL\_scaff\_7063488  & Cadenza0097 &       6360 & C         & T        & het            & het         & gcctctcacCttAatttgaagctgC & gcctctcacCttAatttgaagctgT & aggcagtggagtatgtgaagttT   \\
 IWGSC\_CSS\_4AL\_scaff\_7091701  & Cadenza0097 &       5050 & G         & A        & het            & het         & catgagcatctgggaggaaaatG   & catgagcatctgggaggaaaatA   & agcaagggaAtaatgaacggaaA   \\
 IWGSC\_CSS\_4DS\_scaff\_1845841  & Cadenza0097 &       7110 & G         & A        & hom            & hom         & aatgTAgctccccatacCgG      & aatgTAgctccccatacCgA      & actgaaacTgcaatcgtTtatggA  \\
 IWGSC\_CSS\_5AL\_scaff\_2767581  & Cadenza0097 &       3737 & G         & A        & het            & het         & gagaggtcctcactAtcggC      & gagaggtcctcactAtcggT      & cgTcatcacaaatattgctggG    \\
 IWGSC\_CSS\_5BL\_scaff\_10784643 & Cadenza0097 &       1568 & C         & T        & hom            & hom         & agaaaTAcatggatggatggaCG   & agaaaTAcatggatggatggaCA   & catctcCCttccaCgGaaaG      \\
 IWGSC\_CSS\_1AL\_scaff\_3952258  & Cadenza2092 &       8107 & C         & T        & het            & ---         & tgagtagagaaattgacagtgtgG  & tgagtagagaaattgacagtgtgA  & tgccaccattgacatgagaG      \\
 IWGSC\_CSS\_1BL\_scaff\_3858008  & Cadenza2092 &      10278 & G         & A        & hom            & hom         & tttgagcaggcaggatcgC       & tttgagcaggcaggatcgT       & actcacggcctatatcActattC   \\
 IWGSC\_CSS\_1DL\_scaff\_2265172  & Cadenza2092 &       9094 & C         & T        & hom            & hom         & tgcaTGTcatttgttcttatcagC  & tgcaTGTcatttgttcttatcagT  & agtgtccaacttccGttcatC     \\
 IWGSC\_CSS\_2AL\_scaff\_6435867  & Cadenza2092 &      16201 & G         & A        & hom            & hom         & tttctgTaccttaacgtcaattgaC & tttctgTaccttaacgtcaattgaT & gtgaggatgatgaggtaagacC    \\
 IWGSC\_CSS\_2AL\_scaff\_6439430  & Cadenza2092 &      25101 & C         & T        & het            & ---         & caagaaagggCagCtCagC       & caagaaagggCagCtCagT       & tcGttAcTctttcActggtgaA    \\
 IWGSC\_CSS\_2DL\_scaff\_9760848  & Cadenza2092 &       4733 & C         & T        & het            & het         & gcaccatgggtctcaggtaC      & gcaccatgggtctcaggtaT      & tcagtcagtttGCTCtgTCTG     \\
 IWGSC\_CSS\_3AL\_scaff\_4407012  & Cadenza2092 &       2785 & C         & T        & hom            & hom         & acatatAgtgttctcatccaccatC & acatatAgtgttctcatccaccatT & acctctctcatgttaataggtttgT \\
 IWGSC\_CSS\_3AS\_scaff\_3441108  & Cadenza2092 &        541 & G         & A        & het            & het         & GtgatgaccttgagacGgaG      & GtgatgaccttgagacGgaA      & aggcaTgacaaCgcgcaA        \\
 IWGSC\_CSS\_3B\_scaff\_10449827  & Cadenza1551 &       4779 & G         & A        & hom            & hom         & ggcaaggtcaagaaacGgtC      & ggcaaggtcaagaaacGgtT      & aCagaGtgggttagaggcaG      \\
 IWGSC\_CSS\_3B\_scaff\_10550638  & Cadenza1551 &       3250 & C         & T        & het            & het         & ctccttcacttgttgcggC       & ctccttcacttgttgcggT       & gcaacAtTttgatactgcaaagG   \\
 IWGSC\_CSS\_3DL\_scaff\_6945816  & Cadenza1551 &        589 & C         & T        & hom            & hom         & agcatctcacctgcaaCaataC    & agcatctcacctgcaaCaataT    & TgtgcccTctgaAtattttcaTG   \\
 IWGSC\_CSS\_3DL\_scaff\_6954177  & Cadenza1551 &       3508 & C         & T        & het            & het         & tgtagcatcacattaactttcctG  & tgtagcatcacattaactttcctA  & gcttggtataaaccCttacgacA   \\
 IWGSC\_CSS\_4AS\_scaff\_5938272  & Cadenza1551 &      19080 & G         & A        & hom            & hom         & agAcCccgAtcgccatgG        & agAcCccgAtcgccatgA        & GggAgatAcaggtaaaActcTtcG  \\
 IWGSC\_CSS\_4AS\_scaff\_5977594  & Cadenza1551 &      11092 & C         & T        & het            & het         & gccttgattcggaacaacaaaC    & gccttgattcggaacaacaaaT    & gcgtctctcagtcctgcA        \\
 IWGSC\_CSS\_5AL\_scaff\_2671035  & Cadenza1551 &       5859 & C         & T        & het            & het         & cggtgatattTttagacttcgacgC & cggtgatattTttagacttcgacgT & ggcagttcagcGacccatT       \\
 IWGSC\_CSS\_5BL\_scaff\_10889480 & Cadenza1551 &       2530 & G         & A        & hom            & hom         & gagcttaactcgcagatggaG     & gagcttaactcgcagatggaA     & tccatgCAacGccttggT        \\
 IWGSC\_CSS\_3B\_scaff\_10528396  & Cadenza2088 &       8059 & G         & A        & hom            & ---         & cttttccgtccgtaagcaataG    & cttttccgtccgtaagcaataA    & gtgcactgttcaggcctgA       \\
 IWGSC\_CSS\_3B\_scaff\_10637573  & Cadenza2088 &      16815 & G         & A        & het            & het         & agcaagcttaccGgtctgC       & agcaagcttaccGgtctgT       & cgagcAactacgagcagctT      \\
 IWGSC\_CSS\_4AL\_scaff\_7086469  & Cadenza2088 &       6697 & G         & A        & het            & het         & gccgtctacttcaacgcG        & gccgtctacttcaacgcA        & ccaGaggcttgtTGcattttT     \\
 IWGSC\_CSS\_4AL\_scaff\_7126302  & Cadenza2088 &       3627 & G         & A        & hom            & hom         & gttcaaaaacaagtggctAatttgC & gttcaaaaacaagtggctAatttgT & cacaaggatatgaagcTcttctagA \\
 IWGSC\_CSS\_4BL\_scaff\_7041808  & Cadenza2088 &      10234 & G         & A        & hom            & hom         & tcaatggatgagggtgcttC      & tcaatggatgagggtgcttT      & ccatagcagcatcagccacA      \\
 IWGSC\_CSS\_5AL\_scaff\_2794167  & Cadenza2088 &      13162 & G         & A        & het            & ---         & agtattcaggacaagcatCttCaG  & agtattcaggacaagcatCttCaA  & caatgaaacctctcgaagaaGaG   \\
 IWGSC\_CSS\_5BL\_scaff\_10889232 & Cadenza2088 &       3885 & G         & A        & het            & het         & cTcaaccacaatgggcaAatC     & cTcaaccacaatgggcaAatT     & tccttcatcaatcatcaattgttgG \\
 IWGSC\_CSS\_5BS\_scaff\_2267405  & Cadenza2088 &      11113 & C         & T        & hom            & hom         & ctttgatgatcctaggcctctTG   & ctttgatgatcctaggcctctTA   & tgatttggtCtggttAgagtttGA  \\
 IWGSC\_CSS\_3B\_scaff\_10475354  & Cadenza1409 &       2203 & G         & A        & hom            & hom         & agCgaacaagagGtcaaacG      & agCgaacaagagGtcaaacA      & ctgaaacacaCtagaCAattAccG  \\
 IWGSC\_CSS\_3B\_scaff\_10674115  & Cadenza1409 &       4555 & C         & T        & het            & het         & gcttcagtgcatgccttcaG      & gcttcagtgcatgccttcaA      & cttcacacccGagataatGtattG  \\
 IWGSC\_CSS\_4AL\_scaff\_7153568  & Cadenza1409 &      13073 & C         & T        & hom            & hom         & tccgaccgAtcaaccttgG       & tccgaccgAtcaaccttgA       & gaccggaactcctcggcC        \\
 IWGSC\_CSS\_4DL\_scaff\_14314966 & Cadenza1409 &       2010 & G         & A        & het            & hom         & gtaggtcccctcctCAggG       & gtaggtcccctcctCAggA       & cggcgTcacaAgttgCcT        \\
 IWGSC\_CSS\_4DS\_scaff\_2324074  & Cadenza1409 &       7606 & G         & A        & het            & het         & tGcatgaaaatgtgtGcaGaG     & tGcatgaaaatgtgtGcaGaA     & gggtaAgttcAaaactGaagtgaaG \\
 IWGSC\_CSS\_5AS\_scaff\_1517889  & Cadenza1409 &       3561 & G         & A        & het            & het         & tctcgacatcttcccgtgtaC     & tctcgacatcttcccgtgtaT     & gtgcctggaacattgcttatttA   \\
 IWGSC\_CSS\_5AS\_scaff\_1523866  & Cadenza1409 &       8054 & G         & A        & hom            & ---         & ggtgatctaccgccaGgaC       & ggtgatctaccgccaGgaT       & tcctgcagCcTctcctcA        \\
 IWGSC\_CSS\_5BL\_scaff\_10917655 & Cadenza1409 &      19073 & G         & A        & hom            & hom         & caaatgacatgcaaaagaagttgC  & caaatgacatgcaaaagaagttgT  & cgcttcatcactacaAaatatgtcT \\
 IWGSC\_CSS\_1AL\_scaff\_3886649  & Cadenza1599 &       5204 & C         & T        & het            & het         & tgatgccaaccacaatGcC       & tgatgccaaccacaatGcT       & ggactgactgctgaccatatttaG  \\
 IWGSC\_CSS\_1BL\_scaff\_3810267  & Cadenza1599 &       6634 & C         & T        & hom            & hom         & ccCaggaaatgagcacctC       & ccCaggaaatgagcacctT       & cgcaggcgaagatgtgaTtG      \\
 IWGSC\_CSS\_1DL\_scaff\_2291677  & Cadenza1599 &      12856 & C         & T        & hom            & hom         & GgtagacaagtcgccgaG        & GgtagacaagtcgccgaA        & cctcctccttcaacGCcG        \\
 IWGSC\_CSS\_2AL\_scaff\_6354492  & Cadenza1599 &       7566 & G         & A        & het            & het         & gGagaatgcaCAgtAacTtctgG   & gGagaatgcaCAgtAacTtctgA   & ttccgaagaaccacaTccTG      \\
 IWGSC\_CSS\_2AS\_scaff\_5282937  & Cadenza1599 &       9736 & G         & A        & het            & het         & gctgtagattttatagctgctatgC & gctgtagattttatagctgctatgT & cacCagaattgttCactgatttTC  \\
 IWGSC\_CSS\_2BL\_scaff\_7952427  & Cadenza1599 &      19249 & G         & A        & hom            & hom         & cgTccctCcctagcacgaC       & cgTccctCcctagcacgaT       & aTcactccattagcgcgAG       \\
 IWGSC\_CSS\_2DL\_scaff\_9897981  & Cadenza1599 &       5627 & C         & T        & het            & het         & cttggtgctTgattgcttactC    & cttggtgctTgattgcttactT    & gTttgctCtctctgatctTtgtG   \\
 IWGSC\_CSS\_3AL\_scaff\_4446105  & Cadenza1599 &       1765 & G         & A        & hom            & ---         & aaatgctttcctaCcgctagtG    & aaatgctttcctaCcgctagtA    & ttctAgaggcaatagctTatatgcT \\
\end{longtable}

%%!TEX root = ../../Main.tex

\begin{tabular}{llrlllllll}
\toprule
 IWGSC contig                 & Line       &   Pos & WT   & Mut   & Predicted   & Called on $M_{4}$    & Primer 1 (Kronos)        & Primer 2 (mutant)        & Common Primer            \\
\midrule
 IWGSC\_CSS\_1AS\_scaff\_3284790  & Kronos3085 &  7449 & G    & A     & Het    & Het   & ccacaccttgagcctcgC       & ccacaccttgagcctcgT       & gtgattttgccaggggagA      \\
 IWGSC\_CSS\_1BL\_scaff\_3897513  & Kronos3085 &  1515 & C    & T     & Het    & Het   & gcttccactGggtcctgC       & gcttccactGggtcctgT       & acAaggactgcttcagaGaC     \\
 IWGSC\_CSS\_2AL\_scaff\_6434745  & Kronos3085 &  3424 & C    & T     & Het    & Het   & cctcGgttttgcaaatttctatgC & cctcGgttttgcaaatttctatgT & gGCaaTggcataacaacagatA   \\
 IWGSC\_CSS\_3AS\_scaff\_3408995  & Kronos3085 &   732 & C    & T     & Het    & Het   & aggccatttcgaattccgC      & aggccatttcgaattccgT      & ggTgttaTccagAacctgagTG   \\
 IWGSC\_CSS\_3B\_scaff\_10708748  & Kronos3085 &  2675 & G    & A     & Het    & Het   & gttgcatgcttcacccagG      & gttgcatgcttcacccagA      & gtaacaatctgagttcgtagcaC  \\
 IWGSC\_CSS\_4AL\_scaff\_7132733  & Kronos3085 &  1799 & C    & T     & Hom    & Hom   & cacccgtgagtgaccctC       & cacccgtgagtgaccctT       & aCcGcctaGaaagaaagcttC    \\
 IWGSC\_CSS\_5AS\_scaff\_1534693  & Kronos3085 &  4605 & C    & T     & Het    & Het   & cagcttcctggccctcAtC      & cagcttcctggccctcAtT      & gtaCctcacgAgtcaTgagAG    \\
 IWGSC\_CSS\_6AS\_scaff\_4361911  & Kronos3085 &  8857 & G    & A     & Het    & Het   & tcacgaaagacgacttcaacctcC & tcacgaaagacgacttcaacctcT & catgaggtgctgcatctccatcA  \\
 IWGSC\_CSS\_6BS\_scaff\_3008326  & Kronos3085 &  1528 & G    & A     & Het    & Het   & ccatgttgtactggtggtgC     & ccatgttgtactggtggtgT     & ggaagcatggCaagtgcA       \\
 IWGSC\_CSS\_7AS\_scaff\_4214385  & Kronos3085 & 27835 & C    & T     & Hom    & Hom   & cgtaccttcgttgggaaagG     & cgtaccttcgttgggaaagA     & ctcttggtcagctgtataagacT  \\
 IWGSC\_CSS\_1AL\_scaff\_3929964  & Kronos3191 &  1336 & C    & T     & Het    & Het   & tttcggccatacctgacatC     & tttcggccatacctgacatT     & attgcctccagttcttgcaG     \\
 IWGSC\_CSS\_1BL\_scaff\_3899789  & Kronos3191 &  7925 & C    & T     & Het    & Het   & actctcacTggcagcagC       & actctcacTggcagcagT       & caacgtggtgcccatcGtA      \\
 IWGSC\_CSS\_2AL\_scaff\_6426728  & Kronos3191 &  1481 & G    & A     & Hom    & Hom   & gaaActgccgcagctCgC       & gaaActgccgcagctCgT       & ccaGcaGctcgtgagaaA       \\
 IWGSC\_CSS\_2BL\_scaff\_7960273  & Kronos3191 &   690 & C    & T     & Hom    & Hom   & gccattcatccttaggcgC      & gccattcatccttaggcgT      & acatgcaattgctgatgactG    \\
 IWGSC\_CSS\_3AS\_scaff\_3286603  & Kronos3191 &  2975 & G    & A     & Het*   & Hom   & ccgtgtggtttgttgtggG      & ccgtgtggtttgttgtggA      & gaaaggaacgtgTcaTgcaG     \\
 IWGSC\_CSS\_5AL\_scaff\_2694249  & Kronos3191 &  2399 & C    & T     & Het    & Het   & gccttccagatagagccGC      & gccttccagatagagccGT      & cgccacatcgacattcctG      \\
 IWGSC\_CSS\_5BL\_scaff\_10923577 & Kronos3191 &  3713 & C    & T     & Het    & Het   & gtggattgcctgagcttgC      & gtggattgcctgagcttgT      & tggtggccttcttgggaC       \\
 IWGSC\_CSS\_6AL\_scaff\_5823017  & Kronos3191 & 13225 & C    & T     & Hom    & Hom   & ccctttcgagcctctggaG      & ccctttcgagcctctggaA      & ttcgagaaggcccatcgA       \\
 IWGSC\_CSS\_6BS\_scaff\_2955394  & Kronos3191 &  1622 & C    & T     & Het*   & Hom   & gtggagatgaaggtctagcaaG   & gtggagatgaaggtctagcaaA   & gatactcgTgcaatgggtgT     \\
 IWGSC\_CSS\_7BL\_scaff\_6739382  & Kronos3191 & 12261 & G    & A     & Hom    & Hom   & gagacaagctttgaattgctcC   & gagacaagctttgaattgctcT   & CgagtgacctTcatttcccG     \\
 IWGSC\_CSS\_1AS\_scaff\_3276389  & Kronos3288 &  9720 & C    & T     & Hom    & Hom   & aCcaGcaggaccAatgtctC     & aCcaGcaggaccAatgtctT     & atgatgcaacctcagccaT      \\
 IWGSC\_CSS\_2AL\_scaff\_6367515  & Kronos3288 &  6976 & G    & A     & Het    & Het   & caggtcgagTgtctccgG       & caggtcgagTgtctccgA       & ggggtgatCtggaagggC       \\
 IWGSC\_CSS\_2AL\_scaff\_6422019  & Kronos3288 &  4523 & G    & A     & Het    & Het   & cgctaggtccctgcatagG      & cgctaggtccctgcatagA      & acgcAcgctaagccgtaC       \\
 IWGSC\_CSS\_3AL\_scaff\_4284850  & Kronos3288 &  7901 & C    & T     & Hom    & Hom   & tggctttggacaacatcgG      & tggctttggacaacatcgA      & tgtcAgcatcgacagccaG      \\
 IWGSC\_CSS\_4AS\_scaff\_5962359  & Kronos3288 & 13049 & G    & A     & Het    & Hom   & ccatcaagaagtacgagttcgaC  & ccatcaagaagtacgagttcgaT  & accatgcccagcttgtcA       \\
 IWGSC\_CSS\_6AL\_scaff\_5778773  & Kronos3288 &  6853 & G    & A     & Het    & Het   & gagtgaccttcccgtctttC     & gagtgaccttcccgtctttT     & ggagaacagctactcggcT      \\
 IWGSC\_CSS\_6AS\_scaff\_4392100  & Kronos3288 &  3434 & C    & T     & Het    & Het   & atggaagcacaggtgaccG      & atggaagcacaggtgaccA      & ggAagcgaaagtgaacaaacA    \\
 IWGSC\_CSS\_7BL\_scaff\_6744240  & Kronos3288 &  9772 & G    & A     & Het    & Het   & agctgttcttctcctacttcaaG  & agctgttcttctcctacttcaaA  & caggtcgttcttgagctcC      \\
 IWGSC\_CSS\_1AL\_scaff\_3887185  & Kronos3413 &  9708 & C    & T     & Hom    & Hom   & gcacgcctttatcgaggtaaaG   & gcacgcctttatcgaggtaaaA   & AgaaacagcagagcgcaA       \\
 IWGSC\_CSS\_2BS\_scaff\_3381362  & Kronos3413 &  5160 & C    & T     & Het*   & Hom   & caacttctgggctgtagtgtG    & caacttctgggctgtagtgtA    & tgAgaattctgacGcaaaagaC   \\
 IWGSC\_CSS\_3AS\_scaff\_3296605  & Kronos3413 &  6154 & G    & A     & Het    & Het   & ctggtcacgggctctagC       & ctggtcacgggctctagT       & cagcactgagagacatggaC     \\
 IWGSC\_CSS\_3B\_scaff\_10693516  & Kronos3413 & 12632 & C    & T     & Het    & Het   & ctaggcttggacaaacaggC     & ctaggcttggacaaacaggT     & agcttgcatctatgggcatT     \\
 IWGSC\_CSS\_5AS\_scaff\_1547699  & Kronos3413 &  2686 & G    & A     & Het    & Het   & gCtacaaccttcaccaatcgC    & gCtacaaccttcaccaatcgT    & gacggctttgaagtgtcatC     \\
 IWGSC\_CSS\_5BL\_scaff\_10856077 & Kronos3413 &  5853 & G    & A     & Het    & Het   & agagcttcaccccatgctC      & agagcttcaccccatgctT      & acgCacatttAatagctgaagC   \\
 IWGSC\_CSS\_6AL\_scaff\_5750718  & Kronos3413 & 11046 & G    & A     & Hom    & Hom   & cacgcTtcccgacttcttataG   & cacgcTtcccgacttcttataA   & AgacgatgtgatcaggattcaG   \\
 IWGSC\_CSS\_7AL\_scaff\_4433177  & Kronos3413 &  3511 & C    & T     & Het    & Het   & GaTgctccGtcaggctgG       & GaTgctccGtcaggctgA       & cactactggacaagctcttgG    \\
 IWGSC\_CSS\_7BL\_scaff\_6742567  & Kronos3413 &   667 & C    & T     & Het    & Het   & gttgcttgcgtggcagaC       & gttgcttgcgtggcagaT       & cattttgcaccgtgtgtcTG     \\
 IWGSC\_CSS\_1AL\_scaff\_3976389  & Kronos3935 & 10941 & C    & T     & Hom    & Hom   & ggtgaggagatcggCgatG      & ggtgaggagatcggCgatA      & cagtcatctacatgagaggtcaG  \\
 IWGSC\_CSS\_1BL\_scaff\_3873362  & Kronos3935 &  1392 & G    & A     & Het    & Het   & cagatctgaagcctaGcacatG   & cagatctgaagcctaGcacatA   & actaccagaatcagcacaaaaAC  \\
 IWGSC\_CSS\_2BL\_scaff\_7882382  & Kronos3935 &  2721 & C    & T     & Het    & Het   & gcaagctaagatgtaccgtagC   & gcaagctaagatgtaccgtagT   & gccacagtaggagaaagactT    \\
 IWGSC\_CSS\_3AL\_scaff\_4242376  & Kronos3935 &  2410 & C    & T     & Het    & Het   & agaacccaaaacccgTacttaG   & agaacccaaaacccgTacttaA   & gtagGgtCcatcCtaaagcttG   \\
 IWGSC\_CSS\_3B\_scaff\_10485067  & Kronos3935 &  3349 & C    & T     & Hom    & Hom   & gcttgagcaactactccaactG   & gcttgagcaactactccaactA   & gcaatttcctttaTccgcagT    \\
 IWGSC\_CSS\_4AS\_scaff\_5984153  & Kronos3935 &  6006 & G    & A     & Het    & Het   & agCaggtctggccaagttG      & agCaggtctggccaagttA      & cgaatGtatgaGtaggcgcT     \\
 IWGSC\_CSS\_4BL\_scaff\_7019402  & Kronos3935 &  9081 & C    & T     & Het    & Het   & tgcaatcatgtagtgagctgG    & tgcaatcatgtagtgagctgA    & agcatgatccctagaaCcataC   \\
 IWGSC\_CSS\_5BL\_scaff\_10842786 & Kronos3935 &  3304 & G    & A     & Het    & Het   & tggttcccGaagcctgaaC      & tggttcccGaagcctgaaT      & cgcatacttgaaacaTGagcAC   \\
 IWGSC\_CSS\_6BS\_scaff\_3045205  & Kronos3935 &  2293 & G    & A     & Het    & Het   & aaggaccaagcccaaactctcG   & aaggaccaagcccaaactctcA   & agtgatcaagcccaatgtcgcA   \\
 IWGSC\_CSS\_7AL\_scaff\_4555249  & Kronos3935 &  4487 & C    & T     & Het    & Het   & cAgtgctcgagatggcgC       & cAgtgctcgagatggcgT       & cCttgcaaccctcctgatT      \\
 IWGSC\_CSS\_1BL\_scaff\_3918498  & Kronos4240 &  6096 & G    & A     & Het    & Het   & ttgcatgccccaagaagaG      & ttgcatgccccaagaagaA      & tgggcgaactggtaatgtgG     \\
 IWGSC\_CSS\_2BS\_scaff\_5131713  & Kronos4240 &  5900 & G    & A     & Het    & Het   & cctttatcgaggaaagagacacC  & cctttatcgaggaaagagacacT  & caccattgtagggttccttTttC  \\
 IWGSC\_CSS\_5AL\_scaff\_2769540  & Kronos4240 &  9626 & C    & T     & Het    & Het   & tgCagtgtgggaaacggaG      & tgCagtgtgggaaacggaA      & catgagtGagatcttcctgcT    \\
 IWGSC\_CSS\_5BL\_scaff\_10871091 & Kronos4240 &  7062 & G    & A     & Het    & Het   & gccaaggAaccataacctgC     & gccaaggAaccataacctgT     & GgactcttggcAaccggA       \\
 IWGSC\_CSS\_6AL\_scaff\_5800333  & Kronos4240 &  2360 & G    & A     & Het    & Het   & cgacaggattgtgagCgC       & cgacaggattgtgagCgT       & tcagatgctgcaagattcatcT   \\
 IWGSC\_CSS\_7BL\_scaff\_6716931  & Kronos4240 &  2613 & G    & A     & Het    & Het   & gGtgGgtattTgcttggtgaG    & gGtgGgtattTgcttggtgaA    & tgGtggactcgacaGtGtA      \\
 IWGSC\_CSS\_2BL\_scaff\_8029221  & Kronos4346 &  2860 & G    & A     & Het    & Het   & tgcttccgctcttgctcC       & tgcttccgctcttgctcT       & atTtgcatTCgAtcgggcC      \\
 IWGSC\_CSS\_3B\_scaff\_10460714  & Kronos4346 & 14359 & C    & T     & Hom    & Hom   & ctaccttgccatgcgacatG     & ctaccttgccatgcgacatA     & agcaccccagtctttgacG      \\
 IWGSC\_CSS\_4AS\_scaff\_5989735  & Kronos4346 &  6404 & G    & A     & Hom    & Hom   & acgcatgctaacatcagcC      & acgcatgctaacatcagcT      & actcaagataccaCcgcacG     \\
 IWGSC\_CSS\_5BL\_scaff\_7648030  & Kronos4346 &  6893 & C    & T     & Het    & Het   & taccctttcctactggcagG     & taccctttcctactggcagA     & ttttcagaggaacacaggtatcA  \\
 IWGSC\_CSS\_6AL\_scaff\_5755840  & Kronos4346 &   778 & C    & T     & Het    & Het   & atcgagtaagctgtcacCgC     & atcgagtaagctgtcacCgT     & acctgcatgtcaCatccaC      \\
 IWGSC\_CSS\_6BS\_scaff\_2972151  & Kronos4346 &  7876 & G    & A     & Hom    & Hom   & gcagcaatgtcActgtttgG     & gcagcaatgtcActgtttgA     & gcttggactgggcatttatG     \\
 IWGSC\_CSS\_7AL\_scaff\_4542983  & Kronos4346 & 18700 & G    & A     & Het    & Het   & gcagggctAccggatacC       & gcagggctAccggatacT       & catctgccGgttaaacatgC     \\
 IWGSC\_CSS\_7BS\_scaff\_3098098  & Kronos4346 &  5183 & C    & T     & Het    & Het   & gCgatatggtacttgcaatgaG   & gCgatatggtacttgcaatgaA   & ttacattgcttataGTttgCcgG  \\
 IWGSC\_CSS\_1AS\_scaff\_3259804  & Kronos4485 &   219 & C    & T     & Het    & Het   & gtcggcacaaccccttgC       & gtcggcacaaccccttgT       & gcttctttaaggagggcgA      \\
 IWGSC\_CSS\_2AL\_scaff\_6315418  & Kronos4485 & 10490 & G    & A     & Hom    & Hom   & gcccctctcaaCcttctcagC    & gcccctctcaaCcttctcagT    & ttcagacgctCgaggaatttccC  \\
 IWGSC\_CSS\_2BS\_scaff\_5181092  & Kronos4485 &  3742 & G    & A     & Het    & Het   & TggccagcacacctgcaG       & TggccagcacacctgcaA       & tggacgatgagTgatggAaaT    \\
 IWGSC\_CSS\_3B\_scaff\_10425015  & Kronos4485 &  2372 & C    & T     & Het    & Het   & gctactgaagttggCtcGG      & gctactgaagttggCtcGA      & cttcacatccttgggggTtC     \\
 IWGSC\_CSS\_3B\_scaff\_10775915  & Kronos4485 &  4701 & C    & T     & Het    & Het   & ccaagggctgcagagagG       & ccaagggctgcagagagA       & agacctcacgatGtcctcC      \\
 IWGSC\_CSS\_5AL\_scaff\_2754304  & Kronos4485 &  2301 & G    & A     & Het    & Het   & taacccTgccatcgcccG       & taacccTgccatcgcccA       & cattgGccagccaTgacT       \\
 IWGSC\_CSS\_5BL\_scaff\_10919959 & Kronos4485 &  1867 & C    & T     & Hom    & Hom   & gatgccctttgtggagaagG     & gatgccctttgtggagaagA     & tcttgttcccgaaacatgtcA    \\
 IWGSC\_CSS\_7AS\_scaff\_4245431  & Kronos4485 &  3402 & G    & A     & Hom    & Hom   & aaggcgcctggtgtttcC       & aaggcgcctggtgtttcT       & agtaagtggaAcagctaagatcaT \\
 IWGSC\_CSS\_7BL\_scaff\_6667357  & Kronos4485 &   641 & C    & T     & Het    & Het   & gatcAgctgctcattcgagG     & gatcAgctgctcattcgagA     & ttccctgtcaattgatgccC     \\
\bottomrule
\end{tabular}

%\end{localsize}
%\end{landscape}


%N79298:SupplementalTables ramirezr$ tabulate -s, -1 -f latex  -o CadenzaMutValidation3.tex CadenzaMutValidation3.csv
%N79298:SupplementalTables ramirezr$ tabulate -s, -1 -f latex  -o CadenzaMutValidation2.tex CadenzaMutValidation2.csv

%!TEX root = ../Main.tex

\chapter{PolyMarker validation}

\label{app:PolyMarkerValidation}
\begin{sidewaystable}
\section{Validation of mutations on $M_{4}$ on Kronos}
\label{app:PolyMarkerM4ValidationKronos}

\begin{localsize}{6}{7}

%N79298:SupplementalTables ramirezr$ tabulate -s, -1 -f latex  -o CadenzaMutValidation3.tex CadenzaMutValidation3.csv
%N79298:SupplementalTables ramirezr$ tabulate -s, -1 -f latex  -o CadenzaMutValidation2.tex CadenzaMutValidation2.csv

%!TEX root = ../../Main.tex

\begin{tabular}{llrlllllll}
\toprule
 IWGSC contig                 & Line       &   Pos & WT   & Mut   & Predicted   & Called on $M_{4}$    & Primer 1 (Kronos)        & Primer 2 (mutant)        & Common Primer            \\
\midrule
 IWGSC\_CSS\_1AS\_scaff\_3284790  & Kronos3085 &  7449 & G    & A     & Het    & Het   & ccacaccttgagcctcgC       & ccacaccttgagcctcgT       & gtgattttgccaggggagA      \\
 IWGSC\_CSS\_1BL\_scaff\_3897513  & Kronos3085 &  1515 & C    & T     & Het    & Het   & gcttccactGggtcctgC       & gcttccactGggtcctgT       & acAaggactgcttcagaGaC     \\
 IWGSC\_CSS\_2AL\_scaff\_6434745  & Kronos3085 &  3424 & C    & T     & Het    & Het   & cctcGgttttgcaaatttctatgC & cctcGgttttgcaaatttctatgT & gGCaaTggcataacaacagatA   \\
 IWGSC\_CSS\_3AS\_scaff\_3408995  & Kronos3085 &   732 & C    & T     & Het    & Het   & aggccatttcgaattccgC      & aggccatttcgaattccgT      & ggTgttaTccagAacctgagTG   \\
 IWGSC\_CSS\_3B\_scaff\_10708748  & Kronos3085 &  2675 & G    & A     & Het    & Het   & gttgcatgcttcacccagG      & gttgcatgcttcacccagA      & gtaacaatctgagttcgtagcaC  \\
 IWGSC\_CSS\_4AL\_scaff\_7132733  & Kronos3085 &  1799 & C    & T     & Hom    & Hom   & cacccgtgagtgaccctC       & cacccgtgagtgaccctT       & aCcGcctaGaaagaaagcttC    \\
 IWGSC\_CSS\_5AS\_scaff\_1534693  & Kronos3085 &  4605 & C    & T     & Het    & Het   & cagcttcctggccctcAtC      & cagcttcctggccctcAtT      & gtaCctcacgAgtcaTgagAG    \\
 IWGSC\_CSS\_6AS\_scaff\_4361911  & Kronos3085 &  8857 & G    & A     & Het    & Het   & tcacgaaagacgacttcaacctcC & tcacgaaagacgacttcaacctcT & catgaggtgctgcatctccatcA  \\
 IWGSC\_CSS\_6BS\_scaff\_3008326  & Kronos3085 &  1528 & G    & A     & Het    & Het   & ccatgttgtactggtggtgC     & ccatgttgtactggtggtgT     & ggaagcatggCaagtgcA       \\
 IWGSC\_CSS\_7AS\_scaff\_4214385  & Kronos3085 & 27835 & C    & T     & Hom    & Hom   & cgtaccttcgttgggaaagG     & cgtaccttcgttgggaaagA     & ctcttggtcagctgtataagacT  \\
 IWGSC\_CSS\_1AL\_scaff\_3929964  & Kronos3191 &  1336 & C    & T     & Het    & Het   & tttcggccatacctgacatC     & tttcggccatacctgacatT     & attgcctccagttcttgcaG     \\
 IWGSC\_CSS\_1BL\_scaff\_3899789  & Kronos3191 &  7925 & C    & T     & Het    & Het   & actctcacTggcagcagC       & actctcacTggcagcagT       & caacgtggtgcccatcGtA      \\
 IWGSC\_CSS\_2AL\_scaff\_6426728  & Kronos3191 &  1481 & G    & A     & Hom    & Hom   & gaaActgccgcagctCgC       & gaaActgccgcagctCgT       & ccaGcaGctcgtgagaaA       \\
 IWGSC\_CSS\_2BL\_scaff\_7960273  & Kronos3191 &   690 & C    & T     & Hom    & Hom   & gccattcatccttaggcgC      & gccattcatccttaggcgT      & acatgcaattgctgatgactG    \\
 IWGSC\_CSS\_3AS\_scaff\_3286603  & Kronos3191 &  2975 & G    & A     & Het*   & Hom   & ccgtgtggtttgttgtggG      & ccgtgtggtttgttgtggA      & gaaaggaacgtgTcaTgcaG     \\
 IWGSC\_CSS\_5AL\_scaff\_2694249  & Kronos3191 &  2399 & C    & T     & Het    & Het   & gccttccagatagagccGC      & gccttccagatagagccGT      & cgccacatcgacattcctG      \\
 IWGSC\_CSS\_5BL\_scaff\_10923577 & Kronos3191 &  3713 & C    & T     & Het    & Het   & gtggattgcctgagcttgC      & gtggattgcctgagcttgT      & tggtggccttcttgggaC       \\
 IWGSC\_CSS\_6AL\_scaff\_5823017  & Kronos3191 & 13225 & C    & T     & Hom    & Hom   & ccctttcgagcctctggaG      & ccctttcgagcctctggaA      & ttcgagaaggcccatcgA       \\
 IWGSC\_CSS\_6BS\_scaff\_2955394  & Kronos3191 &  1622 & C    & T     & Het*   & Hom   & gtggagatgaaggtctagcaaG   & gtggagatgaaggtctagcaaA   & gatactcgTgcaatgggtgT     \\
 IWGSC\_CSS\_7BL\_scaff\_6739382  & Kronos3191 & 12261 & G    & A     & Hom    & Hom   & gagacaagctttgaattgctcC   & gagacaagctttgaattgctcT   & CgagtgacctTcatttcccG     \\
 IWGSC\_CSS\_1AS\_scaff\_3276389  & Kronos3288 &  9720 & C    & T     & Hom    & Hom   & aCcaGcaggaccAatgtctC     & aCcaGcaggaccAatgtctT     & atgatgcaacctcagccaT      \\
 IWGSC\_CSS\_2AL\_scaff\_6367515  & Kronos3288 &  6976 & G    & A     & Het    & Het   & caggtcgagTgtctccgG       & caggtcgagTgtctccgA       & ggggtgatCtggaagggC       \\
 IWGSC\_CSS\_2AL\_scaff\_6422019  & Kronos3288 &  4523 & G    & A     & Het    & Het   & cgctaggtccctgcatagG      & cgctaggtccctgcatagA      & acgcAcgctaagccgtaC       \\
 IWGSC\_CSS\_3AL\_scaff\_4284850  & Kronos3288 &  7901 & C    & T     & Hom    & Hom   & tggctttggacaacatcgG      & tggctttggacaacatcgA      & tgtcAgcatcgacagccaG      \\
 IWGSC\_CSS\_4AS\_scaff\_5962359  & Kronos3288 & 13049 & G    & A     & Het    & Hom   & ccatcaagaagtacgagttcgaC  & ccatcaagaagtacgagttcgaT  & accatgcccagcttgtcA       \\
 IWGSC\_CSS\_6AL\_scaff\_5778773  & Kronos3288 &  6853 & G    & A     & Het    & Het   & gagtgaccttcccgtctttC     & gagtgaccttcccgtctttT     & ggagaacagctactcggcT      \\
 IWGSC\_CSS\_6AS\_scaff\_4392100  & Kronos3288 &  3434 & C    & T     & Het    & Het   & atggaagcacaggtgaccG      & atggaagcacaggtgaccA      & ggAagcgaaagtgaacaaacA    \\
 IWGSC\_CSS\_7BL\_scaff\_6744240  & Kronos3288 &  9772 & G    & A     & Het    & Het   & agctgttcttctcctacttcaaG  & agctgttcttctcctacttcaaA  & caggtcgttcttgagctcC      \\
 IWGSC\_CSS\_1AL\_scaff\_3887185  & Kronos3413 &  9708 & C    & T     & Hom    & Hom   & gcacgcctttatcgaggtaaaG   & gcacgcctttatcgaggtaaaA   & AgaaacagcagagcgcaA       \\
 IWGSC\_CSS\_2BS\_scaff\_3381362  & Kronos3413 &  5160 & C    & T     & Het*   & Hom   & caacttctgggctgtagtgtG    & caacttctgggctgtagtgtA    & tgAgaattctgacGcaaaagaC   \\
 IWGSC\_CSS\_3AS\_scaff\_3296605  & Kronos3413 &  6154 & G    & A     & Het    & Het   & ctggtcacgggctctagC       & ctggtcacgggctctagT       & cagcactgagagacatggaC     \\
 IWGSC\_CSS\_3B\_scaff\_10693516  & Kronos3413 & 12632 & C    & T     & Het    & Het   & ctaggcttggacaaacaggC     & ctaggcttggacaaacaggT     & agcttgcatctatgggcatT     \\
 IWGSC\_CSS\_5AS\_scaff\_1547699  & Kronos3413 &  2686 & G    & A     & Het    & Het   & gCtacaaccttcaccaatcgC    & gCtacaaccttcaccaatcgT    & gacggctttgaagtgtcatC     \\
 IWGSC\_CSS\_5BL\_scaff\_10856077 & Kronos3413 &  5853 & G    & A     & Het    & Het   & agagcttcaccccatgctC      & agagcttcaccccatgctT      & acgCacatttAatagctgaagC   \\
 IWGSC\_CSS\_6AL\_scaff\_5750718  & Kronos3413 & 11046 & G    & A     & Hom    & Hom   & cacgcTtcccgacttcttataG   & cacgcTtcccgacttcttataA   & AgacgatgtgatcaggattcaG   \\
 IWGSC\_CSS\_7AL\_scaff\_4433177  & Kronos3413 &  3511 & C    & T     & Het    & Het   & GaTgctccGtcaggctgG       & GaTgctccGtcaggctgA       & cactactggacaagctcttgG    \\
 IWGSC\_CSS\_7BL\_scaff\_6742567  & Kronos3413 &   667 & C    & T     & Het    & Het   & gttgcttgcgtggcagaC       & gttgcttgcgtggcagaT       & cattttgcaccgtgtgtcTG     \\
 IWGSC\_CSS\_1AL\_scaff\_3976389  & Kronos3935 & 10941 & C    & T     & Hom    & Hom   & ggtgaggagatcggCgatG      & ggtgaggagatcggCgatA      & cagtcatctacatgagaggtcaG  \\
 IWGSC\_CSS\_1BL\_scaff\_3873362  & Kronos3935 &  1392 & G    & A     & Het    & Het   & cagatctgaagcctaGcacatG   & cagatctgaagcctaGcacatA   & actaccagaatcagcacaaaaAC  \\
 IWGSC\_CSS\_2BL\_scaff\_7882382  & Kronos3935 &  2721 & C    & T     & Het    & Het   & gcaagctaagatgtaccgtagC   & gcaagctaagatgtaccgtagT   & gccacagtaggagaaagactT    \\
 IWGSC\_CSS\_3AL\_scaff\_4242376  & Kronos3935 &  2410 & C    & T     & Het    & Het   & agaacccaaaacccgTacttaG   & agaacccaaaacccgTacttaA   & gtagGgtCcatcCtaaagcttG   \\
 IWGSC\_CSS\_3B\_scaff\_10485067  & Kronos3935 &  3349 & C    & T     & Hom    & Hom   & gcttgagcaactactccaactG   & gcttgagcaactactccaactA   & gcaatttcctttaTccgcagT    \\
 IWGSC\_CSS\_4AS\_scaff\_5984153  & Kronos3935 &  6006 & G    & A     & Het    & Het   & agCaggtctggccaagttG      & agCaggtctggccaagttA      & cgaatGtatgaGtaggcgcT     \\
 IWGSC\_CSS\_4BL\_scaff\_7019402  & Kronos3935 &  9081 & C    & T     & Het    & Het   & tgcaatcatgtagtgagctgG    & tgcaatcatgtagtgagctgA    & agcatgatccctagaaCcataC   \\
 IWGSC\_CSS\_5BL\_scaff\_10842786 & Kronos3935 &  3304 & G    & A     & Het    & Het   & tggttcccGaagcctgaaC      & tggttcccGaagcctgaaT      & cgcatacttgaaacaTGagcAC   \\
 IWGSC\_CSS\_6BS\_scaff\_3045205  & Kronos3935 &  2293 & G    & A     & Het    & Het   & aaggaccaagcccaaactctcG   & aaggaccaagcccaaactctcA   & agtgatcaagcccaatgtcgcA   \\
 IWGSC\_CSS\_7AL\_scaff\_4555249  & Kronos3935 &  4487 & C    & T     & Het    & Het   & cAgtgctcgagatggcgC       & cAgtgctcgagatggcgT       & cCttgcaaccctcctgatT      \\
 IWGSC\_CSS\_1BL\_scaff\_3918498  & Kronos4240 &  6096 & G    & A     & Het    & Het   & ttgcatgccccaagaagaG      & ttgcatgccccaagaagaA      & tgggcgaactggtaatgtgG     \\
 IWGSC\_CSS\_2BS\_scaff\_5131713  & Kronos4240 &  5900 & G    & A     & Het    & Het   & cctttatcgaggaaagagacacC  & cctttatcgaggaaagagacacT  & caccattgtagggttccttTttC  \\
 IWGSC\_CSS\_5AL\_scaff\_2769540  & Kronos4240 &  9626 & C    & T     & Het    & Het   & tgCagtgtgggaaacggaG      & tgCagtgtgggaaacggaA      & catgagtGagatcttcctgcT    \\
 IWGSC\_CSS\_5BL\_scaff\_10871091 & Kronos4240 &  7062 & G    & A     & Het    & Het   & gccaaggAaccataacctgC     & gccaaggAaccataacctgT     & GgactcttggcAaccggA       \\
 IWGSC\_CSS\_6AL\_scaff\_5800333  & Kronos4240 &  2360 & G    & A     & Het    & Het   & cgacaggattgtgagCgC       & cgacaggattgtgagCgT       & tcagatgctgcaagattcatcT   \\
 IWGSC\_CSS\_7BL\_scaff\_6716931  & Kronos4240 &  2613 & G    & A     & Het    & Het   & gGtgGgtattTgcttggtgaG    & gGtgGgtattTgcttggtgaA    & tgGtggactcgacaGtGtA      \\
 IWGSC\_CSS\_2BL\_scaff\_8029221  & Kronos4346 &  2860 & G    & A     & Het    & Het   & tgcttccgctcttgctcC       & tgcttccgctcttgctcT       & atTtgcatTCgAtcgggcC      \\
 IWGSC\_CSS\_3B\_scaff\_10460714  & Kronos4346 & 14359 & C    & T     & Hom    & Hom   & ctaccttgccatgcgacatG     & ctaccttgccatgcgacatA     & agcaccccagtctttgacG      \\
 IWGSC\_CSS\_4AS\_scaff\_5989735  & Kronos4346 &  6404 & G    & A     & Hom    & Hom   & acgcatgctaacatcagcC      & acgcatgctaacatcagcT      & actcaagataccaCcgcacG     \\
 IWGSC\_CSS\_5BL\_scaff\_7648030  & Kronos4346 &  6893 & C    & T     & Het    & Het   & taccctttcctactggcagG     & taccctttcctactggcagA     & ttttcagaggaacacaggtatcA  \\
 IWGSC\_CSS\_6AL\_scaff\_5755840  & Kronos4346 &   778 & C    & T     & Het    & Het   & atcgagtaagctgtcacCgC     & atcgagtaagctgtcacCgT     & acctgcatgtcaCatccaC      \\
 IWGSC\_CSS\_6BS\_scaff\_2972151  & Kronos4346 &  7876 & G    & A     & Hom    & Hom   & gcagcaatgtcActgtttgG     & gcagcaatgtcActgtttgA     & gcttggactgggcatttatG     \\
 IWGSC\_CSS\_7AL\_scaff\_4542983  & Kronos4346 & 18700 & G    & A     & Het    & Het   & gcagggctAccggatacC       & gcagggctAccggatacT       & catctgccGgttaaacatgC     \\
 IWGSC\_CSS\_7BS\_scaff\_3098098  & Kronos4346 &  5183 & C    & T     & Het    & Het   & gCgatatggtacttgcaatgaG   & gCgatatggtacttgcaatgaA   & ttacattgcttataGTttgCcgG  \\
 IWGSC\_CSS\_1AS\_scaff\_3259804  & Kronos4485 &   219 & C    & T     & Het    & Het   & gtcggcacaaccccttgC       & gtcggcacaaccccttgT       & gcttctttaaggagggcgA      \\
 IWGSC\_CSS\_2AL\_scaff\_6315418  & Kronos4485 & 10490 & G    & A     & Hom    & Hom   & gcccctctcaaCcttctcagC    & gcccctctcaaCcttctcagT    & ttcagacgctCgaggaatttccC  \\
 IWGSC\_CSS\_2BS\_scaff\_5181092  & Kronos4485 &  3742 & G    & A     & Het    & Het   & TggccagcacacctgcaG       & TggccagcacacctgcaA       & tggacgatgagTgatggAaaT    \\
 IWGSC\_CSS\_3B\_scaff\_10425015  & Kronos4485 &  2372 & C    & T     & Het    & Het   & gctactgaagttggCtcGG      & gctactgaagttggCtcGA      & cttcacatccttgggggTtC     \\
 IWGSC\_CSS\_3B\_scaff\_10775915  & Kronos4485 &  4701 & C    & T     & Het    & Het   & ccaagggctgcagagagG       & ccaagggctgcagagagA       & agacctcacgatGtcctcC      \\
 IWGSC\_CSS\_5AL\_scaff\_2754304  & Kronos4485 &  2301 & G    & A     & Het    & Het   & taacccTgccatcgcccG       & taacccTgccatcgcccA       & cattgGccagccaTgacT       \\
 IWGSC\_CSS\_5BL\_scaff\_10919959 & Kronos4485 &  1867 & C    & T     & Hom    & Hom   & gatgccctttgtggagaagG     & gatgccctttgtggagaagA     & tcttgttcccgaaacatgtcA    \\
 IWGSC\_CSS\_7AS\_scaff\_4245431  & Kronos4485 &  3402 & G    & A     & Hom    & Hom   & aaggcgcctggtgtttcC       & aaggcgcctggtgtttcT       & agtaagtggaAcagctaagatcaT \\
 IWGSC\_CSS\_7BL\_scaff\_6667357  & Kronos4485 &   641 & C    & T     & Het    & Het   & gatcAgctgctcattcgagG     & gatcAgctgctcattcgagA     & ttccctgtcaattgatgccC     \\
\bottomrule
\end{tabular}

\end{localsize}
\end{sidewaystable}

\begin{sidewaystable}

\section{Validation of mutations on $M_{4}$ on Cadenza}

\begin{localsize}{6}{7}

\label{app:PolyMarkerM4ValidationCadenza}
\begin{longtable}{llrlllllll}
\caption{Validation of mutations on $M_{4}$ on Cadenza}\\
\label{app:PolyMarkerM4ValidationCadenza}\\
\toprule
 IWGSC contig                 & Line       &   Pos & WT   & Mut   & Predicted   & $M_{4}$      & Primer 1 (Cadenza)        & Primer 2 (mutant)         & Common Primer             \\
\midrule
\endfirsthead
\toprule
 IWGSC contig                 & Line       &   Pos & WT   & Mut   & Predicted   & $M_{4}$      & Primer 1 (Cadenza)        & Primer 2 (mutant)         & Common Primer             \\
\midrule
\endhead
\bottomrule
\endfoot
\bottomrule
\endlastfoot
 IWGSC\_CSS\_3B\_scaff\_10445294  & Cadenza1772 &       6019 & C         & T        & het            & het         & caggatAgtGggactgtcaaaG    & caggatAgtGggactgtcaaaA    & ggagacGGctGtggacatT       \\
 IWGSC\_CSS\_3DL\_scaff\_6955403  & Cadenza1772 &       2418 & C         & T        & het*           & hom         & tcagCggattgtcgggatG       & tcagCggattgtcgggatA       & tgtcCatgaaTcttgtccacG     \\
 IWGSC\_CSS\_4AL\_scaff\_7106846  & Cadenza1772 &      11277 & G         & A        & hom            & hom         & tgggatccatgcctacactG      & tgggatccatgcctacactA      & gatggtGgatttgccgctA       \\
 IWGSC\_CSS\_4AS\_scaff\_5991335  & Cadenza1772 &      15710 & G         & A        & hom            & hom         & ctggccctgcgctgctaC        & ctggccctgcgctgctaT        & gtggaaGttcagaaggaccaG     \\
 IWGSC\_CSS\_4BS\_scaff\_4956646  & Cadenza1772 &        252 & G         & A        & het*           & hom         & gcaggttgacttcccggaG       & gcaggttgacttcccggaA       & tGaggtacgaGcTaaagAaagC    \\
 IWGSC\_CSS\_4DS\_scaff\_1715962  & Cadenza1772 &       1225 & G         & A        & hom            & hom         & cagctgtggTatctcaactgG     & cagctgtggTatctcaactgA     & CcCtGaaACACcGtttggaT      \\
 IWGSC\_CSS\_5AL\_scaff\_2763407  & Cadenza1772 &       2119 & G         & A        & hom            & hom         & gcgacGaacctcgagatctG      & gcgacGaacctcgagatctA      & gaTggcaAtcgtCgtgcA        \\
 IWGSC\_CSS\_5AS\_scaff\_1548786  & Cadenza1772 &      12625 & C         & T        & het            & het         & AtaggcacattgctagactgaG    & AtaggcacattgctagactgaA    & ggattgggtgttgcacgC        \\
 IWGSC\_CSS\_5BL\_scaff\_10849226 & Cadenza1772 &       2289 & C         & T        & het*           & hom         & cctgacatcattgttcacgatC    & cctgacatcattgttcacgatT    & cactccgaggtgtccatgaT      \\
 IWGSC\_CSS\_5BS\_scaff\_2270737  & Cadenza1772 &       2262 & G         & A        & hom            & ---         & attcCTgtgttgtggCaaatgaG   & attcCTgtgttgtggCaaatgaA   & taaGcacaaAccctccagctgG    \\
 IWGSC\_CSS\_1AL\_scaff\_3022915  & Cadenza1661 &        891 & C         & T        & hom            & hom         & ccacagtgagactcctattgaCG   & ccacagtgagactcctattgaCA   & atgtctgattcGtcGtagtcC     \\
 IWGSC\_CSS\_1AS\_scaff\_3297240  & Cadenza1661 &       1970 & C         & T        & het            & het         & catcccgccGtttcctcC        & catcccgccGtttcctcT        & gctcgccgatgaagagcT        \\
 IWGSC\_CSS\_1BL\_scaff\_3828996  & Cadenza1661 &       1340 & G         & A        & hom            & hom         & agccggatgttagtgttaacC     & agccggatgttagtgttaacT     & agcagcttgTcgcgttaaC       \\
 IWGSC\_CSS\_1DS\_scaff\_1884529  & Cadenza1661 &      10575 & G         & A        & hom            & hom         & aCagatacaAttgtcatgcaggC   & aCagatacaAttgtcatgcaggT   & acctgggTTgtccaatacttC     \\
 IWGSC\_CSS\_2AL\_scaff\_6318370  & Cadenza1661 &      19142 & C         & T        & het            & ---         & cgtggcCgaatCtcGacG        & cgtggcCgaatCtcGacA        & ttcttgtgggagccgggC        \\
 IWGSC\_CSS\_2AS\_scaff\_5213460  & Cadenza1661 &       1358 & G         & A        & hom            & hom         & gtcacgaaCccgctcagG        & gtcacgaaCccgctcagA        & aggaaagagaggaaaagaGcG     \\
 IWGSC\_CSS\_2BS\_scaff\_5179331  & Cadenza1661 &       5604 & G         & A        & het            & het         & actctcgtcaagaactgatacaG   & actctcgtcaagaactgatacaA   & gcaGagaatgttcttgcaacT     \\
 IWGSC\_CSS\_2DS\_scaff\_5341235  & Cadenza1661 &       4673 & G         & A        & het            & het         & ggtgaggatctcggagctG       & ggtgaggatctcggagctA       & gcgcggtcgtacgagttG        \\
 IWGSC\_CSS\_3AL\_scaff\_4250995  & Cadenza1661 &       7046 & G         & A        & hom            & hom         & cCaagaaacgggtggtccaG      & cCaagaaacgggtggtccaA      & ctgcagctgtcccatcatcgT     \\
 IWGSC\_CSS\_3B\_scaff\_10404421  & Cadenza1661 &       4303 & G         & A        & het            & het         & ccttcgtcgaCaggacctG       & ccttcgtcgaCaggacctA       & GCcagtactCacAtgctctC      \\
 IWGSC\_CSS\_5DL\_scaff\_2390496  & Cadenza1538 &       2125 & C         & T        & hom            & het         & gcagttttatcctcagtagtcttgG & gcagttttatcctcagtagtcttgA & ttctgagaaTgtaatgtgcGatG   \\
 IWGSC\_CSS\_6AL\_scaff\_5753680  & Cadenza1538 &       3920 & C         & T        & hom            & hom         & tgctccaaatttgagcacaaTaaC  & tgctccaaatttgagcacaaTaaT  & aaatgcaaggggtaagtttttgT   \\
 IWGSC\_CSS\_6AS\_scaff\_4425792  & Cadenza1538 &       4307 & G         & A        & hom            & het         & agatgcttgtCggGccaG        & agatgcttgtCggGccaA        & gctgaagcaacgcgatcaaT      \\
 IWGSC\_CSS\_6BS\_scaff\_3003630  & Cadenza1538 &       6933 & C         & T        & het            & het         & ggcagtaatgtggtgctgagC     & ggcagtaatgtggtgctgagT     & tTgaCttctggtttggtggcA     \\
 IWGSC\_CSS\_6DL\_scaff\_3246988  & Cadenza1538 &       9186 & G         & A        & het            & het         & gctaaagaagagcttgagagaattC & gctaaagaagagcttgagagaattT & aatttctgaagagaggtgttgtatG \\
 IWGSC\_CSS\_7AL\_scaff\_4480114  & Cadenza1538 &       3446 & C         & T        & het            & ---         & gatatctcccacacggcgG       & gatatctcccacacggcgA       & tgagccactcttgcagtttT      \\
 IWGSC\_CSS\_7AS\_scaff\_4193541  & Cadenza1538 &       8359 & C         & T        & hom            & het         & agcaattctttggctatcaattagC & agcaattctttggctatcaattagT & tcatctGtcttaactctactgctG  \\
 IWGSC\_CSS\_7BL\_scaff\_6721572  & Cadenza1538 &       9223 & C         & T        & het            & het         & gctCagggaggaagacaagaaG    & gctCagggaggaagacaagaaA    & tgctatgaagaattccgacctC    \\
 IWGSC\_CSS\_7BS\_scaff\_3152545  & Cadenza1538 &       3960 & G         & A        & hom            & ---         & tcagcaaaatcacctgcCgC      & tcagcaaaatcacctgcCgT      & gCtgccccatcatcgtttaT      \\
 IWGSC\_CSS\_7DS\_scaff\_3963838  & Cadenza1538 &       2913 & G         & A        & het            & het         & tCgttgcaagcCttTtgtgC      & tCgttgcaagcCttTtgtgT      & agaGttaTcaagcTactgtcacA   \\
 IWGSC\_CSS\_1AL\_scaff\_3903380  & Cadenza1469 &       6193 & G         & A        & hom            & hom         & ctcttcAgagatgaacgcgG      & ctcttcAgagatgaacgcgA      & tcGtGagatgGtggtttGTtA     \\
 IWGSC\_CSS\_1AS\_scaff\_3287728  & Cadenza1469 &       3817 & C         & T        & het*           & hom         & ccgaccaAttcactaaccgG      & ccgaccaAttcactaaccgA      & accctctttcccAgacatgaT     \\
 IWGSC\_CSS\_1BL\_scaff\_3815304  & Cadenza1469 &        513 & G         & A        & hom            & hom         & aacatttgcctTaCcaaaacGC    & aacatttgcctTaCcaaaacGT    & acacagcaagttataatgCAAgC   \\
 IWGSC\_CSS\_1DL\_scaff\_2266648  & Cadenza1469 &       5926 & C         & T        & het            & het         & caacatgagacacaacaccttC    & caacatgagacacaacaccttT    & gtcaacgcgtgaggattgtC      \\
 IWGSC\_CSS\_1DS\_scaff\_1906671  & Cadenza1469 &       3697 & C         & T        & hom            & hom         & tggTGtagacacttggcgaG      & tggTGtagacacttggcgaA      & catggcgaccaccAcctG        \\
 IWGSC\_CSS\_2AL\_scaff\_6337088  & Cadenza1469 &       7334 & G         & A        & het*           & hom         & acaatgccAagttgacaggttG    & acaatgccAagttgacaggttA    & gggagtgttggttCagaacaT     \\
 IWGSC\_CSS\_2BL\_scaff\_7972799  & Cadenza1469 &       8995 & C         & T        & het            & hom         & gTgCtcctcGgcatccttC       & gTgCtcctcGgcatccttT       & gatccgGgcaaactacgTG       \\
 IWGSC\_CSS\_2DL\_scaff\_9832343  & Cadenza1469 &       3262 & G         & A        & het            & het         & TtgtctaAcagcacCGcagG      & TtgtctaAcagcacCGcagA      & agatctcggtcagcctttcT      \\
 IWGSC\_CSS\_2DS\_scaff\_5327939  & Cadenza1469 &       3889 & G         & A        & het            & het         & ttttTgccttatgtgactctagtaC & ttttTgccttatgtgactctagtaT & gaggccatcacagatagcG       \\
 IWGSC\_CSS\_3B\_scaff\_10395219  & Cadenza1469 &       1292 & G         & A        & hom            & ---         & aggtgcttgtgcttgctgG       & aggtgcttgtgcttgctgA       & cctcttctgggggctttataC     \\
 IWGSC\_CSS\_3B\_scaff\_10592217  & Cadenza0580 &       2994 & C         & T        & het            & ---         & acagcagtatcaagcccctC      & acagcagtatcaagcccctT      & tgatactgttgTggCggagG      \\
 IWGSC\_CSS\_3DS\_scaff\_2596771  & Cadenza0580 &       1037 & G         & A        & het            & het         & tggttatgCAcaggataatCagG   & tggttatgCAcaggataatCagA   & tggcaaatgtgatgtcattaggT   \\
 IWGSC\_CSS\_4AL\_scaff\_7093953  & Cadenza0580 &       9881 & C         & T        & hom            & hom         & GacaggaagccggtaacaC       & GacaggaagccggtaacaT       & ctccAgcaggcatgggaT        \\
 IWGSC\_CSS\_4BL\_scaff\_7037448  & Cadenza0580 &       1837 & C         & T        & hom            & hom         & CgttgaaaaGctgcaagaacttaaC & CgttgaaaaGctgcaagaacttaaT & cagttcttccTtCaGagcagataT  \\
 IWGSC\_CSS\_4BS\_scaff\_4929479  & Cadenza0580 &      10668 & G         & A        & hom            & ---         & tggattttcccgcactgttC      & tggattttcccgcactgttT      & gtaaacaaggcatttcaagagtcA  \\
 IWGSC\_CSS\_4DL\_scaff\_14359838 & Cadenza0580 &       1408 & G         & A        & hom            & ---         & gCtcAttcagggatTGTcCtaTatG & gCtcAttcagggatTGTcCtaTatA & tgaCagaacagttggtcatacT    \\
 IWGSC\_CSS\_4DS\_scaff\_2276484  & Cadenza0580 &       8034 & G         & A        & hom            & hom         & gccgtggttgatggAgaG        & gccgtggttgatggAgaA        & cgtccagattactgatacttgcA   \\
 IWGSC\_CSS\_5AL\_scaff\_2756579  & Cadenza0580 &       5278 & G         & A        & het            & het         & tgaatggatttttcgtcccgttC   & tgaatggatttttcgtcccgttT   & ggAAtCCTATgCAgaAgAaaCTG   \\
 IWGSC\_CSS\_5BL\_scaff\_10787208 & Cadenza0580 &      10627 & G         & A        & het            & ---         & gcctctcacatgcggagaC       & gcctctcacatgcggagaT       & acgatgtcAggtggGcgT        \\
 IWGSC\_CSS\_5BS\_scaff\_2282179  & Cadenza0580 &       5267 & G         & A        & het            & ---         & tgatgggctacgacgtgC        & tgatgggctacgacgtgT        & tcggcgcccttgaaAtcC        \\
 IWGSC\_CSS\_5DL\_scaff\_4498073  & Cadenza0423 &       4937 & C         & T        & hom            & hom         & gcaccctctggttggtcatC      & gcaccctctggttggtcatT      & tgagcagcaAagcagccG        \\
 IWGSC\_CSS\_5DS\_scaff\_2738970  & Cadenza0423 &       2319 & C         & T        & het            & ---         & cgtgaggtgggtgatttgC       & cgtgaggtgggtgatttgT       & tggaactagttacactgcagtTC   \\
 IWGSC\_CSS\_6AL\_scaff\_5757109  & Cadenza0423 &       2788 & G         & A        & hom            & hom         & caggaGcctggcaaataaaGG     & caggaGcctggcaaataaaGA     & ctttcGcagtctcttagtttcG    \\
 IWGSC\_CSS\_6AS\_scaff\_4387871  & Cadenza0423 &       2543 & G         & A        & hom            & hom         & gcatgctaacaggcgaaaagG     & gcatgctaacaggcgaaaagA     & ctcatgctcctgatcttaaggtT   \\
 IWGSC\_CSS\_6BL\_scaff\_4271391  & Cadenza0423 &       4660 & C         & T        & hom            & hom         & tacgtgcatgatgtggtagtcgtaC & tacgtgcatgatgtggtagtcgtaT & gtttgaagtgcatcagatgTaccA  \\
 IWGSC\_CSS\_6DS\_scaff\_1880206  & Cadenza0423 &       9159 & G         & A        & het            & het         & ctgCgaaggctccacaaG        & ctgCgaaggctccacaaA        & ggatgagaagtttgcattgctC    \\
 IWGSC\_CSS\_7AS\_scaff\_4227506  & Cadenza0423 &        952 & G         & A        & het            & ---         & ccatgtgtttccaatgttagagC   & ccatgtgtttccaatgttagagT   & tgccctagctggtatgcT        \\
 IWGSC\_CSS\_7BL\_scaff\_6681782  & Cadenza0423 &       1486 & C         & T        & hom            & hom         & agtaagCGtgacagcaatggG     & agtaagCGtgacagcaatggA     & AtgtctTtgGtggaagtacatcA   \\
 IWGSC\_CSS\_7BS\_scaff\_3160328  & Cadenza0423 &       7801 & C         & T        & het            & het         & tgttaaatGatacagCctgcagC   & tgttaaatGatacagCctgcagT   & tggaatggtgCgttgttttT      \\
 IWGSC\_CSS\_7DS\_scaff\_407428   & Cadenza0423 &       2051 & G         & A        & het            & het         & gtcGCgccatcctgacaG        & gtcGCgccatcctgacaA        & actcatcAggtcagcccaA       \\
 IWGSC\_CSS\_3AL\_scaff\_442479   & Cadenza0364 &       3198 & C         & T        & het            & het         & gagtcaTtaagttggtaagattggC & gagtcaTtaagttggtaagattggT & GCaGaTaaCaacaggatcacG     \\
 IWGSC\_CSS\_3AL\_scaff\_4447942  & Cadenza0364 &      11917 & G         & A        & het            & het         & gtcataaagattgctcctgtgaaG  & gtcataaagattgctcctgtgaaA  & ctcGgatgtgggaggaagA       \\
 IWGSC\_CSS\_3AS\_scaff\_1557483  & Cadenza0364 &       2547 & C         & T        & het            & het         & aaagtcacatcatgcttaccataaG & aaagtcacatcatgcttaccataaA & cgaaatccaacgcctcatcA      \\
 IWGSC\_CSS\_3AS\_scaff\_2648747  & Cadenza0364 &       2688 & G         & A        & het            & het         & tggAagcAcaaggggccC        & tggAagcAcaaggggccT        & GccgccgatggagactcG        \\
 IWGSC\_CSS\_3AS\_scaff\_3304956  & Cadenza0364 &       1017 & G         & A        & het            & het         & gtcccttgcacacagctttG      & gtcccttgcacacagctttA      & cctgctggactacaacttcaaT    \\
 IWGSC\_CSS\_3AS\_scaff\_3321091  & Cadenza0364 &       4585 & C         & T        & het            & het         & caagaatgATgctgatgttggaG   & caagaatgATgctgatgttggaA   & acatgctgaatcgccgaatC      \\
 IWGSC\_CSS\_3AS\_scaff\_3371333  & Cadenza0364 &        538 & G         & A        & het            & het         & gggaaaCgAgAcgagcgG        & gggaaaCgAgAcgagcgA        & ccgtgccttcctcacccT        \\
 IWGSC\_CSS\_3AS\_scaff\_3371815  & Cadenza0364 &       1061 & C         & T        & het            & het         & atccccacggcacagagG        & atccccacggcacagagA        & aAttggcccttggtgattcC      \\
 IWGSC\_CSS\_3AS\_scaff\_3440912  & Cadenza0364 &       4498 & G         & A        & het            & het         & ccgtaaaactttctgtgcttgC    & ccgtaaaactttctgtgcttgT    & atActgacaaactacatgatgtgC  \\
 IWGSC\_CSS\_3B\_scaff\_10343586  & Cadenza0364 &       2242 & G         & A        & het            & ---         & ggttcTgTcctctcttccactG    & ggttcTgTcctctcttccactA    & tgtgttgaacccgcaagcA       \\
IWGSC\_CSS\_3AL\_scaff\_442479   & Cadenza0364 &       3198 & C         & T        & het            & het         & gagtcaTtaagttggtaagattggC & gagtcaTtaagttggtaagattggT & GCaGaTaaCaacaggatcacG     \\
 IWGSC\_CSS\_3AL\_scaff\_4447942  & Cadenza0364 &      11917 & G         & A        & het            & het         & gtcataaagattgctcctgtgaaG  & gtcataaagattgctcctgtgaaA  & ctcGgatgtgggaggaagA       \\
 IWGSC\_CSS\_3AS\_scaff\_1557483  & Cadenza0364 &       2547 & C         & T        & het            & het         & aaagtcacatcatgcttaccataaG & aaagtcacatcatgcttaccataaA & cgaaatccaacgcctcatcA      \\
 IWGSC\_CSS\_3AS\_scaff\_2648747  & Cadenza0364 &       2688 & G         & A        & het            & het         & tggAagcAcaaggggccC        & tggAagcAcaaggggccT        & GccgccgatggagactcG        \\
 IWGSC\_CSS\_3AS\_scaff\_3304956  & Cadenza0364 &       1017 & G         & A        & het            & het         & gtcccttgcacacagctttG      & gtcccttgcacacagctttA      & cctgctggactacaacttcaaT    \\
 IWGSC\_CSS\_3AS\_scaff\_3321091  & Cadenza0364 &       4585 & C         & T        & het            & het         & caagaatgATgctgatgttggaG   & caagaatgATgctgatgttggaA   & acatgctgaatcgccgaatC      \\
 IWGSC\_CSS\_3AS\_scaff\_3371333  & Cadenza0364 &        538 & G         & A        & het            & het         & gggaaaCgAgAcgagcgG        & gggaaaCgAgAcgagcgA        & ccgtgccttcctcacccT        \\
 IWGSC\_CSS\_3AS\_scaff\_3371815  & Cadenza0364 &       1061 & C         & T        & het            & het         & atccccacggcacagagG        & atccccacggcacagagA        & aAttggcccttggtgattcC      \\
 IWGSC\_CSS\_3AS\_scaff\_3440912  & Cadenza0364 &       4498 & G         & A        & het            & het         & ccgtaaaactttctgtgcttgC    & ccgtaaaactttctgtgcttgT    & atActgacaaactacatgatgtgC  \\
 IWGSC\_CSS\_3B\_scaff\_10343586  & Cadenza0364 &       2242 & G         & A        & het            & ---         & ggttcTgTcctctcttccactG    & ggttcTgTcctctcttccactA    & tgtgttgaacccgcaagcA       \\
 IWGSC\_CSS\_5DL\_scaff\_242342   & Cadenza0281 &       2433 & C         & T        & hom            & hom         & catggCgacggtGtcctG        & catggCgacggtGtcctA        & aAccctcatTTtggCTACTtCT    \\
 IWGSC\_CSS\_5DL\_scaff\_4538822  & Cadenza0281 &       1208 & G         & A        & hom            & ---         & acgtcagaacaaccgtttgaC     & acgtcagaacaaccgtttgaT     & ttaaattggttggcgccacC      \\
 IWGSC\_CSS\_6AL\_scaff\_5813297  & Cadenza0281 &       4532 & C         & T        & hom            & ---         & gggagagggacgtctcgG        & gggagagggacgtctcgA        & ttcttctgccaacgattccG      \\
 IWGSC\_CSS\_6AS\_scaff\_4378990  & Cadenza0281 &       6748 & C         & T        & hom            & hom         & cccaggttctgcttcttttcC     & cccaggttctgcttcttttcT     & caagtatcaagaaaatgaagggTgT \\
 IWGSC\_CSS\_6BL\_scaff\_4360781  & Cadenza0281 &       5426 & C         & T        & het            & het         & aCtactcaaatggcttGgtgtaG   & aCtactcaaatggcttGgtgtaA   & tcagtccaacatgTcaagagatT   \\
 IWGSC\_CSS\_7AL\_scaff\_4488310  & Cadenza0281 &       3808 & G         & A        & hom            & hom         & gttctcttgtagtagcagccG     & gttctcttgtagtagcagccA     & ggcgctttcttcggcctA        \\
 IWGSC\_CSS\_7BL\_scaff\_6696509  & Cadenza0281 &       9232 & G         & A        & het            & het         & gctctaggGgtggcaaAagG      & gctctaggGgtggcaaAagA      & ggcttGaGgtcGcagtgT        \\
 IWGSC\_CSS\_7BS\_scaff\_3143575  & Cadenza0281 &       1866 & C         & T        & het            & het         & agatgttgagagggcgcttC      & agatgttgagagggcgcttT      & gcttggAtggtggcaagtT       \\
 IWGSC\_CSS\_7DL\_scaff\_3346250  & Cadenza0281 &       1663 & G         & A        & het            & het         & acgtgcagcaacatcctaaC      & acgtgcagcaacatcctaaT      & TttcccaccaggcccaagA       \\
 IWGSC\_CSS\_7DS\_scaff\_3933917  & Cadenza0281 &       1243 & C         & T        & het            & het         & tgCtgagcCttTcaccttgC      & tgCtgagcCttTcaccttgT      & agaggtttggttccatcGG       \\
 IWGSC\_CSS\_3B\_scaff\_10626860  & Cadenza0148 &       7847 & G         & A        & het            & het         & gcagctctgggaaggagG        & gcagctctgggaaggagA        & gttaatgtacCTcctagcctcG    \\
 IWGSC\_CSS\_3DL\_scaff\_6915683  & Cadenza0148 &       6904 & C         & T        & het            & het         & cgtcaaCctgtgggcaattG      & cgtcaaCctgtgggcaattA      & tcatgctcataatgTcatagggT   \\
 IWGSC\_CSS\_4AS\_scaff\_5929057  & Cadenza0148 &       4238 & G         & A        & hom            & hom         & gcgcaacgtagCacctacC       & gcgcaacgtagCacctacT       & ttatctggtgaagtgacaggttCA  \\
 IWGSC\_CSS\_4AS\_scaff\_5950625  & Cadenza0148 &      10590 & C         & T        & het            & het         & agaTattCaaaTcggtggAttggC  & agaTattCaaaTcggtggAttggT  & cctgCtcccctcacgtcC        \\
 IWGSC\_CSS\_4AS\_scaff\_5967119  & Cadenza0148 &      11626 & C         & T        & hom            & hom         & cgtGgacaccccgagctG        & cgtGgacaccccgagctA        & gacgacgcactgcacgaC        \\
 IWGSC\_CSS\_4DL\_scaff\_14455742 & Cadenza0148 &       1946 & C         & T        & hom            & hom         & gCctgagggagatcgcgC        & gCctgagggagatcgcgT        & aaccgGtAaCTGtGgGcA        \\
 IWGSC\_CSS\_4DS\_scaff\_2318993  & Cadenza0148 &       4000 & C         & T        & hom            & hom         & tccagtttgacacagattgaatggG & tccagtttgacacagattgaatggA & tgagaTtctgtttcctttcacAttG \\
 IWGSC\_CSS\_5AL\_scaff\_2750707  & Cadenza0148 &       4603 & G         & A        & het            & het         & ccttggtgctagccatttcaagTaG & ccttggtgctagccatttcaagTaA & ccaggaTgcAgtgcaatatttcaaG \\
 IWGSC\_CSS\_5BL\_scaff\_10794137 & Cadenza0148 &       9235 & C         & T        & hom            & hom         & gaagctgcttctgcgttG        & gaagctgcttctgcgttA        & agtatcccttccatataagcagtG  \\
 IWGSC\_CSS\_5BS\_scaff\_1646558  & Cadenza0148 &       2916 & C         & T        & het            & het         & gccGtacactcacctAtcctttG   & gccGtacactcacctAtcctttA   & gcaaTgtccacttAtcatcccT    \\
 IWGSC\_CSS\_1AL\_scaff\_3883106  & Cadenza0110 &      27536 & C         & T        & het            & het         & accttccatcactggctgG       & accttccatcactggctgA       & gtgaagaacaacaggttgaagC    \\
 IWGSC\_CSS\_1BL\_scaff\_3812829  & Cadenza0110 &      10770 & G         & A        & het*           & hom         & cccccactccattccagG        & cccccactccattccagA        & gGatgttgttctgtgctggaA     \\
 IWGSC\_CSS\_1DL\_scaff\_2266648  & Cadenza0110 &       6156 & G         & A        & het            & het         & actgcgtggttatgggacC       & actgcgtggttatgggacT       & ccccatcactgaacacaacA      \\
 IWGSC\_CSS\_1DS\_scaff\_1889435  & Cadenza0110 &       8826 & C         & T        & hom            & hom         & aaccatgaattactcggacagG    & aaccatgaattactcggacagA    & gccctgaagaattgtatcaaaacaG \\
 IWGSC\_CSS\_2AS\_scaff\_5268634  & Cadenza0110 &       4636 & G         & A        & het            & het         & gatccatgtgattggcatgtttG   & gatccatgtgattggcatgtttA   & TgctgtTggatatgcagttacT    \\
 IWGSC\_CSS\_2BL\_scaff\_7965110  & Cadenza0110 &      15801 & C         & T        & hom            & hom         & cattgaagcAtacacAattgcAtaC & cattgaagcAtacacAattgcAtaT & gccagagtatccagataaggTttA  \\
 IWGSC\_CSS\_2DL\_scaff\_9852812  & Cadenza0110 &      13788 & G         & A        & hom            & hom         & atttttgtatggtctcaatcttcgC & atttttgtatggtctcaatcttcgT & gaacgtTcattcttgtacttgcT   \\
 IWGSC\_CSS\_2DS\_scaff\_5371379  & Cadenza0110 &       2166 & C         & T        & hom            & hom         & agacacaaaactagtGatgcgC    & agacacaaaactagtGatgcgT    & gctgctgagaatgttTtgtatttG  \\
 IWGSC\_CSS\_3AL\_scaff\_4384278  & Cadenza0110 &       1276 & C         & T        & het            & het         & agcTgaactgccccTgtaG       & agcTgaactgccccTgtaA       & agggacctCgGtggatgaA       \\
 IWGSC\_CSS\_3AS\_scaff\_3340122  & Cadenza0110 &       1467 & C         & T        & hom            & hom         & attcctAgtgttgtcggaacatG   & attcctAgtgttgtcggaacatA   & gagaagactagaaagttttcAgcaT \\
 IWGSC\_CSS\_5DL\_scaff\_4554222  & Cadenza2103 &       6528 & C         & T        & het*           & hom         & gctgccctacaaagaaacaaaattG & gctgccctacaaagaaacaaaattA & aTcccaactatCGaTtttgtcataC \\
 IWGSC\_CSS\_6AL\_scaff\_5833640  & Cadenza2103 &       7346 & C         & T        & hom            & hom         & aagaaaagccacaatggtttctC   & aagaaaagccacaatggtttctT   & aCTctgTcagtgtttcccagC     \\
 IWGSC\_CSS\_6AS\_scaff\_4429974  & Cadenza2103 &       3867 & G         & A        & hom            & hom         & GagatgaAtttattgagcatgtggC & GagatgaAtttattgagcatgtggT & ggttccggctgcataagT        \\
 IWGSC\_CSS\_6DL\_scaff\_3307626  & Cadenza2103 &       4970 & C         & T        & hom            & hom         & tgcagatgttgtcctgtgtaG     & tgcagatgttgtcctgtgtaA     & ctaggaaggtgattttgtactGtC  \\
 IWGSC\_CSS\_6DS\_scaff\_2059604  & Cadenza2103 &       5224 & G         & A        & het            & ---         & gctcaatgcatgcTgagtgG      & gctcaatgcatgcTgagtgA      & tgtcaagtattattttcctgctctG \\
 IWGSC\_CSS\_7AL\_scaff\_4552322  & Cadenza2103 &       1412 & C         & T        & het            & het         & gcaaaggcTgatactccaacaG    & gcaaaggcTgatactccaacaA    & ggcAAGccAgtataaaagtaaGC   \\
 IWGSC\_CSS\_7BS\_scaff\_3147455  & Cadenza2103 &       4607 & G         & A        & het            & ---         & gcaccttaggatgtgagTtatgC   & gcaccttaggatgtgagTtatgT   & gcatgtagggtttatttgactgttA \\
 IWGSC\_CSS\_7DL\_scaff\_3382467  & Cadenza2103 &       3473 & C         & T        & hom            & ---         & GGTtctgCaGTTCATAActcatC   & GGTtctgCaGTTCATAActcatT   & attgaatcaactgatacGaaGactC \\
 IWGSC\_CSS\_3B\_scaff\_10457010  & Cadenza0277 &      10599 & G         & A        & het            & het         & aaccttggccgcagaacaC       & aaccttggccgcagaacaT       & actggctgcacgagaggG        \\
 IWGSC\_CSS\_3B\_scaff\_10593852  & Cadenza0277 &      10124 & C         & T        & het            & het         & tgacaggggacgctatacaG      & tgacaggggacgctatacaA      & gtctaaCTtACattAcccatcagC  \\
 IWGSC\_CSS\_3DS\_scaff\_2583390  & Cadenza0277 &        663 & G         & A        & hom            & hom         & actgcactcatacaatActtCtgC  & actgcactcatacaatActtCtgT  & tcCacctggacagcaagtG       \\
 IWGSC\_CSS\_4AL\_scaff\_7093953  & Cadenza0277 &      10004 & C         & T        & hom            & hom         & ccttgtattcaatggaTtgTtttgG & ccttgtattcaatggaTtgTtttgA & ttccccaaaTaaaaaggaagagC   \\
 IWGSC\_CSS\_4AL\_scaff\_7176064  & Cadenza0277 &       6220 & C         & T        & het            & het         & gtgccgtaTtcCgcctgG        & gtgccgtaTtcCgcctgA        & atgttcgaggggatgggG        \\
 IWGSC\_CSS\_4DL\_scaff\_14122349 & Cadenza0277 &       1010 & C         & T        & hom            & hom         & gtcgctgctgCttgtgaG        & gtcgctgctgCttgtgaA        & ggaacaggcccaaggagG        \\
 IWGSC\_CSS\_5AL\_scaff\_2736916  & Cadenza0277 &       4296 & G         & A        & het            & het         & aagaactATgAaaGtaacacacgaC & aagaactATgAaaGtaacacacgaT & ttcGcTttTaagGcAttCtcG     \\
 IWGSC\_CSS\_5BL\_scaff\_10883744 & Cadenza0277 &       2080 & C         & T        & hom            & hom         & gcctctttCtgttTagcctcaG    & gcctctttCtgttTagcctcaA    & cgacaaggttcgtgatTgcA      \\
 IWGSC\_CSS\_1AL\_scaff\_3932013  & Cadenza0548 &      11765 & C         & T        & hom            & hom         & accgccaaCccaagacaG        & accgccaaCccaagacaA        & cccattaGccgTgcAacG        \\
 IWGSC\_CSS\_1BS\_scaff\_3417505  & Cadenza0548 &        373 & C         & T        & het            & het         & gtggtgaggaGGgtgGaG        & gtggtgaggaGGgtgGaA        & tggtcgGccagttgttgA        \\
 IWGSC\_CSS\_2AS\_scaff\_5305619  & Cadenza0548 &       2786 & C         & T        & hom            & hom         & atacagatgccctAAgtggTtC    & atacagatgccctAAgtggTtT    & ggaagacaAtGctccaggtaC     \\
 IWGSC\_CSS\_2AS\_scaff\_5306489  & Cadenza0548 &      46953 & T         & G        & het            & wt          & aggttccatgtccatagaagGT    & aggttccatgtccatagaagGG    & aggctaTAgactcctgtACAgT    \\
 IWGSC\_CSS\_2BL\_scaff\_7984123  & Cadenza0548 &      11660 & G         & A        & het            & het         & cattgtggcatagtaatcagtacaG & cattgtggcatagtaatcagtacaA & aatacattgaggaatcaaagccC   \\
 IWGSC\_CSS\_2DL\_scaff\_9907477  & Cadenza0548 &       1363 & C         & T        & hom            & hom         & tgcctccctttgccagaaC       & tgcctccctttgccagaaT       & ggcaaacctgatgtggcatC      \\
 IWGSC\_CSS\_2DS\_scaff\_5330886  & Cadenza0548 &       5449 & G         & A        & hom            & hom         & gcatgtccatttatactgaaCgtG  & gcatgtccatttatactgaaCgtA  & catgctgcttcttctggacC      \\
 IWGSC\_CSS\_3AL\_scaff\_4449951  & Cadenza0548 &        633 & C         & T        & het            & het         & tccaaacctaacagtctaacactaG & tccaaacctaacagtctaacactaA & gtctgcagTGCaatgtgC        \\
 IWGSC\_CSS\_3B\_scaff\_10479889  & Cadenza0097 &       3339 & C         & T        & hom            & ---         & ttgTttctGgagaagatgcCG     & ttgTttctGgagaagatgcCA     & ggtgctcattcaAcGgcA        \\
 IWGSC\_CSS\_3B\_scaff\_10562262  & Cadenza0097 &       7819 & C         & T        & het            & het         & agaggggtgctatccatAttgG    & agaggggtgctatccatAttgA    & agcgatgccaaggcttcC        \\
 IWGSC\_CSS\_4AL\_scaff\_7040796  & Cadenza0097 &      10772 & G         & A        & hom            & hom         & acacaacattgccaccagaG      & acacaacattgccaccagaA      & CAatCgattgcttgctTctcC     \\
 IWGSC\_CSS\_4AL\_scaff\_7063488  & Cadenza0097 &       6360 & C         & T        & het            & het         & gcctctcacCttAatttgaagctgC & gcctctcacCttAatttgaagctgT & aggcagtggagtatgtgaagttT   \\
 IWGSC\_CSS\_4AL\_scaff\_7091701  & Cadenza0097 &       5050 & G         & A        & het            & het         & catgagcatctgggaggaaaatG   & catgagcatctgggaggaaaatA   & agcaagggaAtaatgaacggaaA   \\
 IWGSC\_CSS\_4DS\_scaff\_1845841  & Cadenza0097 &       7110 & G         & A        & hom            & hom         & aatgTAgctccccatacCgG      & aatgTAgctccccatacCgA      & actgaaacTgcaatcgtTtatggA  \\
 IWGSC\_CSS\_5AL\_scaff\_2767581  & Cadenza0097 &       3737 & G         & A        & het            & het         & gagaggtcctcactAtcggC      & gagaggtcctcactAtcggT      & cgTcatcacaaatattgctggG    \\
 IWGSC\_CSS\_5BL\_scaff\_10784643 & Cadenza0097 &       1568 & C         & T        & hom            & hom         & agaaaTAcatggatggatggaCG   & agaaaTAcatggatggatggaCA   & catctcCCttccaCgGaaaG      \\
 IWGSC\_CSS\_1AL\_scaff\_3952258  & Cadenza2092 &       8107 & C         & T        & het            & ---         & tgagtagagaaattgacagtgtgG  & tgagtagagaaattgacagtgtgA  & tgccaccattgacatgagaG      \\
 IWGSC\_CSS\_1BL\_scaff\_3858008  & Cadenza2092 &      10278 & G         & A        & hom            & hom         & tttgagcaggcaggatcgC       & tttgagcaggcaggatcgT       & actcacggcctatatcActattC   \\
 IWGSC\_CSS\_1DL\_scaff\_2265172  & Cadenza2092 &       9094 & C         & T        & hom            & hom         & tgcaTGTcatttgttcttatcagC  & tgcaTGTcatttgttcttatcagT  & agtgtccaacttccGttcatC     \\
 IWGSC\_CSS\_2AL\_scaff\_6435867  & Cadenza2092 &      16201 & G         & A        & hom            & hom         & tttctgTaccttaacgtcaattgaC & tttctgTaccttaacgtcaattgaT & gtgaggatgatgaggtaagacC    \\
 IWGSC\_CSS\_2AL\_scaff\_6439430  & Cadenza2092 &      25101 & C         & T        & het            & ---         & caagaaagggCagCtCagC       & caagaaagggCagCtCagT       & tcGttAcTctttcActggtgaA    \\
 IWGSC\_CSS\_2DL\_scaff\_9760848  & Cadenza2092 &       4733 & C         & T        & het            & het         & gcaccatgggtctcaggtaC      & gcaccatgggtctcaggtaT      & tcagtcagtttGCTCtgTCTG     \\
 IWGSC\_CSS\_3AL\_scaff\_4407012  & Cadenza2092 &       2785 & C         & T        & hom            & hom         & acatatAgtgttctcatccaccatC & acatatAgtgttctcatccaccatT & acctctctcatgttaataggtttgT \\
 IWGSC\_CSS\_3AS\_scaff\_3441108  & Cadenza2092 &        541 & G         & A        & het            & het         & GtgatgaccttgagacGgaG      & GtgatgaccttgagacGgaA      & aggcaTgacaaCgcgcaA        \\
 IWGSC\_CSS\_3B\_scaff\_10449827  & Cadenza1551 &       4779 & G         & A        & hom            & hom         & ggcaaggtcaagaaacGgtC      & ggcaaggtcaagaaacGgtT      & aCagaGtgggttagaggcaG      \\
 IWGSC\_CSS\_3B\_scaff\_10550638  & Cadenza1551 &       3250 & C         & T        & het            & het         & ctccttcacttgttgcggC       & ctccttcacttgttgcggT       & gcaacAtTttgatactgcaaagG   \\
 IWGSC\_CSS\_3DL\_scaff\_6945816  & Cadenza1551 &        589 & C         & T        & hom            & hom         & agcatctcacctgcaaCaataC    & agcatctcacctgcaaCaataT    & TgtgcccTctgaAtattttcaTG   \\
 IWGSC\_CSS\_3DL\_scaff\_6954177  & Cadenza1551 &       3508 & C         & T        & het            & het         & tgtagcatcacattaactttcctG  & tgtagcatcacattaactttcctA  & gcttggtataaaccCttacgacA   \\
 IWGSC\_CSS\_4AS\_scaff\_5938272  & Cadenza1551 &      19080 & G         & A        & hom            & hom         & agAcCccgAtcgccatgG        & agAcCccgAtcgccatgA        & GggAgatAcaggtaaaActcTtcG  \\
 IWGSC\_CSS\_4AS\_scaff\_5977594  & Cadenza1551 &      11092 & C         & T        & het            & het         & gccttgattcggaacaacaaaC    & gccttgattcggaacaacaaaT    & gcgtctctcagtcctgcA        \\
 IWGSC\_CSS\_5AL\_scaff\_2671035  & Cadenza1551 &       5859 & C         & T        & het            & het         & cggtgatattTttagacttcgacgC & cggtgatattTttagacttcgacgT & ggcagttcagcGacccatT       \\
 IWGSC\_CSS\_5BL\_scaff\_10889480 & Cadenza1551 &       2530 & G         & A        & hom            & hom         & gagcttaactcgcagatggaG     & gagcttaactcgcagatggaA     & tccatgCAacGccttggT        \\
 IWGSC\_CSS\_3B\_scaff\_10528396  & Cadenza2088 &       8059 & G         & A        & hom            & ---         & cttttccgtccgtaagcaataG    & cttttccgtccgtaagcaataA    & gtgcactgttcaggcctgA       \\
 IWGSC\_CSS\_3B\_scaff\_10637573  & Cadenza2088 &      16815 & G         & A        & het            & het         & agcaagcttaccGgtctgC       & agcaagcttaccGgtctgT       & cgagcAactacgagcagctT      \\
 IWGSC\_CSS\_4AL\_scaff\_7086469  & Cadenza2088 &       6697 & G         & A        & het            & het         & gccgtctacttcaacgcG        & gccgtctacttcaacgcA        & ccaGaggcttgtTGcattttT     \\
 IWGSC\_CSS\_4AL\_scaff\_7126302  & Cadenza2088 &       3627 & G         & A        & hom            & hom         & gttcaaaaacaagtggctAatttgC & gttcaaaaacaagtggctAatttgT & cacaaggatatgaagcTcttctagA \\
 IWGSC\_CSS\_4BL\_scaff\_7041808  & Cadenza2088 &      10234 & G         & A        & hom            & hom         & tcaatggatgagggtgcttC      & tcaatggatgagggtgcttT      & ccatagcagcatcagccacA      \\
 IWGSC\_CSS\_5AL\_scaff\_2794167  & Cadenza2088 &      13162 & G         & A        & het            & ---         & agtattcaggacaagcatCttCaG  & agtattcaggacaagcatCttCaA  & caatgaaacctctcgaagaaGaG   \\
 IWGSC\_CSS\_5BL\_scaff\_10889232 & Cadenza2088 &       3885 & G         & A        & het            & het         & cTcaaccacaatgggcaAatC     & cTcaaccacaatgggcaAatT     & tccttcatcaatcatcaattgttgG \\
 IWGSC\_CSS\_5BS\_scaff\_2267405  & Cadenza2088 &      11113 & C         & T        & hom            & hom         & ctttgatgatcctaggcctctTG   & ctttgatgatcctaggcctctTA   & tgatttggtCtggttAgagtttGA  \\
 IWGSC\_CSS\_3B\_scaff\_10475354  & Cadenza1409 &       2203 & G         & A        & hom            & hom         & agCgaacaagagGtcaaacG      & agCgaacaagagGtcaaacA      & ctgaaacacaCtagaCAattAccG  \\
 IWGSC\_CSS\_3B\_scaff\_10674115  & Cadenza1409 &       4555 & C         & T        & het            & het         & gcttcagtgcatgccttcaG      & gcttcagtgcatgccttcaA      & cttcacacccGagataatGtattG  \\
 IWGSC\_CSS\_4AL\_scaff\_7153568  & Cadenza1409 &      13073 & C         & T        & hom            & hom         & tccgaccgAtcaaccttgG       & tccgaccgAtcaaccttgA       & gaccggaactcctcggcC        \\
 IWGSC\_CSS\_4DL\_scaff\_14314966 & Cadenza1409 &       2010 & G         & A        & het            & hom         & gtaggtcccctcctCAggG       & gtaggtcccctcctCAggA       & cggcgTcacaAgttgCcT        \\
 IWGSC\_CSS\_4DS\_scaff\_2324074  & Cadenza1409 &       7606 & G         & A        & het            & het         & tGcatgaaaatgtgtGcaGaG     & tGcatgaaaatgtgtGcaGaA     & gggtaAgttcAaaactGaagtgaaG \\
 IWGSC\_CSS\_5AS\_scaff\_1517889  & Cadenza1409 &       3561 & G         & A        & het            & het         & tctcgacatcttcccgtgtaC     & tctcgacatcttcccgtgtaT     & gtgcctggaacattgcttatttA   \\
 IWGSC\_CSS\_5AS\_scaff\_1523866  & Cadenza1409 &       8054 & G         & A        & hom            & ---         & ggtgatctaccgccaGgaC       & ggtgatctaccgccaGgaT       & tcctgcagCcTctcctcA        \\
 IWGSC\_CSS\_5BL\_scaff\_10917655 & Cadenza1409 &      19073 & G         & A        & hom            & hom         & caaatgacatgcaaaagaagttgC  & caaatgacatgcaaaagaagttgT  & cgcttcatcactacaAaatatgtcT \\
 IWGSC\_CSS\_1AL\_scaff\_3886649  & Cadenza1599 &       5204 & C         & T        & het            & het         & tgatgccaaccacaatGcC       & tgatgccaaccacaatGcT       & ggactgactgctgaccatatttaG  \\
 IWGSC\_CSS\_1BL\_scaff\_3810267  & Cadenza1599 &       6634 & C         & T        & hom            & hom         & ccCaggaaatgagcacctC       & ccCaggaaatgagcacctT       & cgcaggcgaagatgtgaTtG      \\
 IWGSC\_CSS\_1DL\_scaff\_2291677  & Cadenza1599 &      12856 & C         & T        & hom            & hom         & GgtagacaagtcgccgaG        & GgtagacaagtcgccgaA        & cctcctccttcaacGCcG        \\
 IWGSC\_CSS\_2AL\_scaff\_6354492  & Cadenza1599 &       7566 & G         & A        & het            & het         & gGagaatgcaCAgtAacTtctgG   & gGagaatgcaCAgtAacTtctgA   & ttccgaagaaccacaTccTG      \\
 IWGSC\_CSS\_2AS\_scaff\_5282937  & Cadenza1599 &       9736 & G         & A        & het            & het         & gctgtagattttatagctgctatgC & gctgtagattttatagctgctatgT & cacCagaattgttCactgatttTC  \\
 IWGSC\_CSS\_2BL\_scaff\_7952427  & Cadenza1599 &      19249 & G         & A        & hom            & hom         & cgTccctCcctagcacgaC       & cgTccctCcctagcacgaT       & aTcactccattagcgcgAG       \\
 IWGSC\_CSS\_2DL\_scaff\_9897981  & Cadenza1599 &       5627 & C         & T        & het            & het         & cttggtgctTgattgcttactC    & cttggtgctTgattgcttactT    & gTttgctCtctctgatctTtgtG   \\
 IWGSC\_CSS\_3AL\_scaff\_4446105  & Cadenza1599 &       1765 & G         & A        & hom            & ---         & aaatgctttcctaCcgctagtG    & aaatgctttcctaCcgctagtA    & ttctAgaggcaatagctTatatgcT \\
\end{longtable}

\end{localsize}
\end{sidewaystable}


\begin{sidewaystable}
\begin{localsize}{6}{7}

\begin{tabular}{llrlllllll}
\toprule
IWGSC contig                 & Line       &   Pos & WT   & Mut   & Predicted   & Called on $M_{4}$    & Primer 1 (Cadenza)        & Primer 2 (mutant)         & Common Primer             \\
\midrule
 IWGSC\_CSS\_3AL\_scaff\_442479   & Cadenza0364 &       3198 & C         & T        & het            & het         & gagtcaTtaagttggtaagattggC & gagtcaTtaagttggtaagattggT & GCaGaTaaCaacaggatcacG     \\
 IWGSC\_CSS\_3AL\_scaff\_4447942  & Cadenza0364 &      11917 & G         & A        & het            & het         & gtcataaagattgctcctgtgaaG  & gtcataaagattgctcctgtgaaA  & ctcGgatgtgggaggaagA       \\
 IWGSC\_CSS\_3AS\_scaff\_1557483  & Cadenza0364 &       2547 & C         & T        & het            & het         & aaagtcacatcatgcttaccataaG & aaagtcacatcatgcttaccataaA & cgaaatccaacgcctcatcA      \\
 IWGSC\_CSS\_3AS\_scaff\_2648747  & Cadenza0364 &       2688 & G         & A        & het            & het         & tggAagcAcaaggggccC        & tggAagcAcaaggggccT        & GccgccgatggagactcG        \\
 IWGSC\_CSS\_3AS\_scaff\_3304956  & Cadenza0364 &       1017 & G         & A        & het            & het         & gtcccttgcacacagctttG      & gtcccttgcacacagctttA      & cctgctggactacaacttcaaT    \\
 IWGSC\_CSS\_3AS\_scaff\_3321091  & Cadenza0364 &       4585 & C         & T        & het            & het         & caagaatgATgctgatgttggaG   & caagaatgATgctgatgttggaA   & acatgctgaatcgccgaatC      \\
 IWGSC\_CSS\_3AS\_scaff\_3371333  & Cadenza0364 &        538 & G         & A        & het            & het         & gggaaaCgAgAcgagcgG        & gggaaaCgAgAcgagcgA        & ccgtgccttcctcacccT        \\
 IWGSC\_CSS\_3AS\_scaff\_3371815  & Cadenza0364 &       1061 & C         & T        & het            & het         & atccccacggcacagagG        & atccccacggcacagagA        & aAttggcccttggtgattcC      \\
 IWGSC\_CSS\_3AS\_scaff\_3440912  & Cadenza0364 &       4498 & G         & A        & het            & het         & ccgtaaaactttctgtgcttgC    & ccgtaaaactttctgtgcttgT    & atActgacaaactacatgatgtgC  \\
 IWGSC\_CSS\_3B\_scaff\_10343586  & Cadenza0364 &       2242 & G         & A        & het            & ---         & ggttcTgTcctctcttccactG    & ggttcTgTcctctcttccactA    & tgtgttgaacccgcaagcA       \\
 IWGSC\_CSS\_5DL\_scaff\_242342   & Cadenza0281 &       2433 & C         & T        & hom            & hom         & catggCgacggtGtcctG        & catggCgacggtGtcctA        & aAccctcatTTtggCTACTtCT    \\
 IWGSC\_CSS\_5DL\_scaff\_4538822  & Cadenza0281 &       1208 & G         & A        & hom            & ---         & acgtcagaacaaccgtttgaC     & acgtcagaacaaccgtttgaT     & ttaaattggttggcgccacC      \\
 IWGSC\_CSS\_6AL\_scaff\_5813297  & Cadenza0281 &       4532 & C         & T        & hom            & ---         & gggagagggacgtctcgG        & gggagagggacgtctcgA        & ttcttctgccaacgattccG      \\
 IWGSC\_CSS\_6AS\_scaff\_4378990  & Cadenza0281 &       6748 & C         & T        & hom            & hom         & cccaggttctgcttcttttcC     & cccaggttctgcttcttttcT     & caagtatcaagaaaatgaagggTgT \\
 IWGSC\_CSS\_6BL\_scaff\_4360781  & Cadenza0281 &       5426 & C         & T        & het            & het         & aCtactcaaatggcttGgtgtaG   & aCtactcaaatggcttGgtgtaA   & tcagtccaacatgTcaagagatT   \\
 IWGSC\_CSS\_7AL\_scaff\_4488310  & Cadenza0281 &       3808 & G         & A        & hom            & hom         & gttctcttgtagtagcagccG     & gttctcttgtagtagcagccA     & ggcgctttcttcggcctA        \\
 IWGSC\_CSS\_7BL\_scaff\_6696509  & Cadenza0281 &       9232 & G         & A        & het            & het         & gctctaggGgtggcaaAagG      & gctctaggGgtggcaaAagA      & ggcttGaGgtcGcagtgT        \\
 IWGSC\_CSS\_7BS\_scaff\_3143575  & Cadenza0281 &       1866 & C         & T        & het            & het         & agatgttgagagggcgcttC      & agatgttgagagggcgcttT      & gcttggAtggtggcaagtT       \\
 IWGSC\_CSS\_7DL\_scaff\_3346250  & Cadenza0281 &       1663 & G         & A        & het            & het         & acgtgcagcaacatcctaaC      & acgtgcagcaacatcctaaT      & TttcccaccaggcccaagA       \\
 IWGSC\_CSS\_7DS\_scaff\_3933917  & Cadenza0281 &       1243 & C         & T        & het            & het         & tgCtgagcCttTcaccttgC      & tgCtgagcCttTcaccttgT      & agaggtttggttccatcGG       \\
 IWGSC\_CSS\_3B\_scaff\_10626860  & Cadenza0148 &       7847 & G         & A        & het            & het         & gcagctctgggaaggagG        & gcagctctgggaaggagA        & gttaatgtacCTcctagcctcG    \\
 IWGSC\_CSS\_3DL\_scaff\_6915683  & Cadenza0148 &       6904 & C         & T        & het            & het         & cgtcaaCctgtgggcaattG      & cgtcaaCctgtgggcaattA      & tcatgctcataatgTcatagggT   \\
 IWGSC\_CSS\_4AS\_scaff\_5929057  & Cadenza0148 &       4238 & G         & A        & hom            & hom         & gcgcaacgtagCacctacC       & gcgcaacgtagCacctacT       & ttatctggtgaagtgacaggttCA  \\
 IWGSC\_CSS\_4AS\_scaff\_5950625  & Cadenza0148 &      10590 & C         & T        & het            & het         & agaTattCaaaTcggtggAttggC  & agaTattCaaaTcggtggAttggT  & cctgCtcccctcacgtcC        \\
 IWGSC\_CSS\_4AS\_scaff\_5967119  & Cadenza0148 &      11626 & C         & T        & hom            & hom         & cgtGgacaccccgagctG        & cgtGgacaccccgagctA        & gacgacgcactgcacgaC        \\
 IWGSC\_CSS\_4DL\_scaff\_14455742 & Cadenza0148 &       1946 & C         & T        & hom            & hom         & gCctgagggagatcgcgC        & gCctgagggagatcgcgT        & aaccgGtAaCTGtGgGcA        \\
 IWGSC\_CSS\_4DS\_scaff\_2318993  & Cadenza0148 &       4000 & C         & T        & hom            & hom         & tccagtttgacacagattgaatggG & tccagtttgacacagattgaatggA & tgagaTtctgtttcctttcacAttG \\
 IWGSC\_CSS\_5AL\_scaff\_2750707  & Cadenza0148 &       4603 & G         & A        & het            & het         & ccttggtgctagccatttcaagTaG & ccttggtgctagccatttcaagTaA & ccaggaTgcAgtgcaatatttcaaG \\
 IWGSC\_CSS\_5BL\_scaff\_10794137 & Cadenza0148 &       9235 & C         & T        & hom            & hom         & gaagctgcttctgcgttG        & gaagctgcttctgcgttA        & agtatcccttccatataagcagtG  \\
 IWGSC\_CSS\_5BS\_scaff\_1646558  & Cadenza0148 &       2916 & C         & T        & het            & het         & gccGtacactcacctAtcctttG   & gccGtacactcacctAtcctttA   & gcaaTgtccacttAtcatcccT    \\
 IWGSC\_CSS\_1AL\_scaff\_3883106  & Cadenza0110 &      27536 & C         & T        & het            & het         & accttccatcactggctgG       & accttccatcactggctgA       & gtgaagaacaacaggttgaagC    \\
 IWGSC\_CSS\_1BL\_scaff\_3812829  & Cadenza0110 &      10770 & G         & A        & het*           & hom         & cccccactccattccagG        & cccccactccattccagA        & gGatgttgttctgtgctggaA     \\
 IWGSC\_CSS\_1DL\_scaff\_2266648  & Cadenza0110 &       6156 & G         & A        & het            & het         & actgcgtggttatgggacC       & actgcgtggttatgggacT       & ccccatcactgaacacaacA      \\
 IWGSC\_CSS\_1DS\_scaff\_1889435  & Cadenza0110 &       8826 & C         & T        & hom            & hom         & aaccatgaattactcggacagG    & aaccatgaattactcggacagA    & gccctgaagaattgtatcaaaacaG \\
 IWGSC\_CSS\_2AS\_scaff\_5268634  & Cadenza0110 &       4636 & G         & A        & het            & het         & gatccatgtgattggcatgtttG   & gatccatgtgattggcatgtttA   & TgctgtTggatatgcagttacT    \\
 IWGSC\_CSS\_2BL\_scaff\_7965110  & Cadenza0110 &      15801 & C         & T        & hom            & hom         & cattgaagcAtacacAattgcAtaC & cattgaagcAtacacAattgcAtaT & gccagagtatccagataaggTttA  \\
 IWGSC\_CSS\_2DL\_scaff\_9852812  & Cadenza0110 &      13788 & G         & A        & hom            & hom         & atttttgtatggtctcaatcttcgC & atttttgtatggtctcaatcttcgT & gaacgtTcattcttgtacttgcT   \\
 IWGSC\_CSS\_2DS\_scaff\_5371379  & Cadenza0110 &       2166 & C         & T        & hom            & hom         & agacacaaaactagtGatgcgC    & agacacaaaactagtGatgcgT    & gctgctgagaatgttTtgtatttG  \\
 IWGSC\_CSS\_3AL\_scaff\_4384278  & Cadenza0110 &       1276 & C         & T        & het            & het         & agcTgaactgccccTgtaG       & agcTgaactgccccTgtaA       & agggacctCgGtggatgaA       \\
 IWGSC\_CSS\_3AS\_scaff\_3340122  & Cadenza0110 &       1467 & C         & T        & hom            & hom         & attcctAgtgttgtcggaacatG   & attcctAgtgttgtcggaacatA   & gagaagactagaaagttttcAgcaT \\
 IWGSC\_CSS\_5DL\_scaff\_4554222  & Cadenza2103 &       6528 & C         & T        & het*           & hom         & gctgccctacaaagaaacaaaattG & gctgccctacaaagaaacaaaattA & aTcccaactatCGaTtttgtcataC \\
 IWGSC\_CSS\_6AL\_scaff\_5833640  & Cadenza2103 &       7346 & C         & T        & hom            & hom         & aagaaaagccacaatggtttctC   & aagaaaagccacaatggtttctT   & aCTctgTcagtgtttcccagC     \\
 IWGSC\_CSS\_6AS\_scaff\_4429974  & Cadenza2103 &       3867 & G         & A        & hom            & hom         & GagatgaAtttattgagcatgtggC & GagatgaAtttattgagcatgtggT & ggttccggctgcataagT        \\
 IWGSC\_CSS\_6DL\_scaff\_3307626  & Cadenza2103 &       4970 & C         & T        & hom            & hom         & tgcagatgttgtcctgtgtaG     & tgcagatgttgtcctgtgtaA     & ctaggaaggtgattttgtactGtC  \\
 IWGSC\_CSS\_6DS\_scaff\_2059604  & Cadenza2103 &       5224 & G         & A        & het            & ---         & gctcaatgcatgcTgagtgG      & gctcaatgcatgcTgagtgA      & tgtcaagtattattttcctgctctG \\
 IWGSC\_CSS\_7AL\_scaff\_4552322  & Cadenza2103 &       1412 & C         & T        & het            & het         & gcaaaggcTgatactccaacaG    & gcaaaggcTgatactccaacaA    & ggcAAGccAgtataaaagtaaGC   \\
 IWGSC\_CSS\_7BS\_scaff\_3147455  & Cadenza2103 &       4607 & G         & A        & het            & ---         & gcaccttaggatgtgagTtatgC   & gcaccttaggatgtgagTtatgT   & gcatgtagggtttatttgactgttA \\
 IWGSC\_CSS\_7DL\_scaff\_3382467  & Cadenza2103 &       3473 & C         & T        & hom            & ---         & GGTtctgCaGTTCATAActcatC   & GGTtctgCaGTTCATAActcatT   & attgaatcaactgatacGaaGactC \\
 IWGSC\_CSS\_3B\_scaff\_10457010  & Cadenza0277 &      10599 & G         & A        & het            & het         & aaccttggccgcagaacaC       & aaccttggccgcagaacaT       & actggctgcacgagaggG        \\
 IWGSC\_CSS\_3B\_scaff\_10593852  & Cadenza0277 &      10124 & C         & T        & het            & het         & tgacaggggacgctatacaG      & tgacaggggacgctatacaA      & gtctaaCTtACattAcccatcagC  \\
 IWGSC\_CSS\_3DS\_scaff\_2583390  & Cadenza0277 &        663 & G         & A        & hom            & hom         & actgcactcatacaatActtCtgC  & actgcactcatacaatActtCtgT  & tcCacctggacagcaagtG       \\
 IWGSC\_CSS\_4AL\_scaff\_7093953  & Cadenza0277 &      10004 & C         & T        & hom            & hom         & ccttgtattcaatggaTtgTtttgG & ccttgtattcaatggaTtgTtttgA & ttccccaaaTaaaaaggaagagC   \\
 IWGSC\_CSS\_4AL\_scaff\_7176064  & Cadenza0277 &       6220 & C         & T        & het            & het         & gtgccgtaTtcCgcctgG        & gtgccgtaTtcCgcctgA        & atgttcgaggggatgggG        \\
 IWGSC\_CSS\_4DL\_scaff\_14122349 & Cadenza0277 &       1010 & C         & T        & hom            & hom         & gtcgctgctgCttgtgaG        & gtcgctgctgCttgtgaA        & ggaacaggcccaaggagG        \\
 IWGSC\_CSS\_5AL\_scaff\_2736916  & Cadenza0277 &       4296 & G         & A        & het            & het         & aagaactATgAaaGtaacacacgaC & aagaactATgAaaGtaacacacgaT & ttcGcTttTaagGcAttCtcG     \\
 IWGSC\_CSS\_5BL\_scaff\_10883744 & Cadenza0277 &       2080 & C         & T        & hom            & hom         & gcctctttCtgttTagcctcaG    & gcctctttCtgttTagcctcaA    & cgacaaggttcgtgatTgcA      \\
 IWGSC\_CSS\_1AL\_scaff\_3932013  & Cadenza0548 &      11765 & C         & T        & hom            & hom         & accgccaaCccaagacaG        & accgccaaCccaagacaA        & cccattaGccgTgcAacG        \\
 IWGSC\_CSS\_1BS\_scaff\_3417505  & Cadenza0548 &        373 & C         & T        & het            & het         & gtggtgaggaGGgtgGaG        & gtggtgaggaGGgtgGaA        & tggtcgGccagttgttgA        \\
 IWGSC\_CSS\_2AS\_scaff\_5305619  & Cadenza0548 &       2786 & C         & T        & hom            & hom         & atacagatgccctAAgtggTtC    & atacagatgccctAAgtggTtT    & ggaagacaAtGctccaggtaC     \\
 IWGSC\_CSS\_2AS\_scaff\_5306489  & Cadenza0548 &      46953 & T         & G        & het            & wt          & aggttccatgtccatagaagGT    & aggttccatgtccatagaagGG    & aggctaTAgactcctgtACAgT    \\
 IWGSC\_CSS\_2BL\_scaff\_7984123  & Cadenza0548 &      11660 & G         & A        & het            & het         & cattgtggcatagtaatcagtacaG & cattgtggcatagtaatcagtacaA & aatacattgaggaatcaaagccC   \\
 IWGSC\_CSS\_2DL\_scaff\_9907477  & Cadenza0548 &       1363 & C         & T        & hom            & hom         & tgcctccctttgccagaaC       & tgcctccctttgccagaaT       & ggcaaacctgatgtggcatC      \\
 IWGSC\_CSS\_2DS\_scaff\_5330886  & Cadenza0548 &       5449 & G         & A        & hom            & hom         & gcatgtccatttatactgaaCgtG  & gcatgtccatttatactgaaCgtA  & catgctgcttcttctggacC      \\
 IWGSC\_CSS\_3AL\_scaff\_4449951  & Cadenza0548 &        633 & C         & T        & het            & het         & tccaaacctaacagtctaacactaG & tccaaacctaacagtctaacactaA & gtctgcagTGCaatgtgC        \\
 IWGSC\_CSS\_3B\_scaff\_10479889  & Cadenza0097 &       3339 & C         & T        & hom            & ---         & ttgTttctGgagaagatgcCG     & ttgTttctGgagaagatgcCA     & ggtgctcattcaAcGgcA        \\
 IWGSC\_CSS\_3B\_scaff\_10562262  & Cadenza0097 &       7819 & C         & T        & het            & het         & agaggggtgctatccatAttgG    & agaggggtgctatccatAttgA    & agcgatgccaaggcttcC        \\
 IWGSC\_CSS\_4AL\_scaff\_7040796  & Cadenza0097 &      10772 & G         & A        & hom            & hom         & acacaacattgccaccagaG      & acacaacattgccaccagaA      & CAatCgattgcttgctTctcC     \\
 IWGSC\_CSS\_4AL\_scaff\_7063488  & Cadenza0097 &       6360 & C         & T        & het            & het         & gcctctcacCttAatttgaagctgC & gcctctcacCttAatttgaagctgT & aggcagtggagtatgtgaagttT   \\
 IWGSC\_CSS\_4AL\_scaff\_7091701  & Cadenza0097 &       5050 & G         & A        & het            & het         & catgagcatctgggaggaaaatG   & catgagcatctgggaggaaaatA   & agcaagggaAtaatgaacggaaA   \\
 IWGSC\_CSS\_4DS\_scaff\_1845841  & Cadenza0097 &       7110 & G         & A        & hom            & hom         & aatgTAgctccccatacCgG      & aatgTAgctccccatacCgA      & actgaaacTgcaatcgtTtatggA  \\
 IWGSC\_CSS\_5AL\_scaff\_2767581  & Cadenza0097 &       3737 & G         & A        & het            & het         & gagaggtcctcactAtcggC      & gagaggtcctcactAtcggT      & cgTcatcacaaatattgctggG    \\
 IWGSC\_CSS\_5BL\_scaff\_10784643 & Cadenza0097 &       1568 & C         & T        & hom            & hom         & agaaaTAcatggatggatggaCG   & agaaaTAcatggatggatggaCA   & catctcCCttccaCgGaaaG      \\
\bottomrule
\end{tabular}

\end{localsize}
\end{sidewaystable}

\begin{sidewaystable}
\begin{localsize}{6}{7}

\begin{tabular}{llrlllllll}
\toprule
IWGSC contig                 & Line       &   Pos & WT   & Mut   & Predicted   & Called on $M_{4}$     & Primer 1 (Cadenza)        & Primer 2 (mutant)         & Common Primer             \\
\midrule
 IWGSC\_CSS\_1AL\_scaff\_3952258  & Cadenza2092 &       8107 & C         & T        & het            & ---         & tgagtagagaaattgacagtgtgG  & tgagtagagaaattgacagtgtgA  & tgccaccattgacatgagaG      \\
 IWGSC\_CSS\_1BL\_scaff\_3858008  & Cadenza2092 &      10278 & G         & A        & hom            & hom         & tttgagcaggcaggatcgC       & tttgagcaggcaggatcgT       & actcacggcctatatcActattC   \\
 IWGSC\_CSS\_1DL\_scaff\_2265172  & Cadenza2092 &       9094 & C         & T        & hom            & hom         & tgcaTGTcatttgttcttatcagC  & tgcaTGTcatttgttcttatcagT  & agtgtccaacttccGttcatC     \\
 IWGSC\_CSS\_2AL\_scaff\_6435867  & Cadenza2092 &      16201 & G         & A        & hom            & hom         & tttctgTaccttaacgtcaattgaC & tttctgTaccttaacgtcaattgaT & gtgaggatgatgaggtaagacC    \\
 IWGSC\_CSS\_2AL\_scaff\_6439430  & Cadenza2092 &      25101 & C         & T        & het            & ---         & caagaaagggCagCtCagC       & caagaaagggCagCtCagT       & tcGttAcTctttcActggtgaA    \\
 IWGSC\_CSS\_2DL\_scaff\_9760848  & Cadenza2092 &       4733 & C         & T        & het            & het         & gcaccatgggtctcaggtaC      & gcaccatgggtctcaggtaT      & tcagtcagtttGCTCtgTCTG     \\
 IWGSC\_CSS\_3AL\_scaff\_4407012  & Cadenza2092 &       2785 & C         & T        & hom            & hom         & acatatAgtgttctcatccaccatC & acatatAgtgttctcatccaccatT & acctctctcatgttaataggtttgT \\
 IWGSC\_CSS\_3AS\_scaff\_3441108  & Cadenza2092 &        541 & G         & A        & het            & het         & GtgatgaccttgagacGgaG      & GtgatgaccttgagacGgaA      & aggcaTgacaaCgcgcaA        \\
 IWGSC\_CSS\_3B\_scaff\_10449827  & Cadenza1551 &       4779 & G         & A        & hom            & hom         & ggcaaggtcaagaaacGgtC      & ggcaaggtcaagaaacGgtT      & aCagaGtgggttagaggcaG      \\
 IWGSC\_CSS\_3B\_scaff\_10550638  & Cadenza1551 &       3250 & C         & T        & het            & het         & ctccttcacttgttgcggC       & ctccttcacttgttgcggT       & gcaacAtTttgatactgcaaagG   \\
 IWGSC\_CSS\_3DL\_scaff\_6945816  & Cadenza1551 &        589 & C         & T        & hom            & hom         & agcatctcacctgcaaCaataC    & agcatctcacctgcaaCaataT    & TgtgcccTctgaAtattttcaTG   \\
 IWGSC\_CSS\_3DL\_scaff\_6954177  & Cadenza1551 &       3508 & C         & T        & het            & het         & tgtagcatcacattaactttcctG  & tgtagcatcacattaactttcctA  & gcttggtataaaccCttacgacA   \\
 IWGSC\_CSS\_4AS\_scaff\_5938272  & Cadenza1551 &      19080 & G         & A        & hom            & hom         & agAcCccgAtcgccatgG        & agAcCccgAtcgccatgA        & GggAgatAcaggtaaaActcTtcG  \\
 IWGSC\_CSS\_4AS\_scaff\_5977594  & Cadenza1551 &      11092 & C         & T        & het            & het         & gccttgattcggaacaacaaaC    & gccttgattcggaacaacaaaT    & gcgtctctcagtcctgcA        \\
 IWGSC\_CSS\_5AL\_scaff\_2671035  & Cadenza1551 &       5859 & C         & T        & het            & het         & cggtgatattTttagacttcgacgC & cggtgatattTttagacttcgacgT & ggcagttcagcGacccatT       \\
 IWGSC\_CSS\_5BL\_scaff\_10889480 & Cadenza1551 &       2530 & G         & A        & hom            & hom         & gagcttaactcgcagatggaG     & gagcttaactcgcagatggaA     & tccatgCAacGccttggT        \\
 IWGSC\_CSS\_3B\_scaff\_10528396  & Cadenza2088 &       8059 & G         & A        & hom            & ---         & cttttccgtccgtaagcaataG    & cttttccgtccgtaagcaataA    & gtgcactgttcaggcctgA       \\
 IWGSC\_CSS\_3B\_scaff\_10637573  & Cadenza2088 &      16815 & G         & A        & het            & het         & agcaagcttaccGgtctgC       & agcaagcttaccGgtctgT       & cgagcAactacgagcagctT      \\
 IWGSC\_CSS\_4AL\_scaff\_7086469  & Cadenza2088 &       6697 & G         & A        & het            & het         & gccgtctacttcaacgcG        & gccgtctacttcaacgcA        & ccaGaggcttgtTGcattttT     \\
 IWGSC\_CSS\_4AL\_scaff\_7126302  & Cadenza2088 &       3627 & G         & A        & hom            & hom         & gttcaaaaacaagtggctAatttgC & gttcaaaaacaagtggctAatttgT & cacaaggatatgaagcTcttctagA \\
 IWGSC\_CSS\_4BL\_scaff\_7041808  & Cadenza2088 &      10234 & G         & A        & hom            & hom         & tcaatggatgagggtgcttC      & tcaatggatgagggtgcttT      & ccatagcagcatcagccacA      \\
 IWGSC\_CSS\_5AL\_scaff\_2794167  & Cadenza2088 &      13162 & G         & A        & het            & ---         & agtattcaggacaagcatCttCaG  & agtattcaggacaagcatCttCaA  & caatgaaacctctcgaagaaGaG   \\
 IWGSC\_CSS\_5BL\_scaff\_10889232 & Cadenza2088 &       3885 & G         & A        & het            & het         & cTcaaccacaatgggcaAatC     & cTcaaccacaatgggcaAatT     & tccttcatcaatcatcaattgttgG \\
 IWGSC\_CSS\_5BS\_scaff\_2267405  & Cadenza2088 &      11113 & C         & T        & hom            & hom         & ctttgatgatcctaggcctctTG   & ctttgatgatcctaggcctctTA   & tgatttggtCtggttAgagtttGA  \\
 IWGSC\_CSS\_3B\_scaff\_10475354  & Cadenza1409 &       2203 & G         & A        & hom            & hom         & agCgaacaagagGtcaaacG      & agCgaacaagagGtcaaacA      & ctgaaacacaCtagaCAattAccG  \\
 IWGSC\_CSS\_3B\_scaff\_10674115  & Cadenza1409 &       4555 & C         & T        & het            & het         & gcttcagtgcatgccttcaG      & gcttcagtgcatgccttcaA      & cttcacacccGagataatGtattG  \\
 IWGSC\_CSS\_4AL\_scaff\_7153568  & Cadenza1409 &      13073 & C         & T        & hom            & hom         & tccgaccgAtcaaccttgG       & tccgaccgAtcaaccttgA       & gaccggaactcctcggcC        \\
 IWGSC\_CSS\_4DL\_scaff\_14314966 & Cadenza1409 &       2010 & G         & A        & het            & hom         & gtaggtcccctcctCAggG       & gtaggtcccctcctCAggA       & cggcgTcacaAgttgCcT        \\
 IWGSC\_CSS\_4DS\_scaff\_2324074  & Cadenza1409 &       7606 & G         & A        & het            & het         & tGcatgaaaatgtgtGcaGaG     & tGcatgaaaatgtgtGcaGaA     & gggtaAgttcAaaactGaagtgaaG \\
 IWGSC\_CSS\_5AS\_scaff\_1517889  & Cadenza1409 &       3561 & G         & A        & het            & het         & tctcgacatcttcccgtgtaC     & tctcgacatcttcccgtgtaT     & gtgcctggaacattgcttatttA   \\
 IWGSC\_CSS\_5AS\_scaff\_1523866  & Cadenza1409 &       8054 & G         & A        & hom            & ---         & ggtgatctaccgccaGgaC       & ggtgatctaccgccaGgaT       & tcctgcagCcTctcctcA        \\
 IWGSC\_CSS\_5BL\_scaff\_10917655 & Cadenza1409 &      19073 & G         & A        & hom            & hom         & caaatgacatgcaaaagaagttgC  & caaatgacatgcaaaagaagttgT  & cgcttcatcactacaAaatatgtcT \\
 IWGSC\_CSS\_1AL\_scaff\_3886649  & Cadenza1599 &       5204 & C         & T        & het            & het         & tgatgccaaccacaatGcC       & tgatgccaaccacaatGcT       & ggactgactgctgaccatatttaG  \\
 IWGSC\_CSS\_1BL\_scaff\_3810267  & Cadenza1599 &       6634 & C         & T        & hom            & hom         & ccCaggaaatgagcacctC       & ccCaggaaatgagcacctT       & cgcaggcgaagatgtgaTtG      \\
 IWGSC\_CSS\_1DL\_scaff\_2291677  & Cadenza1599 &      12856 & C         & T        & hom            & hom         & GgtagacaagtcgccgaG        & GgtagacaagtcgccgaA        & cctcctccttcaacGCcG        \\
 IWGSC\_CSS\_2AL\_scaff\_6354492  & Cadenza1599 &       7566 & G         & A        & het            & het         & gGagaatgcaCAgtAacTtctgG   & gGagaatgcaCAgtAacTtctgA   & ttccgaagaaccacaTccTG      \\
 IWGSC\_CSS\_2AS\_scaff\_5282937  & Cadenza1599 &       9736 & G         & A        & het            & het         & gctgtagattttatagctgctatgC & gctgtagattttatagctgctatgT & cacCagaattgttCactgatttTC  \\
 IWGSC\_CSS\_2BL\_scaff\_7952427  & Cadenza1599 &      19249 & G         & A        & hom            & hom         & cgTccctCcctagcacgaC       & cgTccctCcctagcacgaT       & aTcactccattagcgcgAG       \\
 IWGSC\_CSS\_2DL\_scaff\_9897981  & Cadenza1599 &       5627 & C         & T        & het            & het         & cttggtgctTgattgcttactC    & cttggtgctTgattgcttactT    & gTttgctCtctctgatctTtgtG   \\
 IWGSC\_CSS\_3AL\_scaff\_4446105  & Cadenza1599 &       1765 & G         & A        & hom            & ---         & aaatgctttcctaCcgctagtG    & aaatgctttcctaCcgctagtA    & ttctAgaggcaatagctTatatgcT \\
\bottomrule
\end{tabular}

\end{localsize}
\end{sidewaystable}

%!TEX root = ../Main.tex
\chapter[Quality control]{Quality Control}

% the \\ insures the section title is centered below the phrase: Appendix B
\section{Sequence read quality}
 \label{App:AppendixQCRead}
 
\begin{center}
\begin{tabular}{ccc}
\toprule
Sample  & Read 1 & Read 2 \\ \midrule 
\\
\begin{sideways}LIB1721 AvocetS\end{sideways} & \includegraphics[height=5cm]{Appendices/images/Sample_LIB1721_base_quality_R1.png} & \includegraphics[height=5cm]{Appendices/images/Sample_LIB1721_base_quality_R2.png} \\ \midrule  \\
\begin{sideways}LIB1722 AvocetS(Yr15)\end{sideways} & \includegraphics[height=5cm]{Appendices/images/Sample_LIB1722_base_quality_R1.png} & \includegraphics[height=5cm]{Appendices/images/Sample_LIB1722_base_quality_R2.png} \\   

\bottomrule
\end{tabular}
\end{center}
\begin{center}
\begin{tabular}{ccc}
\toprule
Sample  & Read 1 & Read 2 \\ \midrule 
\\
\begin{sideways}LIB1715 Bulk R1\end{sideways} & \includegraphics[height=5cm]{Appendices/images/Sample_LIB1715_base_quality_R1.png} & \includegraphics[height=5cm]{Appendices/images/Sample_LIB1715_base_quality_R2.png} \\ \midrule  \\
\begin{sideways}LIB1716 Bulk R2\end{sideways} & \includegraphics[height=5cm]{Appendices/images/Sample_LIB1716_base_quality_R1.png} & \includegraphics[height=5cm]{Appendices/images/Sample_LIB1716_base_quality_R2.png} \\ \midrule  \\
\begin{sideways}LIB1717 Bulk R3\end{sideways} & \includegraphics[height=5cm]{Appendices/images/Sample_LIB1717_base_quality_R1.png} & \includegraphics[height=5cm]{Appendices/images/Sample_LIB1717_base_quality_R2.png} \\ \midrule  \\

\end{tabular}
\end{center}

\begin{center}
\begin{tabular}{ccc}
\toprule
Sample  & Read 1 & Read 2 \\ \midrule 
\\
\begin{sideways}LIB1718 Bulk S1\end{sideways} & \includegraphics[height=5cm]{Appendices/images/Sample_LIB1718_base_quality_R1.png} & \includegraphics[height=5cm]{Appendices/images/Sample_LIB1718_base_quality_R2.png} \\ \midrule  \\
\begin{sideways}LIB1719 Bulk S2\end{sideways} & \includegraphics[height=5cm]{Appendices/images/Sample_LIB1719_base_quality_R1.png} & \includegraphics[height=5cm]{Appendices/images/Sample_LIB1719_base_quality_R2.png} \\ \midrule  \\
\begin{sideways}LIB1720 Bulk S3 \end{sideways} & \includegraphics[height=5cm]{Appendices/images/Sample_LIB1720_base_quality_R1.png} & \includegraphics[height=5cm]{Appendices/images/Sample_LIB1720_base_quality_R2.png} \\ \midrule  \\

\end{tabular}
\end{center}



\newpage


\section{Sequence GC content}
 \label{App:AppendixQCGC}
\begin{center}
\begin{tabular}{ccc}
\toprule
Sample  & Read 1 & Read 2 \\ \midrule 
\\
\begin{sideways}LIB1721 AvocetS\end{sideways} & \includegraphics[height=5cm]{Appendices/images/Sample_LIB1721_base_gc_R1.png} & \includegraphics[height=5cm]{Appendices/images/Sample_LIB1721_base_gc_R2.png} \\ \midrule  \\
\begin{sideways}LIB1722 AvocetS(Yr15)\end{sideways} & \includegraphics[height=5cm]{Appendices/images/Sample_LIB1722_base_gc_R1.png} & \includegraphics[height=5cm]{Appendices/images/Sample_LIB1722_base_gc_R2.png} \\   

\bottomrule
\end{tabular}
\end{center}

\begin{center}
\begin{tabular}{ccc}
\toprule
Sample  & Read 1 & Read 2 \\ \midrule 
\\
\begin{sideways}LIB1715 Bulk R1\end{sideways} & \includegraphics[height=5cm]{Appendices/images/Sample_LIB1715_base_gc_R1.png} & \includegraphics[height=5cm]{Appendices/images/Sample_LIB1715_base_gc_R2.png} \\ \midrule  \\
\begin{sideways}LIB1716 Bulk R2\end{sideways} & \includegraphics[height=5cm]{Appendices/images/Sample_LIB1716_base_gc_R1.png} & \includegraphics[height=5cm]{Appendices/images/Sample_LIB1716_base_gc_R2.png} \\ \midrule  \\
\begin{sideways}LIB1717 Bulk R3\end{sideways} & \includegraphics[height=5cm]{Appendices/images/Sample_LIB1717_base_gc_R1.png} & \includegraphics[height=5cm]{Appendices/images/Sample_LIB1717_base_gc_R2.png} \\ \midrule  \\

\end{tabular}
\end{center}

\begin{center}
\begin{tabular}{ccc}
\toprule
Sample  & Read 1 & Read 2 \\ \midrule 
\\
\begin{sideways}LIB1718 Bulk S1\end{sideways} & \includegraphics[height=5cm]{Appendices/images/Sample_LIB1718_base_gc_R1.png} & \includegraphics[height=5cm]{Appendices/images/Sample_LIB1718_base_gc_R2.png} \\ \midrule  \\
\begin{sideways}LIB1719 Bulk S2\end{sideways} & \includegraphics[height=5cm]{Appendices/images/Sample_LIB1719_base_gc_R1.png} & \includegraphics[height=5cm]{Appendices/images/Sample_LIB1719_base_gc_R2.png} \\ \midrule  \\
\begin{sideways}LIB1720 Bulk S3 \end{sideways} & \includegraphics[height=5cm]{Appendices/images/Sample_LIB1720_base_gc_R1.png} & \includegraphics[height=5cm]{Appendices/images/Sample_LIB1720_base_gc_R2.png} \\ \midrule  \\

\end{tabular}
\end{center}



%!TEX root = ../Main.tex

\chapter{expVIP tutorial}
\label{exp:tutorial}
\includepdf[pages={1-},scale=0.90]{expVIP/tutorial/TutorialSinglePage.pdf}

\bibliographystyle{plainnat_short}
\bibliography{References}

\end{document}
